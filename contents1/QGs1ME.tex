%量子气体(单能级巨正则系综法)

我们可以把一个包含许多粒子的系统看做热池, 把每个能级 ${\varepsilon_i}$ 看做一个系统. 为了便于理解, 可以把能级想象成一个盒子, 所有处于该能级的粒子都在盒内, 都具有能量 ${\varepsilon_i}$.  当 ${\varepsilon_i}$ 系统中粒子数为 ${n_i}$ 时, 系统的总能量为 $E = {n_i}{\varepsilon_i}$.  注意对于一个 ${n_i}$,  由于同种粒子不可区分, 系统只有一种状态, 所以在当前系统的巨配分函数中, 对能量的求和只有一项.
\begin{equation}
  \begin{aligned}
  \Xi & = \sum\limits_{n_i}^{} {\sum\limits_{E_j}^{} {\E^{\left( {{n_i}\mu  - {E_j}} \right)\beta }} }  = \sum\limits_{n_i}^{} {\E^{\left( {{n_i}\mu  - E} \right)\beta }}  \\
  & = \sum\limits_{n_i}^{} {\E^{\left( {{n_i}\mu  - {n_i}{\varepsilon_i}} \right)\beta }}  = \sum\limits_{n_i}^{} {{\left[ {\E^{\left( {\mu  - {\varepsilon_i}} \right)\beta }} \right]}^{n_i}}
  \end{aligned}
\end{equation}
系统( ${\varepsilon_i}$ 能级)中的平均粒子数为
\begin{equation}
\begin{aligned}
  \left\langle {n_i} \right\rangle  & = \frac{1}{\Xi }\sum\limits_{n_i}^{} {\sum\limits_{E_j}^{} {{n_i}{\E^{\left( {{n_i}\mu  - {E_j}} \right)\beta }}} }  \\
  & = \frac{1}{\Xi }\sum\limits_{n_i}^{} {{n_i}{{\left[ {\E^{\left( {\mu  - {\varepsilon_i}} \right)\beta }} \right]}^{n_i}}}
\end{aligned}
\end{equation}
\subsection{费米子}
由于泡利不相容原理, 一个能级只能存在 $0$ 或 $1$ 个费米子(这里忽略自旋).
\begin{equation}
  \Xi  = \sum\limits_{{n_i} = 0}^1 {{\left[ {\E^{\left( {\mu  - {\varepsilon_i}} \right)\beta }} \right]}^{n_i}}  = 1 + {\E^{\left( {\mu  - {\varepsilon_i}} \right)\beta }}
\end{equation}
 ${\varepsilon_i}$ 能级的平均粒子数为
\begin{equation}
\begin{aligned}
\left\langle {n_i} \right\rangle & = \frac{1}{\Xi }\sum\limits_{{n_i} = 0}^1 {{n_i}{{\left[ {\E^{\left( {\mu  - {\varepsilon_i}} \right)\beta }} \right]}^{n_i}}} = \frac{0 + {\E^{\left( {\mu  - {\varepsilon_i}} \right)\beta }}}{1 + {\E^{\left( {\mu  - {\varepsilon_i}} \right)\beta }}}  = \frac{1}{{\E^{\left( {{\varepsilon_i} - \mu } \right)\beta }} + 1}
\end{aligned}
\end{equation}
  这就是著名的\bb{费米—狄拉克分布}.
\subsection{波色子} 
任何能级都允许同时存在任意数量的玻色子, 所以上面两式中对 ${n_i}$ 的求和上限变为正无穷即可(见等比数列求和%链接未完成
以及类等比数列求和%连接未完成
). 但为了使求和收敛, 必须要求 ${\E^{\left( {{\varepsilon_i} - \mu } \right)\beta }} - 1 > 0$,  或者 $\mu  < {\varepsilon_i}$. 
\begin{equation}
  \Xi  = \sum\limits_{{n_i} = 0}^\infty  {{\left[ {\E^{\left( {\mu  - {\varepsilon_i}} \right)\beta }} \right]}^{n_i}}  = \frac{1}{1 - {\E^{\left( {\mu  - {\varepsilon_i}} \right)\beta }}}
\end{equation}
\begin{equation}
\begin{aligned}
  \left\langle {n_i} \right\rangle & = \frac{1}{\Xi }\sum\limits_{{n_i} = 0}^1 {{n_i}{{\left[ {\E^{\left( {\mu  - {\varepsilon_i}} \right)\beta }} \right]}^{n_i}}} = \left[ {1 - {\E^{\left( {\mu  - {\varepsilon_i}} \right)\beta }}} \right]\frac{\E^{\left( {\mu  - {\varepsilon_i}} \right)\beta }}{{\left[ {1 - {\E^{\left( {\mu  - {\varepsilon_i}} \right)\beta }}} \right]}^2}  = \frac{1}{{\E^{\left( {{\varepsilon_i} - \mu } \right)\beta }} - 1}
  \end{aligned}
\end{equation}
这就是著名的\bb{波色—爱因斯坦分布}.\\
当每个能级的平均粒子数 $\left\langle {{n_i}} \right\rangle $ 都很小时, 即 $\left\langle {{n_i}} \right\rangle \ll 1$ 时, $\left\langle {{n_i}} \right\rangle  = {1}/({{{\E^{\left( {{\varepsilon_i} - \mu } \right)\beta }} \pm 1}})$ 的分母 $ \gg 1$,  分布可以近似为
\begin{equation}
  \left\langle {n_i} \right\rangle  = \frac{1}{\E^{\left( {{\varepsilon_i} - \mu } \right)\beta }} = {\E^{\left( {\mu  - {\varepsilon_i}} \right)\beta }}
\end{equation}
这就是\bb{麦克斯韦—玻尔兹曼分布}, 对应理想气体. 由此可见, 当 % 未完成???
该分布的总粒子数为
\begin{equation}
  N = \sum\limits_i^\infty  {\left\langle {n_i} \right\rangle }  = {\E^{\mu \beta }}\sum\limits_i^\infty  {\E^{ - {\varepsilon_i}\beta }}  = z{Q_1}
\end{equation}
为了验证该式的正确性, 代入理想气体的化学势和单粒子配分函数, 上式成立.
\begin{equation}
  \mu  = kT\ln \frac{N{\lambda ^3}}{V} 
  \qquad
  {Q_1} = \frac{V}{\lambda ^3}
\end{equation}
这种方法虽然可以简单地求出分布函数, 但却不能求出其他物理量, 例如量子气体的压强, 熵, 等. 因为我们的系统只包含一个能级, 而不是大量粒子. 要使用标准的巨正则系综, 必须把包含大量粒子的量子气体作为系统, 并考虑每个粒子数对应的所有可能的能级分布.\\