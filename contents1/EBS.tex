%电磁场的能量守恒 坡印廷矢量

%未完成
%(这个概念太神奇了, 要好好介绍其实电荷的势能不是由电荷本身携带, 而是通过电磁场传播的). 举例: 同轴电缆. 不要说介质, 太复杂了, 或者顺带提一下就行)

%“转换速率+流出速率+增加速率=0”
\pentry{麦克斯韦方程组,%未完成
电场的能量,%未完成
磁场的能量%未完成
}


\subsection{结论}
\begin{enumerate}
\item 坡印廷矢量

真空中电磁场的能流密度为
\begin{equation}
\vec s = \frac{1}{{{\mu _0}}}\vec E \cross \vec B
\end{equation} 
 $\vec s$ 就是坡印廷矢量.


\item 电磁场能量守恒积分形式
\begin{equation}
\int\limits_V {\frac{{\D w}}{{\D t}}\D V}  + \frac{\partial }{{\partial t}}\int\limits_V {{\rho _E}\D V}  + \oint\limits_\Omega  {\vec s \vdot \D \vec a}  = 0
\end{equation} 
选取任意的一个闭合曲面 $\Omega $, 内部空间记为 $V$, 以下三者之和为零.
\begin{enumerate}
\item 电磁场对 $V$ 中所有电荷做功的功率
\item $V$ 中电磁场能量增加的速率
\item 以及通过曲面 $\Omega $ 流出的能量的速率
\end{enumerate}

\item 电磁场能量守恒微分形式
\begin{equation}
\frac{{\D w}}{{\D t}} + \frac{{\partial \rho }}{{\partial t}} + \div \vec s = 0
\end{equation} 
空间中选取任意一点, 以下三者之和为零.
\begin{enumerate}
\item 电磁场对电荷的功率密度
\item 电磁场能量密度增量
\item 能流密度散度
\end{enumerate}
\end{enumerate}

\subsection{推导}
类比电流的连续性方程%未完成:引用
(即电荷守恒),若电磁场不对电荷做功,能量守恒可以写成
\begin{equation}
\frac{{\partial {\rho _E}}}{{\partial t}} + \div \vec s = 0
\end{equation} 
的形式.其中 $\vec s$ 是电磁场的能流密度(也叫\textbf{坡印廷矢量})(参考流密度%未完成:引用
).但若再考虑上电磁场对电荷做功, 则还需要加上一项做功做功功率密度 $\partial w/\partial t$, 即单位时间单位体积电磁场对电荷做的功).
\begin{equation}\label{EBS_eq1}
\frac{{\partial w}}{{\partial t}} + \frac{{\partial {\rho _E}}}{{\partial t}} + \div \vec s = 0
\end{equation} 

第一项中电磁场对电荷做功即广义洛伦兹力%未完成:引用
做功(功率密度)
\begin{equation}
\pdv{w}{t} = \vec f \vdot \vec v = \rho \left( {\vec E + \vec v \cross \vec B} \right) \vdot \vec v = \vec j \vdot \vec E
\end{equation} 
假设电磁场的能量守恒\autoref{EBS_eq1} 成立, 那么 $\vec j \vdot \vec E =  - \pdv*{\rho_E}{t} - \div \vec s$. 等式右边只与场有关, 所以应该把电流密度 $\vec j$ 用麦克斯韦方程组%未完成:引用
替换成场的表达式, 即
\begin{equation}
\vec j = \frac{1}{{{\mu _0}}}\curl \vec B - {\epsilon _0}\frac{{\partial \vec E}}{{\partial t}}
\end{equation} 
代入得
\begin{equation}\label{EBS_eq2}
\begin{aligned}
\pdv{w}{t} &= \left( {\frac{1}{{{\mu _0}}}\curl \vec B - {\epsilon _0}\frac{{\partial \vec E}}{{\partial t}}} \right) \vdot \vec E\\
&= \frac{1}{{{\mu _0}}}\left( {\curl \vec B} \right)\vec E - {\epsilon _0}\frac{{\partial \vec E}}{{\partial t}}\vec E
\end{aligned}
\end{equation} 
\autoref{EBS_eq1} 第二项中, ${\rho _E}$ 是电场能量密度和磁场能量密度之和, 即
\begin{equation}\label{EBS_eq3}
\begin{aligned}
\curl \left( {\vec B \cross \uvec x} \right) &= \vec B\left( {\div \uvec x} \right) + \left( {x\vec\nabla } \right)\vdot\vec B - \uvec x\left( {\div \vec B} \right) - \left( {\vec B\vdot\vec\nabla } \right)\uvec x \\
&= \left( {x\vec\nabla } \right)\vdot\vec B = \frac{{\partial \vec B}}{{\partial x}}
\end{aligned}\end{equation} 
现在我们可以把\autoref{EBS_eq2},\autoref{EBS_eq3} 代入\autoref{EBS_eq1} 中, 求出 $\div \vec s$. 
\begin{equation}
\begin{aligned}
\div \vec s &=  - \frac{{\partial w}}{{\partial t}} - \frac{{\partial {\rho _E}}}{{\partial t}}\\
&=  - \frac{1}{{{\mu _0}}}\left( {\curl \vec B} \right)\vec E + {\epsilon _0}\frac{{\partial \vec E}}{{\partial t}}\vec E - \frac{\partial }{{\partial t}}\left( {\frac{1}{2}{\epsilon _0}{{\vec E}^2} + \frac{1}{2}\frac{{{{\vec B}^2}}}{{{\mu _0}}}} \right)\\
&=  - \frac{1}{{{\mu _0}}}\left( {\curl \vec B} \right)\vec E - \frac{1}{{{\mu _0}}}\frac{{\partial \vec B}}{{\partial t}}\vec B
\end{aligned}
\end{equation} 
其中 $\left( {\curl \vec B} \right) \vdot \vec E = \left( {\curl \vec E} \right) \vdot \vec B - \div \left( {\vec E \cross \vec B} \right)$, 因为 $\div \left( {\vec E \cross \vec B} \right) = \left( {\curl \vec E} \right) \vdot \vec B - \left( {\curl \vec B} \right) \vdot \vec E$(Gibbs算子相关公式%未完成:引用
).代入得
\begin{equation}
\begin{aligned}
\div \vec s &=  - \frac{{\partial w}}{{\partial t}} - \frac{{\partial {\rho _E}}}{{\partial t}}\\
&=  - \frac{1}{{{\mu _0}}}\left( {\curl \vec B} \right)\vec E + {\epsilon _0}\frac{{\partial \vec E}}{{\partial t}}\vec E - \frac{\partial }{{\partial t}}\left( {\frac{1}{2}{\epsilon _0}{{\vec E}^2} + \frac{1}{2}\frac{{{{\vec B}^2}}}{{{\mu _0}}}} \right)\\
&=  - \frac{1}{{{\mu _0}}}\left( {\curl \vec E} \right) \vdot \vec B - \frac{1}{{{\mu _0}}}\frac{{\partial \vec B}}{{\partial t}}\vec B + \frac{1}{{{\mu _0}}}\div \left( {\vec E \cross \vec B} \right)
\end{aligned}
\end{equation} 
其中 $\curl \vec E =  -\pdv*{\vec B}{t}$, 代入得
\begin{equation}\label{EBS_eq4}
\div \vec s = \div \left( {\frac{1}{{{\mu _0}}}\vec E \cross \vec B} \right)
\end{equation} 
即
\begin{equation}\label{EBS_eq5}
\vec s = \frac{1}{{{\mu _0}}}\vec E \cross \vec B
\end{equation} 
这就是电磁场的能流密度.

事实上, 给 $\vec s$ 再加上任意一个散度为零的场,\autoref{EBS_eq4} 都能满足, 但为了简洁起见, 一般写成\autoref{EBS_eq5}. 


% \subsection{电荷的势能传递}

%未完成
% 其实所谓电荷的势能根本就不是由电荷携带并传递, 而是通过电磁场来传递的呀!


