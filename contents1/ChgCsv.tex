%电荷守恒 电流连续性方程
总电荷守恒: 一个封闭系统的总电荷保持不变(正反粒子带有不同的电荷, 可以成对创造或湮灭, 但总电荷仍然保持不变).

然而这并没有阻止一些电荷从一个地点凭空消失同时在另一个地点出现, 所以我们还可以描述得更精确一些. 在空间中任意选取一个闭合曲面, 曲面内的总电荷的减少的速率等于由内向外通过曲面的速率.
\begin{equation}
\oint {\vec j \vdot \D \vec s}  =  - \frac{\D }{{\D t}}\int {\rho {\kern 1pt} \D V} 
\end{equation} 
由于不同变量的积分和求导可以交换 $\frac{\D }{{\D t}}\oint {\rho {\kern 1pt} \D V}  = \oint {\frac{{\D \rho }}{{\D t}}{\kern 1pt} \D V} $, 且由散度定理%未完成:引用
 . 上式可写为
\begin{equation}
\int {\div \vec j \D V}  + \int {\frac{{\D \rho }}{{\D t}}\D V}  = 0
\end{equation} 
由于该体积分对任何闭合曲面都成立, 所以
\begin{equation}
\div \vec j + \frac{{\D \rho }}{{\D t}} = 0
\end{equation} 