% 简谐振子(级数)

\subsection{结论}

简谐振子的能级为
\begin{equation}
{E_n} = \left( {\frac12 + n} \right)\hbar \omega 
\end{equation}
波函数为
\begin{equation}
{\psi_n}\left( x \right) = \frac{1}{\sqrt {{2^n}n!} }{\left( {\frac{\alpha ^2}{\pi }} \right)^{1/4}}{H_n}\left( u \right){\E^{ - {u^2}/2}}
\end{equation}
其中
\begin{equation}
\alpha  \equiv \sqrt {\frac{m\omega }{\hbar }} , u\, \equiv \,\alpha x, {H_n}\left( u \right) \equiv {\left( { - 1} \right)^n}{\E^{u^2}}\frac{\D^n}{\D{u^n}}\left( {\E^{ - {u^2}}} \right)
\end{equation}
${H_n}\left( u \right)$ 叫做Hermite Polynomials. 前6个分别为
\begin{equation}
\begin{array}{l}
{H_0}\left( u \right) = 1\\
{H_1}\left( u \right) = 2u\\
{H_2}\left( u \right) = 4{u^2} - 2
\end{array}
\qquad
\begin{array}{l}
{H_3}\left( u \right) = 8{u^3} - 12u\\
{H_4}\left( u \right) = 16{u^4} - 48{u^2} + 12\\
{H_5}\left( u \right) = 32{u^5} - 160{u^3} + 120u
\end{array}
\end{equation}
前4个波函数分别为(注意函数的奇偶性与角标的奇偶性相同)
\begin{equation}\begin{aligned}
{\psi_0}\left( x \right) &= {\left( {\frac{\alpha ^2}{\pi }} \right)^{1/4}}{\E^{ - {u^2}/2}}\\
{\psi_1}\left( x \right) &= {\left( {\frac{\alpha ^2}{\pi }} \right)^{1/4}}\sqrt 2 u{\E^{ - {u^2}/2}} \\
{\psi_2}\left( x \right) &= {\left( {\frac{\alpha ^2}{\pi }} \right)^{1/4}}\frac{1}{\sqrt 2 }\left( {2{u^2} - 1} \right){\E^{ - {u^2}/2}}\\
{\psi_3}\left( x \right) &= {\left( {\frac{\alpha ^2}{\pi }} \right)^{1/4}}\frac{1}{\sqrt 3 }u\left( {2{u^2} - 3} \right){\E^{ - {u^2}/2}}
\end{aligned}\end{equation}
\subsection{推导}%未完成

薛定谔方程为
\begin{equation}
- \frac{\hbar ^2}{2m}\frac{{\D^2}\psi }{\D{x^2}} + \frac12  m{\omega ^2}{x^2}\psi  = E\psi
\end{equation}