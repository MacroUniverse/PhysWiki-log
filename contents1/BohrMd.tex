% 玻尔模型

\pentry{圆周运动\upref{CircM}%未完成(总结篇)
,向心力%未完成
}
\subsection{结论}
\subsubsection{能级公式}
\begin{equation}
E_n =  - \frac{m Z^2 e^4}{32 \pi^2 \epsilon_0^2 \hbar ^2} \frac{1}{n^2} \approx - 13.6\Si{eV} \frac{Z^2}{n^2}
\end{equation}
\subsubsection{玻尔半径}
\begin{equation}
a_0 = \frac{4\pi \epsilon_0 \hbar^2}{e^2 m}
\end{equation}
玻尔原子模型是量子力学发展的早期被提出的一种解释类氢原子光谱的模型.该词条讨论其最简单的版本.

所有原子中最简单的一类叫类氢原子,类氢原子只有一个电核外电子,以及一个带 $Z$ 个元电荷的原子核.以下的计算假设二者为质点和点电荷,原子核不动,电子绕原子核做圆周运动.运用经典力学和库仑力公式,可求出电子在不同半径下做圆周运动的能量.库伦定律与牛顿定律(圆周运动)分别为
\begin{equation}
F = \frac{1}{4\pi \epsilon_0} \frac{(Ze)e}{r^2}
\qquad
F = ma = m\frac{v^2}{r}
\end{equation}
解得电子速度为
\begin{equation}\label{BohrMd_eq2}
v = e\sqrt{\frac{Z}{4\pi \epsilon_0 mr}} 
\end{equation}
动能与势能分别为
\begin{equation}
E_K = \frac12 m v^2 = \frac{1}{8\pi\epsilon_0} \frac{Z e^2}{r}
\qquad
E_P =  -\frac{1}{4\pi\epsilon_0} \frac{Ze^2}{r}
\end{equation}   
总能量为
\begin{equation}\label{BohrMd_eq4}
E = E_K + E_P =  -\frac{Z e^2}{8\pi\epsilon_0 r}
\end{equation}
到此为止,我们还没有用到量子力学.然而这样的模型与真实的类氢原子相比有两个致命的缺陷: 第一,根据电动力学,圆周运动的电子会向外辐射电磁波,能量减少,最终坠入原子核; 第二,该模型允许氢原子的能量具有连续值(因为 $r$ 可连续变化),而实验中氢原子只能放出特定能量的光子,说明只能取特定的能量,即存在离散的\bb{能级},我们把能级由低到高记为 $E$  $(n = 1,2,3\dots)$. 

以上矛盾说明围观世界的例子不遵守经典力学和电磁学.玻尔为了解释实验,在经典力学和电磁学上加入了一个条件: 角动量量子化.

以原子核为原点,电子轨道平面的法向量为 $z$ 轴,由于电子的位矢 $\vec r$ 与动量 $\vec p$ 始终垂直,电子的角动量为
\begin{equation}
\vec L = \vec r \cross \vec p = mvr \vdot \uvec z
\end{equation}
玻尔引入的角动量量子化条件为
\begin{equation}\label{BohrMd_eq6}
mvr = n\hbar 
\end{equation}
其中 $n$ 可以取任意正整数, $\hbar$ 为\bb{约化普朗克常量}
\begin{equation}\label{BohrMd_eq7}
\hbar  = \frac{h}{2\pi}
\end{equation}
该条件也可以等效理解为驻波条件,即允许的圆形轨道长度是德布罗意波长的整数倍.但注意只有原子核不动是可以这样理解.
\begin{equation}\label{BohrMd_eq8}
2\pi r \vdot n = \frac{h}{mv}
\end{equation}
注意\autoref{BohrMd_eq6} 与\autoref{BohrMd_eq8} 等效.把式\autoref{BohrMd_eq2} 带入了该条件,解得可能的轨道半径为
\begin{equation}\label{BohrMd_eq9}
\vec L = \vec r \cross \vec p = mvr \vdot \hat z
\end{equation}
可见轨道与 $n^2$ 成正比.

氢原子 $(Z=1)$ 的 $r_1$ 是量子力学中一个重要常数,叫\bb{玻尔半径},一般记为 $a_0$. 把 $r_n$ 代入式\autoref{BohrMd_eq2}, 得到对的电子速度为
\begin{equation}\label{BohrMd_eq10}
v_n = \frac{Z e^2}{4\pi\epsilon_0\hbar} \vdot \frac{1}{n}
\end{equation}
代入式\autoref{BohrMd_eq4}, 得到能级表达式为
\begin{equation}\label{BohrMd_eq11}
E_n =  - \frac{mZ^2 e^4}{32\pi^2\epsilon_0^2 \hbar ^2} \frac{1}{n^2} \approx  - 13.6\Si{eV}\frac{Z^2}{n^2}
\end{equation}
最简单的原子是氢原子, $Z = 1$, 最低的能级为 $n = 1$, 所以氢原子\bb{基态}的能级约为 $-13.6\Si{eV}$. 这是一个著名的常数,建议熟记.









