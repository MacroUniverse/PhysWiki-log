%算符的矩阵表示
%50min
%未完成
%(一定要强调一下计算矩阵元的方法 \[{Q_{ij}} = \left\langle {{\psi _i}} \right|\Q Q\left| {{\psi _j}} \right\rangle \])
考虑一个比较基本的问题, 算符的“功能”是什么呢? 算符就是对函数的一种操作方法.给出一个波函数, 经过算符作用, 可以得到一个新的波函数.以下给出量子力学中算符的两个重要性质
\begin{enumerate}
\item 算符都是线性的, 即对任意 $n$ 个波函数 ${\psi _1},{\psi _2}\dots{\psi _n}$,  算符 $\Q Q$ 满足
\begin{equation}
\Q Q\left( {{c_1} {\psi _1} + {c_2} {\psi _2}\dots{c_n}{\psi _n}} \right) = {c_1}\Q Q  {\psi _1} + {c_2}\Q Q  {\psi _2}\dots{c_n}\Q Q  {\psi _n}
\end{equation}
\item 算符的本征方程的本征值都是实数.因为根据测量理论, 本征值就是可能出现的测量结果, 所以本征值一定是实数.
\end{enumerate}

我们已经知道, 波函数可以用列向量表示%链接未完成,词条在哪里?
.既然算符都是线性的, 而矩阵可以表示列向量的线性变换, 是否可以用矩阵代替算符, 从而作用于列向量呢?根据性质 $1$,  若 是算符 $\Q Q$ 的本征函数, ${\lambda _1}\dots{\lambda _n}$ 是对应的本征值(实数), 则
\begin{equation}
\begin{aligned}
\Q Q  \psi & = \Q Q\left( {{c_1}{\psi _1} + \dots + {c_n}{\psi _n}} \right)\\
 & = {c_1}\Q Q  {\psi _1} + \dots + {c_n}\Q Q  {\psi _n}\\
 & = {\lambda _1}{c_1}{\psi _1} + \dots + {\lambda _n}{c_n}{\psi _n}
\end{aligned}
\end{equation}
若把上面的波函数表示成列矢量,就相当于在算符 $\Q Q$ 的作用下任意一个列矢量 $\ket{\psi}  = ({{c_1}}, \dots ,{{c_n}})\Tr$ 总是会变成 $({{\lambda _1}{c_1}}, \dots, {{\lambda _n}{c_n}})\Tr$. 这个变换可以用矩阵
\begin{equation}
\mat Q = \pmat{
{{\lambda _1}}&{}&{}\\
{}& \ddots &{}\\
{}&{}&{{\lambda _n}}} 
\end{equation}
来表示,即
\begin{equation}
\pmat{{{\lambda _1}}&{}&{}\\ {}& \ddots &{}\\ {}&{}&{{\lambda _n}}}
\pmat{{{c_1}}\\ \vdots \\{{c_n}}} 
= \pmat{{{\lambda _1}{c_1}}\\  \vdots \\ {{\lambda _n}{c_n}} }
\end{equation}
所以矩阵 $\mat Q$ 就是算符 $\Q Q$ 的矩阵形式, 把算符作用在波函数上得到新的波函数, 等效于把算符对应的矩阵作用在波函数对应的列矢量上, 得到新的波函数对应的列矢量.\\
用矩阵和列向量表示的本征方程如下
\begin{equation}
\mat Q \ket{\psi}  = \lambda \ket{\psi} 
\end{equation}
解得 $\lambda  = {\lambda _i}$ 时, $\ket{\psi}  = \ket{\psi _1}  = (0,\dots,1,\dots,0)\Tr$ (只有第 $i$ 个分量等于 1,其余分量等于 0),而 $\left| {{\psi _i}} \right\rangle $ 正是波函数 ${\psi _i}$ 对应的列向量.

\subsection{在任意基底中的矩阵}
上面的讨论中用矩阵 $\mat Q$ 表示算符 $\Q Q$, 其局限性在于,只能使用 $\Q Q$ 的本征函数 ${\psi _1}\dots{\psi _n}$ 作为基底.现在若用其他基底(正交归一的) ${\phi _1}\dots{\phi _n}$, 能否求出算符 $\Q Q$ 对应的矩阵 ${\mat Q_1}$ 呢?\\
下面讨论中, 为了避免混淆, 用 ${\left| f \right\rangle _\phi }$ 表示波函数 $f$ 以 ${\phi _1}\dots{\phi _n}$ 为基底的列矢量, ${\left| f \right\rangle _\psi }$ 表示波函数 $f$ 以 ${\psi _1}\dots{\psi _n}$ 为基底的列矢量.\\
现取任意一波函数 $f$,  ${\left| f \right\rangle _\psi } = \left( {\begin{aligned}
{{c_1}}\\
 \vdots \\
{{c_n}}
\end{aligned}} \right)$,   ${\left| f \right\rangle _\phi } = \left( {\begin{aligned}
{{d_1}}\\
 \vdots \\
{{d_n}}
\end{aligned}} \right)$. 虽然他们表示同一个波函数 $f$,  但是由于选取的基底不同, 列向量也不同.下面讨论他们之间的变换关系.\\
若把 $f$ 按 ${\phi _i}$ 展开, 有
\begin{equation}
\begin{aligned}
\int {\phi _i^*f   \D x} & = \int {\phi _i^*\left( {{d_1}{\phi _1} + \dots + {d_n}{\phi _n}} \right)   \D x}  = \sum\limits_{j = 1}^n {{d_j}} \int {\phi _i^*{\phi _j} \D x}  \\
&= \sum\limits_{j = 1}^n {{d_j}} {\delta _{ij}} = {d_i}
\end{aligned}
\end{equation}
若把 $f$ 按 $\psi _i$ 展开, 有
\begin{equation}
\begin{aligned}
{d_i} & = \int {\phi _i^*f   \D x}   = \int {\phi _i^*\left( {{c_1}{\psi _1} + \dots + {c_n}{\psi _n}} \right)   \D x} = \sum\limits_{j = 1}^n {{c_j}} \int {\phi _i^*{\psi _j} \D x}
\end{aligned}
\end{equation}
上式用矩阵和列矢量表示, 即 $\left( {\begin{aligned}
{{d_1}}\\
 \vdots \\
{{d_n}}
\end{aligned}} \right) = P\left( {\begin{aligned}
{{c_1}}\\
 \vdots \\
{{c_n}}
\end{aligned}} \right)$,  即 ${\left| f \right\rangle _\phi } = P{\left| f \right\rangle _\psi }$.\\
其中 $P$ 矩阵的矩阵元 ${P_{ij}} = \int {\phi _i^*{\psi _j}dx} $.  $P$ 叫做基底变换矩阵(或表象变换矩阵).\\
若令 $\Q Q f = g$,  根据前面的内容, $Q{\left| f \right\rangle _\psi } = {\left| g \right\rangle _\psi }$ 其中 $Q = \left( {\begin{aligned}
{{\lambda _1}}&{}&{}\\
{}& \ddots &{}\\
{}&{}&{{\lambda _n}}
\end{aligned}} \right)$. \\
下面应用基底变换矩阵, 有 ${\left| f \right\rangle _\psi } = {P^{ - 1}}{\left| f \right\rangle _\phi }$ ; ${\left| g \right\rangle _\psi } = {P^{ - 1}}{\left| g \right\rangle _\phi }$. 代入上式得
\begin{equation}
  Q{P^{ - 1}}{\left| f \right\rangle _\phi } = {P^{ - 1}}{\left| g \right\rangle _\phi }
\end{equation}
两边左乘 $P$ 得
\begin{equation}
  PQ{P^{ - 1}}{\left| f \right\rangle _\phi } = {\left| g \right\rangle _\phi }
\end{equation}
令 ${\mat Q_1} = \mat P\mat Q{\mat P^{ - 1}}$,  得
\begin{equation}
  {Q_1}{\left| f \right\rangle _\phi } = {\left| g \right\rangle _\phi }
\end{equation}
所以 ${\mat Q_1}$ 就是要求的矩阵.\\
下面证明 ${\mat Q_1}$ 是厄米矩阵\\
我们先学习所谓幺正矩阵.这里给出幺正矩阵的一种定义:\\
若把矩阵 $P$ 的每一列划分成一个列向量, 从左到右分别为$\ket{p_1} \dots\ket{p_n}$,  若满足 $\bra{p_i}\ket{p_j}=\delta_{ij}$ 则矩阵 $P$ 叫做幺正矩阵.\\
容易证明式 $*$ 中的 $P$ 就是幺正矩阵(证明略).\\
性质 $1$ : 幺正矩阵一个很重要的性质就是其厄米共轭等于其逆矩阵, ${P^*} = {P^{ - 1}}$ \\
证明:\\
要证明, ${P^*} = {P^{ - 1}}$,  只需证明 ${P^*}P$ 是单位矩阵即可.\\
根据矩阵乘法的定义,
\begin{equation}
{\left( {{P^*}P} \right)_{ij}} = \sum\limits_{k = 1}^n {{{\left( {{P^*}} \right)}_{ik}}{P_{kj}}}
\end{equation}
根据厄米共轭的定义,
\begin{equation}
 \sum\limits_{k = 1}^n {{{\left( {{P^*}} \right)}_{ik}}{P_{kj}}}  = \sum\limits_{k = 1}^n {{{\left( {{P_{ki}}} \right)}^*}{P_{kj}}}  = \left\langle {{{p_i}}}
 \mathrel{\left | {\vphantom {{{p_i}} {{p_j}}}}
 \right. \kern-\nulldelimiterspace}
 {{{p_j}}} \right\rangle  = {\delta _{ij}}
\end{equation}
所以 ${P^*}P$ 是 $n$ 阶的单位矩阵.  证毕.

在上文中, $Q$ 是所谓的实数元的对角矩阵, 所以 ${Q^*} = Q$ ;\\
另外容易证明, ${\left( {AB} \right)^*} = {B^*}{A^*}$.  所以
\begin{equation}
Q_1^* = {\left( {PQ{P^{ - 1}}} \right)^*} = {\left( {{P^{ - 1}}} \right)^*}{\left( {PQ} \right)^*} = P\left( {{Q^*}{P^*}} \right) = PQ{P^{ - 1}} = {Q_1}
\end{equation}
所以 ${Q_1}$ 是厄米矩阵.

