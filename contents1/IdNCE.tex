%理想气体(微正则系综法)
%10min
\pentry{理想气体的状态密度(相空间)}%未完成链接
\subsection{$N$ 粒子相空间}

  由理想气体的状态密度(相空间)%未完成链接
  中的结论, $\Delta E$ 能量内的状态数为
  \begin{equation}
    \Omega \left( {E,V,N;{\kern 1pt} {\kern 1pt} \D E} \right) = g\left( E \right)\D E = \frac{{{V^N}}}{{N!{h^3}}}\frac{{{{\left( {2\pi m} \right)}^{3N/2}}}}{{\left( {3N/2 - 1} \right)!}}{E^{3N/2 - 1}}\D E
  \end{equation}
  根据熵的定义%未完成链接
  \begin{equation}
    S = kT\ln \Omega  = Nk\left( {\ln \frac{V}{{N{\lambda ^3}}} + \frac{5}{2}} \right)
  \end{equation}
  其中用到了Stirling近似%未完成链接
   $\ln N! = N\ln N - N$ . \\
   根据熵的微分关系
   \begin{equation}
     \D S = \frac{1}{T}\D E + \frac{P}{T}\D V - \frac{\mu }{T}\D N
   \end{equation}
   可求出温度, 压强, 化学势和能量之间的关系.
