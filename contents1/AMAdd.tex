%角动量加法
%40min

考虑两个系统,总角动量分别为 ${l_1}$  和 ${l_2}$, 可能的状态分别为 $\left| {{l_1},{m_1}} \right\rangle $,   $\left| {{l_2},{m_2}} \right\rangle $. 角动量算符分别为  $\Q L_1^2$,   ${\Q L_{1x}}$,   ${\Q L_{1y}}$,   ${\Q L_{1z}}$,   $\Q L_1^2$,   $\Q L_2^2$,    ${\Q L_{2y}}$,  ${\Q L_{2z}}$.  

现在定义总角动量算符
 \begin{equation}
{\Q J^2} = {\left( {{{\Q L}_1} + {{\Q L}_2}} \right)^2} = {\left( {{{\Q L}_{1x}} + {{\Q L}_{2x}}} \right)^2} + {\left( {{{\Q L}_{1x}} + {{\Q L}_{2x}}} \right)^2} + {\left( {{{\Q L}_{1x}} + {{\Q L}_{2x}}} \right)^2}
\end{equation}
 \begin{equation}
  {\Q J_z} = {\Q L_{1z}} + {\Q L_{2z}}
\end{equation}
令量子数分别为 $J$  和 $M$.  所有新增的对易关系为
\begin{equation}
[{\Q J^2},{J_z}] = 0
\text{,}
[{\Q J^2},L_1^2] = 0
\text{,}  
[{\Q J^2},L_2^2] = 0
\text{,}
[{J_z},{L_{1z}}] = 0
\text{,}
[{J_z},{L_{2z}}] = 0
\end{equation}
若限制  ${l_1}$ 和 ${l_2}$ 为常数,原来和现在的Complete Set of Commutable Operators (CSCO) 是
 \begin{equation}
  \left\{ {{{\Q L}_{1z}},{{\Q L}_{2z}}} \right\}
  \text{,}
  \left\{ {{{\Q J}^2},{{\Q J}_z}} \right\}
\end{equation}
现在已知前一组的本征基底 $\left| {{l_1},{m_1}} \right\rangle \left| {{l_2},{m_2}} \right\rangle $,  要求后者的基底 $\left| {J,M} \right\rangle $.  首先由对易关系, $\left| {{l_1},{m_1}} \right\rangle \left| {{l_2},{m_2}} \right\rangle $ 已经是 ${\Q J_z}$ 的本征矢,每个 $M$ 可以有一个子空间,维数 $N$ 是 ${m_1} + {m_2} = M$ 的不同组合数.当 $\left| M \right| = {l_1} + {l_2}$ 时 $N = 1$ (唯一的非简并情况), $\left| M \right| = {l_1} + {l_2} - 1$ 时 $N = 2$, 以此类推(但注意 ${m_1}$,  ${m_2}$ 不能超出范围)
  \begin{equation}
  N = \left\{ \begin{gathered}
  {l_1} + {l_2} + 1 - \left| M \right|\quad \left( {\left| M \right| > \left| {{l_1} - {l_2}} \right|} \right) \hfill \\
  \min \left\{ {{l_1},{l_2}} \right\}\quad (otherwise) \hfill \\ 
\end{gathered}  \right.
\end{equation}
我们只需要在每个子空间 $M$ 中把 ${\Q J^2}$  对角化即可.
  \begin{equation}
  {\Q J^2} = \Q L_1^2 + \Q L_2^2 + 2\left( {{{\Q L}_{1x}}{{\Q L}_{2x}} + {{\Q L}_{1y}}{{\Q L}_{2y}} + {{\Q L}_{1z}}{{\Q L}_{2z}}} \right)
\end{equation}
其中只有 ${\Q L_{1x}}{\Q L_{2x}} + {\Q L_{1y}}{\Q L_{2y}}$  不是对角矩阵.利用升降算符表示
  \begin{equation}
  2\left( {{{\Q L}_{1x}}{{\Q L}_{2x}} + {{\Q L}_{1y}}{{\Q L}_{2y}}} \right) = {\Q L_{1 + }}{\Q L_{2 - }} + {\Q L_{1 - }}{\Q L_{2 + }}
\end{equation} 
\begin{equation}\begin{aligned}
 &\left\langle {{l_2},{{m'}_2}} \right|\left\langle {{l_1},{{m'}_1}} \right|{\Q J^2}\left| {{l_1},{m_1}} \right\rangle \left| {{l_2},{m_2}} \right\rangle  \\
&= {\hbar ^2}\left[ \begin{gathered}
  {\delta _{{{m'}_1},{m_1}}}{\delta _{{{m'}_2},{m_2}}}[{l_1}({l_1} + 1) + {l_2}({l_2} + 1) + 2{m_1}{m_2}] \hfill \\
   + {\delta _{{{m'}_1},{m_1} + 1}}{\delta _{{{m'}_2},{m_2} - 1}}\sqrt {{l_1}({l_1} + 1) - {m_1}({m_1} + 1)} \sqrt {{l_2}({l_2} + 1) - {m_2}({m_2} - 1)}  \hfill \\
   + {\delta _{{{m'}_1},{m_1} - 1}}{\delta _{{{m'}_2},{m_2} + 1}}\sqrt {{l_1}({l_1} + 1) - {m_1}({m_1} - 1)} \sqrt {{l_2}({l_2} + 1) - {m_2}({m_2} + 1)}  \hfill \\ 
\end{gathered}  \right]
\end{aligned}\end{equation}
 
一般在一个 $M$ 空间中, ${m_1}$ 用降序排列, ${m_2} = M - {m_1}$,  ${m_1}$ 的最大值为
  \begin{equation}
  \max \left\{ {{m_1}} \right\} = \left\{ \begin{gathered}
  {l_1}\quad (M \geqslant {l_1} - {l_2}) \hfill \\
  {l_2} + M\quad (otherwise) \hfill \\ 
\end{gathered}  \right.
\end{equation}
这样, ${J^2}$ 就是一个三对角矩阵,其本征矢矩阵就是从 $\left| {J,M} \right\rangle $ 表象到 $\left| {{l_1},{m_1}} \right\rangle \left| {{l_2},{m_2}} \right\rangle $ 表象的幺正变换矩阵 ${U_M}$ ( $\left| {J,M} \right\rangle $ 的顺序取 $J$ 从大到小).矩阵的输入矢量可以用 ${m_1}$ 为角标,输出矢量可以用 $J$ 为角标.查CG表时,CG系数通常以 ${U_M}$ 矩阵的形式给出(如Griffiths).可以证明, ${U_M}$ 的 $N$ 个本征值为 $J(J + 1){\hbar ^2}$,  其中 $J = {l_1} + {l_2},{l_1} + {l_2} - 1,...$ (共 $N$ 项%,不会证明
)(注意最小值大于但不一定等于 $\left| M \right|$ ). $J$ 在所有子空间的最小值是 $\left| {{l_1} - {l_2}} \right|$ (当 $\left| M \right| = \left| {{l_1} - {l_2}} \right|$ 时取得),所以 $J$ 在所有子空间的范围是
  \begin{equation}
  J = \left| {{l_1} - {l_2}} \right|,...,{l_1} + {l_2}
\end{equation}
现在我们已经知道了每个子空间 $M$ 的变换,那么如何求总变换呢?先把总矩阵列表,行标题是所有的 $\left| {{l_1},{m_1}} \right\rangle \left| {{l_2},{m_2}} \right\rangle $, 列标题是所有的 $\left| {J,M} \right\rangle $, 对每个空间,找到对应的 $N$ 行和 $N$ 列,把 $N \cross N$  的 ${U_M}$ 矩阵照抄上去即可.


