% 原子单位
\pentry{玻尔原子模型} % 链接未完成

在量子力学的许多理论或数值计算中,选用\bb{原子单位(atomic unit)}会更方便.原子单位中,不同物理量的符号都记为 “a.u.”,而事实上,可以把“a.u.”等效为数值 1(无量纲),例如角频率单位“Rad/s”中单位“Rad”可等效为数值 1.以后我们只有在想要强调原子单位时,才加“a.u.”.

单位转换的一般方法为定义一个转换常数 $\beta$,例如原子单位中,我们把一个电子的质量定义为“1 a.u.”,把以 a.u. 为单位的质量 $m_a$ 转换为以 kg 为单位的质量 $m$,可先定义转换常数为
\begin{equation}
{\beta _m} = {m_e}/\text{a.u.} \approx 9.109 \times {10^{ - 31}}\text{kg}
\end{equation}
则有
\begin{equation}
m = {\beta _m}{m_a}
\end{equation}
再次注意 $m_a$ 是一个没有量纲的变量.

\autoref{AU_Table1} %链接未完成
是基本单位的转换常数表.其中许多常量都是基于氢原子的玻尔模型的基态% 链接未完成
(原子核不动)的参数定义的(表中简称基态).

用精细结构常数可以化简以下部分公式.注意这是一个无量纲常数.
\begin{equation}
\alpha  = \frac{{{e^2}}}{{4\pi {\epsilon _0}\hbar c}} = 7.2973525664 \times {10^{ - 3}} \approx \frac{1}{{137}}
\end{equation}

\begin{table}[ht]
\caption{原子单位转换常数表}\label{AU_Table1}
\zihao{5}
\centering
\begin{tabular}{|c|c|c|c|}
\hline
物理量 & $\beta$(单位:$SI/a.u.$) & 描述 & 数值\\
\hline
质量 $m$ & $m_e$ & 电子质量 & $9.10938215 \times {10^{ - 31}}$ \\
\hline
\dfracH 长度 $x$ & $\dfrac{4\pi {\epsilon _0}{\hbar ^2}}{m_e e^2}=\dfrac{\hbar}{\alpha m_e c}$ & 玻尔半径 $a_0$ & $5.2917721067 \times {10^{ - 11}}$ \\
\hline
\dfracH 速度 $v$ & $\dfrac{e^2}{4\pi {\epsilon _0}\hbar } = \alpha c$ & 基态电子速度 $v_0$ & ${2.187691263} \times {10^6}$\\
\hline
电荷 $q$ & $e$ 或 $q_e$ & 电子电荷 & ${1.6021766208} \times {10^{ - 19}}$\\
\hline
时间 $t$ & $a_0/v_0$ & 长度除以速度 & $2.418884326 \times {10^{ - 17}}$\\
\hline
%\dfracH 频率 & $\dfrac{v_0}{2\pi a_0}$ & 基态频率 & $6.579683921 \times {10^{15}}$ \\
%\hline
角动量 $L$ & $m_e v_0 a_0 = \hbar$ & 长度乘以动量 & ${\rm{1}}{\rm{.054571800}} \times {10^{ - 34}}$ \\
\hline
\dfracH 电势 $V$ & $\dfrac{e}{4\pi {\epsilon _0}{a_0}}$ & 基态轨道电势 & 27.211386019 \\
\hline
\dfracH 能量 $E$ & $\dfrac{\hbar^2}{m_e a_0^2} = \dfrac{{{e^2}}}{{4\pi {\epsilon _0}{a_0}}} = {\alpha ^2}m{c^2}$ & 基态电子势能大小 & ${\rm{4}}{\rm{.3597446499}} \times {10^{ - 18}}$ \\
\hline
\dfracH 电场强度 $\mathcal{E}$ & $\dfrac{e}{{(4\pi {\epsilon _0})a_0^2}}$ & 基态轨道电场强度 & ${\rm{5}}{\rm{.1422067070}} \times {\rm{1}}{{\rm{0}}^{11}}$ \\
\hline
\dfracH 磁感应强度 $B$ & $\dfrac{\hbar}{e a_0^2}$ & 无 & ${\rm{2}}{\rm{.350517550}} \times {\rm{1}}{{\rm{0}}^5}$ \\
\hline
\end{tabular}
\end{table}

使用原子单位可以简化许多公式,同时表达微观尺度的物理量变得更容易
\begin{itemize}
\item 点电荷电势
\begin{equation}
V(\vec r) = q/r
\end{equation}
\item 薛定谔方程  
\begin{equation}
-\frac{1}{2m}\laplacian\psi(\vec r) + V(\vec r)\psi(\vec r) = \I\frac{\partial}{\partial t} \psi(\vec r)
\end{equation}
\end{itemize}


%频率与角频率
%\begin{equation}
%f = \frac{1}{{2\pi T}}
%\end{equation}

\begin{exam}{电场中电子的薛定谔方程} % 链接未完成
SI 单位下电子的薛定谔方程为(偶极子近似) \footnote{为了区别能量与电场,以下用 $E$ 表示能量,用 $\mathcal{E}$ 表示电场.} 
\begin{equation}
-\frac{\hbar^2}{2m_e}\frac{\D^2}{\D x^2}\psi + e x\mathcal{E}(t)\psi= \I\hbar\frac{\D}{\D t}\psi
\end{equation}
其中 $\mathcal{E}(t)$ 是电场.注意到上式中每一项都具有能量量纲,为了化为更简单的无量纲方程,把等式两边除以能量单位 $\beta_E= \hbar^2/(m_e\beta_x^2)$,得
\begin{equation}\label{AU_eq3}
-\frac 12 \frac{\D^2}{\D (x/\beta_x)^2}\psi + e m_e\beta_x^3\beta_\mathcal{E}/\hbar^2 \vdot\left[\frac{\mathcal{E}(t)}{\beta_\mathcal{E}}\right]\left(\frac{x}{\beta_x}\right)\psi= m_e\beta_x^2/(\beta_t\hbar)\vdot\I\frac{\D}{\D (t/\beta_t)}\psi
\end{equation}
现在定义一些无量纲的物理量,$x_a \equiv x/\beta_x$,$t_a \equiv t/\beta_t$,$\mathcal{E}_a \equiv \mathcal{E}/\beta_\mathcal{E}$,代入得\footnote{严格来说,波函数也要变为 $\psi=\beta_\psi\psi_a$ 但注意到薛定谔方程为齐次方程,两边可约去 $\beta_\psi$,故波函数无需转换.但要注意 $\psi(x_a) = \psi(x/\beta_x)$. } \footnote{注意这里定义 $\mathcal{E}_a(t_a)=\beta_\mathcal{E}\mathcal{E}(t)=\beta_\mathcal{E}\mathcal{E}(\beta_t t_a)$}
\begin{equation}\label{AU_eq2}
-\frac 12\frac{\D^2}{\D x_a^2}\psi + \frac{e m_e\beta_x^3\beta_\mathcal{E}}{\hbar^2}\vdot \mathcal{E}_a(t_a) x_a\psi= \frac{m_e\beta_x^2}{\beta_t\hbar}\vdot\I\frac{\D}{\D t_a}\psi
\end{equation}
如果我们适当选择不同的 $\beta$, 就有可能将上式简化为
\begin{equation}\label{AU_eq1}
-\frac 12\frac{\D^2}{\D x_a^2}\psi +\mathcal{E}_a(t_a) x_a\psi= \I\frac{\D}{\D t_a}\psi
\end{equation}
不难验证,表中给出的 $\beta$ 就满足这样的条件.

在量子力学问题的数值计算中,通常先使用原子单位来化简公式.
\end{exam}

事实上,原子单位有不同的版本,以适用不同的问题.

\begin{exam}{另一种原子单位}

当问题涉及一基本角频率 $\omega$ 的时候,可选择 $\beta_E = \hbar\omega$ 做能量单位.为了让薛定谔方程仍然保持\autoref{AU_eq1} 的形式,在\autoref{AU_eq3} 的推导中,令能量单位为
\begin{equation}
\beta_E = \frac{\hbar^2}{m_e\beta_x^2}=\hbar\omega
\end{equation}
得长度单位为
\begin{equation}
\beta_x = \sqrt{\frac{\hbar}{m_e\omega}}
\end{equation}
令\autoref{AU_eq2} 中两个含有 $\beta$ 的因子为 1,得
\begin{equation}
\beta_\mathcal{E} = \frac{\hbar\omega}{e \beta_x} \qquad \beta_t = \frac{1}{\omega}
\end{equation}
一种常见的情况是平面波电场 $\mathcal{E}(t) = \mathcal{E}_0\cos(\omega t)$, 定义 $\mathcal{E}_{a0} = \mathcal{E}_0/\beta_\mathcal{E}$,则该原子单位下的电场为
\begin{equation}
\mathcal{E}_a(t_a) = \mathcal{E}(t)/\beta_\mathcal{E} =
\frac{\mathcal{E}_{0}}{\beta_\mathcal{E}}\cos(\omega t) = \mathcal{E}_{a0}\cos(t_a)
\end{equation}
注意右边不含 $\omega$, 形式更简洁.
\end{exam}



