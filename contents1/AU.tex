% 原子单位

\pentry{无量纲的物理公式\upref{NoUnit}, 玻尔原子模型\upref{BohrMd}}

在量子力学的许多理论或数值计算中,选用\bb{原子单位(atomic unit)}会更方便. 事实上所谓的原子单位并不是一套量纲, 而是将量子力学公式变为无量纲公式过程中定义的一系列 $\beta$ 常数.

有时候为了强调我们使用原子单位, 我们会在数值后面加上 “a.u.”, “a.u.” 可等效为数值 1, 就像弧度单位 “Rad” 一样\footnote{例如在圆的面积公式中, $S = \pi R^2$, 其中 $\pi$ 可以看作具有量纲 “Rad”, 但面积的量纲却只是 “\Si{m^2}” 而无需记为 “\Si{Rad\times m^2}”.}.

\subsection{无量纲的薛定谔方程}
国际单位下的一维含时薛定谔方程为
\begin{equation}\label{AU_eq4}
-\frac{\hbar^2}{2m} \pdv[2]{\Psi}{x} + V\Psi= \I\hbar \pdv{\Psi}{t}
\end{equation}
与“无量纲的物理公式\upref{NoUnit}” 中的方法类似, 我们需要给公式中出现的每个量纲定义一个常数, 分别记为 $\beta_x$, $\beta_m$, $\beta_t$, $\beta_E$, $\beta_\Psi$, 且令 $x = x_a\beta_x$, $m = m\beta_m$, $t = t_a\beta_t$, $V = V_a\beta_E$, $\Psi = \Psi_a \beta_\Psi$.
代入\autoref{AU_eq4}, 各项同除 $\beta_E\beta_\Psi$, 得\footnote{根据偏微分的定义, 常数可以移到偏微分算符外, 如 $\pdv*[2]{(\beta_x x_a)} = (1/\beta_x^2) \pdv*[2]{x_a}$}
\begin{equation}
-\qty(\frac{\hbar^2}{\beta_m\beta_x^2\beta_E})\frac{1}{2m_a} \pdv[2]{\Psi_a}{x_a} + V_a\Psi_a= \I\qty(\frac{\hbar}{\beta_E\beta_t})\pdv{\Psi_a}{t_a}
\end{equation}
为了让公式尽可能简洁, 我们令两个括号都为 1, 得
\begin{equation}\label{AU_eq6}
\beta_E = \frac{\hbar^2}{\beta_m\beta_x^2}
\qquad
\beta_t = \frac{\beta_m\beta_x^2}{\hbar}
\end{equation}
于是无量纲的薛定谔方程为
\begin{equation}\label{AU_eq4}
-\frac{1}{2m_a} \pdv[2]{\Psi_a}{x_a} + V_a\Psi_a= \I\pdv{\Psi_a}{t_a}
\end{equation}
我们再看波函数的归一化公式
\begin{equation}
1 = \int \abs{\Psi}^2 \dd{x} = \beta_\Psi^2 \beta_x \int \abs{\Psi_a}^2 \dd{x_a}
\end{equation}
为了使归一化公式的形式不变, 必须令
\begin{equation}\label{AU_eq5}
\beta_\Psi = \beta_x^{-1/2}
\end{equation}
同理, 对 $N$ 维波函数有 $\beta_\Psi = \beta_x^{-N/2}$.
由\autoref{AU_eq5} 和 \autoref{AU_eq6} 可知我们只剩下两个自由度, 也就是说只要确定 $\beta_x$ 和 $\beta_m$, 剩下的 $\beta$ 也就确定了.

\subsection{原子单位}
事实上, 原子单位也不同的定义, 但都能得到\autoref{AU_eq4} 形式的薛定谔方程. 最常见的情况是定义 $\beta_m$ 等于电子的质量, $\beta_x$ 等于玻尔半径, 再由 \autoref{AU_eq6} 和 \autoref{AU_eq5} 确定 $\beta_E, \beta_t, \beta_\Psi$, 如\autoref{AU_tab1} 所示\footnote{为了区别能量与电场,以下用 $E$ 表示能量,用 $\mathcal{E}$ 表示电场.} . 注意许多常数都与氢原子的玻尔模型\upref{BohrMd}(原子核不动)的基态(表中简称基态)有关.

\begin{table}[ht]
\caption{原子单位转换常数表}\label{AU_tab1}
\zihao{5}
\centering
\begin{tabular}{|c|c|c|c|}
\hline
物理量 & $\beta$ & 描述 & 数值(国际单位)\\
\hline
质量 $m$ & $m_e$ & 电子质量 & $9.10938215 \times 10^{-31}$ \\
\hline
\dfracH 长度 $x$ & $a_0 = \dfrac{4\pi \epsilon_0 \hbar ^2}{m_e e^2}$ & 玻尔半径 & $5.2917721067 \times 10^{-11}$ \\
\hline
时间 $t$ & $m_e a_0^2/\hbar$ & 长度除以速度 & $2.418884326 \times 10^{-17}$\\
\hline
\dfracH 角频率 $\omega$ & $\dfrac{\hbar}{m_e a_0^2}$ & 基态运动频率 & $6.579683921 \times {10^{15}}$ \\
\hline
\dfracH 能量 $E$ & $\dfrac{\hbar^2}{m_e a_0^2} = \dfrac{e^2}{4\pi \epsilon_0 a_0}$ & 基态电子势能大小 & $4.3597446499 \times 10^{-18}$ \\
\hline
\dfracH 速度 $v$ & $\dfrac{\hbar}{m_e a_0}$ & 基态电子速度 & $4.3597446499 \times 10^{-18}$ \\
\hline
角动量 $L$ & $m_e v_0 a_0 = \hbar$ & 长度乘以动量 & $1.054571800 \times 10^{-34}$ \\
\hline
电荷 $q$ & $e$ 或 $q_e$ & 电子电荷 & $1.6021766208 \times 10^{-19}$\\
\hline
\dfracH 电场强度 $\mathcal{E}$ & $\dfrac{e}{(4\pi \epsilon_0) a_0^2}$ & 基态轨道电场强度 & $5.1422067070 \times 10^{11}$ \\
\hline
\dfracH 电势 $V$ & $\dfrac{e}{4\pi\epsilon_0 a_0}$ & 基态轨道电势 & 27.211386019 \\
\hline
\end{tabular}
\end{table}

表中还定义了一些其他的物理量的转换常数, 它们的定义可以使以下无量纲公式成立(以后我们在不至于混淆的情况下省略无量纲物理量的角标 $a$)
\begin{align}
\omega &= \frac{2\pi}{T}\\
x &= v t \\
\vec L &= m\vec r \cross \vec v  \qquad \text{(角动量)}\\
\mathcal{E} &= \frac{q}{r^2} \qquad \text{(点电荷电场)}\\
U &= \frac{q}{r} = \mathcal{E} x \qquad \text{(点电荷电势)}\\
V &= qU = q\mathcal{E} x \qquad \text{(匀强电场电势能)} \label{AU_eq11}
\end{align}
薛定谔方程为
\begin{equation}\label{AU_eq12}
-\frac{1}{2m} \pdv[2]{\Psi}{x} + V\Psi= \I\pdv{\Psi}{t}
\end{equation}
注意当考察对象为电子时, 式中 $m = 1$, 可省略.

\begin{exam}{匀强电场中电子的薛定谔方程}
令\autoref{AU_eq12} 中 $m = 1$, $q = 1$, 再将\autoref{AU_eq11} 代入, 得
\begin{equation}
-\frac12 \pdv[2]{\Psi}{x} + \mathcal{E} x \Psi= \I\pdv{\Psi}{t}
\end{equation}
\end{exam}

\begin{exer}{氢原子的基态能量}
计算玻尔模型中氢原子基态的能量(答案:$-1/2$).
\end{exer}

\subsection{另一种原子单位}

当问题涉及一基本角频率 $\omega$ 的时候,可选择 $\beta_E = \hbar\omega$ 做能量单位. 同样令 $\beta_m$ 等于电子质量, $\beta_q$ 等于元电荷, 由\autoref{AU_eq6} 得
\begin{equation}\label{AU_eq15}
\beta_x = \sqrt{\frac{\hbar}{m_e\omega}}
\qquad
\beta_t = \frac{1}{\omega}
\end{equation}
为了使\autoref{AU_eq11} 成立,得
\begin{equation}
\beta_\mathcal{E} = \frac{\hbar\omega}{e \beta_x}
\end{equation}
一种常见的情况是平面波电场用国际单位表示为 $\mathcal{E}(t) = \mathcal{E}_0\cos(\omega t)$, 而原子单位下该式为
\begin{equation}
\mathcal{E}(t) = \mathcal{E}_0\cos t
\end{equation}
注意右边不含 $\omega$, 形式更简洁.

另一个常见的例子是简谐振子\upref{QSHOxn}, 在原子单位下, 其薛定谔方程为
\begin{equation}\label{AU_eq18}
-\frac12 \pdv[2]{\Psi}{x} + \frac12 x^2 \Psi= \I\pdv{\Psi}{t}
\end{equation}
能级为
\begin{equation}\label{AU_eq19}
E_n = \frac12 + n \qquad (n = 0, 1, 2\dots)
\end{equation}
归一化的基态波函数为
\begin{equation}
\psi_0(x) = \pi^{-1/4} \E^{-x^2/2}
\end{equation}


\begin{exam}{转换为含量刚的公式}
现在我们按照“无量纲的物理公式\upref{NoUnit}” 中介绍的方法将\autoref{AU_eq18} 转换为含量纲的公式. 即先把所有无量纲的物理量替换成有量纲的物理量除以对应的 $\beta$ 常数, 得(两边已同乘 $\beta_\Psi$)
\begin{equation}
-\beta_x^2\frac12 \pdv[2]{\Psi}{x} + \frac{1}{\beta_x^2}\frac12 x^2 \Psi= \beta_t\I\pdv{\Psi}{t}
\end{equation}
将 \autoref{AU_eq15} 代入, 两边乘以 $\omega\hbar$ 得
\begin{equation}
-\frac{\hbar^2}{2m} \pdv[2]{\Psi}{x} + \frac12 m\omega^2 x^2 \Psi= \I\hbar\pdv{\Psi}{t}
\end{equation}

类似地, 也可以将\autoref{AU_eq19} 变为
\begin{equation}
E =  \qty(\frac12 + n)\beta_E = \qty(\frac12 + n)\omega\hbar \qquad (n = 0, 1, 2\dots)
\end{equation}
\end{exam}
