% 安培力

\pentry{洛伦兹力%未完成:词条
}

在匀强磁场中,一段笔直的细导线中通有电流 $I$, 导线的长度和正方向用矢量 $\vec L$ 表示\footnote{根据电流连续性定理,电流不可能在端点凭空出现或消失,所以我们可以认为 $\vec L$ 是一个回路中的一段.},若电流与 $\vec L$ 的方向相同则取正值,若相反则取负值.导线 $\vec L$ 受到的安培力为
\begin{equation}
\vec F = I\,\vec L \cross \vec B
\end{equation}
该式可由洛伦兹力推导,见下文.

当磁场分布不均匀,或导线是弯曲的,可用“微元法”的思想,把该导线分为许多小段
然后对每小段的安培力 $\D\vec F = I\vec\D L\cross\vec B$ 进行矢量求和,即曲线积分%链接未完成
).
\begin{equation}
\vec F = I\int\limits_L {\D \vec r \cross \vec B}
\end{equation}
$L$ 为导线所在的曲线,积分方向沿电流方向.

\subsection{推导(匀强磁场中的直导线)}
假设导线中正电荷运动而负电荷不动\footnote{事实上是负电荷 $-\lambda$ 以 $-\vec v$ 运动而正电荷不动,但这样假设得到的安培力相同,且方便记忆和推导.}, 运动的正电荷线密度(单位长度的电荷量)为 $\lambda$,速度为 $\vec v$.那么电流为% 未完成: 链接到“电流”词条中的该公式
\begin{equation}
I = \lambda v
\end{equation}
所有运动的正电荷受到的洛伦兹力%链接未完成
为
\begin{equation}\label{FAmp_eq4}
\vec F =  q\vec v \cross \vec B
\end{equation}
当电流方向 $\vec v$ 与 $\vec L$ 相同时有 $L \vec v = v \vec L$,此时定义电流为正,$I = \lambda v$, 相反时有 $L \vec v = -v\vec L$,定义电流为负,$I = -\lambda v$.所以
\begin{equation}
q\vec v = \lambda L \vec v = \pm \lambda v\vec L= I \vec L
\end{equation}
代入\autoref{FAmp_eq4} 得
\begin{equation}
\vec F = I\,\vec L \cross \vec B
\end{equation}


















