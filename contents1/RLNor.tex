% 简谐振子升降算符归一化

所以, 记住, 一般来说, 算符满足的一个条件是 $\left\langle {g}
 \mathrel{\left | {\vphantom {g {\Q Qf}}}
 \right. \kern-\nulldelimiterspace}
 {\Q Qf} \right\rangle  = \left\langle {{{\Q Q}^ * }g}
 \mathrel{\left | {\vphantom {{{{\Q Q}^ * }g} f}}
 \right. \kern-\nulldelimiterspace}
 {f} \right\rangle $.   但是对于厄米算符, ${\Q Q^ * } = \Q Q$,   所以有 $\left\langle {g}
 \mathrel{\left | {\vphantom {g {\Q Qf}}}
 \right. \kern-\nulldelimiterspace}
 {\Q Qf} \right\rangle  = \left\langle {\Q Qg}
 \mathrel{\left | {\vphantom {{\Q Qg} f}}
 \right. \kern-\nulldelimiterspace}
 {f} \right\rangle $  

对于角动量升算符, 
 \begin{equation}
\begin{aligned}
  {{\Q L}_ + }{{\Q L}_ - } &= \left( {{{\Q L}_x} + i{{\Q L}_y}} \right)\left( {{{\Q L}_x} - i{{\Q L}_y}} \right) = \Q L_x^2 + \Q L_y^2 - i[{{\Q L}_x},{{\Q L}_y}] \hfill \\
   &= {{\Q L}^2} - \Q L_z^2 + \hbar {{\Q L}_z} \hfill \\ 
\end{aligned} 
\end{equation}
所以
 \begin{equation}
\begin{aligned}
  {{\Q L}_ + }{{\Q L}_ - }{\varphi_{l,m}}&= {\hbar ^2}\left( {l + 1} \right)l{\varphi_{l,m}} - {\hbar ^2}{m^2}{\varphi_{l,m}} + m{\hbar ^2}{\varphi_{l,m}} \hfill \\
  & = {\hbar ^2}\left[ {l\left( {l + 1} \right) - m\left( {m - 1} \right)} \right]{\varphi_{l,m}} \hfill \\ 
\end{aligned} 
\end{equation}

所以  $\left\langle {{{\Q L}_ - }{\varphi_{l,m}}}
 \mathrel{\left | {\vphantom {{{{\Q L}_ - }{\varphi_{l,m}}} {{{\Q L}_ + }{\varphi_{l,m}}}}}
 \right. \kern-\nulldelimiterspace}
 {{{\Q L}_ + }{\varphi_{l,m}}} \right\rangle  = \left\langle {\varphi_{l,m}}
 \mathrel{\left | {\vphantom {{\varphi_{l,m}} {{{\Q L}_ + }{{\Q L}_ - }{\varphi_{l,m}}}}}
 \right. \kern-\nulldelimiterspace}
 {{{\Q L}_ + }{{\Q L}_ - }{\varphi_{l,m}}} \right\rangle  = {\hbar ^2}\left[ {l\left( {l + 1} \right) - m\left( {m - 1} \right)} \right]$  
 
所以 ${\Q L_ - }{\varphi_{l,m}} = \hbar \sqrt {l\left( {l + 1} \right) - m\left( {m - 1} \right)} {\varphi_{l,m - 1}}$ 

同理可证 ${\Q L_ + }{\varphi_{l,m}} = \hbar \sqrt {l\left( {l + 1} \right) - m\left( {m + 1} \right)} {\varphi_{l,m + 1}}$ 


对于谐振子的升降算符 ${\hat a_ \pm } = \frac{1}{\sqrt {2m\omega \hbar } }\left( {m\omega \hat x \mp i\hat p} \right)$.   
 \begin{equation}
  \begin{aligned}
  {{\Q a}_ - }{{\Q a}_ + } &= \frac{1}{2m\omega \hbar }\left( {{m^2}{\omega ^2}{{\Q x}^2} + {{\Q p}^2} - im\omega {\kern 1pt} {\kern 1pt} [\Q x,\Q p]} \right) \hfill \\
   &= \frac{1}{\omega \hbar }\left( {\frac{1}{2m}\left( {{m^2}{\omega ^2}{{\Q x}^2} + {{\Q p}^2}} \right) + \frac{\omega \hbar }{2}} \right) \hfill \\
  & = \frac{1}{\omega \hbar }\Q H + \frac12 \hfill \\ 
\end{aligned} 
\end{equation}
 \begin{equation}
  \begin{aligned}
  \left\langle {{{\Q a}_ + }{\varphi_n}}
 \mathrel{\left | {\vphantom {{{{\Q a}_ + }{\varphi_n}} {{{\Q a}_ + }{\varphi_n}}}}
 \right. \kern-\nulldelimiterspace}
 {{{\Q a}_ + }{\varphi_n}} \right\rangle  &= \left\langle {\varphi_n}
 \mathrel{\left | {\vphantom {{\varphi_n} {{{\Q a}_ - }{{\Q a}_ + }{\varphi_n}}}}
 \right. \kern-\nulldelimiterspace}
 {{{\Q a}_ - }{{\Q a}_ + }{\varphi_n}} \right\rangle  \hfill \\
  & = \left( {n + \frac12} \right) + \frac12\\
  &= n + 1 \hfill \\ 
\end{aligned} 
\end{equation}
\begin{equation}
  {\Q a_ + }{\varphi_n} = \sqrt {n + 1} {\varphi_{n + 1}}
\end{equation}
 
同理 ${\Q a_ - }{\varphi_n} = \sqrt n {\varphi_{n - 1}}$   

