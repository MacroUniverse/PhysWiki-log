% 二维随机走动

\pentry{中心极限定理\upref{CLT}}
\subsection{结论}
若平面上某点从坐标运点出发, 每一步沿随机方向走动一个随机步长, 步长的分布函数为 $f(r)$, $N\ \ (N \gg 1)$ 步之后, 该点的位置分布可以用\bb{圆高斯分布}表示
\begin{equation}\label{RW2D_eq1}
P(X,Y) = \frac{a}{\pi} \E^{-a(X^2 + Y^2)} \quad \text{或} \quad
P(R) = 2aR \E^{-aR^2}
\end{equation}
其中
\begin{equation}\label{RW2D_eq2}
a = \frac{1}{N\expval{r^2}} \qquad \expval{r^2} = \int_0^{\infty} r^2 f(r)\dd{r}
\end{equation}
由分布函数可得, 随机点最终离原点的距离的平均值和方均根为
\begin{equation}
\expval{R} = \frac{\sqrt{\pi}}{2}\sqrt{N\expval{r^2}} \qquad
\sqrt{\expval{R^2}} = \sqrt{N\expval{r^2}}
\end{equation}

\subsection{推导}
我们先来分析随机点的 $x$ 坐标. 假设每一步在 $x$ 方向投影的长度为 $x_i$, $N$ 步以后, 该点的 $x$ 坐标为 $X$, 则
\begin{equation}
\expval{x^2} = \int_0^{\infty} \int_0^{2\pi}  (r\cos\theta)^2 \cdot f(r)\dd{r} \cdot \frac{1}{2\pi}\dd{\theta} = \frac12 \expval{r^2}
\end{equation}
根据中心极限定理\upref{CLT}, $X$ 满足高斯分布, 且
\begin{equation}\label{RW2D_eq5}
\expval{X^2} = N\expval{x^2} = \frac12 N\expval{r^2}
\end{equation}
对 $y$ 轴分析也有类似的结果, 将 $P(X,Y)$ 分布归一化后, 可以得到\autoref{RW2D_eq1}. 我们由\autoref{RW2D_eq1} 求 $\expval{X^2}$, 得
\begin{equation}\label{RW2D_eq6}
\expval{X^2} = \int_0^\infty \int_0^\infty X^2 P(X,Y) \dd{X}\dd{Y}
= \int_0^\infty  \sqrt{\frac{a}{\pi}} X^2\E^{-a X^2} \dd{X} =\frac{1}{2a}
\end{equation}
对比\autoref{RW2D_eq5} 和\autoref{RW2D_eq6} 即可得到\autoref{RW2D_eq2}. 证毕.