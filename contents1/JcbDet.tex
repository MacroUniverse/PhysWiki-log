%雅可比行列式

若有坐标系变换
 \begin{equation}
\left\{ \begin{array}{l}
x = x\left( {u,v,w} \right)\\
y = y\left( {u,v,w} \right)\\
z = z\left( {u,v,w} \right)
\end{array} \right.
\end{equation}
根据全微分关系%(链接未完成)
 \begin{equation}
\pmat{{\D x}\\{\D y}\\{\D z}} =
\pmat{
{\partial x/\partial u}&{\partial x/\partial v}&{\partial x/\partial w} \\ 
{\partial y/\partial u}&{\partial y/\partial v}&{\partial y/\partial w} \\ 
{\partial z/\partial u}&{\partial z/\partial v}&{\partial z/\partial w} }
\end{equation}
其中 $J$ 叫做雅可比矩阵.

考虑 $uvw$ 坐标系中的一个体积元 $\left( {u,v,w} \right)\left( {u + \D u,v + \D v,w + \D w} \right)$,  一般情况下(不需要是正交曲线坐标系), 体积元为平行六面体, 起点为 $\left( {u,v,w} \right)$  的三条棱对应的矢量分别为
 \begin{equation}
\pmat{{\D {x_1}}\\{\D {y_1}}\\{\D {z_1}}} = 
J\pmat{{\D u}\\0\\0} = 
\pmat{{{J_{11}}}\\{{J_{21}}}\\{{J_{31}}}}\D u
\end{equation} 
\begin{equation}
\pmat{{\D {x_2}}\\{\D {y_2}}\\{\D {z_2}}} = 
J\pmat{0\\{\D v}\\0} = 
\pmat{{{J_{12}}}\\{{J_{22}}}\\{{J_{32}}}}\D v
\end{equation} 
\begin{equation}
\pmat{{\D {x_3}}\\{\D {y_3}}\\{\D {z_3}}} = 
J\pmat{0\\0\\{\D w}} = 
\pmat{{{J_{13}}}\\{{J_{23}}}\\{{J_{33}}}}\D w
\end{equation} 
由于平行六面体的体积是同一起点三条矢量的混合积%(链接未完成)
\begin{equation}
\D V = \vmat{
{\D {x_1}}&{\D {y_1}}&{\D {z_1}}\\
{\D {x_2}}&{\D {y_2}}&{\D {z_2}}\\
{\D {x_3}}&{\D {y_3}}&{\D {z_3}}}
= \vmat{
{\D {x_1}}&{\D {x_2}}&{\D {x_3}}\\
{\D {y_1}}&{\D {y_2}}&{\D {y_3}}\\
{\D {z_1}}&{\D {z_2}}&{\D {z_3}}
}
= \abs{J}\D u \D v \D w
\end{equation}
其中 $\abs{J}$  叫做\bb{雅可比行列式}.

