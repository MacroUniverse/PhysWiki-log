%直积空间

\subsubsection{1}
两个矢量空间的直积空间就是两个空间所有可能的基底组合. 整个直积空间中任意矢量可以表示为 $\ket{\psi} = \sum\limits_{i = 1}^m  \sum\limits_{j = 1}^n c_{ij}\ket{u_i}\ket{v_j} $,  从而可以把所有系数写成一个列矢量表示. 然而顺序非常重要, 计算中要保持一致.这里规定顺序为 $\{ \ket{u_1 v_1}, \ket{u_1 v_2}\dots \ket{u_2 v_1}, \ket{u_2 v_2}\dots \}$. 在以下讨论中, 我们假设所有的基底都是正交归一的. 一个直积空间可以用两种方法分成子空间, 一种是根据 $\ket{u_i v_i}$ 中 $u_i$  的不同来划分, 另一种根据 $\ket{v_i}$ 不同来划分. 姑且分别叫做 $\ket{u_i}$ 子空间和 $\ket{v_i}$ 子空间.

\subsubsection{2}
两个矢量(分别来自两空间)的直积定义为: 先把它们分别在各自的基底上展开, 然后用乘法分配律进行相乘. 两空间的基底相乘得到直积空间中新的基底. 直积空间中, 只有一些矢量可以表示成两个空间中的矢量的一次直积运算. 这种矢量的特征是, 若投影到不同子空间, 则对应分量成正比.

\subsubsection{3}
两个算符的直积变成的(线性)算符可以作用在直积空间中的任意矢量. 先定义作用在任意直积矢量上的结果为
\begin{equation}
(\Q A \otimes \Q B) (\ket{u} \otimes \ket{v}) = (\Q A \ket{u}) \otimes (\Q B \ket{v})
\end{equation}
要对任意矢量作用, 只需将其拆成直积基底的线性组合, 然后再分别对直积基底作用即可. 特殊地, 可以用 $\Q A \otimes \Q I$ 运算将 $\{\ket{u_i}\}$  空间中的 $\Q A$ 拓展到直积空间中来
\begin{equation}
(\Q A \otimes \Q I)(\ket{u_i} \otimes \ket{v_j}) = (\Q A \ket{u_i}) \otimes \ket{v_j} 
\end{equation}
等式右边的矢量仍然落在 $\ket{v_i}$ 子空间中. 所以, 算符 $\Q A \otimes \Q I$ 作用在直积空间的任意矢量上, 相当于 $\Q A \otimes \Q I$ 对各个子空间中的分量作用. $\Q I \otimes \Q B$ 的作用类似. 根据定义, 不难证明
\begin{equation}
(\Q A \otimes \Q I)(\Q I \otimes \Q B)(\ket{u_i} \otimes \ket{v_j}) = (\Q I \otimes \Q B)(\Q A \otimes \Q I)(\ket{u_i} \otimes \ket{v_j}) = (\Q A \ket{u}) \otimes (\Q B \ket{v})
\end{equation}
即两算符对易且等于 $(\Q A \otimes \Q B)$. 

\subsubsection{矢量的内积}

定义两个直积矢量的内积分别为每个空间中对应矢量的内积的乘积.
\begin{equation}
(\bra{u'}\otimes\bra{v'})(\ket{u}\otimes\ket{v}) = \braket{u}{u'}\vdot \braket{v}{v'}
\end{equation}
计算任意两个矢量的内积, 只需分解成直积空间基底之间的内积再运用以上定律即可. 如果要求直积空间的基底正交归一, 任意两基底必须满足
\begin{equation}
(\bra{u_{i'}} \otimes \bra{v_{j'}})(\ket{u_i} \otimes \ket{v_j}) = \braket{u_{i'}}{u_i} \braket{v_{i'}}{v_i} = \delta_{i,i'}\delta_{j,j'}
\end{equation}
这就等效于要求 $\{\ket{u_i}\}$ 和 $\{\ket{v_i}\}$  分别都是正交归一基底.

\subsubsection{矩阵元的计算}
若直积空间中的基底正交归一, 求矩阵元只需用
\begin{equation}
\bra{u_{i'} v_{j'}} (\Q A \otimes \Q B) \ket{u_i v_j} = \bra{u_{i'} v_{j'}} (\Q A \ket{u_i} \otimes \Q B \ket{v_j}) = \bra{u_{i'}} \Q A \ket{u_i} \bra{v_{j'}} \Q B \ket{v_j}
\end{equation}
现在用分块矩阵的概念, 若把矢量分成一段段, 每一段是 $u$ 子空间中的系数, 矩阵也会分成一些小块. 在 $(m, n)$ 小块中, 根据式, 这个分块中的矩阵元为
\begin{equation}
A_{mn} \vdot B
\end{equation}
所以, $\Q A \otimes \Q B$ 的矩阵是把 $\mat A$ 的每个矩阵元 $A_{mn}$ 拓展成矩阵分块 $A_{mn} \mat B$.  注意这是以 $u$ 空间来划分列矩阵. 反之, 如果是根据 $v$ 空间来划分列矩阵, 那么就是把 $\mat B$ 的每个矩阵元 $B_{mn}$ 拓展成矩阵分块 $B_{mn} \mat A$. 

\subsection{关于本征问题的定理}%(证明未完成)
	
*如果考虑直积空间中的本征问题, $A_1$  的本征矢 $\ket{eig_i}$ 具有 $n$ 重简并, 简并空间的基底分别为 $\ket{eig_i}\ket{v_1}, \ket{eig_i}\ket{v_2}\dots$. 

* $A_1 \otimes B_2$ 的本征值共有 $m \times n$ 个, $m$ 和 $n$ 分别是 $A$ 和 $B$ 的维度, 若 $a_1, a_2,\dots, a_m$ 和 $b_1, b_2, \dots, b_n$ 分别是 $A$ 和 $B$ 的本征值, 那么 $A \otimes B$ 的本征值分别为 $a_1 b_1, a_1 b_2, \dots, a_2 b_1, a_2 b_2, \dots, a_m b_{n-1}, a_m b_n$.  本征矢为 $\ket{u_i} \otimes \ket{v_j}$ 

* $A \otimes I + I \otimes B$ 的本征值分别为 $a_i + b_j$,  本征矢同样为 $\ket{u_i} \otimes \ket{v_j}$. 

*单个空间的厄米矩阵拓展到直积空间仍然是厄米矩阵. 两个空间分别的厄米矩阵直积仍然是厄米矩阵. 

*厄米矩阵加厄米矩阵仍然是厄米矩阵.

