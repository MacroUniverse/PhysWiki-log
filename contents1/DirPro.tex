%直积空间
%35min

\begin{enumerate}
\item 两个矢量空间的直积空间就是两个空间所有可能的基底组合. 整个直积空间中任意矢量可以表示为 $\left| \psi  \right\rangle  = \sum\limits_{i = 1}^m {\left( {\sum\limits_{j = 1}^n {{c_{ij}}\left| {{u_i}} \right\rangle \left| {{v_j}} \right\rangle } } \right)} $,  从而可以把所有系数写成一个列矢量表示. 然而顺序非常重要, 计算中要保持一致.这里规定顺序为 $\left\{ {\left| {{u_1}{v_1}} \right\rangle ,\left| {{u_1}{v_2}} \right\rangle ...\left| {{u_2}{v_1}} \right\rangle ,\left| {{u_2}{v_2}} \right\rangle ......} \right\}$ 
.  在以下讨论中, 我们假设所有的基底都是正交归一的. 一个直积空间可以用两种方法分成子空间, 一种是根据 $\left| {{u_i}{v_i}} \right\rangle $ 中 ${u_i}$  的不同来划分, 另一种根据 $\left| {{v_i}} \right\rangle$ 不同来划分. 姑且分别叫做 $\left| {{u_i}} \right\rangle $ 子空间和 $\left| {{v_i}} \right\rangle $ 子空间.

\item 两个矢量(分别来自两空间)的直积定义为: 先把它们分别在各自的基底上展开, 然后用乘法分配律进行相乘. 两空间的基底相乘得到直积空间中新的基底. 直积空间中, 只有一些矢量可以表示成两个空间中的矢量的一次直积运算. 这种矢量的特征是, 若投影到不同子空间, 则对应分量成正比.

\item 两个算符的直积变成的(线性)算符可以作用在直积空间中的任意矢量. 先定义作用在任意直积矢量上的结果为
 \begin{equation}
  (\Q A \otimes \Q B)(\left| u \right\rangle  \otimes \left| v \right\rangle ) = (\Q A\left| u \right\rangle ) \otimes (\Q B\left| v \right\rangle )
\end{equation}
要对任意矢量作用, 只需将其拆成直积基底的线性组合, 然后再分别对直积基底作用即可. 特殊地, 可以用 $\Q A \otimes \Q I$ 运算将 $\left\{ {\left| {{u_i}} \right\rangle } \right\}$  空间中的 $\Q A$ 拓展到直积空间中来
 \begin{equation}
  (\Q A \otimes \Q I)(\left| {{u_i}} \right\rangle  \otimes \left| {{v_j}} \right\rangle ) = (\Q A\left| {{u_i}} \right\rangle ) \otimes \left| {{v_j}} \right\rangle 
\end{equation}
等式右边的矢量仍然落在 $\left| {{v_i}} \right\rangle $ 子空间中. 所以, 算符 $\Q A \otimes \Q I$ 作用在直积空间的任意矢量上, 相当于 $\Q A \otimes \Q I$ 对各个子空间中的分量作用. $\Q I \otimes \Q B$ 的作用类似. 根据定义, 不难证明
 \begin{equation}
  (\Q A \otimes \Q I)(\Q I \otimes \Q B)(\left| {{u_i}} \right\rangle  \otimes \left| {{v_j}} \right\rangle ) = (\Q I \otimes \Q B)(\Q A \otimes \Q I)(\left| {{u_i}} \right\rangle  \otimes \left| {{v_j}} \right\rangle ){\text{ = }}(\Q A\left| u \right\rangle ) \otimes (\Q B\left| v \right\rangle )
\end{equation}
即两算符对易且等于 $(\Q A \otimes \Q B)$. 

\item 矢量的内积

定义两个直积矢量的内积分别为每个空间中对应矢量的内积的乘积.
 \begin{equation}
 \left( {\left\langle {u'} \right| \otimes \left\langle {v'} \right|} \right)\left( {\left| u \right\rangle  \otimes \left| v \right\rangle } \right) = \left\langle {u}
 \mathrel{\left | {\vphantom {u {u'}}}
 \right. \kern-\nulldelimiterspace}
 {{u'}} \right\rangle  \vdot \left\langle {v}
 \mathrel{\left | {\vphantom {v {v'}}}
 \right. \kern-\nulldelimiterspace}
 {{v'}} \right\rangle 
\end{equation}
计算任意两个矢量的内积, 只需分解成直积空间基底之间的内积再运用以上定律即可. 如果要求直积空间的基底正交归一, 任意两基底必须满足
 \begin{equation}
 \left( {\left\langle {{u_{i'}}} \right| \otimes \left\langle {{v_{j'}}} \right|} \right)\left( {\left| {{u_i}} \right\rangle  \otimes \left| {{v_j}} \right\rangle } \right) = \left\langle {{{u_{i'}}}}
 \mathrel{\left | {\vphantom {{{u_{i'}}} {{u_i}}}}
 \right. \kern-\nulldelimiterspace}
 {{{u_i}}} \right\rangle  \vdot \left\langle {{{v_{i'}}}}
 \mathrel{\left | {\vphantom {{{v_{i'}}} {{v_i}}}}
 \right. \kern-\nulldelimiterspace}
 {{{v_i}}} \right\rangle  = {\delta _{i,i'}}{\delta _{j,j'}}
\end{equation}
这就等效于要求 $\left\{ {\left| {{u_i}} \right\rangle } \right\}$ 和 $\left\{ {\left| {{v_i}} \right\rangle } \right\}$  分别都是正交归一基底.

\item 矩阵元的计算. 若直积空间中的基底正交归一, 求矩阵元只需用
 \begin{equation}
 \left\langle {{u_{i'}}{v_{j'}}} \right|\left( {\Q A \otimes \Q B} \right)\left| {{u_i}{v_j}} \right\rangle  = \left\langle {{u_{i'}}{v_{j'}}} \right|\left( {\Q A\left| {{u_i}} \right\rangle  \otimes \Q B\left| {{v_j}} \right\rangle } \right) = \left\langle {{u_{i'}}} \right|\Q A\left| {{u_i}} \right\rangle  \vdot \left\langle {{v_{j'}}} \right|\Q B\left| {{v_j}} \right\rangle 
\end{equation}
现在用分块矩阵的概念, 若把矢量分成一段段, 每一段是 $u$ 子空间中的系数, 矩阵也会分成一些小块. 在 $m$,  $n$ 小块中, 根据式, 这个分块中的矩阵元为
 \begin{equation}
  {A_{mn}} \vdot B
\end{equation}
所以,  $\Q A \otimes \Q B$ 的矩阵是把 $A$ 的每个矩阵元 ${A_{mn}}$ 拓展成矩阵分块 ${A_{mn}}B$.  注意这是以 $u$ 空间来划分列矩阵. 反之, 如果是根据 $v$ 空间来划分列矩阵, 那么就是把 $B$ 的每个矩阵元 ${B_{mn}}$ 拓展成矩阵分块 ${B_{mn}}A$. 
\end{enumerate}

\subsection{关于本征问题的定理}%(证明未完成)
	
*如果考虑直积空间中的本征问题, ${A_1}$  的本征矢 $\left| {ei{g_i}} \right\rangle $ 具有 $n$ 重简并, 简并空间的基底分别为 $\left| {ei{g_i}} \right\rangle \left| {{v_1}} \right\rangle ,\left| {ei{g_i}} \right\rangle \left| {{v_2}} \right\rangle ....$. 

* ${A_1} \otimes {B_2}$ 的本征值共有 $m \cross n$ 个 $m$ 和 $n$ 分别是 $A$ 和 $B$ 的维度, 若 ${a_1},{a_2}\,...\,{a_m}$ 和 ${b_1},{b_2}\,...\,{b_n}$ 分别是 $A$ 和 $B$ 的本征值, 那么 $A \otimes B$ 的本征值分别为 ${a_1}{b_1},\,{a_1}{b_2}\,...\,{a_2}{b_1},{a_2}{b_2}\,...\,{a_m}{b_{n - 1}},{a_m}{b_n}$.  本征矢为 $\left| {{u_i}} \right\rangle  \otimes \left| {{v_j}} \right\rangle $ 

* $A \otimes I + I \otimes B$ 的本征值分别为 ${a_i} + {b_j}$,  本征矢同样为 $\left| {{u_i}} \right\rangle  \otimes \left| {{v_j}} \right\rangle $. 

*单个空间的厄米矩阵拓展到直积空间仍然是厄米矩阵. 两个空间分别的厄米矩阵直积仍然是厄米矩阵. 

*厄米矩阵加厄米矩阵仍然是厄米矩阵.

