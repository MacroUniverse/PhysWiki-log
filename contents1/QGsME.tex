%量子气体(巨正则系综法)
%10min
假设 $N$ 个玻色子(不可区分)之间没有互相作用, 每个粒子具有能级 ${\varepsilon _i}$ (这里先假设每个粒子都只有 $m$ 个能级而不是无穷多个, 最后再说明 $m \to \infty $ 时成立).化学势为 $\mu $,  则巨分配函数为
\begin{equation}
  \begin{aligned}
\Xi & = \sum\limits_{N = 0}^\infty  {\sum\limits_{\{ {n_i}\} }^ *  {{e^{(N\mu  - {\varepsilon _i})\beta }}} }  \\
& = \sum\limits_{N = 0}^\infty  {\sum\limits_{\{ {n_i}\} }^ *  {{z^N}\exp \left( { - \beta \sum\limits_i^m {{n_i}{\varepsilon _i}} } \right)} }  \\
& = \sum\limits_{N = 0}^\infty  {\sum\limits_{\{ {n_i}\} }^ *  {\left( {\prod\limits_{i = 1}^\infty  {{z^{{n_i}}}} } \right)\left( {\prod\limits_{i = 1}^\infty  {{{({e^{ - \beta {\varepsilon _i}}})}^{{n_i}}}} } \right)} } \\
& = \sum\limits_{N = 0}^\infty  {\sum\limits_{\{ {n_i}\} }^ *  {\prod\limits_{i = 1}^\infty  {{{(z{e^{ - \beta {\varepsilon _i}}})}^{{n_i}}}} } }
\end{aligned}
\end{equation}
其中 $\sum\limits_{\{ {n_i}\} }^ *  {} $ 求和时, 由于同一个能级上的玻色子数量不限, 限制条件仅为 $\sum\limits_i^m {{n_i}}  = N$ (对于费米子, 由于不相容原理, 要求 $\sum\limits_i^m {{n_i}}  = N$ 以及 ${n_i} \le 1$ ).\\
巨正则系综的最大优势就是可以利用关系
\begin{equation}
  \sum\limits_{N = 0}^\infty  {\sum\limits_{\{ {n_i}\} }^ *  {({\kern 1pt} {\kern 1pt} {\kern 1pt} )} }  = \sum\limits_{{n_1} = 0}^\infty  {\sum\limits_{{n_2} = 0}^\infty  {...\sum\limits_{m = 0}^\infty  {({\kern 1pt} {\kern 1pt} {\kern 1pt} )} } } 
\end{equation}
于是
\begin{equation}
  \begin{aligned}
\Xi & = \sum\limits_{N = 0}^\infty  {\sum\limits_{\{ {n_i}\} }^ *  {\prod\limits_{i = 1}^\infty  {{{(z{e^{ - \beta {\varepsilon _i}}})}^{{n_i}}}} } }  \\
& = \sum\limits_{{n_1} = 0}^\infty  {\sum\limits_{{n_2} = 0}^\infty  {...\sum\limits_{{n_m} = 0}^\infty  {\prod\limits_{i = 1}^\infty  {{{(z{e^{ - \beta {\varepsilon _i}}})}^{{n_i}}}} } } }  \\
& = \sum\limits_{{n_1} = 0}^\infty  {{{(z{e^{ - \beta {\varepsilon _1}}})}^{{n_1}}}\sum\limits_{{n_2} = 0}^\infty  {{{(z{e^{ - \beta {\varepsilon _2}}})}^{{n_2}}}...\sum\limits_{{n_m} = 0}^\infty  {{{(z{e^{ - \beta {\varepsilon _m}}})}^{{n_m}}}} } } \\
& = \prod\limits_{i = 1}^m {\sum\limits_{{n_i}}^\infty  {{{(z{e^{ - \beta {\varepsilon _i}}})}^{{n_i}}}} }
\end{aligned}
\end{equation}
