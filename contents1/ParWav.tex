% 量子散射 分波展开

散射截面 $\sigma$ 等于一定时间内被散射的粒子数除以单位截面入射的粒子数.那么从经典力学的角度,如果想象入射粒子流密度是均匀的, $\sigma$ 可以看做是一个障碍物(无远程作用)的最大横截面面积,微分截面 $\D \sigma/\D\Omega$ 可以理解为单位立体角的散射截面.量子力学中,如果考虑单粒子以平面波入射,那么 $\sigma$ 等于被散射的概率流(概率/时间)除以入射的概率流密度(概率/时间/面积).概率流定义为
\begin{equation}
\vec j = \frac{{i\hbar }}{{2m}}(\psi \grad {\psi ^ * } - {\psi ^ * }\grad \psi )
\end{equation}
\begin{equation}\label{ParWav_eq3}
\sigma  = \mathop {\lim }\limits_{r \to \infty } \int {\frac{{{\text{(}}{{\vec j}_{sc}} \vdot \hat r{\text{)}}}}{{\left| {{{\vec j}_{inc}}} \right|}}{r^2}\D \Omega } 
\qquad
\frac{{\D \sigma }}{{\D \Omega }} = \mathop {\lim }\limits_{r \to \infty } \frac{{{\text{(}}{{\vec j}_{sc}} \vdot \hat r{\text{)}}{r^2}}}{{\left| {{{\vec j}_{inc}}} \right|}}
\end{equation}
在球坐标中解定态薛定谔方程,能量和角动量的本征基底为
\begin{equation}
{\psi _{k,l,m}}(\vec r) = {R_{k,l}}(r)Y_l^m(\theta ,\phi )
\end{equation}
其中径向波函数满足径向方程
\begin{equation}\label{ParWav_eq4}
- \frac{{{\hbar ^2}}}{{2m}}\frac{{{\D ^2}}}{{\D {r^2}}}\left( {r{R_{k,l}}} \right) + \left[ {V(r) + \frac{{{\hbar ^2}}}{{2m}}\frac{{l(l + 1)}}{{{r^2}}}} \right]\left( {r{R_{k,l}}} \right) = E\left( {r{R_{k,l}}} \right)
\end{equation}
原则上我们只需要把初始波包在这个基底上展开,加上时间因子即可得到 $t =  + \infty$ 时概率的分布.现在我们假设势能没有角分布( $m = 0$ ),且无穷远处势能为0,则基底化简为
\begin{equation}
{\psi _{k,l}}(\vec r) = {R_{k,l}}(r)Y_l^0(\theta ) = \sqrt {\frac{{2l + 1}}{{4\pi }}} {R_{k,l}}(r){P_l}(\cos \theta )
\end{equation}
\begin{equation}\begin{aligned}\label{ParWav_eq6}
{R_{k,l}}(r \to  + \infty ) &= {A_l}{j_l}(kr) + {B_l}{n_l}(kr) \\
&= [{A_l}\sin (kr - l\pi /2) - {B_l}\cos(kr - l\pi /2)]/r \\
&= \sin [kr - l\pi /2 + {\delta _l}(k)]/r
\end{aligned}\end{equation}
其中 ${\delta _l}(k) = \arctan ( { - {B_l}/{A_l}} )$. 注意径向函数只能是实数,否则将会有概率流持续流入或流出原点.

然而我们也可以选择能量和无穷远处的线性动量 $\vec k$ 作为本征值(由对称性,令 $\vec k=k\uvec z$),求出本征基底,这样如果初始波包有很窄的动量分布(近似为平面波),我们仅从本征基底的角分布就可求出微分截面而无需分解波包.令该基底为
\begin{equation}\label{ParWav_eq7}
{\psi _k}(\vec r) = {\E^{\I kz}} + \sum\limits_l {{\rho _{k,l}}(r){P_l}(\cos \theta )} 
\end{equation}
由于在无穷远处,平面波就是定态薛定谔方程的解,所以剩下的项也应该是.且由于散射只有向外的概率流,令
\begin{equation}\label{ParWav_eq8}
{\rho _{k,l}}(r \to {\text{ + }}\infty ) = (2l + 1){a_l}(k)\frac{{{\E^{\I kr}}}}{r}
\end{equation}
其中 $(2l + 1)$ 是为了以下计算方便,球面波的相位包含在 ${a_l}(k)$ 中.该基底在无穷远处也可记为
\begin{equation}\label{ParWav_eq9}
{\psi _k}(\vec r) = {\E^{\I kz}} + f(k,\theta )\frac{{{\E^{\I kr}}}}{r}
\end{equation}
其中 $f(k,\theta ) = \sum\limits_l {(2l + 1){a_l}(k){P_l}(\cos \theta )}$.注意\autoref{ParWav_eq9} 在无穷远处是精确成立的.若从波包的角度考虑,入射波包可以看做仅由第一项展开得到,出射波包的 $\theta  \ne 0$ 部分仅由第二项展开得到,所以可以仅用第一项计算 ${\vec j_{inc}}$, 第二项计算 ${\vec j_{sc}}$,代入\autoref{ParWav_eq3} 得
\begin{equation}
\frac{{\D \sigma }}{{\D \Omega }} = {\left| {f(k,\theta )} \right|^2}
\end{equation}
假设我们已经在球坐标中解出了 ${\psi _{k,l}}$,即径向波函数 ${R_{k,l}}(r)$ 与相移, 如何获得 $f(k,\theta )$,即系数 ${a_l}(k)$ 呢? 把 ${\psi _k}$ 用 $\psi_{k,l}$ 基底展开,即对 ${P_l}$ 展开,再逐项对比系数即可.首先展开平面波
\begin{equation}
{\E^{\I kz}}{\text{ = }}\sum\limits_{l = 0}^\infty  {{i^l}(2l + 1){j_l}(kr){P_l}(\cos \theta )} 
\end{equation}
无穷远处
\begin{equation}\label{ParWav_eq12}
{\E^{\I kz}}{\text{ = }}\sum\limits_{l = 0}^\infty  {(2l + 1)\frac{{{\E^{\I kr}} - {\E^{ - \I(kr - l\pi )}}}}{{2ikr}}{P_l}(\cos \theta )} 
\end{equation}
将\autoref{ParWav_eq12} 与\autoref{ParWav_eq8} 代入\autoref{ParWav_eq7},再逐项与\autoref{ParWav_eq6} 对比,得
\begin{equation}
{a_l}(k) = \frac{{{\E^{2\I {\delta _l}(k)}} - 1}}{{2ik}}
\end{equation}
总结起来,轴对称的有心力的散射问题只需通过径向方程\autoref{ParWav_eq4} 获得 ${\delta _l}(k)$,求出 ${a_l}(k)$ 和 $f(k,\theta )$ 即可.

