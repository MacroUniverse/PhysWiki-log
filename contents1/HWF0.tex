%氢原子基态的波函数

\pentry{波函数简介}%未完成链接

由于波函数的统计诠释,统计在量子力学中经常碰到,所以这里举一个例子让你熟悉一下统计的一些常见计算.

氢原子是唯一有解析解的物理实例,因为它结构简单,只有一个核外电子.由于核外电子质量又远小于原子核的质量,忽略核的运动,且不计万有引力.

氢原子基态的波函数为
\begin{equation}
\psi(\vec r) = A\E^{-r/a}
\end{equation}
其中 ${a_0} = {{4\pi {\varepsilon_0}{\hbar ^2}}}/({{{m_e}{e^2}}})$ 是量子理论中一个重要的常数,\bb{玻尔半径}.由于这是个球对称函数,所以氢原子的波函数通常在球坐标中表示,即表示成三个球坐标的函数 $\psi \left( {\vec r} \right) = \psi \left( {r,\theta ,\phi } \right)$. 其模长平方同样表示粒子在某点出现的概率密度.由于氢原子基态的波函数是球对称的,所以只是 $r$ 的函数.

\subsection{归一化}
  
概率密度必须归一化,也就是说,在所有地方找到电子的概率之和为必为 $1$. 所以可以用归一化来确定波函数前面的系数 $A$. 把概率密度对整个空间体积分
\begin{equation}
    1 = \int {{{\left| {\psi \left( {\vec r} \right)} \right|}^2} \D V}  = \int_0^{ + \infty } {(A^2 \E^{-2r/a})(4\pi r^2\D r)}  = {A^2}\pi {a^3}
\end{equation}
所以 $A = {1}/{{\sqrt {\pi {a^3}} }}$, 
\begin{equation}
\psi \left( {\vec r} \right) = \frac{1}{{\sqrt {\pi {a^3}} }}{\E^{-r/a}}
\end{equation}

\subsection{位置的平均值}

 根据连续概率分布中平均值%未完成链接
(或数学期望)的定义
\begin{equation}
\left\langle {\vec r} \right\rangle  = \int {\vec r{{\left| {\psi \left( {\vec r} \right)} \right|}^2} \D V = } \vec 0
\end{equation}
积分显然为零,因为波函数关于中心呈球对称分布,各个方向的 $\vec r$ 互相抵消了.所以如果对足够多个处于基态的氢原子测量电子的位置,并求平均位置(矢量), 一定会在原子核处.

\subsection{电子离原子核距离的平均值}

如果在上题中,求平均值的不是位置矢量,而是位置的大小,那么结果显然是大于零的.
\begin{equation}
 \left\langle r \right\rangle = \int {r{{\left| {\psi \left( {\vec r} \right)} \right|}^2} \D V} = \int {r \left(A^2 \E^{-2r/a}\right)\left( {4\pi {r^2} \D r} \right)}= \frac{3}{8}{a^4}A = \frac{3}{2}a
\end{equation}
注意这比玻尔半径要大.

\subsection{电子最可能出现的位置}

 一个位置的波函数模长平方越大,电子越有可能出现在这个位置.所以现在要求的是概率密度出现最大值的位置.
 
 根据指数函数的性质,最大值 $\left| {\psi \left( {\vec r} \right)} \right|_{\max }^2 ={\left( {\E^{-0/a}}/{{\sqrt {\pi {a^3}} }} \right)^2} = {1}/({{\pi {a^3}}})$, 最大值位置为 $\vec r = \vec 0$. 

\subsection{电子与原子核最有可能的距离}
若定义径向概率密度为
\begin{equation}
f(r) = \lim_{\Delta r \to 0} \frac{{\int_r^{\Delta r} {{{\left| {\psi \left( {\vec r} \right)} \right|}^2}\left( {4\pi {r^2} \D r} \right)} }}{{\Delta r}} = 4\pi {r^2}{\left| {\psi \left( {\vec r} \right)} \right|^2}
  \end{equation}
  要求最可能出现的半径 $f{\left( r \right)_{\max }}$,  可以对其求导(见导数与极值%未完成链接
  ).即 ${{ \D f\left( r \right)}}/{{ \D r}} = 0$ 即 ${8 r{\E^{-2r/a}}}/{{{a^3}}} - ({4}/{{{a^3}}})({2}/{a}){r^2}{\E^{-2r/a}} = 0$, 解得 $a = r$. 
 
这个重要结论说明,\bb{玻尔半径就是氢原子基态中电子与原子核最有可能的距离}.
