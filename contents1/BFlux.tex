% 磁通量的定义
% 8 min
定义通过某曲面的磁通量为
\begin{equation}
\Phi  = \int {\vec B \vdot \D \vec a} 
\end{equation}
利用磁场矢势%未完成:链接
及旋度定理, %未完成:链接
磁通量变为
\begin{equation} \label{BFlux_eq2}
\Phi  = \int {\curl \vec A \vdot \D \vec a}  = \oint {\vec A \vdot \D \vec r} 
\end{equation}
另外, 由于磁场的散度为零, 根据高斯定理, 任何闭合曲面的磁通量都是 0.
用另一种方式来理解: 如果选定一个闭合回路, 以该闭合回路为边界的任何曲面的磁通量都相等.

\subsection{闭合线圈的磁通量}

如何计算一个闭合线圈对自己产生的磁通量呢? 利用磁场矢势公式
\begin{equation}
\vec A\left( {\vec r} \right) = \frac{{{\mu _0}I}}{{4\pi }}\oint {\frac{{\D \vec r'}}{{\left| {\vec r - \vec r'} \right|}}} 
\end{equation}
注意在该积分中, \vec r 视为常量, 积份完后, 积分变量 \vec r 消失. 现在根据\autoref{BFlux_eq2} 再次将上式对 \vec r 进行同一环路积分得到磁通量
\begin{equation}
\Phi  = \frac{{{\mu _0}I}}{{4\pi }}\oint {\oint {\frac{{\D \vec r'\D \vec r}}{{\left| {\vec r - \vec r'} \right|}}} } 
\end{equation}
