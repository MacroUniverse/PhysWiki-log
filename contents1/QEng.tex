% 电势 电势能
% 推导 1/r 势能, 计算连续电荷的势能, 例如带电球壳的势能

\begin{equation}
\Delta V = -\int \vec E(\vec r) \vdot \dd{\vec l}
\end{equation}

连续电荷分布产生的电势
\begin{equation}\label{QEng_eq1}
V(\vec r) = \frac{1}{4\pi\epsilon_0} \int \frac{\rho(\vec r)}{\abs{\vec r - \vec r'}} \dd[3]{r'}
\end{equation}

\begin{exam}{均匀带电球}
令球的半径为 $R$, 电荷密度为 $\rho$, 令无穷远处为零势点, 求均匀带电球内外的电势分布.

%未完成
\end{exam}

\subsection{电势能}
若要将点电荷 $q$ 从电场中从点 1 移动到点 2, 外力需要克服电场力对电荷做功
\begin{equation}\label{QEng_eq2}
W = -\int_1^2 \vec F \vdot \dd{\vec l} = -\int_1^2 \vec E q \vdot \dd{\vec l}
= qV_2 - qV_1
\end{equation}
也就是说, 外力将电荷从电场中的一点移到另一点需要做的功等于末势能减初势能再乘以电荷量. 所以, 我们可以将点电荷的电势能定义为电荷量乘以电势, 这样, 外力做功就等于电势能的增加.

现在来考虑多个点电荷间的电势能(假设不存在额外的电场). 我们定义将这些点电荷各自从无穷远处移动到指定位置所需要能量等于它们的电势能. 若只有两个电荷 $q_1, q_2$ 相距 $r_12$, 把 $q_1$ 从无穷远移动到指定位置都不需要能量, 因为没有额外的电场. 然而移动 $q_2$ 时, 要考虑 $q_1$ 的电场, 由\autoref{QEng_eq2} 可知需要的能量为 $W = q_1q_2/(4\pi\epsilon_0 r_12)$. 以此类推, 当有 $n$ 个点电荷 $q_1, \dots q_n$, 位矢分别为 $\vec r_i$ 时, 总电势能为
\begin{equation}\label{QEng_eq3}
W = \frac12\sum_{i \neq j} \frac{q_i q_j}{4\pi\epsilon_0 \abs{\vec r_j - \vec r_i}}
\end{equation}
在求和前面加 $1/2$ 是因为对于每两个点电荷 $q_m, q_n\quad (m \neq n)$, 它们之间的电势能 $W_{mn} = q_m q_n/(4\pi\epsilon_0 r_mn)$ 被计算了两次, 一次是 $i = m, j = n$, 另一次是 $i = n, j = m$.

我们试图将\label{QEng_eq3} 化成含有电势的形式
\begin{equation}
W = \frac12 \sum_i q_i \sum_{j\neq i} \frac{q_j}{4\pi\epsilon_0\abs{\vec r_j - \vec r_i}}
=\frac12 \sum_i q_i V_i
\end{equation}
其中我们定义 $V_i$ 为 $q_i$ 之外所有点电荷产在 $\vec r_i$ 处产生的电势. 现在, 我们可以写出连续电荷分布的电势能为
\begin{equation}\label{QEng_eq6}
W = \frac 12 \int V(\vec r) \rho(\vec r) \dd[3]{r}
\end{equation}
注意这里的 $V(\vec r)$ 由\autoref{QEng_eq1} 定义\footnote{这里的 $V(\vec r)$ 为什么不排除 $\vec r$ 所在的一小块电荷产生的电势能呢? 粗略地说, 因为当体积元 $\dd[3]{r}$ 为无穷小时, 其中的电荷存不存在对 $V(\vec r)$ 的改变也是无穷小的.}.