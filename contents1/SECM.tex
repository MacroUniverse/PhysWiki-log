% 质心系中的多粒子问题

以一维的 $H_2^+$ 为例,原子核质量为 $M$,电子质量为 $m$,哈密顿量为
\begin{equation}
H = T_{N1} + T_{N2} + T_{e} + V(X_1,X_2,x)
\end{equation}
若势能函数只与位置有关,那么一般我们会把方程化为质心系方程.令总波函数为
\begin{equation}
\Psi(X_1,X_2,x) = \psi_{cm}(X_{cm})\psi(X_1,X_2,x)
\end{equation}

构建不同坐标的哈密顿量.第一可以直接用数学的偏微分变量替换,第二可以通过经典力学的哈密顿量来计算.

用两个质量不同的粒子为例会不会更好一点,rigid rotator!或者氢原子!先解释一下约化质量是怎么来的!为什么玻尔模型用约化质量会有那么高的精度,对应的量子力学原理就在这里!