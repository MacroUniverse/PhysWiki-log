% 波恩近似(散射)

我们还是要解出连续态的不含时波函数,且无穷远处的动量为 $\vec k_i$ (入射平面波的动量).
\begin{gather}
- \frac{{{\hbar ^2}}}{{2m}}{\laplacian}\psi  + V\psi  = E\psi \\
({\laplacian} + {k^2})\psi  = \frac{{2m}}{{{\hbar ^2}}}V\psi  \equiv U(\vec r)\psi
\end{gather}
这是非齐次亥姆霍兹方程,其格林函数% 未完成: 格林函数 亥姆霍兹方程的格林函数
为(球面波)
\begin{equation}
G(R) =  - \frac{{{e^{ikR}}}}{{4\pi R}}
\end{equation}
满足
\begin{equation}
({\laplacian} + {k^2})G(\vec r) = {\delta ^3}(\vec r)
\end{equation}
\textbf{薛定谔方程的积分形式}为
\begin{equation}\label{BornSc_eq5}
\psi (\vec r) = {\psi _0}(\vec r) + \int {G(\left| {\vec r - \vec r'} \right|)U(\vec r')\psi (\vec r'){\D ^3}r'} 
\end{equation}
$\psi_0$ 是自由粒子波函数,由于无穷远处积分项消失($1/r$), $\psi_0(\vec r\to\infty)$ 要求具有动量 $\vec k_i$,唯一的选择是平面波
 \begin{equation}
{\psi _0}(\vec r) = A{e^{i{{\vec k}_i} \vdot \vec r}}
\end{equation}
由于微分截面定义在无穷远处,我们把格林函数取无穷远处的极限(远场),注意这个极限在定义中,所以并不算是一个近似.这是关于 $\vec r'$ 的平面波
\begin{gather}
\left| {\vec r - \vec r'} \right| \approx r - \hat r \vdot \vec r' \approx r\\
G(\vec r,\vec r') =  - \frac{{{e^{ik\left| {\vec r - \vec r'} \right|}}}}{{4\pi \left| {\vec r - \vec r'} \right|}} \to  - \frac{{{e^{ikr}}}}{{4\pi r}}{e^{ - i{{\vec k}_f} \vdot \vec r'}}
\end{gather}
其中 ${\vec k_f} = k\hat r$ 是出射的方向,注意 $\left| {{{\vec k}_i}} \right| = \left| {{{\vec k}_f}} \right|$ 意味着弹性散射.

积分方程求近似解的一般方法是先把一个近似解代入积分内,积分得到一阶修正后的解,再次代入,得到二阶修正后的解,以此类推迭代.波恩近似中,假设势能相对于入射动能较弱,积分项相当于微扰,所以令初始(零阶)波函数为 $\psi_0(\vec r)$.代入\autoref{BornSc_eq5} 得一阶修正的波函数,叫做\textbf{第一波恩近似}
\begin{equation}
{\psi ^{(1)}}(\vec r) = A{e^{i{{\vec k}_i} \vdot \vec r}} - A\frac{m}{{2\pi {\hbar ^2}}}\frac{{{e^{ikr}}}}{r}\int {{e^{i({{\vec k}_i} - {{\vec k}_f}) \vdot \vec r'}}V(\vec r'){\D ^3}r'} 
\end{equation}
根据定义,散射幅为
\begin{equation}
f(k,\hat r) =  - \frac{m}{{2\pi {\hbar ^2}}}\int {{e^{i({{\vec k}_i} - {{\vec k}_f}) \vdot \vec r'}}V(\vec r'){\D ^3}r'}
\end{equation}
这相当于势能函数的空间傅里叶变换.

\subsection{高阶波恩近似}
把\autoref{BornSc_eq5} 多次代入\autoref{BornSc_eq5} 的积分中,得到精确解的“积分级数”形式
 \begin{equation}
\begin{aligned}
  \psi (\vec r) &= {\psi _0}(\vec r) + \int {{\D ^3}r'G(k,\vec r,\vec r')U(\vec r'){\psi _0}(\vec r')}  \hfill \\
&+ \int {{\D ^3}r'G(k,\vec r,\vec r')U(\vec r')\int {{\D ^3}r''G(k,\vec r',\vec r'')U(\vec r''){\psi _0}(\vec r'')} }  \hfill \\
&+ \int {\D ^3}r'G(k,\vec r,\vec r')U(\vec r')\cross\\
&\int {{\D ^3}r''G(k,\vec r',\vec r'')U(\vec r'')\int {{\D ^3}r'''G(k,\vec r'',\vec r''')U(\vec r''')} } {\psi _0}(\vec r''')
  ... 
\end{aligned}
\end{equation}
若只计算是指包含前 $n$ 行,就叫第 $n$ 波恩近似.具体计算时,偶尔会用到二阶,基本不会用到三阶或以上.

非常有趣的是,即使我们不假设零阶波函数是平面波,波函数展开成上式时取前 $n$ 行的结果仍然是相同的.
