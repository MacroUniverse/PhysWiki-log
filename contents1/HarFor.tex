%Hartree-Fork 方法
%80min

Hartree 方法的精髓是假设多粒子波函数 $\Psi ({\vec r_1},{\vec r_2}...{\vec r_N})$ 等于每个单粒子波函数(包括自旋)的乘积 ${u_1}({q_1}){u_2}({q_2})..{u_N}({q_N})$ (Hartree 函数), 其中不同的单粒子波函数要求正交归一. 然后找到最优的单粒子波函数 ${u_1},{u_2},...{u_N}$ 使总哈密顿的平均值最小. 所以该方法属于变分法. 得到的能级大于精确能级.

注意Hartree函数并不满足全同粒子的对易关系, 既不是对称也不是反对称. 对于全同费米子, 方法是令多粒子函数为单粒子函数(包括自旋)的Slater行列式.
 \begin{equation}
\Psi ({q_1},{q_2}...) = \frac{1}{{\sqrt {N!} }}\vmat{ {\begin{array}{*{20}{c}}
  {{u_1}({q_1})}&{{u_2}({q_1})}& \cdots  \\ 
  {{u_1}({q_2})}&{{u_2}({q_2})}& \cdots  \\ 
   \vdots & \vdots & \ddots  
\end{array}} }
\end{equation}
变分法的拉格朗日乘数函数为
 \begin{equation}
L = \bra{\Psi}\Q H\ket{\Psi} - \sum\limits_{ij} \varepsilon _{ij} \bra{u_i}\ket{u_j}
 \end{equation}
为了化简第二项, 令原基底为另一组基底的幺正变换
 \begin{equation}
\pmat{u_1\\ {{u_2}} \\ \vdots }
= \Q U\pmat{ u'_1 \\ u'_2\\  \vdots}
\end{equation}
可以证明第一项不变. 因为
\begin{equation}
\begin{aligned}
\Psi({q_1},{q_2}\dots)& = \frac{1}{\sqrt {N!}}
\vmat{
{{u_1}({q_1})}&{{u_2}({q_1})}& \cdots  \\ 
{{u_1}({q_2})}&{{u_2}({q_2})}& \cdots  \\ 
\vdots & \vdots & \ddots}
= \frac{1}{\sqrt {N!}}
\vmat{\mat U
\pmat{{{{u'}_1}({q_1})}&{{{u'}_1}({q_2})}& \cdots  \\ 
{{{u'}_2}({q_1})}&{{{u'}_2}({q_2})}& \cdots  \\ 
\vdots & \vdots & \ddots}}\\
& = \frac{\abs{\mat U}}{\sqrt {N!}}
\vmat{
{{{u'}_1}({q_1})}&{{{u'}_2}({q_1})}& \cdots  \\ 
{{{u'}_1}({q_2})}&{{{u'}_2}({q_2})}& \cdots  \\ 
\vdots & \vdots & \ddots}
= \frac{{{e^{i\theta }}}}{{\sqrt {N!} }}
\vmat{
{{{u'}_1}({q_1})}&{{{u'}_2}({q_1})}& \cdots  \\ 
{{{u'}_1}({q_2})}&{{{u'}_2}({q_2})}& \cdots  \\ 
\vdots & \vdots & \ddots}
\end{aligned}
\end{equation}
注意这里用到了 $\abs{\mat A\mat B} = \abs{\mat A}\abs{\mat B}$ 和 $\abs{\mat U}= {e^{i\theta }}$.  这是说, 平均能量关于 $u$ 的公式不受幺正变换的影响. 对第二项,

\begin{equation}
\sum\limits_{ij} { \varepsilon_{ij} \bra{u_i}\ket{u_j} }
= \int {(u'_1{}^* ,u'_2{}^* \dots){\mat U \Her} \varepsilon \mat U
\pmat{ u_1' \\ u_2' \\ \vdots}
\D q}
\end{equation}


若我们选择幺正变换, 使得 ${\mat U^\dag }\mat \varepsilon \mat U = {E_i}{\delta_{ij}}$, %未完成: 这里表达不规范
即把矩阵 $\mat \varepsilon $ 角化(下文可知 $\mat\varepsilon$ 为对称矩阵), 得
 \begin{equation}
\sum\limits_{ij} {\varepsilon _{ij}} \bra{u_i}\ket{u_j}
 = \sum\limits_i {E_i}  \bra{u_i}\ket{u_j}
\end{equation}
现在把所有的撇号省略, 拉格朗日函数为
\begin{equation}
L =\bra{\Psi}\Q H \ket{\Psi}  - \sum\limits_i E_i \bra{u_i}\ket{u_i}
\end{equation}
即约束条件只需要归一化, 正交会自动完成. 现在来化简第一项. 首先把总波函数中的行列式记为求和的形式
 \begin{equation}
\Psi  = \frac{1}{{\sqrt {N!} }}\sum\limits_{N!} {{{( - 1)}^p}\Q P{\Psi _H}}  \equiv \sqrt {N!} \Q A{\Psi _H}
\end{equation}

其中 ${\Psi _H} = {u_1}({q_1}){u_2}({q_2})...$ 是Hartree函数,  $\Q P$ 是置换算符, 相当于做 $p$ 次双粒子置换( $p$ 是逆序数), 行列式展开后共有 $N!$ 项.  $\Q A$ 为反对称化算符, 由于 $\Q H$ 和 $\Q A$ 存在一组共同本征矢, $[\Q H,\Q A] = 0$,  另外可以证明, ${\Q A^2} = \Q A$ (意义是反对称化只需要一次)(可先证明 $N = 2,3$,  高阶行列式同理).
\begin{equation}
\begin{aligned}
  \bra{\Psi}\Q H\ket{\Psi}  &= N!\bra{ {{\Psi _H}}}\Q A\Q H\Q A \ket{{\Psi _H}} = N!\bra{ {{\Psi _H}} }\Q H{\Q A^2}\ket{{{\Psi _H}}} \\
  &= N!\bra{{{\Psi _H}} }\Q H\Q A\ket{{{\Psi _H}}}   = \sum\limits_{N!} {{( - 1)}^p}\bra{\Psi _H}\Q H\Q P\ket{\Psi _H}\\ 
\end{aligned}
\end{equation}
我们现在考虑多电子原子(离子)问题
\begin{equation}
\Q H = \sum\limits_i {{\Q h_i}}  + \sum\limits_{i < j} {\frac{1}{{{r_{ij}}}}} 
\qquad
{\Q h_i} =  - \frac{1}{2}{\laplacian} + \frac{Z}{{{r_i}}}
\end{equation}
\begin{equation}
\begin{aligned}
\bra{\Psi}\sum\limits_i {{\Q h_i}} \ket{\Psi} & = \sum\limits_i {\sum\limits_{N!} {{{( - 1)}^p}\bra{\Psi _H}{\Q h_i}\Q P \ket{{\Psi _H}} } }  = \sum\limits_i {\bra{\Psi _H} {\Q h_i} \ket{{\Psi _H}}}  \\
&= \sum\limits_i {\bra{u_i}{\Q h_i}\ket{{u_i}}} 
\end{aligned}
\end{equation}
 
这是因为只有当 $\Q P$ 为 $1$ 时(行列式的对角项, 逆序数 $p = 0$ )积分才不为零. 同理, $\Q H$ 剩下的部分为
\begin{equation}
\bra{\Psi}\sum\limits_{i < j} {\frac{1}{{{r_{ij}}}}} \ket{\Psi}  = \sum\limits_{i < j} {\sum\limits_{N!} {{{( - 1)}^p}\bra{\Psi _H}\frac{1}{{{r_{ij}}}}\Q P \ket{{\Psi _H}} } } 
\end{equation}
现在 $P = 1$ 或者 $P = {P_{ij}}$ ( $p=1$ )时积分都可能不为零. 所以
 \begin{equation}
\begin{aligned}
  \bra{\Psi}\sum\limits_{i < j} {\frac{1}{{{r_{ij}}}}} \ket{\Psi} & = \sum\limits_{i < j} {\bra{\Psi _H}\frac{1}{{{r_{ij}}}}(1 - {P_{ij}}) \ket{{\Psi _H}} }  \\
  &= \sum\limits_{i < j} {\bra{{u_i}{u_j}}\frac{1}{{{r_{ij}}}} \ket{{u_i}{u_j}}}  - \sum\limits_{i < j} {\bra{{u_i}{u_j}}\frac{1}{{{r_{ij}}}} \ket{{u_j}{u_i}} } \\
  & \equiv \sum\limits_{i < j} {{J_{ij}}}  - \sum\limits_{i < j} {{K_{ij}}} \\ 
\end{aligned}
\end{equation}
注意这里 ${ \ket{{u_i}{u_j}} ^\dag }$ 记为 $\bra{{u_i}{u_j}}$ 而不是 $\bra{{u_j}{u_i}}$.  另外易证 ${J_{ij}} = {J_{ji}}$,  ${K_{ij}} = {K_{ji}}$ (交换积分变量即可) 所以
 \begin{equation}
L = \sum\limits_i {\bra{{u_i}}{\Q h_i} \ket{{u_i}} }  + \sum\limits_{i < j} {\bra{{u_i}{u_j}}\frac{1}{{{r_{ij}}}} \bra{{u_i}{u_j}} }  - \sum\limits_{i < j} {\bra{{u_i}{u_j}} \frac{1}{{{r_{ij}}}} \ket{{u_j}{u_i}}  }  - \sum\limits_i {E_i}\bra{u_i}\ket{u_j}
\end{equation}
现在, 类似于变分法中的过程, 把任意一个 $\bra{u_k}$ 变为 $\bra{{u_k} + \delta {u_k}}$,  减去上式, 令为 $0$,得
 \begin{equation}
\bra{\delta {u_k}}{h_k} \ket{{u_k}}  + \sum\limits_j^{(j \ne k)} {\bra{\delta {u_k}{u_j}}\frac{1}{{{r_{jk}}}} \ket{{u_k}{u_j}} }  - \sum\limits_j^{(j \ne k)} {\bra{\delta {u_k}{u_j}}\frac{1}{{{r_{ij}}}}\ket{{u_j}{u_k}} }  - {E_k}\bra{\delta u_k}\ket{u_k}  = 0
\end{equation}
即
 \begin{equation}
\bra{\delta {u_k}}\left[ {{h_k} \ket{{u_k}}  + \sum\limits_j^{(j \ne k)} {\bra{{u_j}} \frac{1}{{{r_{jk}}}} \ket{{u_j}} \ket{{u_k}} }  - \sum\limits_j^{(j \ne k)} {\bra{{u_j}}\frac{1}{{{r_{ij}}}} \ket{{u_k}} \ket{{u_j}} }  - {E_k} \ket{{u_k}} } \right] = 0
\end{equation}
最后, 由于 $\bra{\delta {u_k}}$ 可以取任意微小函数, 与之相乘的ket必须为零
 \begin{equation}
{h_k} \ket{{u_k}}  + \left[ {\sum\limits_j^{(j \ne k)} {\bra{{u_j}}\frac{1}{{{r_{jk}}}} \ket{{u_j}}} } \right] \ket{{u_k}}  - \sum\limits_j^{(j \ne k)} {\left[ {\bra{{u_j}}\frac{1}{{{r_{ij}}}}  \ket{{u_k}}} \right] \ket{{u_j}} }  = {E_k} \ket{{u_k}} 
\end{equation}
这是所谓的, 非线性耦合微分积分本征方程组.

注意虽然现在trial波函数满足全同费米子的反对称, 一般却不是总自旋角动量的本征函数(其实也不是总轨道角动量的本征函数, 除非把无穷个不同的 ${\Psi _H}$ 求和). 为了实现这点, 可以先指定总自旋角动量 $\ket{S,M}$,  然后通过对行列式线性组合构建自旋部分为 $\ket{S,M} $ 的trial波函数.

\begin{exam}{氦原子}
对于 He 原子的 singlet 自旋态 $( \uparrow  \downarrow  -  \downarrow  \uparrow )/\sqrt 2 $,  可以用一个行列式构建trial波函数. 由于自旋为反对称, 轨道波函数必须为对称, 对于基态, 这意味着两个轨道波函数相同
 \begin{equation}
\Psi  = \frac{1}{{\sqrt 2 }}\vmat{ {\begin{array}{*{20}{c}}
  {\phi ({{\vec r}_1}) \uparrow }&{\phi ({{\vec r}_1}) \downarrow } \\ 
  {\phi ({{\vec r}_2}) \uparrow }&{\phi ({{\vec r}_2}) \downarrow } 
\end{array}} } = \phi ({\vec r_1})\phi ({\vec r_2})( \uparrow  \downarrow  -  \downarrow  \uparrow )/\sqrt 2 
\end{equation}
把 ${u_1} = \phi  \uparrow $,  ${u_2} = \phi  \downarrow $ 代入本征方程组得单个轨道波函数的本征方程
 \begin{equation}
\Q h\ket{\phi}  + \bra{\phi}\frac{1}{{{r_{12}}}}  \ket{\phi} \ket{\phi}  = E \ket{\phi} 
\end{equation}
当然也可以直接把 $\Psi  = \phi ({\vec r_1})\phi ({\vec r_2})( \uparrow  \downarrow  -  \downarrow  \uparrow )/\sqrt 2 $ 代入能量平均值公式, 用分母 $\bra{\phi}\ket{\phi}$ 归一化或者用拉格朗日乘数法做, 得到的方程与上式一样
\begin{equation}
\begin{aligned}
L &= 2\bra{\phi \phi }\Q h \ket{\phi \phi }  + \bra{\phi \phi }\frac{1}{r_{12}}  \ket{\phi \phi }  - \lambda [\bra{phi}\ket{phi} - 1]  \\
&= 2\bra{\phi}\Q h \ket{\phi}  + \bra{\phi \phi }\frac{1}{{{r_{12}}}}  \ket{\phi \phi }  - \lambda [\bra{phi}\ket{phi} - 1] \\ 
\end{aligned}
\end{equation}
注意拉格朗日乘数法中的优化函数可以使用限制条件化简. 令增量为0
 \begin{equation}
2\left\langle {\delta \phi } \right|h \ket{\phi}  + 2\left\langle {\delta \phi \phi } \right|\frac{1}{{{r_{12}}}} \ket{\phi \phi }  - \lambda \left\langle {{\delta \phi }}
 \mathrel{\left | {\vphantom {{\delta \phi } \phi }}
 \right. \kern-\nulldelimiterspace}
 {\phi } \right\rangle  = 0
\end{equation}
其中使用了 $\delta \left\langle {\phi \phi } \right| = \left\langle {\delta \phi \phi } \right| + \left\langle {\phi \delta \phi } \right|$. 
\begin{equation}
\Q h \ket{\phi}  + \left\langle \phi  \right|\frac{1}{{{r_{12}}}} \ket{\phi} \ket{\phi}  = \frac{\lambda }{2}  \ket{\phi} 
\end{equation}
\end{exam}

对于其他的自旋态, 往往需要行列式的线性组合才能构造总自旋本征态, 这个比较复杂, 先来看另一种方法. 我们可以直接指定总自旋态, 如果要求轨道波函数反对称, 可把 ${u_i}({q_i}) = {\phi _i}({\vec r_i})$ 直接代入本征方程组即可, 而当要求轨道波函数对称时, 可以把以上的反对称化算符 $\Q A$ 中的 ${( - 1)^p}$ 去掉改成对称化算符 $\Q B$ 
 \begin{equation}
\Q B \ket{{\Psi _H}}  = \frac{1}{{N!}}\sum\limits_{N!} {\Q P\ket{{\Psi _H}} } 
\end{equation}
 $\Q B$ 同样满足 $[\Q H,\Q B] = 0$,  ${\Q B^2} = \Q B$.  以上推导全部有效(所有负号改成正号即可), 新的本征方程组变为
 \begin{equation}
{h_k}\left| {{u_k}} \right\rangle  + \left[ {\sum\limits_j^{(j \ne k)} {\left\langle {{u_j}} \right|\frac{1}{{{r_{jk}}}}\left| {{u_j}} \right\rangle } } \right]\left| {{u_k}} \right\rangle  + \sum\limits_j^{(j \ne k)} {\left[ {\left\langle {{u_j}} \right|\frac{1}{{{r_{ij}}}}\left| {{u_k}} \right\rangle } \right]\left| {{u_j}} \right\rangle }  = {E_k}\left| {{u_k}} \right\rangle 
\end{equation}

\begin{exam}{氦原子}
考虑 $He$ 的 $1s2s$,  $^1S$ 态, 即自旋为singlet $\ket{0,0}  = ( \uparrow  \downarrow  -  \uparrow  \downarrow )/\sqrt 2 $.  两个空间波函数不相同. 直接把 ${u_1} = {\phi _{1s}}$,  ${u_2} = {\phi _{2s}}$ 代入上面的对称本征方程组, 得
 \begin{equation}
{\Q h_1}\left| {{\phi _{1s}}} \right\rangle  + \left\langle {{\phi _{2s}}} \right|\frac{1}{{{r_{12}}}}\left| {{\phi _{2s}}} \right\rangle \left| {{\phi _{1s}}} \right\rangle  + \left\langle {{\phi _{2s}}} \right|\frac{1}{{{r_{12}}}}\left| {{\phi _{1s}}} \right\rangle \left| {{\phi _{2s}}} \right\rangle  = {E_1}\left| {{\phi _{1s}}} \right\rangle 
\end{equation}
\begin{equation}
{\Q h_2}\left| {{\phi _{2s}}} \right\rangle  + \left\langle {{\phi _{1s}}} \right|\frac{1}{{{r_{12}}}}\left| {{\phi _{1s}}} \right\rangle \left| {{\phi _{2s}}} \right\rangle  + \left\langle {{\phi _{1s}}} \right|\frac{1}{{{r_{12}}}}\left| {{\phi _{2s}}} \right\rangle \left| {{\phi _{1s}}} \right\rangle  = {E_2}\left| {{\phi _{2s}}} \right\rangle 
\end{equation}
 
如果直接用 $\left| \Psi  \right\rangle {\text{ = (}}{\phi _1}{\phi _2}{\text{ + }}{\phi _2}{\phi _1}{\text{)/}}\sqrt 2 $,  结果相同.
\end{exam}
 
