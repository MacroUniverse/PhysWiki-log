% 磁场中闭合电流的合力

% 未完成: 感觉这个推导太复杂了, 如果用匀强磁场 
\pentry{安培力\upref{FAmp}, 斯托克斯定理,%未完成
静磁场的高斯定理,%未完成
% del 算符的化简
}

假设空间中有任意磁场 $\vec B\left( {\vec r} \right)$, 闭合电流回路 $L$ 中有电流 $I$. 则其受到的安培力可以表示为线积分 $\vec F = \oint{Id\vec l \cross \vec B} $.积分方向为电流方向.
若磁场是匀强磁场, 则立即得到 $\vec F = I\left( {\oint {\D \vec l} } \right) \cross \vec B = \vec 0$
若磁场是任意的, 那么
\begin{equation}
\begin{aligned}
\vec F &= \oint\limits_L {I\D \vec l \cross \vec B} \\
&= \uvec xI\oint\limits_L {\D \vec l \cross \vec B}  \vdot \uvec x + \uvec yI\oint\limits_L {\D \vec l \cross \vec B}  \vdot \uvec y + \uvec zI\oint\limits_L {\D \vec l \cross \vec B}  \vdot \uvec z\\
&= \uvec xI\oint\limits_L {\left( {\vec B \cross \uvec x} \right) \vdot \D \vec l}  + \uvec yI\oint\limits_L {\left( {\vec B \cross \uvec y} \right) \vdot \D \vec l}  + \uvec zI\oint\limits_L {\left( {\vec B \cross \uvec z} \right) \vdot \D \vec l} \\
&= \uvec xI\int\limits_\Sigma  {\curl \left( {\vec B \cross \uvec x} \right) \vdot \D \vec s}  + \uvec yI\int\limits_\Sigma  {\curl \left( {\vec B \cross \uvec y} \right) \vdot \D \vec s}  + \uvec zI\int\limits_\Sigma  {\curl \left( {\vec B \cross \uvec z} \right) \vdot \D \vec s} 
\end{aligned}
\end{equation}
其中用到了斯托克斯定理, $\Sigma $ 是以闭合曲线 $L$ 为边界的曲面.上式中
\begin{equation}
\curl (\vec B \cross \uvec x) = \vec B (\div \uvec x) + (\uvec x\vdot\vec\nabla )\vec B - \uvec x (\div \vec B) - (\vec B \vdot\vec\nabla)\uvec x = (\uvec x\vdot\vec\nabla)\vec B = \pdv{\vec B}{x}
\end{equation} 
这里用到了 $\uvec x$ 的任意微分为 0 以及 $\div \vec B = 0$ 的性质.
对称地, 将上式中的 $\uvec x$ 替换成 $\uvec y$ 和 $\uvec z$ , 等式也成立. 所以
\begin{equation}
\vec F = \uvec xI\int\limits_\Sigma  {\pdv{\vec B}{x} \vdot \D \vec s}  + \uvec yI\int\limits_\Sigma  {\pdv{\vec B}{y} \vdot \D \vec s}  + \uvec zI\int\limits_\Sigma  {\pdv{\vec B}{z} \vdot \D \vec s} 
\end{equation} 
写成分量的形式, 就是
\begin{gather}
{F_x} = I\int\limits_\Sigma  {\pdv{\vec B}{x} \vdot \D \vec s} \\
{F_y} = I\int\limits_\Sigma  {\pdv{\vec B}{y} \vdot \D \vec s} \\
{F_z} = I\int\limits_\Sigma  {\pdv{\vec B}{z} \vdot \D \vec s}
\end{gather} 








