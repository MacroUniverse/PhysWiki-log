% 简谐振子升降算符归一化

\pentry{简谐振子(量子)}

首先要提醒的是,一般算符满足的一个条件是 $\braket*{g}{\Q Q f}=\braket*{\Q Q^\dagger g}{ f}$ . 但是对于厄米算符, $\Q Q^\dagger = \Q Q$,  所以有 $\braket*{g}{\Q Q f} = \braket*{\Q Q g}{f}$ .

对于谐振子的升降算符 $a_\pm = (m\omega x \mp \I p)/\sqrt{2m\omega\hbar}$, 有
\begin{equation}\ali{
a_- a_+ &= \frac{1}{2m\omega\hbar} (m^2 \omega ^2 x^2 + p^2 - \I m\omega \comm{x}{p})\\
&= \frac{1}{\omega\hbar} \qty[\frac{1}{2m} (m^2\omega ^2 x^2 + p^2) + \frac{\omega\hbar}{2}]\\
&= \frac{1}{\omega \hbar } H + \frac12
}\end{equation}
\begin{equation}\ali{
\abs{a_+\psi_n}^2 &= \braket{a_+\psi_n} = \braket{\psi_n}{a_- a_+\psi_n}
= \braket{\psi_n}{\qty(\frac{1}{\omega\hbar} H + \frac12)\psi_n} \\
&= \qty(n+ \frac12) + \frac12 = n+1
}\end{equation}
所以有 $a_+ \psi_n = \sqrt{n + 1} \psi_{n+1}$ (同理 $a_- \psi_n = \sqrt n \,\psi_{n - 1}$ ).\\
再次提醒,归一化系数后面可以加上任意相位因子 $\E^{\I\theta}$, 同样能满足归一化条件,但一般省略.