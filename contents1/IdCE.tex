%理想气体(正则系宗法)

\subsection{可区分粒子和不可区分粒子}
对于可区分粒子,从粒子的角度求和,配分函数为(dis=distinguishable)
  \begin{equation}\label{IdCE_eq1}
    \begin{aligned}
    {Q_{dis}} & = \sum\limits_{{i_1} = 0}^\infty  {\sum\limits_{{i_2} = 0}^\infty  {...\sum\limits_{{i_N} = 0}^\infty  {{\E^{ - \beta ({\varepsilon _{{i_1}}} + {\varepsilon _{{i_2}}}...)}}} } } = \sum\limits_{{i_1} = 0}^\infty  {{\E^{ - \beta {\varepsilon _{{i_1}}}}}} \sum\limits_{{i_2} = 0}^\infty  {{\E^{ - \beta {\varepsilon _{{i_2}}}}}} ...\sum\limits_{{i_N} = 0}^\infty  {{\E^{ - \beta {\varepsilon _{{i_N}}}}}} \\
      & = {\left( {\sum\limits_{i = 0}^\infty  {{\E^{ - \beta {\varepsilon _i}}}} } \right)^N}= Q_1^N
\end{aligned}
\end{equation}
从能级的角度求和
\begin{equation}\label{IdCE_eq2}
\begin{aligned}
{Q_{dis}} & = \sum\limits_{\{ {n_i}\} }^{} {\frac{{N!}}{{{n_0}!{n_1}!...}}\exp \left( { - \beta \sum\limits_{i = 0}^\infty  {{n_i}{\varepsilon _i}} } \right)} = \sum\limits_{\{ {n_i}\} }^{} {\frac{{N!}}{{{n_0}!{n_1}!...}}{\E^{ - {n_1}{\varepsilon _1}\beta }}{\E^{ - {n_2}{\varepsilon _2}\beta }}...}
\end{aligned}
\end{equation}
由\autoref{IdCE_eq1} 和\autoref{IdCE_eq2} 物理意义可知, 二者相等.再从能级的角度考虑, 若粒子不可区分(由于这个配分函数是最常用的, 所以不写角标)
\begin{equation}\label{IdCE_eq3}
Q = \sum\limits_{\{ {n_i}\} }^{} {{\E^{ - {n_1}{\varepsilon _1}\beta }}{\E^{ - {n_2}{\varepsilon _2}\beta }}...}
\end{equation}
比较\autoref{IdCE_eq2},  求和的每项少了一个因子 ${{N!}}/({{{n_0}!{n_1}!...}})$

理想气体条件:能级占有率极低, 几乎没有两个粒子在同一个能级上, 所以大部分 ${n_i} = 0$,  $0! = 1$.  个别 ${n_i} = 1$,  $1! = 1$.
可以近似认为
\begin{equation}
\frac{{N!}}{{{n_0}!{n_1}!...}} \approx N!
\end{equation}
所以
  \begin{equation}
    Q = \frac{1}{{N!}}{Q_{dis}} = \frac{1}{{N!}}Q_1^N
  \end{equation}
  那如何求 $Q_1$ 呢? 

\subsection{对单粒子相空间积分}
注意每个量子态对应的相空间体积为 $h$ 的空间维数次方.
\begin{equation}
{Q_1} = \frac{1}{{{h^3}}}\int {{\E^{ - ({p^2}/2m)/kT}}{d^3}p\;{d^3}x}  = \frac{V}{{{h^3}}}\int {{\E^{ - ({p^2}/2m)/kT}}{d^3}p}  = \frac{V}{{{\lambda ^3}}}
\end{equation}
其中 $\lambda $ 叫做热力学波长, 正比与粒子热运动的德布罗意波
\begin{equation}
\lambda  = \frac{h}{{\sqrt {2\pi mkT} }}
\end{equation}

\subsection{对单粒子能级密度积分}
用单粒子能级密度 $a\left( \varepsilon  \right)$ 对玻尔兹曼因子积分.
\begin{equation}
    a\left( \varepsilon  \right) = \frac{{2\pi V{{\left( {2m} \right)}^{3/2}}}}{{{h^3}}}{\varepsilon ^{1/2}}
\end{equation}
\begin{equation}
   {Q_1} = \sum\limits_{i = 0}^\infty  {{\E^{ - \beta {\varepsilon _i}}}}  = \int_0^\infty  {a\left( \varepsilon  \right){\E^{ - \beta \varepsilon }}\D \varepsilon }  \\
   = \frac{{2\pi V{{\left( {2m} \right)}^{3/2}}}}{{{h^3}}}\int_0^\infty  {{\varepsilon ^{1/2}}{\E^{ - \beta \varepsilon }}\D \varepsilon }
\end{equation}
  对积分换元, 令 $x = \beta \varepsilon $, 
\begin{equation}
    \int_0^\infty  {{\varepsilon ^{1/2}}{\E^{ - \beta \varepsilon }}\D \varepsilon } = {\left( {kT} \right)^{3/2}}\int_0^\infty  {{x^{1/2}}{\E^{ - x}}\D x}
    = \Gamma \left( {3/2} \right){\left( {kT} \right)^{3/2}}
    = \frac{{\sqrt \pi  }}{2}{\left( {kT} \right)^{3/2}}
\end{equation}
\begin{equation}
{Q_1} = \sum\limits_{i = 0}^\infty  {{\E^{ - \beta {\varepsilon _i}}}}  = \int_0^\infty  {a\left( \varepsilon  \right){\E^{ - \beta \varepsilon }}\D\varepsilon } = \frac{{2\pi V{{\left( {2m} \right)}^{3/2}}}}{{{h^3}}}\frac{{\sqrt \pi  }}{2}{\left( {kT} \right)^{3/2}}  = \frac{V}{{{\lambda ^3}}}
\end{equation}

\subsection{对系统的能级密度积分}
现在我们试图直接求 $Q$,系统的总能级密度为% 链接未完成
\begin{equation}
g\left( E \right) = \frac{{\D {\Omega _0}}}{{\D E}}  = \frac{{{V^N}}}{{N!{h^3}}}\frac{{{{\left( {2\pi m} \right)}^{3N/2}}}}{{\left( {3N/2 - 1} \right)!}}{\E^{3N/2 - 1}}
\end{equation}
\begin{equation}
Q = \int_0^\infty  {g\left( E \right){\E^{ - E\beta }}\D E}  = \frac{{{V^N}{{\left( {2\pi m} \right)}^{3N/2}}}}{{N!{h^3}\left( {3N/2 - 1} \right)!}}\int_0^\infty  {{\E^{3N/2 - 1}}{\E^{ - \beta E}}\D E}
\end{equation}
令 $x = \beta E$ 对积分换元,
\begin{equation}
\begin{aligned}
\int_0^\infty  {{\E^{3N/2 - 1}}{\E^{ - \beta E}}\D E} & = {\left( {kT} \right)^{3N/2}}\int_0^\infty  {{x^{3N/2 - 1}}{\E^{ - x}}\D x}  \\
& = {\left( {kT} \right)^{3N/2}}\Gamma \left( {3N/2} \right) \\
& = {\left( {kT} \right)^{3N/2}}\left( {3N/2 - 1} \right)!
\end{aligned}
\end{equation}
代入上式得
\begin{equation}
\begin{aligned}
Q & = \frac{{{V^N}{{\left( {2\pi m} \right)}^{3N/2}}}}{{N!{h^3}\left( {3N/2 - 1} \right)!}}{\left( {kT} \right)^{3N/2}}(3N/2 - 1)! \\
   & = \frac{{{V^N}{{\left( {2\pi mkT} \right)}^{3N/2}}}}{{N!{h^3}}} = \frac{1}{{N!}}{\left( {\frac{V}{{{\lambda ^3}}}} \right)^N} \\
   & = \frac{1}{{N!}}Q_1^N
 \end{aligned}
 \end{equation}
 与之前的结果都一样.


\subsection{热力学性质}
 得到系统的配分函数 $Q$ 以后, 可由用亥姆霍兹自由能得到热力学的性质
 \begin{equation}
   F =  - kT\ln Q =  - kT\left( {N\ln {Q_1} - N\ln N + N} \right)
 \end{equation}
 \begin{equation}
   S = Nk\left( {\ln \frac{V}{{N{\lambda ^3}}} + \frac{5}{2}} \right)
 \end{equation}
 \begin{equation}
   P =  - {\left( {\frac{{\partial F}}{{\partial V}}} \right)_{T,N}} = \frac{{NkT}}{V} \qquad \text{(理想气体状态方程)}
 \end{equation}
 \begin{equation}
   \mu  = kT\ln \frac{{N{\lambda ^3}}}{V} = kT\ln \frac{N}{{{Q_1}}}
 \end{equation}
 在巨正则系综里, 定义逸度为 $z = {\E^{\mu/kT}}$,  则 $N = {{zV}}/{{{\lambda ^3}}} = z{Q_1}$. 


\subsection{分布函数}
若有 $N$ 个粒子组成理想气体, 每个能级平均有多少粒子(由于理想气体的条件是能级占有率 $\left\langle {{n_i}} \right\rangle \ll 1$,  但仍然会有分布曲线)

对任何一个粒子来说, 出现在 ${\varepsilon _i}$ 能级(非简并)的概率是 ${{{\E^{ - \beta {\varepsilon _i}}}}}/{{{Q_1}}} = {{{\lambda ^3}}}{\E^{ - \beta {\varepsilon _i}}}/{V}$.  那么 $N$ 个没有相互作用的粒子在该能级的平均粒子数就为
\begin{equation}
  \left\langle {{n_i}} \right\rangle  = \frac{{N{\lambda ^3}}}{V}{\E^{ - \beta {\varepsilon _i}}}
\end{equation}
理想气体的化学能 $\mu  = kT\ln {{N{\lambda ^3}}}/{V}$,  即 ${\E^{\mu/kT}} = {{N{\lambda ^3}}}/{V}$.  代入上式, 得
\begin{equation}
  \left\langle {{n_i}} \right\rangle  = {\E^{\beta \mu }}{\E^{ - \beta {\varepsilon _i}}} = {\E^{(\mu - \varepsilon_i)/kT}}
\end{equation}
这就是麦克斯韦—玻尔兹曼分布.

