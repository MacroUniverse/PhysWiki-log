% 磁旋比 玻尔磁子
% 18 min
%未完成
如果一个带电刚体,质量密度和电荷密度成正比,当它绕轴转动时,角动量为
\begin{equation}
\vec L = I\vec \omega  = \vec \omega \int {{r^2}{\rho _m}\D V} 
\end{equation}
磁矩定义为
\begin{equation}
\vec \mu  = Ia \vec{\uvec\omega}= \int {\frac{{\D Q}}{{{{2\pi } \mathord{\left/
 {\vphantom {{2\pi } \omega }} \right.
 \kern-\nulldelimiterspace} \omega }}} \vdot \pi {r^2}\uvec \omega }  = \frac{1}{2}\vec \omega \int {{r^2}{\rho _e}\D V} 
\end{equation}
其中 $r$ 为质量元到转轴的距离.两式比较,得
\begin{equation}
\vec \mu  = \frac{q}{{2m}}\vec L
\end{equation}
但对基本粒子(例如电子)的实验中,发现上式还需要一个修正因子
\begin{equation}
\vec \mu  = g\frac{q}{{2m}}\vec L
\end{equation}
$g$ 一般就叫做 $g$ 因子.定义磁旋比 $\gamma  = gq/{2m}$


对于粒子的自旋, $L = \hbar \sqrt {l\left( {l + 1} \right)} $. 所以 $\mu  = \sqrt {l\left( {l + 1} \right)}\hbar g{{q}}/{{2m}} $. 
对于电子,实验测得 ${g_e} = 2.0023193043617(15) \approx 2$ 
\begin{equation}
\vec \mu  = {g_e}\frac{{e\hbar }}{{2{m_e}}}\sqrt {\frac{1}{2}\left( {1 + \frac{1}{2}} \right)}  = \frac{{\sqrt 3 }}{2}{g_e}{\mu _B} \approx \sqrt 3 {\mu _B}
\end{equation}
其中 ${\mu _B}$ 为玻尔磁子,定义为
\begin{equation}
{\mu _B} = \frac{{e\hbar }}{{2{m_e}}}
\end{equation}

