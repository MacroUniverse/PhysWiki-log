%自旋角动量
%20 min

\begin{enumerate}
\item 自旋角动量三个分量算符 ${\uvec S_x}{\uvec S_y}{\uvec S_z}$ 的互相对易关系以及自旋模长平方算符 ${\uvec S^2}$ 的对易关系 %(已经不想写了)
\item 与轨道角动量同理,存在一组本征态 $\left| {s,m} \right\rangle $ 

( $s = 0,{1}/{2},1,{3}/{2}...$, $m =  - s, - s + 1...,s - 1,s$ 但是每种粒子都有固有的 $s$ ) 满足

 ${\uvec S^2}\left| {s,m} \right\rangle  = {\hbar ^2}s\left( {s + 1} \right)\left| {s,m} \right\rangle $ 和 ${\uvec S_z}\left| {s,m} \right\rangle  = \hbar m\left| {s,m} \right\rangle $
\item 存在升降算符 ${\uvec S_ \pm } = {\uvec S_x} \pm i{\uvec S_y}$, 且

 ${\uvec S_ \pm }\left| {s,m} \right\rangle  = \hbar \sqrt {s\left( {s + 1} \right) - m\left( {m \pm 1} \right)} \left| {s,m + 1} \right\rangle $ (根号项是归一化系数)
\item 对于 $s = {\textstyle{1 \over 2}}$ 的粒子,一共有2个本征态,分别是 $\left| {{\textstyle{1 \over 2}},{\textstyle{1 \over 2}}} \right\rangle $,  $\left| {{\textstyle{1 \over 2}}, - {\textstyle{1 \over 2}}} \right\rangle $. 它们的角动量模长平方都是 ${\textstyle{3 \over 4}}{\hbar ^2}$, 角动量 $z$ 分量都是 ${\hbar }/{2}$.  以这两个本征态为基底,令第一个为  ${\chi _ + } = \left( {\begin{array}{*{20}{c}}
1\\
0
\end{array}} \right)$, 第二个为 ${\chi _ - } = \left( \begin{array}{l}
0\\
1
\end{array} \right)$. 可以得出角动量平方算符的矩阵为 ${\uvec S^2} = {\hbar ^2}{3}/{4}\left( {\begin{array}{*{20}{c}}
1&0\\
0&1
\end{array}} \right)$,   ${\uvec S_z} = {\hbar }/{2}\left( {\begin{array}{*{20}{c}}
1&0\\
0&{ - 1}
\end{array}} \right)$. 根据 ${\uvec S_ + }{\chi _ - } = \hbar {\chi _ + }$ 和 ${\uvec S_ - }{\chi _ + } = \hbar {\chi _ - }$,   得到

 ${\uvec s_y} = {\hbar }/{2}\left( {\begin{array}{*{20}{c}}
0&{ - i}\\
i&0
\end{array}} \right)$ 和 ${\uvec s_z} = {\hbar }/{2}\left( {\begin{array}{*{20}{c}}
1&0\\
0&{ - 1}
\end{array}} \right)$. 

然后, 定义泡力矩阵.  
\begin{equation}
{\sigma _x} = \left( {\begin{array}{*{20}{c}}
0&1\\
1&0
\end{array}} \right)
\qquad
{\sigma _y} = \left( {\begin{array}{*{20}{c}}
0&{ - i}\\
i&0
\end{array}} \right)
\qquad
{\sigma _z} = \left( {\begin{array}{*{20}{c}}
1&0\\
0&{ - 1}
\end{array}} \right)
\end{equation}
其实, 根据对易关系直接就可以得到泡力矩阵.
\end{enumerate}

