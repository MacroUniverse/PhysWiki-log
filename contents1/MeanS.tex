%平均值的不确定度
%20min

例题: 比如说有一个概率分布 $y = f\left( x \right)$.  它的

平均值(数学期望)为 $\bar x = \int_{ - \infty }^{ + \infty } {xf\left( x \right)\D x} $,  

方差为  $\sigma _x^2 = \int_{ - \infty }^{ + \infty } {{{\left( {x - \bar x} \right)}^2}f\left( x \right)\D x}  = \left( {\int_{ - \infty }^{ + \infty } {{x^2}f\left( x \right)\D x} } \right) - {\bar x^2} = \overline {{x^2}}  - {\bar x^2}$ 

标准差为 ${\sigma _x} = \sqrt {\int_{ - \infty }^{ + \infty } {{{\left( {x - \bar x} \right)}^2}f\left( x \right)\D x} } $.  

如果测量一个数据, 这三个值就可以用来衡量这个数据的特征.

但如果测量 $n$ 次平均值, 那显然平均值显然要比一次测量更可靠, ${\sigma _{\bar x}} < {\sigma _x}$.   各种教科书上都会给出 ${\sigma _{\bar x}} = \frac{1}{{\sqrt n }}{\sigma _x}$ 或者 $\sigma _{\bar x}^2 = \frac{1}{n}\sigma _x^2$.  那么这个公式到底怎么来的呢?

其实在上式中, $\sigma _{\bar x}^2$  的定义是 $\sigma _{\bar x}^2 = \int_{ - \infty }^{ + \infty } {{{\left( {\frac{1}{n}\sum\limits_{i = 1}^n {{x_i}}  - \bar x} \right)}^2}f\left( {{x_1}} \right)f\left( {{x_2}} \right)..f\left( {{x_n}} \right)\D {x_1}...\D {x_n}} $.  下面就从这个定义证明 $\sigma _{\bar x}^2 = \frac{1}{n}\sigma _x^2$. 

先考虑两次测量, 即 $n = 2$ 的情况. 先后得到 ${x_1}$,  ${x_2}$ 的概率密度是 ${f_2}\left( {{x_1},{x_2}} \right) = f\left( {{x_1}} \right)f\left( {{x_2}} \right)$ 
.  不难证明归一化:
 \begin{equation}
\begin{aligned}
  \iint {f\left( {{x_1}} \right)f\left( {{x_2}} \right)\D {x_1}\D {x_2}} &= \int {f\left( {{x_1}} \right)\D {x_1}} \int {f\left( {{x_2}} \right)\D {x_2}}  \hfill \\
   &= 1 \times 1 = 1 \hfill \\ 
\end{aligned}
\end{equation}

先看 ${{{x_1} + {x_2}}}/{2}$ 的平均值, 令 $y = {{{x_1} + {x_2}}}/{2}$. 
  \begin{equation}
\begin{aligned}
  \bar y &= \iint {\left( {\frac{{{x_1} + {x_2}}}{2}} \right)f\left( {{x_1}} \right)f\left( {{x_2}} \right)\D {x_1}\D {x_2}} \hfill \\
  & = \frac{1}{2}\int {{x_1}f\left( {{x_1}} \right)\D {x_1}} \int {f\left( {{x_2}} \right)\D {x_2}}  + \frac{1}{2}\int {f\left( {{x_1}} \right)\D {x_1}} \int {{x_2}f\left( {{x_2}} \right)\D {x_2}}  \hfill \\
   &= \frac{{\bar x}}{2} + \frac{{\bar x}}{2} = \bar x \hfill \\ 
\end{aligned} 
\end{equation}
结论是, 进行两次测量去平均值, 数学期望就是测量一次的数学期望. 这个结论是符合常识的.

根据同样的方法, 可以测量方差.
  \begin{equation}
  \begin{aligned}
\sigma _{\bar x}^2 &= \iint {{{\left( {\frac{{{x_1} + {x_2}}}{2} - \bar x} \right)}^2}f\left( {{x_1}} \right)f\left( {{x_2}} \right)\D {x_1}\D {x_2}} \\
&= \frac{1}{2}\left( {\overline {{x^2}}  - {{\bar x}^2}} \right) = \frac{1}{2}\sigma _x^2
\end{aligned} 
\end{equation}
所以 $\sigma _{\bar x}^2 = \frac{1}{2}\sigma _x^2$,  且  ${\sigma _{\bar x}} = \frac{1}{{\sqrt 2 }}{\sigma _x}$ 

对于 $n > 2$ 的情况, 利用求和符号和积分运算法则, 也很容易证明  ${\sigma _{\bar x}} = \frac{1}{{\sqrt n }}{\sigma _x}$ 
