% 电磁场的动量守恒 动量流密度张量

\pentry{能流密度,%未完成
 张量,%未完成
张量的散度%未完成
}
\subsection{结论}

假设电磁场动量守恒,则动量流密度张量为
\begin{equation}
{T_{ij}} = {\epsilon_0}\left( {\frac{1}{2}{{\vec E}^2}{\delta _{ij}} - {E_i}{E_j}} \right) + \frac{1}{{{\mu _0}}}\left( {\frac{1}{2}{{\vec B}^2}{\delta _{ij}} - {B_i}{B_j}} \right)
\end{equation} 
该张量也成为\bb{麦克斯韦应力张量}(Maxwell Stress Tensor)

\subsection{推导}	

能量是标量,所以能流密度就是矢量.但动量本身就是矢量,要如何表示动量流密度呢? 
我们可以分析动量在某方向分量的流密度.根据张量的散度%未完成:引用
\begin{equation}
{\left( {\div \mathord{\buildrel{\lower3pt\hbox{$\scriptscriptstyle\leftrightarrow$}} 
\over T} } \right)_j} = \sum\limits_i {\frac{\partial }{{\partial {x_i}}}{T_{ji}}}
\end{equation}
假设电磁场满足动量守恒,在闭合空间中,有“转换速率+流出速率+增加速率= 0”(类比电磁场的能量守恒公式).则电磁场的动量守恒会有
\begin{equation}\label{EBP_eq1}
\vec f + \div \mathord{\buildrel{\lower3pt\hbox{$\scriptscriptstyle\leftrightarrow$}} 
\over T}  + {\mu _0}{\epsilon_0}\frac{{\partial \vec s}}{{\partial t}} = \vec 0
\end{equation} 
由广义洛伦兹力%未完成:引用
计算电荷的受力密度\vec f %未完成:引用
\begin{equation}
\vec f = \rho \left( {\vec E + \vec v \cross \vec B} \right) = \rho \vec E + \vec j \cross \vec B
\qquad
(\vec j = \rho \vec v)
\end{equation} 
由于\autoref{EBP_eq1} 的后两项是电磁场的量,不能含有关于电荷的量,所以接下来要通过麦克斯韦方程组 %未完成:引用
把电荷密度 $\rho$ 和电流密度 $\vec j$ 替换成电磁场.
\begin{equation}
\rho  = {\epsilon_0}\div \vec E
\qquad
(\div \vec E = \frac{\rho }{{{\epsilon_0}}})
\end{equation}
\begin{equation}
\vec j = \frac{1}{{{\mu _0}}}\curl \vec B - {\epsilon_0}\frac{{\partial \vec E}}{{\partial t}}
\qquad
(\curl \vec B = {\mu _0}\vec j + {\epsilon_0}{\mu _0}\frac{{\partial \vec E}}{{\partial t}})
\end{equation}
代入上式,得
\begin{equation}
\vec f = {\epsilon_0}\left( {\div \vec E} \right)\vec E + \frac{1}{{{\mu _0}}}\left( {\curl \vec B} \right) \cross \vec B - {\epsilon_0}\frac{{\partial \vec E}}{{\partial t}} \cross \vec B
\end{equation} 
其中 
\begin{equation}
\begin{aligned}
\frac{{\partial \vec E}}{{\partial t}} \cross \vec B &= \frac{\partial }{{\partial t}}\left( {\vec E \cross \vec B} \right) - \vec E \cross \frac{{\partial \vec B}}{{\partial t}}\\ 
&= \frac{\partial }{{\partial t}}\left( {\vec E \cross \vec B} \right) - \left( {\curl \vec E} \right) \cross \vec E
\qquad
(\curl \vec E =  - \frac{{\partial \vec B}}{{\partial t}})
\end{aligned}
\end{equation} 
代入上式得
\begin{equation}
\begin{aligned}
\vec f &= {\epsilon_0}\left( {\div \vec E} \right)\vec E + \frac{1}{{{\mu _0}}}\left( {\curl \vec B} \right) \cross \vec B + {\epsilon_0}\left( {\curl \vec E} \right) \cross \vec E - {\epsilon_0}\frac{\partial }{{\partial t}}\left( {\vec E \cross \vec B} \right)\\
&= {\epsilon_0}\left[ {\left( {\div \vec E} \right)\vec E + \left( {\curl \vec E} \right) \cross \vec E} \right] + \frac{1}{{{\mu _0}}}\left( {\curl \vec B} \right) \cross \vec B - {\epsilon_0}{\mu _0}\frac{{\partial \vec s}}{{\partial t}}
\end{aligned}
\end{equation} 
为了使式中电磁场的公式更加对称,不妨加上一项
\begin{equation}
\frac{1}{{{\mu _0}}}\left( {\div \vec B} \right)\vec B = \vec 0
\qquad
(\div \vec B = 0)
\end{equation} 
得
\begin{equation}
\vec f = {\epsilon_0}\left[ {\left( {\div \vec E} \right)\vec E + \left( {\curl \vec E} \right) \cross \vec E} \right] + \frac{1}{{{\mu _0}}}\left[ {\left( {\div \vec B} \right)\vec B + \left( {\curl \vec B} \right) \cross \vec B} \right] - {\epsilon_0}{\mu _0}\frac{{\partial \vec s}}{{\partial t}}
\end{equation}  
一般来说,凡是出现两个连续的叉乘要尽量化成点乘,下面计算 $\left( {\curl \vec E} \right) \cross \vec E$. 
由吉布斯算子(劈形算符)的相关公式
\begin{equation}
\grad \left( {\vec A \vdot \vec B} \right) = \vec A \cross \left( {\curl \vec B} \right) + \vec B \cross \left( {\curl \vec A} \right) + (\vec A\vdot\vec\nabla )\vec B + (\vec B\vdot\vec\nabla)\vec A
\end{equation} 
令 $\vec A = \vec B = \vec E$,得
\begin{equation}
\grad \left( {{{\vec E}^2}} \right) = 2\vec E \cross \left( {\curl \vec E} \right) + 2(\vec E\vdot\vec\nabla )\vec E
\end{equation} 
即 $\left( {\curl \vec E} \right) \cross \vec E = (\vec E\vdot\vec\nabla)\vec E - \frac{1}{2}\grad \left( {{{\vec E}^2}} \right)$
同理得
\begin{equation}
\left( {\curl \vec B} \right) \cross \vec B = ( \vec B\vdot\vec\nabla )\vec B - \frac{1}{2}\grad \left( {{{\vec B}^2}} \right)
\end{equation} 
代入得
\begin{equation}
\begin{aligned}
\vec f = &{\epsilon_0}\left[ {\left( {\curl E} \right)\vec E + \left( {\vec E\vdot\vec\nabla } \right)\vec E - \frac{1}{2}\grad \left( {{{\vec E}^2}} \right)} \right]\\
&+ \frac{1}{{{\mu _0}}}\left[ {( {\div \vec B} )\vec B + \left( {\vec B\vdot\vec\nabla } \right)\vec B - \frac{1}{2}\grad \left( {{{\vec B}^2}} \right)} \right] - {\epsilon_0}{\mu _0}\frac{{\partial \vec s}}{{\partial t}}
\end{aligned}
\end{equation} 

与\autoref{EBP_eq1} 对比,可以看出动量流密度张量的散度为
\begin{equation}
\begin{aligned}
\div \mathord{\buildrel{\lower3pt\hbox{$\scriptscriptstyle\leftrightarrow$}} 
\over T}  =  &- {\epsilon_0} [ ( \div \vec E )\vec E + (\vec E\vdot\vec\nabla )\vec E - \frac{1}{2}\grad ( {\vec E}^2 ) ]\\
&- \frac{1}{\mu _0} [ \left( {\div \vec B} \right)\vec B + \left( {\vec B\vdot\vec\nabla } \right)\vec B - \frac{1}{2}\grad ({\vec B}^2 ) ]
\end{aligned}
\end{equation} 
接下来由二阶张量的散度计算公式,通过对比系数,就可以求出动量流密度张量 $\vec T$ (三阶矩阵).

下面把等式右边的部分用求和符号表示(求和符号是张量分析中最常见的符号,只有熟练运用才能学号张量分析).下面推导用到了克罗内克 $\delta$ 函数 %未完成:引用
,且定义任意矢量加上下标 表示第 个分量,例如
\begin{equation}
{\vec A_j} = \left\{ \begin{array}{l}
{A_x} \left( {j = 1} \right)\\
{A_y}     \left( {j = 2} \right)\\
{A_z}      \left( {j = 3} \right)
\end{array} \right.
\end{equation} 
 $\left( {\div \vec E} \right)\vec E + \left( {\vec E\vdot\vec\nabla } \right)\vec E - \frac{1}{2}\grad \left( {{{\vec E}^2}} \right)$ 是一个矢量,它的第 $j$ 个分量
\begin{equation}
\begin{aligned}
&\phantom{={}} {\left[ {\left( {\div \vec E} \right)\vec E + \left( {\vec E\vdot\vec\nabla } \right)\vec E - \frac{1}{2}\grad \left( {{{\vec E}^2}} \right)} \right]_j}\\
&= \left( {\sum\limits_{i = x,y,z} {\frac{{\partial {E_i}}}{{\partial i}}} } \right){E_j} + \left( {\sum\limits_{i = x,y,z} {{E_i}\frac{\partial }{{\partial i}}} } \right){E_j} - \frac{1}{2}\sum\limits_{i = x,y,z} {\left( {\frac{{\partial {{\vec E}^2}}}{{\partial i}}{\delta _{ij}}} \right)} \\
&= \sum\limits_{i = x,y,z} {\left( {\frac{{\partial {E_i}}}{{\partial i}}{E_j} + {E_i}\frac{{\partial {E_j}}}{{\partial i}} - \frac{1}{2}\frac{\partial\vec E^2}{{\partial i}}{\delta _{ij}}} \right)} \\
&= \sum\limits_{i = x,y,z} {\left( {\frac{\partial }{{\partial i}}\left( {{E_i}{E_j}} \right) - \frac{1}{2}\frac{{\partial {{\vec E}^2}}}{{\partial i}}{\delta _{ij}}} \right)} \\
&= \sum\limits_{i = x,y,z} {\frac{\partial }{{\partial i}}\left( {{E_i}{E_j} - \frac{1}{2}{{\vec E}^2}{\delta _{ij}}} \right)} 
\end{aligned}
\end{equation} 
同理
\begin{equation}
{\left[ {\left( {\div \vec B} \right)\vec B + \left( {\vec B\vdot\vec\nabla } \right)\vec B - \frac{1}{2}\grad \left( {{{\vec B}^2}} \right)} \right]_j} = \sum\limits_{i = x,y,z} {\frac{\partial }{{\partial i}}\left( {{B_i}{B_j} - \frac{1}{2}{{\vec B}^2}{\delta _{ij}}} \right)} 
\end{equation} 
所以
\begin{equation}
{\left( {\div \mathord{\buildrel{\lower3pt\hbox{$\scriptscriptstyle\leftrightarrow$}} 
\over T} } \right)_j} = \sum\limits_{i = x,y,z} {\frac{\partial }{{\partial i}}\left[ {{\epsilon_0}\left( {\frac{1}{2}{{\vec E}^2}{\delta _{ij}} - {E_i}{E_j}} \right) + \frac{1}{{{\mu _0}}}\left( {\frac{1}{2}{{\vec B}^2}{\delta _{ij}} - {B_i}{B_j}} \right)} \right]} 
\end{equation} 
而由张量散度的定义
\begin{equation}
{\left( {\div \mathord{\buildrel{\lower3pt\hbox{$\scriptscriptstyle\leftrightarrow$}} 
\over T} } \right)_j} = \sum\limits_i {\frac{\partial }{{\partial i}}{T_{ij}}} 
\end{equation} 
得到动量流密度张量为
\begin{equation}
{T_{ij}} = {\epsilon_0}\left( {\frac{1}{2}{{\vec E}^2}{\delta _{ij}} - {E_i}{E_j}} \right) + \frac{1}{{{\mu _0}}}\left( {\frac{1}{2}{{\vec B}^2}{\delta _{ij}} - {B_i}{B_j}} \right)
\end{equation} 

理论上,在 $\vec T$ 上面加上任意一个满足 $\div \mathord{\buildrel{\lower3pt\hbox{$\scriptscriptstyle\leftrightarrow$}} 
\over T}{}'  = \vec 0$ 的张量场,均可以使电磁场动量守恒,但是若规定无穷远处动量流密度为零,则可以证明 ${T'_{ij}} = 0$. 
 
 
 