%统计力学公式大全

%咳咳, 由于太多公式记不住, 有必要弄一篇公式大全
\subsection{微分关系}
\begin{equation}
H = E + PV
\end{equation}
\begin{equation}
G = E + PV - ST
\end{equation}
\begin{equation}
\mu  \D N + N \D \mu  =  \D G = V \D P - S \D T + \mu  \D N
\end{equation}
\subsection{微正则系综}
\begin{equation}
\D S = \frac{1}{T} \D E + \frac{P}{T} \D V - \frac{\mu }{T} \D N
\end{equation}
\subsection{正则系宗}
\begin{equation}
- kT\ln Q = F = E - ST
\end{equation}
\begin{equation}
\D F =  - S \D T - P \D V + \mu  \D N
\end{equation}
\begin{equation}
E =  - \pdv{\beta} \ln Q
\end{equation}
\subsection{巨正则系综}
\begin{equation}
- PV = \Phi  = E - ST - \mu N
\end{equation}
\begin{equation}
\Phi  =  - kT\ln \Xi 
\end{equation}
\begin{equation}
\D \Phi  =  - P \D V - S \D T - N \D \mu
\end{equation}
\begin{equation}
\left\langle {n_i} \right\rangle  = \pdv{\Phi}{\varepsilon_i}
\end{equation}
\subsection{理想气体}
\begin{equation}
{V_n} = \frac{{\pi ^{n/2}}{R^n}}{\Gamma \left( {1 + n/2} \right)}
\qquad
\text{( $N$ 维球体)}
\end{equation}
\begin{equation}
{\Omega_0} = \frac{V^N}{N!{h^3}}\frac{{\left( {2\pi mE} \right)}^{3N/2}}{\left( {3N/2} \right)!}
\qquad
\text{( $N$ 粒子能级密度)}
\end{equation}
\begin{equation}
a\left( \varepsilon  \right) = \frac{2\pi V{{\left( {2m} \right)}^{3/2}}}{h^3}{\varepsilon ^{1/2}}
\qquad
\text{(单粒子能及密度)}
\end{equation}
\begin{equation}
S = Nk\left( {\ln \frac{V}{N{\lambda ^3}} + \frac{5}{2}} \right)
\qquad
\text{(熵)}
\end{equation}
\begin{equation}
N = z{Q_1} \Rightarrow \mu  = kT\ln \frac{N{\lambda ^3}}{V}
\qquad
\text{(化学势)}
\end{equation}
\begin{equation}
\Xi  = \ln N
\qquad
\text{(巨势)}
\end{equation}
\subsection{量子气体}
\begin{equation}
N = {Q_1}{g_{3/2}}\left( z \right) = \frac{V}{\lambda ^3}{g_{3/2}}\left( z \right)
\qquad
\text{($BE$)}
\end{equation}
\begin{equation}
N = {Q_1}{f_{3/2}}\left( z \right)
\qquad
\text{($FD$)}
\end{equation}
\begin{equation}
\frac{PV}{kT} = {Q_1}{g_{5/2}}\left( z \right) = \frac{V}{\lambda ^3}{g_{5/2}}\left( z \right)
\qquad
\text{($BE$)}
\end{equation}
\begin{equation}
\frac{PV}{kT} = {Q_1}{f_{5/2}}\left( z \right)
\qquad
\text{($FD$)}
\end{equation}
\begin{equation}
PV = NkT\frac{{g_{5/2}}(z)}{{g_{3/2}}(z)}
\qquad
\text{($BE$)}
\end{equation}
\begin{equation}
PV = NkT\frac{{f_{5/2}}(z)}{{f_{3/2}}(z)}
\qquad
\text{($FD$)}
\end{equation}
\begin{equation}
E = \frac{3}{2}PV
\qquad
\text{($BE$ 和 $FD$)}
\end{equation}
理论上可以通过三式中的任意两式消去 $z$,  但是不能写成解析形式.
\begin{equation}
\begin{aligned}
{g_n}(z) = z + \frac{z^2}{2^n} + \frac{z^3}{3^n}...\\
{f_n}(z) = z - \frac{z^2}{2^n} + \frac{z^3}{3^n}...
\end{aligned}
\end{equation}
\subsection{$BE$ 凝聚态}
\begin{equation}
N = \frac{V}{\lambda_c^3}{g_{3/2}}\left( 1 \right) \Rightarrow {T_c} = \frac{h^2}{2\pi mk}{\left( {\frac{N}{2.612{\kern 1pt} {\kern 1pt} V}} \right)^{2/3}}
\end{equation}
\begin{equation}
\frac{N_e}{N} = \frac{\lambda ^3}{\lambda_c^3} \Rightarrow {N_e} = N{\left( {\frac{T}{T_c}} \right)^{3/2}} \Rightarrow {N_0} = N\left[ {1 - {{\left( {\frac{T}{T_c}} \right)}^{3/2}}} \right]
\end{equation}
\begin{equation}
{N_0} = \frac{1}{{\E^{({\varepsilon_0} - \mu )/kT}} - 1} = \frac{kT}{{\varepsilon_0} - \mu }
\end{equation}
\begin{equation}
{\varepsilon_0} - \mu  \ll {\varepsilon_1} - {\varepsilon_0} \Rightarrow {\varepsilon_0} - \mu  \ll {\varepsilon_1} - \mu 
\end{equation}
\begin{equation}
{N_1} = \frac{1}{{\E^{({\varepsilon_1} - \mu )/kT}} - 1} < \frac{kT}{{\varepsilon_1} - \mu } \ll \frac{kT}{{\varepsilon_0} - \mu } = {N_0}
\end{equation}
\subsection{范德瓦尔斯方程}
\begin{equation}
\left( {P + \frac{a{N^2}}{V^2}} \right)\left( {V - bN} \right) = NkT
\end{equation}
\subsection{量子转子能级}
角量子数 $l$ 决定能级
\begin{equation}
{E_l} = l\left( {l + 1} \right)\frac{\hbar ^2}{2IkT}
\end{equation}
$2l+1$ 重简并, 其中 $I = {{m_1}{m_2}r_{12}^2}/({{m_1} + {m_2}})$ 为质心转动惯量.\\
当 $l$ 为偶数时, 两粒子的波函数具有交换对称, 奇数时反对称.\\
两原子核的自旋共有 ${s^2} = {\left( {2I + 1} \right)^2}$ 种状态, 其中对称态占 ${s\left( {s + 1} \right)}/{2}$ 种, 反对称太占 ${s\left( {s - 1} \right)}/{2}$ 种.\\
若两粒子都是费米子($I$ 为半整数), 则总波函数反对称, 即 $l$ 为单数核自旋对称, 或 $l$ 为偶数核自旋反对称.\\
\subsection{弹簧振子能级}
\begin{equation}
{E_n} = \hbar \omega \left( {n + \frac{1}{2}} \right)
\end{equation}
非简并.\\
为什么书上说 $m = 0 $ (能级密度与 ${\varepsilon ^m}$ 成正比)不能产生凝聚态, 然而我在模拟中做到了?\\