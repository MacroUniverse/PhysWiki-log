% Frog-Crab 笔记

Frog-Crab 的论文是 Y. Mairesse 和 F. Quere 2005 年的 Frequency-resolved optical gating for complete reconstruction of attosecond bursts. 以下的公式全部使用原子单位。

\subsection{Strong Field Approximation}

主要参考文献是 Lewenstein 1994 的 HHG 论文(见 Carlos Research), 但这里另外做了修改。

令 $\vec E_{XUV}$ 为 XUV 的电场, 而 IR 电场用矢势来表示 $\vec E_{IR} = -\pdv*{\vec A}{t}$。 令末态波函数为
\begin{equation}
\ket{\Psi(t, \tau)} = \E^{\I I_p t} \qty(\ket{0} + \int a(\vec v, \tau) \ket{\vec v} \dd[3]{\vec v})
\end{equation}
强场近似下, $a(\vec v, \tau)$ 的解为

\begin{equation}
a(\vec v, \tau) = -\I\int_{-\infty}^{\infty} \dd{t} \E^{\I\phi(\vec v, t)} \vec d_{\vec v(t)} \vec E_{XUV}(t-\tau) \E^{\I(W+I_p)t}
\end{equation}

\begin{equation}
\phi(\vec v, t) = -\int_t^{+\infty} \dd{t'} [\vec v\vdot \vec A(t') + \vec A^2(t')/2]
\end{equation}
其中 $\vec d_{\vec v} = \mel{\vec v}{\vec r}{0}$, $\ket{\vec v}$ 为速度为 $\vec v$ 的平面波, 由于经典力学中 $\vec p = \vec v - \vec A$ 是一个守恒量\footnote{我实在不明白为什么许多论文上都写 $\vec p = \vec v + \vec A$!}, $\vec v(t) = \vec v(\infty) - \vec A(\infty) + \vec A(t)$。

下面来算 $\vec d_{\vec k}$,已知氢原子基态为 $\ket{0} = \E^{-r}/\sqrt{\pi}$, $\ket{\vec k} = \exp(\I \vec k \vdot \vec x)/(2\pi)^{3/2}$。 由于氢原子基态球对称, 不失一般性, 可以将 $\vec k$ 的方向设为极轴的方向($z$), 同样由对称性可得 $\vec d_{\vec k}$ 只有 $z$ 分量不为零。 

\begin{equation}
\vec d_{\vec k} = \frac{ \uvec k}{\sqrt2\pi} \int_0^{+\infty} \int_0^\pi \E^{-r} \E^{-\I k r \cos\theta} r \cos\theta \cdot r^2 \sin\theta \dd{\theta} \dd{r}
\end{equation}
换元, 令 $u = \cos\theta$, 得
\begin{equation}\ali{
\vec d_{\vec k} &= \frac{\uvec k}{\sqrt2\pi}  \int_0^{+\infty} r^3 \E^{-r} \int_{-1}^1 \E^{-\I k r u} u  \dd{u} \cdot \dd{r}\\
&=  \I\frac{\sqrt2 \uvec k}{\pi k}  \int_0^{+\infty} r^2 \E^{-r} \qty[\cos(kr) - \frac{1}{kr}\sin(kr)] \dd{r}\\
&= -\I \frac{8\sqrt2}{\pi} \frac{\vec k}{(k^2+1)^3}
}\end{equation}
这与 Lewenstein 1994 的论文中给出的多了一个负号(不过我应该是对的)。

用 Matlab 根据上面的公式计算了一下 Frog-Crab trace, 但我的是对称的, 而论文上不对称, 师兄说这是论文上的 XUV pulse 有 chirp, 一看果然是。 想了一下 Phase Gate 是什么原理, 发现 $\E^{\I\phi(t)}$ 的确随 $\vec v$ 变化较小(见“phi change.png”), 但还有一项 $\vec d_{\vec p}$ 似乎随时间变化还是比较明显的, 魏晖的意思是也当成常数就行了, 不过他还给了我另一篇文章专门讨论 dipole 变化的, 存在 ipad 相册里面。 至于最后一项 $\E^{\I(W+I_p)t}$, 说明这是傅里叶变换到能量表象。然而既然到能量表象了, 就不能直接使用 $\abs{a(v, t)}^2$ 了, 而是需要做一个换元, 导致 $a(v)$ 需要除以 $\sqrt{v}$。

至于 dipole 随 $v$ 的变化, 观察了一下发现基本上可以认为
\begin{equation}
d(v, t) \approx -\I \frac{8\sqrt2}{\pi} \frac{v}{(v^2+1)^3} g(t);
\end{equation}
其中 $g(t)$ 可看做一个不随 $v$ 变化的时间函数, 其值约等于 $C_0 + \gamma A(t)$, $\gamma$ 是一个较小的常数(见“g(t) change.png”)。 这样一来, 我们只需要将能量谱(先除以 $\sqrt{v}$)除以上式中的 $v$ 因子, 再用 PCGPA 就可以得到 pulse 和 gate, 而得到的 gate 将会是 $g(t)\exp[\phi(t)]$。

到此为止, Frog-Crab 基本上已经被我吃透了, 写出程序只是时间问题。

新的困难: 实验中产生的 trace 的时候只有 $E + I_p > 0$ 的部分, 然而完整的傅里叶变换  $\exp[\I (E + I_p) t]$ 显示 $-(E + I_p)$ 处也有一条类似的 trace, 而且并不是完全对称的(因为被积函数既不是实函数也不是偶函数)。在论文上这点完全没有被提及。目前计划的处理方法就是在 PCGPA 保留负 efrog 不变, 而只把正 efrog 的模长换成正 trace 的模长。

另一个困难:由于 XUV+IR 至少需要 10000 个格点,Matlab 跑起来已经非常吃力了,现在一是要在 C++ 中实现,二是要进行优化。Numerical Recipes 中已经有 FFT 和 SVD 的算法了,然而 SVD 却不支持复数!其实可以仅认为 V 矩阵含有复数,然后对实部和虚部分别用一次 SVD 即可(这样不能保证 V 矩阵正交,但是并没有关系)。至于优化, 首先算 SVD 的时候输入矩阵只有中间的一横条不为零,完全可以只对这部分做 SVD。 另外 FFT 虽然比 DFT 要快, 但如果用 DFT,就可以只对不为零的部分积分,且只算不为零部分的能谱 ,这样说不定会更快。








