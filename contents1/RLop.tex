% 升降算符
% 未完成: 考虑要不要使用狄拉克符号

\pentry{本征方程} % 未完成

\subsection{结论}

已知某个算符 $\Q Q$,若能找到另一个算符 ${\Q Q_ + }$,使得 $[\Q Q,{\Q Q_ + }] = h{\Q Q_ + }$ 成立( 是大于零的实数),这个算符就是 $\Q Q$ 对应的升算符.同理,若有 ${\Q Q_- }$ 使得 $[\Q Q,{\Q Q_ - }] =  - h{\Q Q_ - }$ 成立,这个算符就是对应的降算符.升降算符的作用是把一个本征函数变为本征值更大或者更小的本征函数.即 $\Q Q\left( {{{\Q Q}_ \pm }\psi } \right) = \left( {q \pm h} \right)\left( {{{\Q Q}_ \pm }\psi } \right)$.

\subsection{意义}
有时候如果算符过于复杂求解本征方程比较困难,就可以尝试寻找升降算符.升降算符可以让我们不用求解本征方程就可以快速地找到本征值.具体见简谐振子和轨道角动量.%未完成

\subsection{证明}
如果 $\psi$ 是 $\Q Q$ 的一个本征函数,且本征值为 $\lambda$,那么根据对易关系 $[\Q Q,{\Q Q_ + }] = h{\Q Q_ + }$ 有
\begin{equation}
\Q Q\left( {{{\Q Q}_ + }\psi } \right) = {\Q Q_ + }\left( {\Q Q\psi } \right) + h{\Q Q_ + }\psi  = {\Q Q_ + }\left( {\lambda \psi } \right) + h{\Q Q_ + }\psi  = \left( {\lambda  + h} \right)\left( {{{\Q Q}_ + }\psi } \right)
\end{equation}
降算符的证明同理.

\subsection{本征函数的归一化}
注意升降算符并不一定能保持函数的归一化.若假设 ${\Q a_ \pm }\left| {{\psi _n}} \right\rangle  = {A_n}\left| {{\psi _{n + 1}}} \right\rangle$, 其中 $\left| {{\psi _n}} \right\rangle$ 和 $\left| {{\psi _{n + 1}}} \right\rangle$ 都是归一化的本征态,那么由归一化条件要求
\begin{equation}
\left\langle {{\psi _n}} \right|\Q a_ \pm ^ * {\Q a_ \pm }\left| {{\psi _n}} \right\rangle  = {\left| {{A_n}} \right|^2}\left\langle {{{\psi _{n + 1}}}}
 \mathrel{\left | {\vphantom {{{\psi _{n + 1}}} {{\psi _{n + 1}}}}}
 \right. \kern-\nulldelimiterspace}
 {{{\psi _{n + 1}}}} \right\rangle  = {\left| {{A_n}} \right|^2}
\end{equation}
习惯上令 $A_n$ 为实数,即上式开方. 
