%简谐振子(升降算符)

%未完成
%(首先要参考”升降算符”这篇, 根据里面的定义证明a+a-就是H的升降算符, 然后就好办了).
\pentry{升降算符}%未完成链接

\subsection{结论}
\begin{equation}
{\Q a_ \pm } = \frac{1}{\sqrt {2m\omega \hbar } }\left( {m\omega \Q x \mp i\Q p} \right)
\quad
{\Q a_ + }{\psi_n} = \sqrt {n + 1}    {\psi_{n + 1}}
\quad
{\Q a_ - }{\psi_n} = \sqrt n    {\psi_{n - 1}}
\end{equation}
\begin{equation}
 {E_n} = \left( {n+\frac12} \right)\omega \hbar
\qquad
\Q n = {\Q a_ + }{\Q a_ - }
\qquad
\Q H = \omega \hbar {\text{ }}\left( {\Q n + \frac12} \right)
\end{equation}

\begin{equation}
\psi_0 (x) = \frac{1}{{\pi ^{1/4}}{\beta ^{1/2}}}{\E^{ - {{(x/\beta )}^2}/2}}  \qquad \beta  = \sqrt {\frac{\hbar }{m\omega }}
\end{equation}

\subsection{推导}
在经典的弹簧振子模型中,若质点沿 $x$ 轴方向振动,且在 $x = 0$ 处平衡,则势能函数 $V\left( x \right) = k{x^2}/{2}$ . 由于自由振动的频率为 $\omega  = \sqrt {{k}/{m}} $, 所以势能可记为
\begin{equation}
  V\left( x \right) = \frac12 m{\omega ^2}{x^2}
\end{equation}
在量子力学中,这个模型要用薛定谔方程来求解. 该模型的哈密顿算符为
\begin{equation}
  \Q H = \frac{{\Q p}^2}{2m} + \Q V = \frac{{\Q p}^2}{2m} + \frac12 m{\omega ^2}{\Q x^2} = \frac{1}{2m}\left[ {{{\Q p}^2} + {{\left( {m\omega \Q x} \right)}^2}} \right]
\end{equation}
定态薛定谔方程(能量的本征方程)为
\begin{equation}\label{QSHOop_eq6}
  \Q H\psi  = E\psi
\end{equation}
由于这个方程需要使用幂级数%链接未完成
,但作为一种巧妙的方法,先利用升降算符%未完成链接
来得到能量的本征值,再求本征函数.这里直接给出 $\Q H$ 的升降算符,它们分别可以把本征值升降 $\omega\hbar$ (证明见下文)
\begin{equation}
{\Q a_ \pm } = \frac{1}{\sqrt {2m\omega \hbar } }\left( {m\omega \Q x \mp i\Q p} \right)
\qquad
\Delta E = \omega \hbar
\end{equation}

根据升降算符的结论,%未完成
对任意一个 $\Q H$ 的本征函数 ${\psi_n}$, 有
\begin{equation}
  \Q H\left( {{{\Q a}_ \pm }{\psi_n}} \right) = \left( {{E_n} \pm \hbar \omega } \right)\left( {{{\Q a}_ \pm }{\psi_n}} \right)
\end{equation}
这也就是说,简谐振子的定态薛定谔方程的解中,本征值 ${E_n}$ 取离散值,且相邻两个能级相差 $\Delta E = \hbar \omega $ . 

类似于无限深势垒% 链接未完成
,谐振子也应该有一个最低能级 ${E_0}$ 和对应的 ${\psi_0}\left( x \right)$ . 所以 ${\Q a_ - }$ 必然对 ${\psi_0}$ 无效,即得到的波函数没有物理意义,所以不妨猜测 ${\Q a_ - }{\psi_0} = 0$ 
即
\begin{equation}
(m\omega \Q x + i\Q p){\psi_0} = 0
\quad \Rightarrow \quad
\frac{d}{dx}{\psi_0} =  - \frac{m\omega x}{\hbar }{\psi_0}
\end{equation}
%(可以根据这个,解出最低能级为 $\frac12 \hbar \omega $, 于是乎,就有了猫猫爬梯梯的图片了).
这是一阶齐次线性微分方程,%通解
通解为
\begin{equation}
{\psi_0}(x) = \frac{1}{{\pi ^{1/4}}\sqrt \beta  }{\E^{ - {{(x/\beta )}^2}/2}}
\end{equation}
其中 $\beta = \sqrt {\hbar /m\omega }$ 具有长度量纲.不难验证上式是定态薛定谔方程\autoref{QSHOop_eq6} 的解,本征值为 $E_0=\omega\hbar/2$.


\subsection{归一化条件}


证明 ${\psi_{n + 1}} = {1}/{\sqrt {n + 1} }{\Q a_ + }{\psi_n}$,  ${\psi_{n - 1}} = {1}/{\sqrt n }{\Q a_ - }{\psi_n}$ 
\begin{equation}
\begin{aligned}
  {A^2} & = \int {{{\left( {{a_ + }{\psi_n}} \right)}^*}\left( {{a_ + }{\psi_n}} \right)dx} \\
   & = \int {{{\left( {a_ + ^*{a_ + }{\psi_n}} \right)}^*}\left( {\psi_n} \right)dx}  \\
   & = \int {{{\left( {{a_ - }{a_ + }{\psi_n}} \right)}^*}\left( {\psi_n} \right)dx}
\end{aligned}
\end{equation}
而 ${\Q a_ - }{\Q a_ + }{\psi_n} = {\Q H{\psi_n}}/{\omega \hbar } + {1}/{2}{\psi_n} = \left( {n + 1} \right){\psi_n}$ . 所以
\begin{equation}
  {A^2} = \left( {n + 1} \right)\int {\psi_n^*{\psi_n}dx}  = n + 1. A = \sqrt {n + 1}
\end{equation}
所以 ${\psi_{n + 1}} = {1}/{\sqrt {n + 1} }{\Q a_ + }{\psi_n}$

\subsection{波函数}
现在只需要对基态波函数不断使用升算符和归一化系数%公式链接
就可以得到任意激发态波函数(已归一化)
\begin{equation}
{\psi_n} = \frac{{\Q a}_ + }{\sqrt {n!} }{\psi_0}
\end{equation}


\subsection{升降算符的证明}

根据升降算符的定义, 要证明 $\Q a_\pm$是升降算符只需证明对易关系

\begin{equation}
  [\Q H,{\Q a_ \pm }] =  \pm \omega\hbar  {\Q a_ \pm }
\end{equation}
根据升降算符定义
\begin{equation}
  [\Q H,{\Q a_ \pm }] = \frac{1}{2m\sqrt {2m\hbar \omega } }\left[ {{{\left( {m\omega \Q x} \right)}^2} + {{\Q p}^2},m\omega \Q x \mp i\Q p} \right]
\end{equation}
不难证明
\begin{equation}
  [\Q A + \Q B,\Q C + \Q D] = [\Q A,\Q C] + [\Q A,\Q D] + [\Q B,\Q C] + [\Q B,\Q D]
\end{equation}
和
\begin{equation}
  [{\Q x^2},\Q x] = [{\Q p^2},\Q p] = 0
\end{equation}
将两式代入后,只剩下两个交叉项
\begin{equation}
  [\Q H,{\Q a_ \pm }] = \frac{\omega }{2\sqrt {2m\hbar \omega } }\left( { \mp i[{{\Q x}^2},\Q p] + [{{\Q p}^2},\Q x]} \right)
\end{equation}
同样可以证明
\begin{equation}
  [\Q A\Q B,\Q C] = \Q A[\Q B,\Q C] + [\Q A,\Q C]\Q B
\end{equation}
所以
\begin{equation}
[{{\Q x}^2},\Q p] = \Q x[\Q x,\Q p] + [\Q x,\Q p]\Q x = 2i\hbar \Q x
\end{equation}
\begin{equation}
[{{\Q p}^2},\Q x] =  - \Q p[\Q x,\Q p] - [\Q x,\Q p]\Q p =  - 2i\hbar \Q p
\end{equation}
代入得
\begin{equation}
  [\Q H,{\Q a_ \pm }] = \frac{ \pm \omega \hbar }{\sqrt {2m\hbar \omega } }\left( {\Q x \mp i\Q p} \right) =  \pm \omega\hbar {\Q a_ \pm }
\end{equation}
证毕.
