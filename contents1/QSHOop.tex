%简谐振子(升降算符)

%未完成
%(首先要参考”升降算符”这篇, 根据里面的定义证明a+a-就是H的升降算符, 然后就好办了).
\pentry{升降算符}%未完成链接

\subsection{结论}
\begin{equation}
a_\pm = \frac{1}{\sqrt{2m\omega\hbar}} (m\omega x \mp \I p) \quad
a_+ \psi_n = \sqrt{n + 1} \psi_{n+1}
\quad
a_- \psi_n = \sqrt n \psi_{n-1}
\end{equation}
\begin{equation}
E_n = \qty(n+\frac12)\omega \hbar \qquad
\Q n = a_+ a_ - \qquad
H = \omega\hbar \qty(\Q n + \frac12)
\end{equation}

\begin{equation}
\psi_0 (x) = \frac{1}{\pi^{1/4} \beta^{1/2}} \E^{-(x/\beta)^2/2}  \qquad \beta  = \sqrt{\frac{\hbar}{m\omega}}
\end{equation}

\subsection{推导}
在经典的简谐振子模型中,若质点沿 $x$ 轴方向振动,且在 $x = 0$ 处平衡,则势能函数 $V(x) = k x^2/2$ . 由于自由振动的频率为 $\omega = \sqrt{k/m}$, 所以势能可记为
\begin{equation}
V(x) = \frac12 m \omega^2 x^2
\end{equation}
在量子力学中,这个模型要用薛定谔方程来求解. 该模型的哈密顿算符为
\begin{equation}
H = \frac{p^2}{2m} + V = \frac{p^2}{2m} + \frac12 m\omega^2 x^2 = \frac{1}{2m} [p^2 + (m\omega x)^2]
\end{equation}
定态薛定谔方程(能量的本征方程)为
\begin{equation}\label{QSHOop_eq6}
H\psi  = E\psi
\end{equation}
由于这个方程需要使用幂级数%链接未完成
,但作为一种巧妙的方法,先利用升降算符%未完成链接
来得到能量的本征值,再求本征函数.这里直接给出 $H$ 的升降算符,它们分别可以把本征值升降 $\omega\hbar$ (证明见下文)
\begin{equation}
a_\pm = \frac{1}{\sqrt{2m\omega\hbar}} (m\omega x \mp \I p)
\qquad
\Delta E = \omega \hbar
\end{equation}

根据升降算符的结论,%未完成
对任意一个 $H$ 的本征函数 $\psi_n$, 有
\begin{equation}
H(a_\pm\psi_n) = (E_n\pm\hbar\omega) (a_ \pm \psi_n)
\end{equation}
这也就是说,简谐振子的定态薛定谔方程的解中,本征值 $E_n$ 取离散值,且相邻两个能级相差 $\Delta E = \hbar \omega$ . 

类似于无限深势垒% 链接未完成
,谐振子也应该有一个最低能级 $E_0$ 和对应的 $\psi_0(x)$ . 所以 $a_-$ 必然对 $\psi_0$ 无效,即得到的波函数没有物理意义,所以不妨猜测 $a_- \psi_0 = 0$ 
即
\begin{equation}
(m\omega x + \I p)\psi_0 = 0
\quad \Rightarrow \quad
\dv{x} \psi_0 =  - \frac{m\omega x}{\hbar } \psi_0
\end{equation}
%(可以根据这个,解出最低能级为 $\frac12 \hbar \omega $, 于是乎,就有了猫猫爬梯梯的图片了).
这是一阶齐次线性微分方程,%通解
通解为
\begin{equation}
\psi_0(x) = \frac{1}{\pi^{1/4}\sqrt\beta} \E^{-(x/\beta)^2/2}
\end{equation}
其中 $\beta = \sqrt{\hbar /m\omega}$ 具有长度量纲.不难验证上式是定态薛定谔方程\autoref{QSHOop_eq6} 的解,本征值为 $E_0=\omega\hbar/2$.

\subsection{归一化条件}

证明 $\psi_{n+1} = a_+\psi_n/\sqrt{n+1}$,  $\psi_{n-1} = a_- \psi_n/\sqrt n$ 
\begin{equation}\ali{
A^2 & = \int (a_+ \psi_n)^* (a_+ \psi_n) \dd{x}
= \int (a_+^* a_+ \psi_n)^* (\psi_n) \dd{x}\\
&= \int (a_- a_+ \psi_n)^* (\psi_n) \dd{x}
}\end{equation}
而 $a_- a_+ \psi_n = H\psi_n/(\omega\hbar) + \psi_n/2 = (n + 1)\psi_n$ . 所以
\begin{equation}
A^2 = (n + 1)\int \psi_n^*{\psi_n} \dd{x}  = n + 1\qquad
A = \sqrt{n+1}
\end{equation}
所以 $\psi_{n+1} = a_+ \psi_n/\sqrt{n+1}$

\subsection{波函数}
现在只需要对基态波函数不断使用升算符和归一化系数%公式链接
就可以得到任意激发态波函数(已归一化)
\begin{equation}
\psi_n = \frac{a_+}{\sqrt {n!}} \psi_0
\end{equation}

\subsection{升降算符的证明}

根据升降算符的定义, 要证明 $a_\pm$是升降算符只需证明对易关系

\begin{equation}
\comm{H}{a_\pm} =  \pm \omega\hbar a_\pm
\end{equation}
根据升降算符定义
\begin{equation}
\comm{H}{a_\pm} = \frac{1}{2m\sqrt{2m\hbar \omega}} \comm{(m\omega x)^2 + p^2}{m\omega x \mp \I p}
\end{equation}
不难证明
\begin{equation}
\comm{\Q A + \Q B}{\Q C + \Q D} = \comm{\Q A}{\Q C} + \comm{\Q A}{\Q D} + \comm{\Q B}{\Q C} + \comm{\Q B}{\Q D}
\end{equation}
和
\begin{equation}
\comm{x^2}{x} = \comm{p^2}{p} = 0
\end{equation}
将两式代入后,只剩下两个交叉项
\begin{equation}
\comm{H}{a_\pm} = \frac{\omega}{2\sqrt{2m\hbar \omega}} \qty(\mp \I \comm{x^2}{p} + \comm{p^2}{x})
\end{equation}
同样可以证明
\begin{equation}
\comm{\Q A\Q B}{\Q C} = \Q A \comm{\Q B}{\Q C} + \comm{\Q A}{\Q C}\Q B
\end{equation}
所以
\begin{equation}
\comm{x^2}{p} = x\comm{x}{p} + \comm{x}{p} x = 2\I\hbar  x
\end{equation}
\begin{equation}
\comm{p^2}{x} =  -  p\comm{x}{p} - \comm{x}{p} p =  - 2\I\hbar  p
\end{equation}
代入得
\begin{equation}
\comm{H}{a_\pm} = \frac{\pm\omega\hbar}{\sqrt{2m\hbar\omega}} (x \mp \I p) =  \pm \omega\hbar a_ \pm
\end{equation}
证毕.
