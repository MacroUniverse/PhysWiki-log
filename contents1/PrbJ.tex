% 概率流密度

% 未完成: 结构不完善!
\pentry{含时薛定谔方程}% 未完成

\subsection{结论}
一维情况下,对于某个波函数 $\psi \left( {x,t} \right)$,定义概率流为
\begin{equation}
J\left( {x,t} \right) = \frac{{i\hbar }}{{2m}}\left( {\psi \frac{{\partial {\psi ^ * }}}{{\partial x}} - {\psi ^ * }\frac{{\partial \psi }}{{\partial x}}} \right)
\end{equation}
某个区间中的概率增加率等于流入该区间的概率流
\begin{equation}
\frac{\D}{{\D t}}{P_{ab}}(t) = J(a,t) - J(a,t)
\end{equation}
三维情况下,概率流的定义变为
\begin{equation}
\vec J\left( {\vec r,t} \right) = \frac{{i\hbar }}{{2m}}\left( {\psi \grad {\psi ^ * } - {\psi ^ * }\grad \psi } \right)
\end{equation}
且有
\begin{equation}
\frac{\D}{{\D t}}{P_\mathcal{V}}(t) = \frac{\D}{{\D t}}\int\limits_\mathcal{V} {{{\left| {\psi (\vec r,t)} \right|}^2}\D V}  = \int\limits_\mathcal{S} {\vec J(\vec r,t) \vdot \D\vec s}
\end{equation}
或写为概率守恒公式(类比电荷守恒) %(链接未完成))
\begin{equation}
\frac{\D}{{\D t}}({\psi ^ * }\psi ) + \div\vec J = \vec 0
\end{equation}
平面波的概率流的速度就是例子密度.

\subsection{推导}

对一维情况有
\begin{equation}
\frac{\D}{{\D t}}{P_{ab}} = \frac{\D}{{\D t}}\int_a^b {{\psi ^ * }\psi \D x}  = \int_a^b {\left( {\psi \frac{\partial }{{\partial t}}{\psi ^ * } + {\psi ^ * }\frac{\partial }{{\partial t}}\psi } \right)\D x}
\end{equation}
一维薛定谔方程以及复共轭为
\begin{equation}
i\hbar \frac{{\partial \psi }}{{\partial t}} =  - \frac{{{\hbar ^2}}}{{2m}}\frac{{{\partial ^2}\psi }}{{\partial {x^2}}} + V\psi
\end{equation}
\begin{equation}
 - i\hbar \frac{{\partial {\psi ^ * }}}{{\partial t}} =  - \frac{{{\hbar ^2}}}{{2m}}\frac{{{\partial ^2}{\psi ^ * }}}{{\partial {x^2}}} + V{\psi ^ * }
\end{equation}
代入上式的时间微分,得
\begin{equation}
\begin{aligned}
  \frac{\D}{{\D t}}{P_{ab}} &= \frac{{i\hbar }}{{2m}}\int_a^b {\left( {{\psi ^ * }\frac{{{\partial ^2}\psi }}{{\partial {x^2}}} - \psi \frac{{{\partial ^2}{\psi ^ * }}}{{\partial {x^2}}}} \right)\D x}  = \frac{{i\hbar }}{{2m}}\int_a^b {\frac{\partial }{{\partial x}}\left( {{\psi ^ * }\frac{{\partial \psi }}{{\partial x}} - \psi \frac{{\partial {\psi ^ * }}}{{\partial x}}} \right)\D x} \\
   &= \left. {\frac{{i\hbar }}{{2m}}\left( {{\psi ^ * }\frac{{\partial \psi }}{{\partial x}} - \psi \frac{{\partial {\psi ^ * }}}{{\partial x}}} \right)} \right|_{x = a}^{x = b} = J(a) - J(b)
\end{aligned}
\end{equation}
三维情况的证明可类比.
% 未完成: 真的可以吗?

\subsection{概率流的速度}

类比经典力学或电磁学中的 $\vec j = \rho \vec v$,若定义概率流速度为概率流除以概率密度,则平面波 $\psi (x) = A{e^{i\vec k \vdot \vec r}}$ 的概率流速为
\begin{equation}
\vec v = \vec j/{\left| \psi  \right|^2} = \frac{{i\hbar }}{{2m}}\left( { - {{\left| A \right|}^2}i\vec k - {{\left| A \right|}^2}i\vec k} \right)/{\left| A \right|^2} = \frac{{\hbar \vec k}}{m} = \frac{{\vec p}}{m} = {\vec v_{CM}}
\end{equation}
所以平面波的概率流速度等于具有相同动量的经典粒子的速度.
