% 概率流密度

% 未完成: 结构不完善!
\pentry{含时薛定谔方程}% 未完成

\subsection{结论}
一维情况下,对于某个波函数 $\psi(x,t)$,定义概率流为
\begin{equation}
J(x,t) = \frac{\I\hbar}{2m} \qty(\psi \pdv{\psi^*}{x} - \psi^* \pdv{\psi}{x})
\end{equation}
某个区间中的概率增加率等于流入该区间的概率流
\begin{equation}
\dv{t} P_{ab}(t) = J(a,t) - J(a,t)
\end{equation}
三维情况下,概率流的定义变为
\begin{equation}
\vec J(\vec r,t) = \frac{\I\hbar }{2m} (\psi \grad \psi^* - \psi ^* \grad \psi)
\end{equation}
且有
\begin{equation}
\dv{t} P_\mathcal{V}(t) = \dv{t} \int_\mathcal{V} \abs{\psi (\vec r,t)}^2 \dd{V}
= \int_\mathcal{S} \vec J(\vec r,t) \vdot \dd{\vec s}
\end{equation}
或写为概率守恒公式(类比电荷守恒) %(链接未完成))
\begin{equation}
\dv{t} (\psi^* \psi) + \div\vec J = \vec 0
\end{equation}
平面波的概率流的速度就是例子密度.

\subsection{推导}

对一维情况有
\begin{equation}
\dv{t} P_{ab} = \dv{t} \int_a^b \psi^* \psi \dd{x}  = \int_a^b \qty(\psi \pdv{t} \psi^* + \psi^* \pdv{t} \psi) \dd{x}
\end{equation}
一维薛定谔方程以及复共轭为
\begin{equation}
\I\hbar \pdv{\psi}{t} =  - \frac{\hbar ^2}{2m} \pdv[2]{\psi}{x} + V\psi
\end{equation}
\begin{equation}
- \I\hbar \pdv{\psi^*}{t} =  - \frac{\hbar ^2}{2m} \pdv[2]{\psi^*}{x} + V{\psi^*}
\end{equation}
代入上式的时间微分,得
\begin{equation}\ali{
\dv{t} P_{ab} &= \frac{\I\hbar }{2m} \int_a^b \qty(\psi^* \pdv[2]{\psi}{x} - \psi \pdv[2]{\psi^*}{x})\dd{x} = \frac{\I\hbar }{2m} \int_a^b \pdv{x} \qty(\psi^* \pdv{\psi}{x} - \psi \pdv{\psi^*}{x}) \dd{x}\\
&= \eval{\frac{\I\hbar }{2m} \qty(\psi^*\pdv{\psi}{x} - \psi \pdv{\psi^*}{x})}_{x=a}^{x=b} = J(a) - J(b)
}\end{equation}
三维情况的证明可类比.
% 未完成: 真的可以吗?

\subsection{概率流的速度}

类比经典力学或电磁学中的 $\vec j = \rho \vec v$,若定义概率流速度为概率流除以概率密度,则平面波 $\psi (x) = A\E^{\I \vec k \vdot \vec r}$ 的概率流速为
\begin{equation}
\vec v = \vec j/\abs{\psi}^2 = \frac{\I\hbar}{2m} \qty(-\abs{A}^2 \I\vec k - \abs{A}^2 \I\vec k)/\abs{A}^2 = \frac{\hbar \vec k}{m} = \frac{\vec p}{m} = \vec v_{CM}
\end{equation}
所以平面波的概率流速度等于具有相同动量的经典粒子的速度.
