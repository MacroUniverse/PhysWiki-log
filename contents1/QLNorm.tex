%轨道角动量升降算符归一化

\pentry{轨道角动量%未完成
}


首先要提醒,一般来说,算符满足的一个条件是 $\left\langle {g}
 \mathrel{\left | {\vphantom {g {\Q Qf}}}
 \right. \kern-\nulldelimiterspace}
 {{\Q Qf}} \right\rangle  = \left\langle {{{{\Q Q}^ * }g}}
 \mathrel{\left | {\vphantom {{{{\Q Q}^ * }g} f}}
 \right. \kern-\nulldelimiterspace}
 {f} \right\rangle $.但是对于厄米算符,${\Q Q^ * } = \Q Q$, 所以有 $\left\langle {g}
 \mathrel{\left | {\vphantom {g {\Q Qf}}}
 \right. \kern-\nulldelimiterspace}
 {{\Q Qf}} \right\rangle  = \left\langle {{\Q Qg}}
 \mathrel{\left | {\vphantom {{\Q Qg} f}}
 \right. \kern-\nulldelimiterspace}
 {f} \right\rangle $.

对于角动量升算符
\begin{equation}
{\Q L_ + }{\Q L_ - } = \left( {{{\Q L}_x} + i{{\Q L}_y}} \right)\left( {{{\Q L}_x} - i{{\Q L}_y}} \right) = \Q L_x^2 + \Q L_y^2 - i[{\Q L_x},{\Q L_y}] = {\Q L^2} - \Q L_z^2 + \hbar {\Q L_z}
\end{equation} 
所以
\begin{equation}\begin{aligned}
{\Q L_ + }{\Q L_ - }{\varphi_{l,m}} &= {\hbar ^2}l\left( {l + 1} \right){\varphi_{l,m}} - {\hbar ^2}{m^2}{\varphi_{l,m}} + m{\hbar ^2}{\varphi_{l,m}} \\
&= {\hbar ^2}\left[ {l\left( {l + 1} \right) - m\left( {m - 1} \right)} \right]{\varphi_{l,m}}
\end{aligned}\end{equation} 
所以
\begin{equation}\label{QLNorm_eq1}
\left\langle {{{{\Q L}_ - }{\varphi_{l,m}}}}
 \mathrel{\left | {\vphantom {{{{\Q L}_ - }{\varphi_{l,m}}} {{{\Q L}_ + }{\varphi_{l,m}}}}}
 \right. \kern-\nulldelimiterspace}
 {{{{\Q L}_ + }{\varphi_{l,m}}}} \right\rangle  = \left\langle {{{\varphi_{l,m}}}}
 \mathrel{\left | {\vphantom {{{\varphi_{l,m}}} {{{\Q L}_ + }{{\Q L}_ - }{\varphi_{l,m}}}}}
 \right. \kern-\nulldelimiterspace}
 {{{{\Q L}_ + }{{\Q L}_ - }{\varphi_{l,m}}}} \right\rangle  = {\hbar ^2}\left[ {l\left( {l + 1} \right) - m\left( {m - 1} \right)} \right]
\end{equation} 
所以
\begin{equation}
{\Q L_ - }{\varphi_{l,m}} = \hbar \sqrt {l\left( {l + 1} \right) - m\left( {m - 1} \right)} \,{\varphi_{l,m - 1}}
\end{equation}
同理可证
\begin{equation}
{\Q L_ + }{\varphi_{l,m}} = \hbar \sqrt {l\left( {l + 1} \right) - m\left( {m + 1} \right)} \,{\varphi_{l,m + 1}}
\end{equation} 

严格来说,归一化系数后面加上任意相位因子 ${\E^{\I \theta }}$ 后仍能满足\autoref{QLNorm_eq1}, 但一般省略.



