%介质中的波动方程
%15min

对非磁介质
\begin{equation}
  \curl (\curl \vec E) =  - {\mu _0}\frac{{{\partial ^2}\vec D}}{{\partial {t^2}}}
 \end{equation}
平面波时
 \begin{equation}
  {\laplacian}\vec E - {\mu _0}\frac{{{\partial ^2}\vec D}}{{\partial {t^2}}} = 0
 \end{equation}
 \begin{equation}
\vec D = {\epsilon _0}\vec E + \vec P = {\epsilon _0}(1 + \tilde \chi )\vec E + {\vec P^{NL}} = \tilde \epsilon \vec E + {\vec P^{NL}}
 \end{equation}
 \begin{equation}
{\laplacian}\vec E - {\mu _0}\tilde \epsilon \frac{{{\partial ^2}\vec E}}{{\partial {t^2}}} = {\mu _0}\frac{{{\partial ^2}{{\vec P}^{NL}}}}{{\partial {t^2}}}
 \end{equation}
 
波浪号表示复数,$\tilde \epsilon$ 和 $\tilde \chi $ 都只是 $\omega $ 而不是场强的函数. 先来看线性的情况( ${\vec P^{NL}} = \vec 0$ ), 例如洛伦兹模型.
\begin{equation}
{\laplacian}\vec E - {\mu _0}\tilde \epsilon \frac{{{\partial ^2}\vec E}}{{\partial {t^2}}} = \vec 0
 \end{equation}
(见齐次波动方程, 先做时间傅里叶变换, 再解齐次亥姆霍兹方程, 通解是所有平面波) 然而这里的 ${\tilde k^2} = {\mu _0}\tilde \epsilon \omega$ 是复数, 平面波变为指数衰减的单频单向波. 以 $z$ 方向传播 $x$ 方向极化为例, 令  $\tilde k = k + \kappa $ 
\begin{equation}
{E_x}(z,t) = {\tilde E_{0x}}{e^{i(\tilde kz - \omega t)}} = {\tilde E_{0x}}{e^{ - \kappa z}}{e^{i(kz - \omega t)}}
\end{equation}
\begin{equation}
\tilde k = \omega \sqrt {{\mu _0}\tilde \epsilon }  = \frac{\omega }{c}\sqrt {1 + {\chi ^{(1)}}}
\end{equation}
通解仍然为所有可能的单频单向波的线性组合. 实折射率和吸收系数定义为
\begin{equation}
n = \frac{{ck}}{\omega } = \operatorname{Re} \left[ {\sqrt {1 + {\chi ^{(1)}}} } \right] \approx 1 + \frac{1}{2}\operatorname{Re} [{\chi ^{(1)}}]
\end{equation}
\begin{equation}
\alpha  = 2\kappa  = \frac{{2\omega }}{c}\operatorname{Im} \left[ {\sqrt {1 + {\chi ^{(1)}}} } \right] \approx \frac{\omega }{c}\operatorname{Im} [{\chi ^{(1)}}]
\end{equation}
\begin{equation}
\tilde \epsilon  = {\epsilon _0}{(n + i\alpha c/2\omega )^2}
\end{equation}
 
注意区分 $\epsilon$ 
(permittivity),${\epsilon _r}$ (dielectric constant) 和 $\chi $ (susceptivility). 有时候 ${\epsilon _r}$. 不同的书符号可能不一样,以名称和语境为准.

