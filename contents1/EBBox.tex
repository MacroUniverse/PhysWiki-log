%盒中的电磁波

空间中一个电阻不计的金属盒中有电磁波.
金属盒的大小为 
\begin{equation}
\left\{ \begin{array}{l}
0 \le x \le a\\
0 \le y \le b\\
0 \le z \le c
\end{array} \right.
\end{equation} 
电场的波动方程%(链接未完成)
为
\begin{equation}
{\laplacian} \vdot \vec E = \frac{1}{{{c^2}}}\frac{{{\partial ^2}\vec E}}{{\partial {t^2}}}
\end{equation}  
矢量相等的充要条件是三个分量分别相等
\begin{equation}
{\laplacian}{E_x} = \frac{1}{{{c^2}}}\frac{{{\partial ^2}{E_x}}}{{\partial {t^2}}},
\qquad
{\laplacian}{E_y} = \frac{1}{{{c^2}}}\frac{{{\partial ^2}{E_x}}}{{\partial {t^2}}},
\qquad
{\laplacian}{E_z} = \frac{1}{{{c^2}}}\frac{{{\partial ^2}{E_x}}}{{\partial {t^2}}}
\end{equation}   
下面以 $E_x$ 为例, 用分离变量法得出通解.
先令 ${E_x} = X\left( x \right)Y\left( y \right)Z\left( z \right)T\left( t \right)$. 代入上式, 两边同除 $X\left( x \right)Y\left( y \right)Z\left( z \right)T\left( t \right)$ 得
\begin{equation}
{{\frac{{{d^2}X}}{{d{x^2}}}} \mathord{\left/
 {\vphantom {{\frac{{{\D ^2}X}}{{\D {x^2}}}} X}} \right.
 \kern-\nulldelimiterspace} X} + {{\frac{{{\D ^2}Y}}{{\D {y^2}}}} \mathord{\left/
 {\vphantom {{\frac{{{\D ^2}Y}}{{\D {y^2}}}} Y}} \right.
 \kern-\nulldelimiterspace} Y} + {{\frac{{{\D ^2}Z}}{{\D {z^2}}}} \mathord{\left/
 {\vphantom {{\frac{{{\D ^2}Z}}{{\D {z^2}}}} Z}} \right.
 \kern-\nulldelimiterspace} Z} = \frac{1}{{{c^2}}}{{\frac{{{\D ^2}T}}{{\D {t^2}}}} \mathord{\left/
 {\vphantom {{\frac{{{\D ^2}T}}{{\D {t^2}}}} T}} \right.
 \kern-\nulldelimiterspace} T}
\end{equation} 
由于上式每一项都是一个独立变量的函数, 所以每一项都等于一个常数. 令这些常数为
\begin{equation}\begin{aligned}
&{{\frac{{{\D ^2}X}}{{\D {x^2}}}} \mathord{\left/
 {\vphantom {{\frac{{{\D ^2}X}}{{\D {x^2}}}} X}} \right.
 \kern-\nulldelimiterspace} X} =  - k_x^2 \qquad {{\frac{{{\D ^2}Y}}{{\D {y^2}}}} \mathord{\left/
 {\vphantom {{\frac{{{\D ^2}Y}}{{\D {y^2}}}} Y}} \right.
 \kern-\nulldelimiterspace} Y} =  - k_y^2\\
 &{{\frac{{{\D ^2}Z}}{{\D {z^2}}}} \mathord{\left/
 {\vphantom {{\frac{{{\D ^2}Z}}{{\D {z^2}}}} Z}} \right.
 \kern-\nulldelimiterspace} Z} =  - k_z^2 \qquad \frac{1}{{{c^2}}}{{\frac{{{\D ^2}T}}{{\D {t^2}}}} \mathord{\left/
 {\vphantom {{\frac{{{\D ^2}T}}{{\D {t^2}}}} T}} \right.
 \kern-\nulldelimiterspace} T} =  - {\omega ^2}
\end{aligned}\end{equation} 
(取负号是因为我们只对三角函数解感兴趣, 指数函数解在这里无关)代入上式, 这些常数满足
\begin{equation}
k_x^2 + k_y^2 + k_z^2 = {\omega ^2}
\end{equation} 
上面三式的通解是
\begin{equation}
\left\{ \begin{array}{l}
X = {C_1}\cos \left( {{k_x}x} \right) + {C_2}\sin \left( {{k_x}x} \right)\\
Y = {C_3}\cos \left( {{k_y}y} \right) + {C_4}\sin \left( {{k_y}y} \right)\\
Z = {C_5}\cos \left( {{k_z}z} \right) + {C_6}\sin \left( {{k_z}z} \right)
\end{array} \right.
\end{equation} 
时间函数的解取 $T = C\cos \left( {\omega t} \right)$ (时间函数的相位不重要)
由理想导体的电磁场边界条件%(未完成, 注意包含以下第二条)
\begin{equation}
{E_{//}} = 0
\qquad
 \frac{{\partial {E_ \bot }}}{{\partial n}} = 0
\end{equation}  
 ${{\partial {E_x}}}/{{\partial x}} = 0$ ( $x \to a$ 时);  ${E_x} = 0$ ($y \to b$ 或$z \to c$ 时). 把上面的通解带入条件, 得
\begin{equation}
X = {C_1}\cos \left( {\frac{{{n_x}\pi }}{a}x} \right)
\qquad
Y = {C_4}\sin \left( {\frac{{{n_y}\pi }}{b}y} \right)
\qquad
Z = {C_6}\sin \left( {\frac{{{n_z}\pi }}{c}z} \right)
\end{equation}  
三个变量相乘, 令 ${C_1}{C_4}{C_6} = {E_{x0}}$,  得
\begin{equation}
{E_x} = {E_{x0}}\cos \left( {\frac{{{n_x}\pi }}{a}x} \right)\sin \left( {\frac{{{n_y}\pi }}{b}y} \right)\sin \left( {\frac{{{n_z}\pi }}{c}z} \right)
\end{equation} 
同理对 ${E_y}$,  ${E_z}$ 分析, 得到电场的三个分量在盒内的分布
\begin{equation}
\left\{ \begin{aligned}
{E_x} = {E_{x0}}\cos \left( {\frac{{{n_x}\pi }}{a}x} \right)\sin \left( {\frac{{{n_y}\pi }}{b}y} \right)\sin \left( {\frac{{{n_z}\pi }}{c}z} \right)\\
{E_{\rm{y}}} = {E_{y0}}\sin \left( {\frac{{{n_x}\pi }}{a}x} \right)\cos \left( {\frac{{{n_y}\pi }}{b}y} \right)\sin \left( {\frac{{{n_z}\pi }}{c}z} \right)\\
{E_z} = {E_{z0}}\sin \left( {\frac{{{n_x}\pi }}{a}x} \right)\sin \left( {\frac{{{n_y}\pi }}{b}y} \right)\cos \left( {\frac{{{n_z}\pi }}{c}z} \right)
\end{aligned} \right.
\end{equation} 
\begin{equation}
T = C\cos \left( {\omega t} \right)
\end{equation}
且满足 $\omega  = \pi \sqrt {{{n_x^2}/}{{{a^2}}} + {{n_y^2}}/{{{b^2}}} + {{n_z^2}}/{{{c^2}}}} $
特殊地, 当盒子是立方体的时候, $a = b = c = L$ 时,$\omega  = {\pi }\sqrt {n_x^2 + n_y^2 + n_z^2}/L $.   


%(未完成)
%貌似等一下还会有一个限制条件的(电场散度为零! 磁场散度也为零! 还有磁场的边界条件!),所以电场三个分量的振动幅度并不是独立的, 要求 ${E_x}{n_x} + {E_y}{n_y} + {E_z}{n_z}$, 可能还有更多要求!

%为什么世界如此复杂啊!

