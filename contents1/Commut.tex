%算符对易与共同本征函数
%50min
\pentry{厄米矩阵,%未完成链接
本征函数的简并}%未完成, 要引入希尔伯特子空间的概念, 说明子空间中的任意一个函数都是本征函数, n重简并的子空间是n维的, 即可以有n个线性无关的本征函数张成. 一般选取两两正交的波函数作为基底, 基底有无穷多种选法, 任何基底经过任意幺正变换以后仍然是子空间的基底. 类比一下
\subsection{命题}
以下两个条件互为充分必要条件
\begin{enumerate}
  \item 两个厄米算符 $\Q A$ 和 $\Q B$ 互相对易.
  \item 算符 $\Q A$ 和 $\Q B$ 的本征方程存在一整套共同的本征函数 ${\psi _i}$.
\end{enumerate}
\subsection{证明条件 $2 \to 1$}
设算符 $\Q A$ 和 $\Q B$ 有一组共同的本征函数 ${\psi _i}$,  则它们同时满足 $\Q A$ 和 $\Q B$ 的本征方程
\begin{equation}
  \left\{ \begin{aligned}
\Q A{\psi _i} = {a_i}{\psi _i}\\
\Q B{\psi _i} = {b_i}{\psi _i}
\end{aligned} \right.
\end{equation}
对任何 ${\psi _i}$,  都有
\begin{equation}
  \Q A\left( {\Q B{\psi _i}} \right) = \Q A\left( {{b_i}{\psi _i}} \right) = {b_i}\Q A{\psi _i} = {a_i}{b_i}{\psi _i}
\end{equation}
\begin{equation}
  \Q B\left( {\Q A{\psi _i}} \right) = \Q B\left( {{a_i}{\psi _i}} \right) = {a_i}\Q B{\psi _i} = {a_i}{b_i}{\psi _i}
\end{equation}
所以 $\Q A\Q B{\psi _i} = \Q B\Q A{\psi _i}$ 即
\begin{equation}
[\Q A,\Q B] = \Q A\Q B - \Q B\Q A = 0
\end{equation}
即两算符对易.\\
证毕.
\subsection{证明条件 $1 \to 2$}
要证明 $1 \to 2$,  只需证明 $\Q A$ 的一套本征函数都满足 $\Q B$ 的本征方程即可.
\begin{description}
  \item[(1)] 算符 $\Q A$ 非简并情况( $\Q B$ 是否简并没关系)\\
  先解出算符 $\Q A$ 的本征方程 $\Q A{\psi _i} = {a_i}{\psi _i}$,  如果 $\Q A$ 算符不发生简并(见本征函数的简并%未完成链接
  )那么本征值各不相同, 且给定一个本征值 $a_i$ 其解只可能是 ${\psi _i}$ 或者 ${\psi _i}$ 乘以一个任意复常数(注释:其实也可以再相乘一个算符 $\Q A$ 不涉及的物理量的函数, 例如总能量算符 $\Q H$ 的本征函数还可以再成一个时间因子 ${e^{i\omega t}}$ ).\\
  因为算符对易, 有
  \begin{equation}
    \Q A\left( {\Q B{\psi _i}} \right) = \Q B\left( {\Q A{\psi _i}} \right) = {a_i}\left( {\Q B{\psi _i}} \right)
  \end{equation}
  把式中的 $\Q B{\psi _i}$ 看成一个新的波函数, 上式说明 $\Q B{\psi _i}$ 是算符 $\Q A$ 和本征值 $a_i$ 的另一个本征函数.根据以上分析,  $\Q B{\psi _i}$ 必定是 ${\psi _i}$ 乘以某个复常数(命名为 $b_i$ ), 即
  \begin{equation}
    \Q B{\psi _i} = {b_i}{\psi _i}
  \end{equation}
  而这正是 $\Q B$ 的本征方程(而 $\Q B$ 也是厄米矩阵, 所以作为本征值 $b_i$ 的数域从复数缩小到实数).\\
  证毕.
  \item[(2)] 算符 $\Q A$ 简并情况\\
  假设算符 $\Q A$ 的所有本征值为 $a_i$ (各不相同), 任意一个 $a_i$ 有 $n_i$ 重简并.\\
若 ${n_i} = 1$,  对应唯一一个 ${\psi _i}$,  那么根据上文对非简并情况的推理, ${\psi _i}$ 就已经是 $\Q B$ 的本征函数了.\\
若 ${n_i} > 1$,  存在一个 $n_i$ 维希尔伯特子空间, 里面任何一个函数都是 $a_i$ 对应的本征函数, 所以要在子空间中寻找共同本征函数, 只需在子空间中寻找 $\Q B$ 的本征函数即可. 令 ${\phi _i}$ 为本征值为 $a_i$ 的子空间中的任意函数, 利用对易关系
\begin{equation}
  \Q A\left( {\Q B{\phi _i}} \right) = \Q B\left( {\Q A{\phi _i}} \right) = {a_i}\left( {\Q B{\phi _i}} \right)
\end{equation}
这条式子说明 $\Q B{\phi _i}$ 是 $\Q A$ 和 $a_i$ 的一个本征函数, 即 $\Q B{\phi _i}$ 仍然在 $a_i$ 的简并子空间中.所以 $\Q B$ 对子空间来说是一个闭合的厄米算符, 所以必有 $N$ 个线性无关的本征函数.\\%(厄米算符性质 $x$,  未完成)
证毕.
\end{description}
以下的内容应该归到厄米算符里面讲(厄米算符在希尔伯特空间中是无穷维的矩阵, 但是如果一个厄米算符在一个子空间中闭合,那么就可以通过以下方法找到N个线性无关的本征函数.%厄米算符在
先在空间中任意选取 $n_i$ 个线性无关的正交本征函数 ${\psi _{i1}}, {\psi _{i2}}\dots{\psi _{i{n_i}}}$ 作为子空间的基底(本征函数的简并%未完成链接
)),并可以用基底 ${\psi _{i1}},{\psi _{i2}}\dots{\psi _{i{n_i}}}$ 展开.\\
令 $\Q B{\psi _{ij}} = \sum\limits_{k = 1}^{{n_i}} {{W_{jk}}{\psi _{ik}}} $ ( ${W_{jk}} = \left\langle {{\psi _{ij}}} \right|\Q B\left| {{\psi _{ik}}} \right\rangle $, 可以是复数), 则 $\Q B$ 在该子空间可以表示成一个 $n_i$ 维的方形矩阵(记为 $W$ ).\\
以 ${\psi _{i1}}, {\psi _{i2}}\dots{\psi _{i{n_i}}}$ 为子空间的基底, 子空间内任意函数 $\phi  = {x_1}{\psi _{i1}} + {x_2}{\psi _{i2}}\dots$ 可以记为 $\left| \phi  \right\rangle  = \left( \begin{aligned}
{x_1}\\
{x_2}\\
{\kern 1pt} {\kern 1pt} {\kern 1pt}  \vdots \\
{x_{{n_i}}}
\end{aligned} \right)$. 根据算符的矩阵表示%链接未完成
, $\Q B$ 在子空间的矩阵元就是系数 ${W_{jk}}$, 
\begin{equation}
  W = \left( {\begin{aligned}
{{W_{11}}}& \ldots &{{W_{1{n_i}}}}\\
 \vdots & \ddots & \vdots \\
{{W_{{n_i}1}}}& \ldots &{{W_{{n_i}{n_i}}}}
\end{aligned}} \right)
\end{equation}
所以 $\Q B$ 在子空间范围内的本征方程的矩阵形式就是
\begin{equation}
  W\left| {{\phi _k}} \right\rangle  = {b_{ik}}\left| {{\phi _k}} \right\rangle 
\end{equation}
所以 $\Q B$ 在子空间的本征值就是 $W$ 的本征值, 本征函数就是 $W$ 的本征矢对应的波函数.\\
最后要证明的就是 $W$ 矩阵必然存在 $n_i$ 个本征矢.由于 $\Q B$ 是厄米算符,  $W$ 必然是厄米矩阵, 而 $n_i$ 维的厄米矩阵必然存在 $n_i$ 个两两正交的复数本征矢和实数本征值(厄米接矩阵%链接未完成
).\\
综上所述, 对每一个 $n_i$ 重简并的 $a_i$,  都存在 $n_i$ 个两两正交的本征函数作为 $\Q A$,  $\Q B$ 算符的共同本征函数.\\
证毕.\\
