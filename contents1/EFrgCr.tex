% Exact Frog-Crab 算法
这是我原创的一个算法, 如果已知 $E_{IR}(t)$ 和 Frog-Crab trace, 就可以精确解出
\begin{equation}
a(\vec v, \tau) = -\I\int_{-\infty}^{\infty} \dd{t} \E^{\I\phi(\vec v, t)} \vec d_{\vec v(t)} \vec E_{XUV}(t-\tau) \E^{\I(W+I_p)t}
\end{equation}
\begin{equation}
\phi(\vec v, t) = -\int_t^{+\infty} \dd{t'} [\vec v\vdot \vec A(t') + \vec A^2(t')/2]
\end{equation}
而不需要任何近似.

首先我们把问题写为更简单的数学形式
\begin{equation}\label{EFrgCr_eq3}
S_{i, j} = \int_{-L_t/2}^{L_t/2} \dd{t} f(t) g_i(t+\tau_j)\E^{\I\omega_i t}
\end{equation}
其中 $i$ 取 $N_E$ 个值, $L_t$ 必须要比 $f(t)$ 的范围大, $L_t$ 决定 trace 中必要的 $\Delta E$, 即\footnote{但 $\omega_i = E_i + I_p$ 却不必是 $\Delta \omega$ 的整数倍.} $\Delta \omega$. 如果已知 $E_{IR}(t)$, 就知道了 $g_i(t)$. 所以上式就是一个类似傅里叶变换的变换, 同样是丢失了变换后的 phase 信息.

如果能找到反变换, 那我们的思路就和 PCGPA 差不多, 甚至更简单. 只需要用一个 $f(t)$ 的 guess, 代入公式得到含相位的 $a$, 然后把模长替换成正确的模长, 反变换得到 $f(t)$, 然后再变换得到 $a$, 如此循环即可.

现在假设 $g_i(t + \tau_j)$ 在 $[-L_t/2, L_t/2]$ 上可以近似展开为傅里叶级数
\begin{equation}\label{EFrgCr_eq4}
g_i(t + \tau_j) = \sum_{k=-\infty}^{\infty} c_{ijk} \E^{\I \Delta \omega kt}
\end{equation}
这样\autoref{EFrgCr_eq3}就变为
\begin{equation}\label{EFrgCr_eq5}
S_{i, j} = \sum_k c_{ijk} \int_{-L_t/2}^{L_t/2} \dd{t} f(t) \E^{\I \omega_{i+k} t}
= \sum_k c_{ijk} F_{i+k} = \sum_l c_{i,j,l-i} F_l
\end{equation}
其中 $F$ 是 $f$ 的反傅里叶变换, $F_i$ 对应 $\omega_i$, 所以\autoref{EFrgCr_eq4} 中的 $k$ 也最多只需要 $2N_E$ 个值. 所以 注意 trace 的带宽只需比 $f(t)$ 的大即可.

\autoref{EFrgCr_eq5} 是 $F$ 的 $N_E N_\tau$ 条超定方程, 解超定方程用最小二乘法% 链接未完成
.
\begin{equation}
\vec S = \mat C \vec F
\end{equation}
但事实上我们只知道矢量 $\vec S$ 的每个矩阵元的模长, 我们同样可以用所谓的模长替换法迭代得到正确的 $\vec F$, 然后做一个傅里叶变换再取实部得 $f(t)$.

\subsection{Phase Retrieval}
与真正的 Frog-Crab 一样, 这种幼稚的方法收敛得非常慢。 这类只知道模长的线性方程组已经有相当多成熟的算法。 GitHub 上找到一个包叫做 PhasePack 里面有很多算法而且统一接口。 目前测试了 10 来种算法, 其中最好用的是 Amplitude Flow, Wirt Flow 也不错, 但是貌似初始 guess 不如用上面的模长替代法循环 100 次左右, 用自带的初始 guess generator 反而不好。 收敛的时间从半分钟到几分钟不等, 有时候甚至会停止收敛。
