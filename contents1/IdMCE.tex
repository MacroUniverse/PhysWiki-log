%理想气体(巨正则系综法)

\subsection{理想气体的巨配分函数}
\noindent\bb{结论}
\begin{equation}\label{IdMCE_eq1}
\Xi  = \exp \left( {z{Q_1}} \right) = \exp \left( {\frac{zV}{\lambda ^3}} \right)
\end{equation}
\bb{推导}
\begin{equation}\label{IdMCE_eq2}
\begin{aligned}
\Xi & = \sum\limits_{N = 0}^\infty  {\sum\limits_{i = 1}^\infty  {\E^{\left( {N\mu  - {E_i}} \right)\beta }} }  = \sum\limits_{N = 0}^\infty  {{z^N}Q} \\
& = \sum\limits_{N = 0}^\infty  {{z^N}\frac{1}{N!}Q_1^N} \\
& = \sum\limits_{N = 0}^\infty  {\frac{1}{N!}{{\left( {z{Q_1}} \right)}^N}} \\
& = \exp \left( {z{Q_1}} \right) \\
& = \exp \left( {\frac{zV}{\lambda ^3}} \right){\kern 1pt} {\kern 1pt} {\kern 1pt} {\kern 1pt} {\kern 1pt} {\kern 1pt} {\kern 1pt}
\end{aligned}
\end{equation}
其中用到了指数函数的泰勒展开%未完成链接
\begin{equation}\label{IdMCE_eq3}
\exp \left( x \right) = 1 + x + \frac{1}{2!}{x^2} + \frac{1}{3!}{x^3}... = \sum\limits_{N = 0}^\infty  {\frac{1}{N!}{x^N}}
\end{equation}

\subsection{状态方程推导}
首先求出理想气体的巨势
\begin{equation}\label{IdMCE_eq4}
\Phi  =  - kT\ln \Xi  =  - kT\frac{zV}{\lambda ^3}
\end{equation}
由巨正则系综法%未完成链接
\begin{equation}\label{IdMCE_eq5}
\D \Phi  =  - P \D V - S \D T - N \D \mu
\end{equation}
\begin{equation}\label{IdMCE_eq6}
P =  - {\left( {\pdv{\Phi}{V}} \right)_{T,\mu }} = kT\frac{z}{\lambda ^3}
\end{equation}
注意 $z$ 是 $\mu $ 和 $T$ 的函数( $z = {\E^{{\mu }/{{kT}}}}$ ), $\lambda $ 是 $T$ 的函数, 所以 $z$ 和 $\lambda $ 在该偏微分中都看做常数.
\begin{equation}\label{IdMCE_eq7}
N =  - {\left( {\pdv{\Phi}{\mu}} \right)_{V,T}} = kT\frac{V}{\lambda ^3}{\left( {\pdv{z}{\mu}} \right)_{V,T}} = \frac{Vz}{\lambda ^3} 
\qquad
( = z{Q_1})
\end{equation}
若用上面两式消去 ${z}/{{{\lambda ^3}}}$ 因子, 得到理想气体状态方程 $PV = NkT$ .
  
另外, 想象在巨正则系综的物理情景中, 变化 $T$ 和 $\mu $,  从而使\autoref{IdMCE_eq7} 中的粒子数保持不变, 则 $N$ 不变时 可以看成 $T$ 的函数(而这个函数应该与正则系宗所得到的一样).由\autoref{IdMCE_eq7} 得
\begin{equation}\label{IdMCE_eq8}
\mu  = kT\ln \frac{N{\lambda ^3}}{V}
\end{equation}
再测试一下状态方程 $PV =  - \Phi $,  得到 $PV = kT{{zV}}/{{{\lambda ^3}}}$,  这与上面的压强公式(编号)重复, 没有新的信息.\\
若把粒子数公式 $N = {{Vz}}/{{{\lambda ^3}}}$ (编号)代入理想气体的巨配分函数 $\Xi  = \exp \left( {{{zV}}/{{{\lambda ^3}}}} \right)$(编号)以及巨势 $\Phi  =  - kT{{zV}}/{{{\lambda ^3}}}$(编号), 得到两个个相当简洁的表达式, 可以方便记忆
\begin{equation}\label{IdMCE_eq9}
\Xi  = \exp \left( N \right)
\end{equation}
\begin{equation}\label{IdMCE_eq10}
\Phi  =  - NkT
\end{equation}
理想气体的熵为
\begin{equation}\label{IdMCE_eq11}
\begin{aligned}
S &=  - {\left( {\pdv{\Phi}{T}} \right)_{V,\mu }} \\
& = Vk\frac{T}{\lambda ^3}\pdv{z}{T} + kTz\pdv{T} \left( {\frac{T}{\lambda ^3}} \right) \\
& = Vk\frac{T}{\lambda ^3}\left( { - \frac{\mu z}{k{T^2}}} \right) + kTz\pdv{T}\left( {\frac{{{(2\pi mk)}^{3/2}}{T^{5/2}}}{h^3}} \right)\\
& =  - \frac{\mu zV}{T{\lambda ^3}} + kTz\frac{5}{2}\frac{{(2\pi mkT)}^{3/2}}{h^3} \\
& =  - \frac{\mu zV}{T{\lambda ^3}} + \frac{5}{2}\frac{kTz}{\lambda ^3} \\
& = Nk\left( {\frac{5}{2} - \frac{\mu }{kT}} \right){\kern 1pt} {\kern 1pt} {\kern 1pt} {\kern 1pt} {\kern 1pt} {\kern 1pt} {\kern 1pt}
\end{aligned}
\end{equation}
这里得出的熵是 $\mu $ 和 $T$ 的函数(从巨正则系综的物理情景来看, 得出的所有结果都应该是预先设定的参数 $\mu $ 和 $T$ 的函数).\\
为了和巨正则系综比较, 把\autoref{IdMCE_eq8} 代入\autoref{IdMCE_eq11},  即把粒子数人为保持不变, 一切看成温度的函数. 果然得到了理想气体的熵(Sackur-Tetrode公式)
\begin{equation}\label{IdMCE_eq12}
S = Nk\left( {\ln \frac{V}{N{\lambda ^3}} + \frac{5}{2}} \right)
\end{equation}

\subsection{理解}

巨正则系综法的物理情景是: 让系统(体积 $V$ )与粒子源(化学势 $\mu $ )和热源(温度 $T$ ) 保持平热平衡, 由 $\mu $ 和 $T$ 决定粒子数 $N$,  压强 $P$,  能量 $E$ 等等. 这与微正则系综或正则系宗的物理情景不一样. 但是得到的结论却是一样的.

