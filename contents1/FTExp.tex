% 傅里叶变换(指数)

\pentry{傅里叶级数(指数)\upref{FSExp},傅里叶变换(三角)\upref{FTTri}}

\subsection{结论}
用三角傅里叶变换 \upref{FTTri} 中同样的方法可把指数傅里叶级数拓展为指数傅里叶变换
\begin{align}
g(k) &= \frac{1}{\sqrt{2\pi}} \int_{-\infty}^{+\infty} f(x)\E^{-\I kx} \D x \\
f(x) &= \frac{1}{\sqrt{2\pi}} \int_{-\infty}^{+\infty} g(k)\E^{\I kx} \D x
\end{align}
特殊地,当 $f(x)$ 为实函数时,$g(k)$ 的实部是偶函数,虚部是奇函数.

\subsection{实数函数的情况}

如果实函数 $f(x)$ 的复数傅里叶变换为 $g(k)$, 即
\begin{gather}
g(k) = \frac{1}{\sqrt{2\pi}} \int_{-\infty}^\infty f(x)\E^{-\I kx} \D x\\
f(x) = \frac{1}{\sqrt{2\pi}} \int_{-\infty}^\infty g(k)\E^{\I kx} \D k \label{FouR_eq1}
\end{gather}
$g(k)$ 需要满足什么条件才能使 $f(x)$ 是实数呢?我们从式 \eqref{FouR_eq1} 开始入手.
\begin{equation}\begin{aligned}
f(x) &= \frac{1}{\sqrt{2\pi}} \int_0^\infty [g(k)\E^{\I kx} + g(-k)\E^{-\I kx}] \D k \\
&= \frac{1}{\sqrt{2\pi}} \int_0^\infty [g(k)+g(-k)]\cos(kx)\D k + \frac{\I}{\sqrt{2\pi}} \int_0^\infty [g(k)-g(-k)]\sin(kx) \D k
\end{aligned}\end{equation}
从傅里叶变换(三角函数)% 链接未完成
我们已知对实数函数,方括号项都必须为实函数,即上式第一个方括号中的虚部为零,第二个方括号中的实部为零,即
\begin{equation}\begin{aligned}
g_{Im}(-k) &= -g_{Im}(k) \\
g_{Re}(-k) &= g_{Re}(k)
\end{aligned}\end{equation}
所以结论是,当 $f(x)$ 为实函数时,$g(k)$ 的实部是偶函数,虚部是奇函数.由此可得另一个结论
\begin{equation}
\abs{g(-k)}=\sqrt{g_{Re}^2(-k)+g_{Im}^2(-k)}=\sqrt{g_{Re}^2(k)+g_{Im}^2(k)} = \abs{g(k)}
\end{equation}
即频谱是偶函数.所以对于实数函数,我们只需要 $k>0$ 的频谱.这与三角傅里叶变换的情况一致.