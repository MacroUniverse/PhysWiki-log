%理想气体的状态密度(相空间)
%15min
\subsection{结论}
\begin{equation}
  {\Omega _0} = \frac{{{V^N}}}{{N!{h^3}}}\frac{{{{\left( {2\pi mE} \right)}^{3N/2}}}}{{\left( {3N/2} \right)!}}
\end{equation}
\begin{equation}
  g\left( E \right) = \frac{{{V^N}}}{{N!{h^3}}}\frac{{{{\left( {2\pi m} \right)}^{3N/2}}}}{{\left( {3N/2 - 1} \right)!}}{E^{3N/2 - 1}}
\end{equation}
\pentry{$N$ 维球体的体积}%未完成链接

在 $N$ 个不相干粒子的相空间中, 能量小于 $E$ 的体积为
\begin{equation}
{\Omega _0} = \frac{1}{{N!{h^3}}}\int\limits_{\sum {{p^2}}  \le 2mE} {{ \D ^{3N}}q \cdot { \D ^{3N}}p} = \frac{{{V^N}}}{{N!{h^3}}}\int\limits_{\sum {{p^2}}  \le 2mE} {{ \D ^{3N}}p}
\end{equation}
其中积分 $\int\limits_{\sum {{p^2}}  \le 2mE} {{ \D ^{3N}}p} $ 可以看做 $n=3N$ 维球体的体积, 半径为 $R = \sqrt {2mE} $. 

词条 $N$ 维球体的体积%未完成链接
中的结论为 
\begin{equation}
  {V_n} = \left\{ \begin{aligned}
\frac{{{R^n}}}{{\left( {n/2} \right)!}}{\pi ^{\left( {n - 1} \right)/2}}{\kern 1pt} ({\text{奇数}\rm{n}})\\
\frac{{{R^n}}}{{\left( {n/2} \right)!}}{\pi ^{n/2}}{\kern 1pt} {\kern 1pt} {\kern 1pt} {\kern 1pt} {\kern 1pt} {\kern 1pt} {\kern 1pt} {\kern 1pt} {\kern 1pt} {\kern 1pt} {\kern 1pt} {\kern 1pt} ({\text{偶数}\rm{n}})
\end{aligned} \right. \approx \frac{{n - 1}}{2} = \frac{n}{2}
\end{equation}
(由于系统中粒子数 $N$ 非常多, 可以近似认为 $({{n - 1}})/{2} = {n}/{2}$.  见热力学中的近似%未完成
). 代入 $n=3N$ 和 $R = \sqrt {2mE} $,   得 % 未完成
其中,
\begin{equation}
\int\limits_{\sum {{p^2}}  \le 2mE} {{ \D ^3}p}  = \frac{{{{\left( {2\pi mE} \right)}^{3N/2}}}}{{\left( {3N/2} \right)!}}
\end{equation}
$N$ 个不可区分粒子组成的理想气体, 能量小于 $E$ 的能级个数为
\begin{equation}
{\Omega _0} = \frac{{{V^N}}}{{N!{h^3}}}\frac{{{{\left( {2\pi mE} \right)}^{3N/2}}}}{{\left( {3N/2} \right)!}}
\end{equation}
对能量求导得到状态密度为
\begin{equation}
  g\left( E \right) = \frac{{ \D {\Omega _0}}}{{ \D E}} = \frac{{{V^N}}}{{N!{h^3}}}\frac{{{{\left( {2\pi m} \right)}^{3N/2}}}}{{\left( {3N/2 - 1} \right)!}}{E^{3N/2 - 1}}
\end{equation}
