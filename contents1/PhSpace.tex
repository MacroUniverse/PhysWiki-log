%相空间

%(未完成)
单个粒子的相空间

单个粒子(看做质点)的状态可以由3个位置坐标 $\left( {x,y,z} \right)$ 和三个动量坐标 $\left( {{p_x},{p_y},{p_z}} \right)$ 来描述, 为了便于拓展到一般情况, 我们用 ${q_1} \equiv x$,   ${q_2} \equiv y$, ${q_3} \equiv z$   和 ${p_1} \equiv {p_x}$,   ${p_2} \equiv {p_y}$, ${p_3} \equiv {p_z}$   表示. 想象一个由3个 ${q_i}$  坐标和3个 ${p_i}$  坐标组成的6维空间, 体积元定义为
 \begin{equation}
d{\Omega_1} = \frac{1}{h^3}d{q_1}d{q_2}d{q_3} \vdot d{p_1}d{p_2}d{p_3}
\end{equation} 
为了方便表示, 简写为  $d{\Omega_1} = {d^3}q \vdot {d^3}p$. 
  \begin{equation}
{\Omega_1} = \frac{1}{h^3}\int\limits_{\Omega_1} {{d^3}q \vdot {d^3}p} 
\end{equation} 
积分对所有可能的 ${q_i}$ 和 ${p_i}$ 进行. 例如粒子若被限制在一个长宽高分别为 ${L_x},{\kern 1pt} {L_y},{L_z}$ 的盒子里, 而动量没有限制, 那么上面积分变为
  \begin{equation}
{\Omega_1} = \frac{1}{h^3}\int_0^{L_x} {\int_0^{L_y} {\int_0^{L_z} {\int {d{q_1} \vdot d{q_2} \vdot d{q_3}} } }  \vdot {d^3}p}  = \frac{{L_x}{L_y}{L_z}}{h^3}\int {{d^3}p} 
\end{equation} 
经典力学中, 对于 $N$ 个粒子的系统, 可以用 $3N$ 个位置坐标和 $3N$ 个动量坐标来完全描述系统的状态. 则相空间为 $6N$ 维空间.
