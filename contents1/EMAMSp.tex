%电磁场角动量分解
%13min

电磁场的动量为
\begin{equation}
\vec p = {\epsilon _0}\int {\D V\;\vec E \cross \vec B}
\end{equation}
角动量为
\begin{equation}
\vec J = \vec r \cross \vec p = {\epsilon _0}\int {\D V\;\vec r \cross \vec E \cross (\curl \vec A)} 
\end{equation}
现在假设电磁场只在一定范围内不为零, 且体积分的边界处场强为零. 假设该范围内没有净电荷与电流, 则
\begin{equation}
\begin{aligned}
  \vec r \cross \vec E \cross (\curl \vec A) &= \vec r \cross {[\grad (\vec E \vdot \vec A) - \vec A(\vec E \vdot \vec\nabla )]_{\partial A}} \\
  &= {[(\vec r \cross \vec\nabla )(\vec E \vdot \vec A) - (\vec E \vdot \vec\nabla )(\vec r \cross \vec A)]_{\partial A}}
\end{aligned}
\end{equation}
其中转微分算符 ${[\,]_{\partial A}}$ 的作用是先把方括号内的 $\vec\nabla$ 作为普通矢量进行计算, 再把展开结果中每一项的偏微分作用在 $A$ 的分量上. 上式第一项为 $\sum\limits_i {{E_i}} (\vec r \cross \vec\nabla ){A_i}$, 第二项为
\begin{equation}
   - {[(\vec E \vdot \vec\nabla )(\vec r \cross \vec A)]_{\partial A}} =  - (\vec E \vdot \vec\nabla )(\vec r \cross \vec A) + {[(\vec E \vdot \vec\nabla )(\vec r \cross \vec A)]_{\partial r}}
\end{equation}
其中第二项为 ${[(\vec E \vdot \vec\nabla )(\vec r \cross \vec A)]_{\partial r}} = [(\vec E \vdot \vec\nabla )\vec r] \cross \vec A = \vec E \cross \vec A$, 第一项中
\begin{equation}
\begin{aligned}
  (\vec E \vdot \vec\nabla )(\vec r \cross \vec A) &= {[(\vec E \vdot \vec\nabla )(\vec r \cross \vec A)]_{\partial ErA}} - {[(\vec E \vdot \vec\nabla )(\vec r \cross \vec A)]_{\partial E}}\\
  & = {[(\vec E \vdot \vec\nabla )(\vec r \cross \vec A)]_{\partial ErA}}
  \end{aligned}
\end{equation}
这是因为 ${[(\vec E \vdot \vec\nabla )(\vec r \cross \vec A)]_{\partial E}} = (\div\vec E)(\vec r \cross \vec A) = 0$.  综上,
\begin{equation}
\vec J = {\epsilon _0}\int {\D V\sum\limits_i {{E_i}} (\vec r \cross \vec\nabla ){A_i} + {\epsilon _0}\int {\D V\;\vec E \cross \vec A} }  + {\epsilon _0}\int {\D V\;{{[(\vec E \vdot \vec\nabla )(\vec r \cross \vec A)]}_{\partial ErA}}}
\end{equation}
现在证明最后一项为0. 以 $x$ 分量为例,
\begin{equation}
\begin{aligned}
  \uvec x\int {\D V\;{{[(\vec E \vdot \vec\nabla )(\vec r \cross \vec A)]}_{\partial ErA}}}  &= \int {\D V\; \div [\vec E(\uvec x \vdot \vec r \cross \vec A) ]} \\
  &= \oint {\D \vec s\;\vec E(\uvec x \vdot \vec r \cross \vec A)}  = 0
\end{aligned}
\end{equation}
最后一步是因为边界处场强为零. 现在我们可以看出角动量由两部分组成
\begin{equation}
\vec J = \vec L + \vec S
\qquad
\vec L = {\epsilon _0}\int {\D V\sum\limits_i {{E_i}} (\vec r \cross \vec\nabla ){A_i}}
\qquad
\vec S = {\epsilon _0}\int {\D V\;\vec E \cross \vec A} 
\end{equation}
其中 $\vec L$ 是轨道角动量, $\vec S$ 是自旋角动量.
