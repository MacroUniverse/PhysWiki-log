%量子力学中的变分法

当平均能量是波函数的鞍点时, 波函数就是能量的本征态. 对一维单粒子
 \begin{equation}
E = \left\langle \psi  \right|\Q H\left| \psi  \right\rangle 
\end{equation}
但注意这里的波函数必须已经归一化. 由于变分法需要假设任意的增量函数 $\delta \psi $,  我们只好用一个不要求归一化的能量平均值公式
 \begin{equation}
E = \left\langle \psi  \right|\Q H\left| \psi  \right\rangle /\left\langle {\psi }
 \mathrel{\left | {\vphantom {\psi  \psi }}
 \right. \kern-\nulldelimiterspace}
 {\psi } \right\rangle 
\end{equation}
现在假设波函数增加 $\delta \psi $ 
 \begin{equation}
E\left\langle {\delta \psi }
 \mathrel{\left | {\vphantom {{\delta \psi } \psi }}
 \right. \kern-\nulldelimiterspace}
 {\psi } \right\rangle  + E\left\langle {\psi }
 \mathrel{\left | {\vphantom {\psi  {\delta \psi }}}
 \right. \kern-\nulldelimiterspace}
 {\delta \psi } \right\rangle  = \left\langle {\delta \psi } \right|\Q H\left| \psi  \right\rangle  + \left\langle \psi  \right|\Q H\left| {\delta \psi } \right\rangle 
\end{equation}
由于 $\delta \psi $ 是任意的, 我们也可以使用 $i\delta \psi $ 
 \begin{equation}
 - E\left\langle {\delta \psi }
 \mathrel{\left | {\vphantom {{\delta \psi } \psi }}
 \right. \kern-\nulldelimiterspace}
 {\psi } \right\rangle  + E\left\langle {\psi }
 \mathrel{\left | {\vphantom {\psi  {\delta \psi }}}
 \right. \kern-\nulldelimiterspace}
 {\delta \psi } \right\rangle  =  - \left\langle {\delta \psi } \right|\Q H\left| \psi  \right\rangle  + \left\langle \psi  \right|\Q H\left| {\delta \psi } \right\rangle 
\end{equation}
以上两式等效, 两式相减, 得 (相当于 ${\psi ^ * }$ 与 $\psi $ 是两个独立的变量函数)
 \begin{equation}
E\left\langle {\delta \psi }
 \mathrel{\left | {\vphantom {{\delta \psi } \psi }}
 \right. \kern-\nulldelimiterspace}
 {\psi } \right\rangle  = \left\langle {\delta \psi } \right|\Q H\left| \psi  \right\rangle 
\end{equation}
该式对任意微小函数增量 $\delta \psi $ 都要求成立. 现在如果令 $\delta \psi  = \delta (x)$,  我们得到薛定谔方程
 \begin{equation}
\Q H\left| \psi  \right\rangle  = E\left| \psi  \right\rangle 
\end{equation}
归一化条件下的变分法也可以由拉格朗日乘数法完成, 令
 \begin{equation}
L = \left\langle \psi  \right|\Q H\left| \psi  \right\rangle  - \lambda \left[ {\left\langle {\psi }
 \mathrel{\left | {\vphantom {\psi  \psi }}
 \right. \kern-\nulldelimiterspace}
 {\psi } \right\rangle  - 1} \right]
\end{equation}
类似以上过程, 同样有
 \begin{equation}
\left\langle {\delta \psi } \right|\Q H\left| \psi  \right\rangle  - \lambda \left\langle {\delta \psi }
 \mathrel{\left | {\vphantom {{\delta \psi } \psi }}
 \right. \kern-\nulldelimiterspace}
 {\psi } \right\rangle  = 0
\end{equation}
即
 \begin{equation}
\Q H\left| \psi  \right\rangle  = \lambda \left| \psi  \right\rangle 
\end{equation}
显然, 乘数 $\lambda $ 就是本征态能量.
