% 安培环路定理

在空间中选取一环路(称为\bb{安培环路}) 并定义一个正方向, 那么磁感应强度在该环路上的线积分等于穿过环路的总电流(电流的正方向由右手定则\upref{RHRul} 判断)乘以真空中的磁导率.
\begin{equation}\label{AmpLaw_eq1}
\oint \vec B \vdot \dd{\vec l} = \mu_0 I
\end{equation}

\begin{exam}{无限长直导线的磁场}
若导线的电流为 $I$, 在其周围作一个半径为 $r$ 的安培环路, 由对称性, 环路上任意一点的磁感应强度大小相同且沿正方向. 所以\autoref{AmpLaw_eq1} 等于
\begin{equation}
2\pi r B = \mu_0 I
\end{equation}
所以磁感应强度大小的分布为
\begin{equation}
B(r) = \frac{\mu_0}{2\pi} \frac Ir
\end{equation}
这与使用比奥萨伐尔定律(\autoref{BioSav_exe1}\upref{BioSav}) 得出的结论一致.
\end{exam}