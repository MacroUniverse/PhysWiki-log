%物理学常数定义
%25min

\subsection{国际单位的定义(SI Units)}

物理单位的一个重要性质就是可测量, 至少理论上可测量. 以下的数值除了有特殊说明, 都是精确值(无限位小数用省略号表示), 不存在误差.

\subsubsection{秒(s)的定义}
铯原子133基态的超精细能级之间的跃迁辐射的电磁波周期的9,192,631,770倍. 
说明: 我们知道原子中的电子具有不同的能级, 当电子从一个能级跃迁到一个更低的能级时, 会放出一个光子. 光子的频率为 $\nu  = \varepsilon /h$,   其中 $\varepsilon $ 是光子的能量, $h$ 为普朗克常数.

\subsubsection{米(m)的定义}
真空中, 光在 $1/299792458$ 秒内传播的距离
说明: 由于真空中的光速是物质和信息能传播的最快速度, 且在任何参考系中都相同, 所以可以作为一个精确的标准. 结合秒的定义, 就可以通过实验得到一米的长度.

\subsubsection{光速(c)的定义}
 \begin{equation}
c = 299792458{m \mathord{\left/
 {\vphantom {m s}} \right.
 \kern-\nulldelimiterspace} s}
\end{equation} 
说明: 根据米的定义, 一秒中光可以在真空中传播 $299792459  m$.  

\subsubsection{千克(kg)的定义}
等于国际公斤原器的质量
说明: 千克是现有的唯一一个由特定的物品所定义单位. 国际公斤原器是国际计量大会制造的, 并复制若干份分别存放, 但经过长时间后被发现和复制品存在细微误差. 国际计量大会最终在2014年决定原则上千克应该由普朗克常数所决定, 但是最终的定义再次被推迟.

\subsubsection{牛顿(N)的定义}
等于使 $1kg$ 物体获得 $1{m \mathord{\left/
 {\vphantom {m {{s^2}}}} \right.
 \kern-\nulldelimiterspace} {{s^2}}}$ 加速度的力.

\subsubsection{真空磁导率( ${\mu _0}$ )的定义}
  \begin{equation}
{\mu _0} = 4\pi  \times {10^{ - 7}}{N \mathord{\left/
 {\vphantom {N {{A^2}}}} \right.
 \kern-\nulldelimiterspace} {{A^2}}}
\end{equation}

\subsubsection{真空介电常数( ${\epsilon _0}$ )的定义}
   \begin{equation}
{\epsilon _0} = 1/({c^2}{\mu _0}) = 8.8541878176... \times {10^{ - 12}}{F \mathord{\left/
 {\vphantom {F m}} \right.
 \kern-\nulldelimiterspace} m}
\end{equation}
说明: 其中 $c$ 是光速, ${\mu _0}$ 是真空磁导率. 根据该定义,$c = 1/\sqrt {{\epsilon _0}{\mu _0}} $.  

\subsubsection{库伦(C)的定义}
若真空中两个相同的点电荷相距一米, 产生的相互作用力为 ${1}/{{4\pi {\epsilon _0}}}$,   则该点电荷为1库伦.

\subsubsection{安培(A)的定义}
以下两种定义等效:
\begin{enumerate}
\item 每秒钟经过横截面的电荷量为 1 库伦的电流就是1安培.
\item 两根相距一米的无限长平行细导线流入 1 安培电流后, 相互作用力是 $2 \times {10^{ - 7}}N/m$. 
\end{enumerate}



