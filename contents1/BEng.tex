% 磁场的能量
%未完成 不要写那么深! 先从两片面电流开始推导! (包括介质)

\pentry{磁场矢势}%未完成

\subsection{结论} 
\begin{equation}\label{BEng_eq1}
W = \frac{1}{2\mu_0} \int \vec B^2 \dd{V}
\qquad 
\text{或}
\qquad
W = \frac12 \int \vec A \vdot \vec J \dd{V}
\end{equation} 
其中 $\mu_0$ 是真空中的磁导率, $\vec A$ 是磁场矢势, $\vec J$ 是电流密度,积分是对全空间积分(或者对被积函数不为零的空间积分).
\subsection{幼稚的推导}
首先我们根据能量守恒的思想,假设给一个电感 $L$ 充电的能量都以“磁场能”的形式储存起来,且能量密度只是磁场的函数.

在无限长圆柱螺线管中 % 未完成

\subsection{简单的推导}
 我们首先考虑一个单匝线圈的磁场能量%(图未完成, 画单股线圈, 画出大概的磁场分布, 画出两根相邻的电缆外接电源). 
 假设线圈的电阻为零

假设 $t = 0$ 时 $I = 0$, 此时没有磁场,磁场能量为零.接下来 $I$ 随着 $t$ 慢慢增加.反向电动势为(定义与电流相同的方向为正)
\begin{equation}\label{BEng_eq2}
\varepsilon  =  - \dv{\Phi}{t} =  - \dv{t} \int \vec B \vdot \dd{\vec s} 
=  - \dv{t} \int \qty(\curl \vec A) \vdot \dd{\vec s} 
=  - \dv{t} \oint {\vec A \vdot \D\vec l}
\end{equation}
电源克服反电动势的功率为
\begin{equation}\label{BEng_eq3}
\dv{W}{t} =  - \varepsilon I = I\dv{t} \int \vec B \vdot \dd{\vec s} = \int I \dv{\vec B}{t} \vdot \dd{\vec s}
\end{equation}
由于磁场与电流成正比(见比奥萨伐尔定律%未完成:链接
,不妨设 $\vec B = \vec bI$ .则
\begin{equation}\label{BEng_eq4}
I \dv{\vec B}{t} = \vec bI \dv{I}{t} = \frac{\vec b}{2} \dv{I^2}{t} = \dv{t} (\frac12 I\vec B)
\end{equation}
所以
\begin{equation}\label{BEng_eq5}
\dv{W}{t} = \dv{t} \qty(\frac12 \int I\vec B \vdot \D\vec s)
\end{equation}
注意两边都是对时间的导数.两边对时间积分,得
\begin{equation}\label{BEng_eq6}
W = \frac12 \int I\vec B \vdot \dd{\vec s}
\end{equation}
注意当 $I = 0$ 时 $ W = 0$, 所以积分常数为零.注意在上述过程中,并没有假设电流以什么样的函数随时间变化(只要是缓慢变化即可).
\begin{equation}\label{BEng_eq7}
W = \frac{I}{2} \int \vec B \vdot \dd{\vec s}  = \frac{I}{2} \int \curl \vec A \vdot \dd{\vec s}  = \frac12 \oint I\vec A \vdot \dd{\vec l}
\end{equation}
这就是%未完成:编号
式

 
 
 
 
 
 
 
 
 