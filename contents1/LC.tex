% LC振荡电路
% 15 min
%未完成

电容公式
\begin{equation}\label{LC_eq1}
I = C\frac{{\D {U_c}}}{{\D t}}
\end{equation}
电感公式
\begin{equation}\label{LC_eq2}
{U_L} = L\frac{{\D I}}{{\D t}}
\end{equation}
由回路电压合为零,得
\begin{equation}\label{LC_eq3}
{U_L} + {U_c} = 0
\end{equation}
对\autoref{LC_eq1} 求导得 ${\D I}/{\D t} = C{{{\D^2}{U_c}}}/{{\D{t^2}}}$, 代入\autoref{LC_eq2} 得 ${U_L} = LC {{{\D^2}{U_c}}}/{{\D{t^2}}}$. 代入  \autoref{LC_eq3}  得关于 ${U_c}$ 的微分方程
\begin{equation}
\frac{{{\D^2}{U_c}}}{{\D{t^2}}} + \frac{1}{{LC}}{U_c} = 0
\end{equation}
根据二阶线性常系数齐次微分方程\upref{Ode2}的第3种情况,其通解为
\begin{equation}
{U_c} = {U_0}\sin \left( {\omega {\kern 1pt} {\kern 1pt} t + {\phi_0}} \right)
\qquad
\omega  = \frac{1}{{\sqrt {LC} }}
\end{equation}
\begin{equation}
I = C\frac{{d{U_c}}}{{dt}} = \sqrt {\frac{C}{L}} {U_0}\cos \left( {\omega {\kern 1pt} {\kern 1pt} t + {\phi_0}} \right)
\end{equation}
 
