%氢原子的波函数
%10min
\pentry{波函数简介}%未完成链接
氢原子的波函数在球坐标中表示为
\begin{equation}
  {\psi _{nlm}}\left( {r,\theta ,\phi } \right) = {R_{nl}}\left( r \right)Y_l^n\left( {\theta ,\phi } \right)
\end{equation}
其中 $n$ 是\textbf{主量子数}, $l$ 是\textbf{角量子数}, $m$ 是\textbf{磁量子数}. ${R_{nl}}$ 是归一化的\textbf{径向波函数}, $Y_l^m$ 是归一化的\textbf{球谐函数}.
\begin{enumerate}
  \item \textbf{径向波函数} ${R_{nl}}(r)$ 
  \begin{equation}
    R\left( r \right) = \sqrt {{{\left( {\frac{2}{{na}}} \right)}^3}\frac{{\left( {n - l - 1} \right)!}}{{2n{{\left[ {\left( {n + l} \right)!} \right]}^3}}}} {e^{ - r/na}}{\left( {\frac{{2r}}{{na}}} \right)^l}\left[ {L_{n - l - 1}^{2l + 1}\left( {2r/na} \right)} \right]
  \end{equation}
  \begin{equation}
    n = 1 
    \qquad
    {R_{10}} = 2{a^{ - 3/2}}\exp \left( { - r/a} \right)
  \end{equation}
  \begin{equation}
    n = 2
    \qquad
    \left\{ \begin{aligned}
{R_{20}} &= \frac{1}{{\sqrt 2 }}{a^{ - 3/2}}\left( {1 - \frac{1}{2}\frac{r}{a}} \right)\exp \left( { - r/2a} \right)\\
{R_{21}} &= \frac{1}{{\sqrt {24} }}{a^{ - 3/2}}\frac{r}{a}\exp \left( { - r/2a} \right)
\end{aligned} \right.
  \end{equation}
  \begin{equation}
    n = 3
    \qquad
    \left\{ \begin{aligned}
{R_{30}} &= \frac{2}{{\sqrt {27} }}{a^{ - 3/2}}\left( {1 - \frac{2}{3}\frac{r}{a} + \frac{2}{{27}}{{\left( {\frac{r}{a}} \right)}^2}} \right)\exp \left( { - r/3a} \right)\\
{R_{31}} &= \frac{8}{{27\sqrt 6 }}{a^{ - 3/2}}\left( {1 - \frac{1}{6}\frac{r}{a}} \right)\left( {\frac{r}{a}} \right)\exp \left( { - r/3a} \right)\\
{R_{32}} &= \frac{4}{{81\sqrt {30} }}{a^{ - 3/2}}{\left( {\frac{r}{a}} \right)^2}\exp \left( { - r/3a} \right)
\end{aligned} \right.
  \end{equation}\\
  \item \textbf{球谐函数} $Y_l^m$ 
  \begin{equation}
    Y_l^m\left( {\theta ,\phi } \right) = \varepsilon \sqrt {\frac{{\left( {2l + 1} \right)}}{{4\pi }}\frac{{\left( {l - \left| m \right|} \right)!}}{{\left( {l + \left| m \right|} \right)!}}} {e^{im{\kern 1pt} {\kern 1pt} \phi }}P_l^m\left( {\cos \theta } \right)
  \end{equation}
  \begin{equation}
    l = 0
    \qquad
    Y_0^0 = {\left( {\frac{1}{{4\pi }}} \right)^{1/2}}
  \end{equation}
  \begin{equation}
    l = 1
    \qquad
   \left\{ \begin{aligned}
Y_1^0 &= {\left( {\frac{3}{{4\pi }}} \right)^{1/2}}\cos \theta \\
Y_1^{ \pm 1} &=  \mp {\left( {\frac{3}{{8\pi }}} \right)^{1/2}}\sin \theta  \vdot {e^{ \pm i\phi }}
\end{aligned} \right.
  \end{equation}
  \begin{equation}
    l = 2
    \qquad
    \left\{ \begin{aligned}
Y_2^0 &= {\left( {\frac{5}{{16\pi }}} \right)^{1/2}}\left( {3{{\cos }^2}\theta  - 1} \right)\\
Y_2^{ \pm 1} &=  \mp {\left( {\frac{{15}}{{8\pi }}} \right)^{1/2}}\sin \theta \cos \theta  \vdot {e^{ \pm i\phi }}\\
Y_2^{ \pm 2} &= {\left( {\frac{{15}}{{32\pi }}} \right)^{1/2}}{\sin ^2}\theta  \vdot {e^{ \pm 2i\phi }}
\end{aligned} \right.
  \end{equation}
\end{enumerate}
