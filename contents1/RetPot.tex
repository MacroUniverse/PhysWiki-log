%非齐次亥姆霍兹方程 推迟势

波动方程为
\begin{equation}\label{RetPot_eq1}
\laplacian\Psi  - \frac{1}{c^2} \pdv[2]{\Psi}{t} = f(\vec r, t)
\end{equation} 
用格林函数法求解,令格林函数满足
\begin{equation}\label{RetPot_eq2}
\laplacian G - \frac{1}{c^2} \pdv[2]{G}{t} =  - \delta (\vec r - \vec r')\delta (t - t')
\end{equation} 
这个方程的意义是,如果在时刻 $t$, $\vec r$ 处出现一个极短的脉冲,会生成怎样的波函数 $G$. 求出 $G$ 以后,我们可以把非齐次项 $f(\vec r, t)$ 看成由许许多多这样的脉冲组成,对空间和时间进行积分.即可得到\autoref{RetPot_eq1} 的解.
\begin{equation}\label{RetPot_eq3}
\Psi(\vec r, t) = \iint f(\vec r', t')G(\vec r,t, \vec r',t') \dd{V'} \dd{t'} 
\end{equation} 
下文用傅里叶变换法解\autoref{RetPot_eq2} 解出格林函数为
\begin{equation}\label{RetPot_eq4}
G(\vec r, t, \vec r', t') = \frac{1}{4\pi} \frac{\delta [(t - \abs{\vec r - \vec r'}/c) - t']}{\abs{\vec r - \vec r'}}
\end{equation} 
\autoref{RetPot_eq4}, \autoref{RetPot_eq3} 得\autoref{RetPot_eq1} 的解为
\begin{equation}\label{RetPot_eq5}
\Psi(\vec r, t) = \frac{1}{4\pi } \int \frac{f(\vec r',t - \abs{\vec r - \vec r'}/c)}{\abs{\vec r - \vec r'}} \dd{V}
\end{equation} 
\autoref{RetPot_eq5} 与静电场的势能公式很像,但是场源的时间做了修正.其中 $t - \abs{\vec r - \vec r'}/c \equiv t_{ret}$ 定义为\bb{推迟时间( retarded time )}, $\abs{\vec r - \vec r'}/c$ 是波从 $\vec r'$ 到 $\vec r$ 所需的时间.也就是说,$f(\vec r, t)$ 并不能马上影响 $\Psi(\vec r, t)$,而是需要一个“信号传播时间” 才行.在静电场中,由于场源不随时间变化,所以不需要考虑时间延迟.

\subsection{具体过程}

\subsubsection{解格林函数}  

从物理意义上,要求格林函数在以 $\vec r'$ 为中心的任意方向都相同,即只是 $\abs{\vec r - \vec r'}$ 的函数.以下为了方便,令 $R = \abs{\vec r - \vec r'}$. 
由于等式右边除了原点外都是零,方程为齐次方程.齐次解为平面波
\begin{equation}
G(\vec r, t) = \sum A(\vec k,\omega) \E^{\I \vec k \vdot \vec r} \E^{\I\omega t} 
\end{equation} 
求和是对所有满足 $\omega/k = c$ 和边界条件的 $\omega$ 和 $\vec k$ 求和(或积分).但显然该解在原点不满足要求.为了排除原点,且满足对称性,以 $\vec r'$ 为原点建立球坐标,改用球坐标中的拉普拉斯方程,方程变为
\begin{equation}\label{RetPot_eq6}
\frac{1}{R^2}\dv{r} \qty(R^2\dv{G}{R}) - \frac{1}{c^2} \dv[2]{G}{t} = -\delta (\vec r - \vec r')\delta (t - t')
\end{equation} 
在 $R \ne 0$ 的条件下解齐次方程,首先分离变量,得到分离变量解
(提示: $\frac{1}{r^2} \dv{r} \qty(r^2 \dv{G}{r}) = \frac1r \dv[2]{r^2}{(rG)}$ )
\begin{equation}\label{RetPot_eq7}
\frac1R \qty[C_1(\omega) \E^{\I\omega R/c} + C_2(\omega)\E^{-\I\omega R/c}]\E^{\I\omega t}
\end{equation} 
所以通解为\autoref{RetPot_eq7} 对不同的 $\omega$ 求和.然而 $\omega$ 是连续的,所以改用傅里叶变换法解方程\autoref{RetPot_eq6} 
\begin{equation}\label{RetPot_eq8}
\leftgroup{
A(R, \omega) = \int_{-\infty}^{+\infty } G(R, t, t') \E^{\I\omega t} \dd{t}\\
G(R,t, t') = \frac{1}{2\pi} \int_{-\infty}^{+\infty } A(R, \omega) {\E^{-\I\omega t}} \dd{\omega}
}\end{equation} 
\autoref{RetPot_eq8} 就是\autoref{RetPot_eq7} 对连续 $\omega$ 的求和(积分)得到的通解,待定系数包含在 $A$ 里面,下面的\autoref{RetPot_eq11} 验证了这点.另外,\autoref{RetPot_eq6} 右边含时 $\delta$ 函数的傅里叶变换为
\begin{equation}\label{RetPot_eq9}
\delta (t - t') = \frac{1}{2\pi } \int_{-\infty }^{+\infty} \E^{\I\omega t'} \E^{ - \I\omega t} \dd{\omega} 
\end{equation} 
\autoref{RetPot_eq8}, \autoref{RetPot_eq9} 代入方程\autoref{RetPot_eq6} 得到经过时间傅里叶变换的偏微分方程,与\autoref{RetPot_eq6} 等效. 
\begin{equation}\label{RetPot_eq10}
\laplacian A(R,\omega) + \frac{\omega^2}{c^2} A(R,\omega) =  -\E^{\I\omega t'} \delta (\vec r - \vec r')
\end{equation} 
注意这是关于位置的偏微分方程,$\omega$ 视为常数.解这条方程,就相当于解出了固定振动频率 $\omega$ 的波源所产生的同频率的波动方程.齐次解为
\begin{equation}\label{RetPot_eq11}
A(R, \omega) = \frac1R \left[ {{C_1}\left( \omega  \right){\E^{\I\omega R/c}} + {C_2}\left( \omega  \right){\E^{ -\I\omega R/c}}} \right]
\end{equation} 
(解方程提示: $\frac{1}{R^2}\frac{\D }{\D R}\left( {{R^2}\frac{\D A}{\D R}} \right) = \frac{1}{R}\frac{\D ^2}{\D {R^2}}\left( {RA} \right)$, 令 $\frac{1}{R^2}\frac{\D }{\D R}\left( {{R^2}\frac{\D A}{\D R}} \right) = \frac{1}{R}\frac{\D ^2}{\D {R^2}}\left( {RA} \right)$ )\\
由于这是方程 $\vec r \ne \vec r'$ 的通解,而\autoref{RetPot_eq10} 的右边可以看做 $\vec r = \vec r'$ 时的边界条件,接下来利用边界条件找到适合的待定系数 ${C_1}\left( \omega  \right)$ 和 $C_2(\omega)$.  
首先当 $\vec r \to \vec r'$ 时,$R \to 0$, $A \to (C_1 + C_2)/R$. 所以
\begin{equation}
\laplacian A(R, \omega) = (C_1 + C_2) \laplacian\frac1R =  - 4\pi (C_1 + C_2) \delta (\vec r - \vec r')
\end{equation} 
(关于 $\laplacian\frac1R$ 见空间狄拉克 $delta$ 函数,%链接未完成
).代入\autoref{RetPot_eq10} 左边第一项,得
\begin{equation}
-4\pi (C_1 + C_2) \delta (\vec r - \vec r') + \frac{\omega^2}{c^2} A(R, \omega) =  - \E^{\I\omega t'} \delta (\vec r - \vec r')
\end{equation} 
由于空间 delta 函数 $\delta (\vec r - \vec r') \sim \frac{1}{R^3}$,所以相比之下 $\omega ^2 A(R, \omega)/c^2 \sim 1/R$ 在 $R \to 0$ 时可以忽略不计.等式两边对比系数得
\begin{equation}
C_1 + C_2 = \frac{1}{4\pi} \E^{\I\omega t'}
\end{equation} 
再考虑\autoref{RetPot_eq11} 所代表的波函数分量,第一项代表波源向外传播的球形波,第二项代表向波源传播的,所以 $C_2 = 0$, \autoref{RetPot_eq11} 变为
\begin{equation}
A(R, \omega) = \frac1R C_1(\omega) \E^{\I\omega R/c} = \frac{1}{4\pi R} \E^{\I\omega t'} \E^{\I\omega R/c}
\end{equation} 
现在可以把上式进行反傅里叶变换\autoref{RetPot_eq8} 得到格林函数
\begin{equation}\ali{
G(R, t, t') &= \frac{1}{2\pi } \int_{-\infty}^{+\infty} A(R,\omega) \E^{ -\I\omega t} \dd{\omega}  = \frac{1}{2\pi} \int_{-\infty}^{+\infty} \frac{1}{4\pi R} \E^{\I\omega t'} \E^{\I\omega R/c} \delta (\vec r - \vec r') \E^{ -\I\omega t} \dd{\omega} \\
&= \frac{1}{4\pi R} \cdot \frac{1}{2\pi }\int_{-\infty}^{+\infty} \E^{\I\omega (t' + R/c - t)} \dd{\omega}
= \frac{1}{4\pi R}\delta (t - R/c - t')\\
&= \frac{1}{4\pi }\frac{\delta [(t - \abs{\vec r - \vec r'}/c) - t']}{\abs{\vec r - \vec r'}}
}\end{equation} 
这就是\autoref{RetPot_eq2} 的解\autoref{RetPot_eq4}. 

