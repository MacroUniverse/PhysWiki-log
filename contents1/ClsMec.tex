%经典力学笔记
%45min

拉格朗日方程为
 \begin{equation}
\frac{\D }{{\D t}}\left( {\frac{{\partial L}}{{\partial {{\dot q}_i}}}} \right) = \frac{{\partial L}}{{\partial {q_i}}}\text{, 其中 }L(q,\dot q) = T - V\text{.}
\end{equation}

正则动量(canonical momentum) ${p_i} = {{\partial L}}/{{\partial {{\dot q}_i}}}$,  广义力(generalized force) ${{\partial L}}/{{\partial {q_i}}}$,  拉氏方程就是广义力与正则动量的牛顿第二定律. 对于任何广义坐标, 拉格朗日方程的形式不变.

拉格朗日变换(略) 后, 得到哈密顿正则(canonical)方程为
 \begin{equation}
 \left\{
\begin{aligned}
  &{\dot q}_i = \partial H/\partial {p_i} \\
  &{\dot p}_i = -\partial H/\partial {q_i}
\end{aligned}
\right.
\end{equation} 
其中 $H(p,q) = T + V$. 

\subsection{泊松括号}

对任意物理量 $\omega (q,p)$,   都有
  \begin{equation}
\dot \omega  = \sum\limits_i {\left[ {\frac{{\partial \omega }}{{\partial {q_i}}}{{\dot q}_i} + \frac{{\partial \omega }}{{\partial {p_i}}}{{\dot p}_i}} \right]}  = \sum\limits_i {\left[ {\frac{{\partial \omega }}{{\partial {q_i}}}\frac{{\partial H}}{{\partial {p_i}}} - \frac{{\partial H}}{{\partial {q_i}}}\frac{{\partial \omega }}{{\partial {p_i}}}} \right]}  = \left\{ {\omega ,H} \right\}
\end{equation}
量子力学中的对易算符对应泊松括号. 注意该物理量不能显含时间, 即不能是 $\omega (q,p,t)$.  所以若泊松括号消失, 则该物理量守恒! 当显含时间时
  \begin{equation}
\dot \omega  = \left\{ {\omega ,H} \right\} + \frac{{\partial \omega }}{{\partial t}}
\end{equation}
对应量子力学中的算符平均值演化方程. 注意若调换泊松括号里面的物理量, 结果取相反数.


\subsection{坐标变换}

若变换到另一套广义坐标 $q'(q)$
\begin{equation}
{\dot q'_i} = \sum\limits_j {\frac{{\partial {{q'}_i}}}{{\partial {q_j}}}{{\dot q}_j}} \qquad
{\dot q_i} = \sum\limits_j {\frac{{\partial {q_i}}}{{\partial {{q'}_j}}}{{\dot q'}_j}}
\end{equation}
拉格朗日量是系统的状态量, 所以 $L(q',\dot q') = L[q(q'),\dot q(q',\dot q')]$,  所以
  \begin{equation}
{p'_i} = \frac{{\partial L}}{{\partial {{\dot q'}_i}}} = \sum\limits_k {\frac{{\partial L}}{{\partial {{\dot q}_k}}}\frac{{\partial {{\dot q}_k}}}{{\partial {{\dot q'}_i}}}}  = \sum\limits_k {\frac{{\partial {q_k}}}{{\partial {{q'}_i}}}{p_k}} 
\end{equation}
这就从坐标变换推出了动量变换. 对于任何广义坐标以及对应的正则动量, 哈密顿方程的形式不变(因为拉格朗日方程的形式不变, 哈密顿是由拉格朗日推出来的). 但是还有其他情况也不变, 所有使正则方程成立的坐标叫做正则坐标(canonical coordinates). 下面推导判断正则坐标的一般条件.

对于不显含时的物理量有
  \begin{equation}
\dot \omega  = \left\{ {\omega ,H} \right\} = \sum\limits_i {\left[ {\frac{{\partial \omega }}{{\partial {q_i}}}\frac{{\partial H}}{{\partial {p_i}}} - \frac{{\partial H}}{{\partial {q_i}}}\frac{{\partial \omega }}{{\partial {p_i}}}} \right]} 
\end{equation}
现在若把 $H$ 看成是 $H[q'(q,p),p'(q,p)]$,  
  \begin{equation}
\frac{{\partial H}}{{\partial {p_i}}} = \sum\limits_k {\frac{{\partial H}}{{\partial {{q'}_k}}}\frac{{\partial {{q'}_k}}}{{\partial {p_i}}} + \frac{{\partial H}}{{\partial {{p'}_k}}}\frac{{\partial {{p'}_k}}}{{\partial {p_i}}}} 
\end{equation}
 \begin{equation}
\frac{{\partial H}}{{\partial {q_i}}} = \sum\limits_k {\frac{{\partial H}}{{\partial {{q'}_k}}}\frac{{\partial {{q'}_k}}}{{\partial {q_i}}} + \frac{{\partial H}}{{\partial {{p'}_k}}}\frac{{\partial {{p'}_k}}}{{\partial {q_i}}}} 
\end{equation}
 
代入得, 并对 $H$ 的偏微分合并同类项得
  \begin{equation}
  \dot \omega  = \left\{ {\omega ,H} \right\} = \sum\limits_k {\left[ {\frac{{\partial H}}{{\partial {{q'}_k}}}\left\{ {\omega ,{{q'}_k}} \right\} + \frac{{\partial H}}{{\partial {{p'}_k}}}\left\{ {\omega ,{{p'}_k}} \right\}} \right]} 
\end{equation}
注意泊松括号是对 $q,p$ 进行偏微分, 记为 ${\left\{ {} \right\}_{q,p}}$.  分别代入 $\omega  = {q'_i},{p'_i}$,  得到转换坐标后的哈密顿方程的一般形式. 为了保持正则方程的形式, 必须要求
  \begin{equation}
{\left\{ {{{q'}_i},{{q'}_k}} \right\}_{q,p}} = 0 = {\left\{ {{{p'}_i},{{p'}_k}} \right\}_{q,p}}
\end{equation}
  \begin{equation}
{\left\{ {{{q'}_i},{{p'}_k}} \right\}_{q,p}} = {\delta _{ik}}
\end{equation}
这就是判断正则变换的一般条件. 可以证明, 用任何正则坐标作为泊松括号的角标, 其值都不变. 下面是证明
  \begin{equation}
{\left\{ {a,b} \right\}_{q,p}} = \sum\limits_i {\left( {\frac{{\partial a}}{{\partial {q_i}}}\frac{{\partial b}}{{\partial {p_i}}} - \frac{{\partial b}}{{\partial {q_i}}}\frac{{\partial a}}{{\partial {p_i}}}} \right)} 
\end{equation}
其中
  \begin{equation}
\frac{{\partial a}}{{\partial {q_i}}}\frac{{\partial b}}{{\partial {p_i}}} = \sum\limits_j {\left( {\frac{{\partial a}}{{\partial {{q'}_j}}}\frac{{\partial {{q'}_j}}}{{\partial {q_i}}} + \frac{{\partial a}}{{\partial {{p'}_j}}}\frac{{\partial {{p'}_j}}}{{\partial {q_i}}}} \right)} \sum\limits_k {\left( {\frac{{\partial b}}{{\partial {{q'}_k}}}\frac{{\partial {{q'}_k}}}{{\partial {p_i}}} + \frac{{\partial b}}{{\partial {{p'}_k}}}\frac{{\partial {{p'}_k}}}{{\partial {p_i}}}} \right)} 
\end{equation}
 \begin{equation}
\frac{{\partial b}}{{\partial {q_i}}}\frac{{\partial a}}{{\partial {p_i}}} = \sum\limits_k {\left( {\frac{{\partial b}}{{\partial {{q'}_k}}}\frac{{\partial {{q'}_k}}}{{\partial {q_i}}} + \frac{{\partial b}}{{\partial {{p'}_k}}}\frac{{\partial {{p'}_k}}}{{\partial {q_i}}}} \right)} \sum\limits_j {\left( {\frac{{\partial a}}{{\partial {{q'}_j}}}\frac{{\partial {{q'}_j}}}{{\partial {p_i}}} + \frac{{\partial a}}{{\partial {{p'}_j}}}\frac{{\partial {{p'}_j}}}{{\partial {p_i}}}} \right)} 
\end{equation}
 
现在我们要得到 ${\left\{ {a,b} \right\}_{q',p'}} = \sum\limits_i {\left( {\frac{{\partial a}}{{\partial {{q'}_i}}}\frac{{\partial b}}{{\partial {{p'}_i}}} - \frac{{\partial b}}{{\partial {{q'}_i}}}\frac{{\partial a}}{{\partial {{p'}_i}}}} \right)} $,  可以把上两式代入后对 $\frac{{\partial a}}{{\partial q'}}\frac{{\partial b}}{{\partial p'}}$ 和 $\frac{{\partial b}}{{\partial {{q'}_i}}}\frac{{\partial a}}{{\partial {{p'}_i}}}$ 合并同类项, 得
  \begin{equation}
\begin{aligned}
  {\left\{ {a,b} \right\}_{q,p}}& = \sum\limits_{jk} {\frac{{\partial a}}{{\partial {{q'}_j}}}\frac{{\partial b}}{{\partial {{p'}_k}}}} \sum\limits_i {\left( {\frac{{\partial {{q'}_j}}}{{\partial {q_i}}}\frac{{\partial {{p'}_k}}}{{\partial {p_i}}} - \frac{{\partial {{p'}_k}}}{{\partial {q_i}}}\frac{{\partial {{q'}_j}}}{{\partial {p_i}}}} \right)}  \\
  &   - \sum\limits_{jk} {\frac{{\partial b}}{{\partial {{q'}_k}}}\frac{{\partial a}}{{\partial {{p'}_j}}}} \sum\limits_i {\left( {\frac{{\partial {{q'}_k}}}{{\partial {q_i}}}\frac{{\partial {{p'}_j}}}{{\partial {p_i}}} - \frac{{\partial {{p'}_j}}}{{\partial {q_i}}}\frac{{\partial {{q'}_k}}}{{\partial {p_i}}}} \right)}  \hfill \\
   &= \sum\limits_{jk} {\frac{{\partial a}}{{\partial {{q'}_j}}}\frac{{\partial b}}{{\partial {{p'}_k}}}{{\left\{ {{{q'}_j},{{p'}_k}} \right\}}_{q,p}}}  - \sum\limits_{jk} {\frac{{\partial b}}{{\partial {{q'}_k}}}\frac{{\partial a}}{{\partial {{p'}_j}}}} {\left\{ {{{q'}_k},{{p'}_j}} \right\}_{q,p}}
\end{aligned}
\end{equation}
代入正则坐标条件, 得
  \begin{equation}
  \begin{aligned}
{\left\{ {a,b} \right\}_{q,p}}& = \sum\limits_{jk} {\frac{{\partial a}}{{\partial {{q'}_j}}}\frac{{\partial b}}{{\partial {{p'}_k}}}{\delta _{jk}}}  - \sum\limits_{jk} {\frac{{\partial b}}{{\partial {{q'}_k}}}\frac{{\partial a}}{{\partial {{p'}_j}}}} {\delta _{jk}} = \sum\limits_j {\left( {\frac{{\partial a}}{{\partial {{q'}_j}}}\frac{{\partial b}}{{\partial {{p'}_j}}} - \frac{{\partial b}}{{\partial {{q'}_j}}}\frac{{\partial a}}{{\partial {{p'}_j}}}} \right)}  \\
&= {\left\{ {a,b} \right\}_{q',p'}}
\end{aligned}
\end{equation}
