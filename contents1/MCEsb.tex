% 巨正则系综法

\subsection{中心思想}
一系统与热源达到平衡, 热源温度为 $T$,  化学势为 $\mu $,  那么系统在任意一个包含 $N$ 个粒子, 能量 $E$ 的非简并状态的概率为
\begin{equation}
\frac{{{\E^{(\mu N - E)/kT}}}}{\Xi }
\end{equation}
其中, 巨配分函数使所有状态的概率之和为一, 起到归一化的作用.
\begin{equation}
\Xi  = \sum\limits_{\rm{N}}^{} {\sum\limits_i^{} {{\E^{(\mu N - {E_i})/kT}}} } \end{equation}
\subsection{推导} %未完成

\subsection{“能级导向”}

\begin{equation}
\begin{aligned}
\Xi & = \sum\limits_{N = 1}^\infty  {\sum\limits_{\{ {n_i}\} }^ *  {\exp \left( {N\mu  - \sum\limits_{i = 1}^\infty  {{n_i}{\varepsilon _i}} } \right)\beta } }  = \sum\limits_{N = 1}^\infty  {\sum\limits_{\{ {n_i}\} }^ *  {{z^N}\prod\limits_{i = 0}^\infty  {{{\left( {{\E^{ - {\varepsilon _i}\beta }}} \right)}^{{n_i}}}} } }  = \sum\limits_{N = 1}^\infty  {\sum\limits_{\{ {n_i}\} }^ *  {\prod\limits_{i = 0}^\infty  {{{\left( {z{\E^{ - {\varepsilon _i}\beta }}} \right)}^{{n_i}}}} } } \\
 &= \sum\limits_{{n_1}}^ *  {\sum\limits_{{n_2}}^ *  {...\prod\limits_{i = 0}^\infty  {{{\left( {z{\E^{ - {\varepsilon _i}\beta }}} \right)}^{{n_i}}}} } }  = \sum\limits_{{n_1}}^ *  {{{\left( {z{\E^{ - {\varepsilon _i}\beta }}} \right)}^{{n_1}}}} \sum\limits_{{n_2}}^ *  {{{\left( {z{\E^{ - {\varepsilon _i}\beta }}} \right)}^{{n_2}}}} ... = \prod\limits_i^\infty  {\sum\limits_{{n_i}}^ *  {{{\left( {z{\E^{ - {\varepsilon _i}\beta }}} \right)}^{{n_i}}}} }
\end{aligned}
\end{equation}
\subsection{系统的热力学性质}
由最大概率项假设,
\begin{equation}
1 = \frac{{\Omega {\E^{{{\left( {\mu N - E} \right)} \mathord{\left/
 {\vphantom {{\left( {\mu N - E} \right)} {kT}}} \right.
 \kern-\nulldelimiterspace} {kT}}}}}}{\Xi } = \frac{{{\E^{S/k}}{\E^{{{\left( {\mu N - E} \right)} \mathord{\left/
 {\vphantom {{\left( {\mu N - E} \right)} {kT}}} \right.
 \kern-\nulldelimiterspace} {kT}}}}}}{\Xi }
\end{equation}
\begin{equation} 
 {\E^{S/k}}{\E^{{{\left( {\mu N - E} \right)} \mathord{\left/
 {\vphantom {{\left( {\mu N - E} \right)} {kT}}} \right.
 \kern-\nulldelimiterspace} {kT}}}} = \Xi 
\end{equation}
\begin{equation} 
 E - ST - \mu N =  - kT\ln \Xi
\end{equation}
令 $\Phi  =  - kT\ln \Xi $ 叫做\bb{巨势}
\begin{equation}
 \Phi  = E - ST - \mu N
\end{equation}
\begin{equation}
\Phi  = E - ST - \mu N = F - G = E - ST - (E - ST + PV) =  - PV
\end{equation}
考虑到 $ \D E = T\D S - P\D V + \mu \D N$
\begin{equation}
\D \Phi  =  - P\D V - S\D T - N\D \mu
\end{equation}
所以
\begin{equation}
S =  - {\left( {\pdv{\Phi}{T}} \right)_{V,\mu }} , N =  - {\left( {\pdv{\Phi}{\mu}} \right)_{V,T}} , P =  - {\left( {\pdv{\Phi}{V}} \right)_{T,\mu }}
\end{equation}
另外有一个求能级分布的公式
\begin{equation}
\left\langle {{n_i}} \right\rangle  = \frac{1}{\Xi }\sum\limits_{N = 1}^\infty  {\sum\limits_{\{ {n_i}\} }^ *  {{n_i}\exp \left( {N\mu  - \sum\limits_{i = 1}^\infty  {{n_i}{\varepsilon _i}} } \right)\beta } }  =  - \frac{1}{{\beta {\kern 1pt} \Xi }} \pdv{\Xi}{\varepsilon_i} =  - kT \pdv{\varepsilon _i} \ln \Xi  = \pdv{\Phi}{\varepsilon_i}
\end{equation}