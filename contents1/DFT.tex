% 离散傅里叶变换(DFT)

\subsection{结论}
\pentry{幺正矩阵,傅里叶变换(指数函数)\upref{FT}} % 链接未完成

对两组有序数列 $f_0,f_1,\dots,f_{N-1}$ 和 $g_0,g_2,\dots,g_{N-1}$,正变换和逆变换分别为\footnote{工程上的定义常常是正变换没有 $1/\sqrt{N}$ 因子,逆变换的 $1/\sqrt{N}$ 因子变为 $1/N$. 这样的好处是节省运算量.本书中使用的定义好处是变换为幺正变换,有保持归一化的特点.}
\begin{align}
g_p &= \frac{1}{\sqrt{N}}\sum\limits_{q=0}^{N-1} \exp\left(-\frac{2\pi\I}{N} p q \right) f_q \\
f_q &= \frac{1}{\sqrt{N}}\sum\limits_{p=0}^{N-1} \exp\left(\frac{2\pi\I}{N} p q\right) g_p
\end{align}

\textbf{离散傅里叶变换(Discrete Fourier Transform)(DFT)}是一个复数域的正交变换.%链接未完成
\textbf{快速傅里叶变换(FFT)}的结果与离散傅里叶变换一样,只是优化了算法使程序运行更高效.

\subsection{矩阵表示}
把变换和逆变换用幺正矩阵 $\mat F$ 和 $\mat F^{-1}$ 来表示,令列矢量 $\vec f = (f_0,f_1,\dots,f_N)\Tr$,$\vec g = (g_1,g_2,\dots,g_N)\Tr$, 变换和逆变换分别为
\begin{equation}
\vec g = \mat F \vec f \qquad
\vec f = \mat F^{-1} \vec g
\end{equation}
其中
\begin{equation}
F_{pq} = \frac{1}{\sqrt{N}} \exp\left(-\frac{2\pi\I}{N} p q \right) \qquad
\mat F^{-1} = \mat F^\dag
\end{equation}
% 未完成:正交矩阵的逆矩阵
% 未完成:凡是证明以前,真的至少应该说明一下这玩意有什么用,从哪来吧!任何词条都是啊!

\subsection{证明 $\mat F$ 的正交性}
根据正交矩阵的定义,我们需要证明
\begin{equation}
\sum\limits_{p=0}^{N-1} F^*_{pq_1}F_{pq_2} = 0 \quad (q_1 \ne q_2)
\end{equation}
而
\begin{equation}\label{DFT_eq1}
\sum\limits_{p=0}^{N-1} F^*_{pq_1}F_{pq_2}
= \sum\limits_{p=0}^{N-1} \exp\left[\frac{2\pi\I}{N} (q_2-q_1) p \right]
\end{equation}
注意到求和的每一项在复平面上都对应模长为1,幅角为 $(q_2-q_1)p/N$ 个圆周的矢量,% 未完成:这个几何意义要在复数的加法以及指数函数(复数)中介绍!
而 $N$ 条矢量恰好向不同方向均匀分布,所以相加为 $0$.证毕.

\subsection{逆矩阵}
先来看矩阵各列的模长平方,即式 \eqref{DFT_eq1} 中 $q_1=q_2$ 的情况,易得任何一列的模长平方都为 $N$.所以矩阵 $\mat F$ 是一个幺正矩阵 $\mat U$ 乘以 $\sqrt{N}$.考虑到幺正矩阵的逆矩阵是其厄米共轭, %链接未完成
有
\begin{equation}
\mat F^\dag \mat F = \sqrt{N}^2 \mat U^\dag \mat U = N \mat U^{-1} \mat U = N
\end{equation}
两边除以 $N$,可得 $\mat F^{-1} = \mat F^\dag /N$. 证毕.
% 未完成:在逆矩阵的介绍中应该说明用常数表示常数乘以单位矩阵

有时为了对称起见,把矩阵元定义为
\begin{equation}
F_{pq} = \frac{1}{\sqrt{N}}\exp\left(-\frac{2\pi\I}{N} p q \right)
\end{equation}
这样,$\mat F$ 的每一列模长为1,使 $\mat F$ 本身就是幺正矩阵,其逆矩阵等于厄米共轭.

\subsection{用离散傅里叶变换近似傅里叶变换}
在数值计算傅里叶变换 %未完成:链接
时,不可能计算连续的频谱 $g(k)$,只能用数值积分计算离散值 $g(k_i)$
\begin{equation}
g(k_i) = \frac{1}{\sqrt{2\pi}} \int_{-\infty}^{+\infty} f(x)\E^{-\I kx} \D x
\end{equation}
为了节省计算量,我们假设
% 未完成:艾玛,这个过程要推出来真的好麻烦啊!!

