% 等间隔能级系统(正则系宗)

\pentry{玻尔兹曼因子% 未完成,
配分函数 %未完成
}

\subsection{结论}
若一个系统的能量只能取一系列离散的值(能级),但相邻能级间距恰好为 $\varepsilon$,那么该系统在温度 $\tau$ 达到热平衡时,平均能量为
\begin{equation}
\left\langle \varepsilon  \right\rangle  = \frac{\varepsilon }{{{\E^{\varepsilon /\tau }} - 1}}
\end{equation}

\subsection{推导1}
令第 $n$ 个能级的能量为 ${\varepsilon_n}$,能量的平均值为
\begin{equation}\label{EqCE_eq2}
\left\langle \varepsilon  \right\rangle  = \frac{{\sum\limits_{n = 0}^\infty  {{\varepsilon_n}\exp \left( { - {\varepsilon_n}/\tau } \right)} }}{{\sum\limits_{n = 0}^\infty  {\exp \left( { - {\varepsilon_n}/\tau } \right)} }}
\end{equation}
其中$\sum\limits_{n = 0}^\infty  {\exp \left( { - {\varepsilon_n}/\tau } \right)}$ 就是配分函数 $Q$.

对于等间距能级,假设等间距能级 ${\varepsilon_n} = n\varepsilon$ (也可以假设 ${\varepsilon_n} = {\varepsilon_0} + n\varepsilon $,上式的分子分母都多出一个因子 $\exp ( - {\varepsilon_0}/\tau )$,最后的结果相同).首先化简配分函数
\begin{equation}
Q = \sum\limits_{n = 0}^\infty  {{\E^{ - n\varepsilon /\tau }}}  = \sum\limits_{n = 0}^\infty  {{{\left( {{\E^{ - \varepsilon /\tau }}} \right)}^n}} 
\end{equation}
由于 ${\E^{ - \varepsilon /\tau }} < 1$,根据等比数列求和公式 %未完成
\begin{equation}\label{EqCE_eq4}
Q = \frac{1}{{1 - {\E^{ - \varepsilon /\tau }}}}
\end{equation}
再化简分子
\begin{equation}\label{EqCE_eq5}
\sum\limits_{n = 0}^\infty  {n\varepsilon \exp \left( { - n\varepsilon /\tau } \right)}  = \varepsilon \sum\limits_{n = 0}^\infty  {n{{\left( {{\E^{ - \varepsilon /\tau }}} \right)}^n}}
\end{equation}
令 $x = {\E^{ - \varepsilon /\tau }}$,有 $x < 1$.同样根据等比数列求和公式
\begin{equation}
\sum\limits_{n = 0}^\infty  {n{x^n}}  = x\dv{x}\sum\limits_{n = 0}^\infty  {{x^n}}  = x\dv{x}\left( {\frac{1}{{1 - x}}} \right) = \frac{x}{{{{\left( {1 - x} \right)}^2}}}
\end{equation}
把 $x$ 换成 ${\E^{ - \varepsilon /\tau }}$,\autoref{EqCE_eq5} 变为
\begin{equation}
\frac{{\varepsilon {\E^{ - \varepsilon /\tau }}}}{{{{\left( {1 - {\E^{ - \varepsilon /\tau }}} \right)}^2}}}
\end{equation}
把分子分母代入平均值公式\autoref{EqCE_eq2} 得到最后结论
\begin{equation}
\langle \varepsilon \rangle = \left. \frac{\varepsilon \E^{-\varepsilon/\tau}}{(1 - \E^{-\varepsilon /\tau})^2} \middle/ \frac{1}{1 - \E^{-\varepsilon/\tau}}  \right. = \frac{\varepsilon \E^{-\varepsilon/\tau}}{1 - \E^{-\varepsilon/\tau}} = \frac{\varepsilon}{\E^{\varepsilon/\tau} - 1}
\end{equation}

\subsection{推导2}
由\autoref{EqCE_eq4} 已知配分函数
\begin{equation}
Q = \frac{1}{{1 - {\E^{ - \varepsilon /\tau }}}} = \frac{1}{{1 - {\E^{ - \varepsilon\beta }}}}
\end{equation}
我们也可以直接用能量均值公式
\begin{equation}
\langle\varepsilon\rangle = -\pdv{\beta} \ln Q
= \pdv{\beta} \ln (1 - {\E^{ - \varepsilon \beta }}) = \frac{{\varepsilon {\E^{ - \varepsilon \beta }}}}{{1 - {\E^{ - \varepsilon \beta }}}} = \frac{\varepsilon }{{{\E^{\varepsilon /\tau }} - 1}}
\end{equation}
结果相同.
