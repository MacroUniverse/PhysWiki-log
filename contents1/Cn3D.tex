% Cn3D 笔记

\subsection{氢原子的数值误差}
解三维氢原子的时候, 为了防止势能为无穷, 中心点要选在八个格点中间。

首先用的 $60\times 60\times 60$ a.u. 的范围, $401\times 401\times 401$ 的格点, $0.032$ a.u. 的时间步长。

TDSE 传播时, 如果把 log 概率灰度图的范围取得很大 ($10^{-25}$ 左右), 即使没有电场, 也会出现一个方形的花纹很快地扩散出来, 然后波函数还会向外温柔地扩散。

比较了用解析的基态波函数和 Imaginary Time 得到的基态波函数, 虽然两者都有类似情况, 但是后者的程度明显要轻。

爱华说 XUV 的强度并不太重要, 频率越大, 氢原子的截面就越小, 所以反而需要更大的强度才能电离出同样的概率。 但只要不超过 $1\times 10^{16} \Si{W/cm^2}$ (0.5338 a.u.)就好。
