% 洛伦兹力

\pentry{矢量的叉乘\upref{Cross}}
磁场 $\vec B$ 中,电荷为 $q$,以速度 $\vec v$ 运动的点电荷受到的洛伦兹力通过叉乘\upref{Cross}定义
\begin{equation}
\vec F = q\vec v \cross \vec B
\end{equation}
即洛伦兹力与速度和磁场的方向垂直,大小等于 $qvB$ 乘以速度与磁场夹角的正弦值.可见当速度与磁场垂直时洛伦兹力最大,平行时没有洛伦兹力.

\subsection{磁场对电荷不做功}
由于任意时刻,磁场力的方向垂直于运动方向,所以静磁场不对电荷做功(类比向心力不对圆周运动做功),证明如下.洛伦兹力的瞬时功率为
\begin{equation}
P = \vec F \vdot \vec v = q\,\vec v \cross \vec B \vdot \vec v
\end{equation}
由矢量混合积的运算 %链接未完成
\begin{equation}
\vec v \cross \vec B \vdot \vec v = \vec v \cross \vec v \vdot \vec B = 0
\end{equation}
因为矢量叉乘本身等于0.


\subsection{广义洛伦兹力}
\pentry{极限\upref{Lim}}
\bb{广义上的洛伦兹力} 是指电磁场给电荷施加的所有作用力,即电场力加洛伦兹力.麦克斯韦方程组%链接未完成
只描述了由电荷的分布及运动情况如何计算电磁场.而广义洛伦兹力则解释了已知电磁场分布如何计算电荷的受力.对于点电荷
\begin{equation}
\vec F = q (\vec E + \vec v \cross \vec B)
\end{equation}
对于连续的电荷分布
\begin{equation}
\vec f = \rho(\vec E + \vec v \cross \vec B)
\end{equation}
其中 $\vec f$ 是受力密度,用极限的方法定义为无穷小体积的受力除以该体积
\begin{equation}
\vec f = \lim_{V \to 0} \frac{\vec F}{V}
\end{equation}

