\documentclass[zihao=-4,openany,fancyhdr,UTF8]{ctexbook}
\usepackage{amsmath,amssymb,amsthm}
\usepackage{physics} % bra ket 符号等
\usepackage{graphicx}
\usepackage{wrapfig}
\usepackage{xcolor}
\usepackage[framemethod=TikZ]{mdframed}
\usepackage[final]{pdfpages}
\usepackage{listings} % 代码排版
\usepackage{bm} % 用于矢量的粗体
% 数学格式

%表示角度的圆圈直接用° (搜狗打du即可)
%\newcommand{\abs}[1]{\left\lvert#1 \right\rvert} %定义绝对值 % physics 宏包已有
\newcommand{\D}{\,\mathrm{d}} %微分算符正体
\newcommand{\I}{\mathrm{i}} %虚数单位正体
\newcommand{\E}{\mathrm{e}} %自然对数底整体
\renewcommand{\vec}[1]{\boldsymbol{\mathbf{#1}}}%矢量用正体加粗字母, 大写也可以(电场) %只用 mathbf 对希腊字母没用
\newcommand{\mat}[1]{\boldsymbol{\mathbf{#1}}}   %矩阵用正体加粗字母, 小写也可以(泡力矩阵)
\newcommand{\ten}[1]{\boldsymbol{\mathbf{#1}}} % 张量与矩阵相同
\newcommand{\Nabla}{\vec\nabla}
\renewcommand{\Tr}{^{\textup{T}}} %矩阵转置 % 覆盖 physics 宏包的 Trace
\newcommand{\uvec}[1]{\hat{\boldsymbol{\mathbf{#1}}}} %单位矢量为黑体加 hat
\renewcommand{\pmat}[1]{\begin{pmatrix}#1\end{pmatrix}} % 矩阵圆括号, 覆盖 physics 宏包 (pauli matrix)
\newcommand{\ali}[1]{\begin{aligned}#1\end{aligned}}
\newcommand{\leftgroup}[1]{\left\{\begin{aligned}#1\end{aligned}\right.}
\newcommand{\vmat}[1]{\begin{vmatrix}#1\end{vmatrix}} % 行列式
\newcommand{\Cj}{^*} % 复共轭
\newcommand{\dfracH}{\rule[-0.5cm]{0pt}{1.3cm}} % 全尺寸分数 dfrac 在表格中的高度
\newcommand{\Her}{^\dag} %厄米共轭
\newcommand{\Q}[1]{\hat #1}
\newcommand{\sinc}{\mathrm{sinc}} % sinc 函数
\newcommand{\Si}[1]{\ \si{#1}} % 国际单位
\newcommand{\x}{\texttt} % 代码字体
\newcommand{\bb}{\textbf} % 粗体
\newcommand{\les}{\leqslant} % 小于等于号
\newcommand{\ges}{\geqslant} % 大于等于号

% 设置 Matlab 格式,用于MatlabStyle.tex
\usepackage{color} %red, green, blue, yellow, cyan, magenta, black, white
\definecolor{mygreen}{RGB}{28,172,0}
\definecolor{mylilas}{RGB}{170,55,241}
\definecolor{string}{RGB}{160,32,240}
\definecolor{comment}{RGB}{34,139,34}
\definecolor{warning}{RGB}{255,100,0}
\definecolor{error}{RGB}{230,0,0}

% 其他
\newcommand{\entry}[2]{\section{#1}\label{#2}\input{./contents/#2}}% 主程序中使用,从 contents 文件夹加入词条
\newcommand{\Entry}[2]{\section{#1}\label{#2}\input{./contents1/#2}}% 从 contents1 中加入词条(contents1 里面用于存放超出力学分册的内容)
\newcommand{\pentry}[1]{\begin{mdframed}\bb{预备知识\ } #1 \end{mdframed}}%预备知识的格式
\newcommand{\rentry}[1]{\begin{mdframed}\bb{拓展阅读\ } #1 \end{mdframed}}%拓展阅读的格式
\newcommand{\eentry}[1]{\begin{mdframed}\bb{应用举例\ } #1 \end{mdframed}}%应用举例的格式

% 图片的格式
%\newcommand{\fig}[3]{\begin{figure}[ht]\centering\includegraphics[width= #2]{./figures/#1.pdf}\caption{#3}\end{figure}}
% 控制行代码环境(注意不能用于段落最后!)
% zihao 命令,小N号用 -N,大N号用 +N
\newenvironment{Command}{\vspace{0.2cm} \\ \ttfamily\zihao{5}}{\vspace{0.2cm}\\}

% 淘宝模板其他设置
\mdfsetup{%
frametitlealignment=\noindent\raggedright,
middlelinecolor=gray,
middlelinewidth=.3pt,
backgroundcolor=white,
roundcorner=6pt}
%-------------------------------------------------------------------------------
% Colors adapted from Material Design
%-------------------------------------------------------------------------------

% Red
\definecolor{red-50}{RGB}{255,235,238}
\definecolor{red-100}{RGB}{255,205,210}
\definecolor{red-200}{RGB}{239,154,154}
\definecolor{red-300}{RGB}{229,115,115}
\definecolor{red-400}{RGB}{239,83,80}
\definecolor{red-500}{RGB}{244,67,54}
\definecolor{red-600}{RGB}{229,57,53}
\definecolor{red-700}{RGB}{211,47,47}
\definecolor{red-800}{RGB}{198,40,40}
\definecolor{red-900}{RGB}{183,28,28}
\definecolor{red-A100}{RGB}{255,138,128}
\definecolor{red-A200}{RGB}{255,82,82}
\definecolor{red-A400}{RGB}{255,23,68}
\definecolor{red-A700}{RGB}{213,0,0}

% Pink
\definecolor{pink-50}{RGB}{252,228,236}
\definecolor{pink-100}{RGB}{248,187,208}
\definecolor{pink-200}{RGB}{244,143,177}
\definecolor{pink-300}{RGB}{240,98,146}
\definecolor{pink-400}{RGB}{236,64,122}
\definecolor{pink-500}{RGB}{233,30,99}
\definecolor{pink-600}{RGB}{216,27,96}
\definecolor{pink-700}{RGB}{194,24,91}
\definecolor{pink-800}{RGB}{173,20,87}
\definecolor{pink-900}{RGB}{136,14,79}
\definecolor{pink-A100}{RGB}{255,128,171}
\definecolor{pink-A200}{RGB}{255,64,129}
\definecolor{pink-A400}{RGB}{245,0,87}
\definecolor{pink-A700}{RGB}{197,17,98}

% Purple
\definecolor{purple-50}{RGB}{243,229,245}
\definecolor{purple-100}{RGB}{225,190,231}
\definecolor{purple-200}{RGB}{206,147,216}
\definecolor{purple-300}{RGB}{186,104,200}
\definecolor{purple-400}{RGB}{171,71,188}
\definecolor{purple-500}{RGB}{156,39,176}
\definecolor{purple-600}{RGB}{142,36,170}
\definecolor{purple-700}{RGB}{123,31,162}
\definecolor{purple-800}{RGB}{106,27,154}
\definecolor{purple-900}{RGB}{74,20,140}
\definecolor{purple-A100}{RGB}{234,128,252}
\definecolor{purple-A200}{RGB}{224,64,251}
\definecolor{purple-A400}{RGB}{213,0,249}
\definecolor{purple-A700}{RGB}{170,0,255}

% Deep-Purple
\definecolor{deep-purple-50}{RGB}{237,231,246}
\definecolor{deep-purple-100}{RGB}{209,196,233}
\definecolor{deep-purple-200}{RGB}{179,157,219}
\definecolor{deep-purple-300}{RGB}{149,117,205}
\definecolor{deep-purple-400}{RGB}{126,87,194}
\definecolor{deep-purple-500}{RGB}{103,58,183}
\definecolor{deep-purple-600}{RGB}{94,53,177}
\definecolor{deep-purple-700}{RGB}{81,45,168}
\definecolor{deep-purple-800}{RGB}{69,39,160}
\definecolor{deep-purple-900}{RGB}{49,27,146}
\definecolor{deep-purple-A100}{RGB}{179,136,255}
\definecolor{deep-purple-A200}{RGB}{124,77,255}
\definecolor{deep-purple-A400}{RGB}{101,31,255}
\definecolor{deep-purple-A700}{RGB}{98,0,234}

% Indigo
\definecolor{indigo-50}{RGB}{232,234,246}
\definecolor{indigo-100}{RGB}{197,202,233}
\definecolor{indigo-200}{RGB}{159,168,218}
\definecolor{indigo-300}{RGB}{121,134,203}
\definecolor{indigo-400}{RGB}{92,107,192}
\definecolor{indigo-500}{RGB}{63,81,181}
\definecolor{indigo-600}{RGB}{57,73,171}
\definecolor{indigo-700}{RGB}{48,63,159}
\definecolor{indigo-800}{RGB}{40,53,147}
\definecolor{indigo-900}{RGB}{26,35,126}
\definecolor{indigo-A100}{RGB}{140,158,255}
\definecolor{indigo-A200}{RGB}{83,109,254}
\definecolor{indigo-A400}{RGB}{61,90,254}
\definecolor{indigo-A700}{RGB}{48,79,254}

% Blue
\definecolor{blue-50}{RGB}{227,242,253}
\definecolor{blue-100}{RGB}{187,222,251}
\definecolor{blue-200}{RGB}{144,202,249}
\definecolor{blue-300}{RGB}{100,181,246}
\definecolor{blue-400}{RGB}{66,165,245}
\definecolor{blue-500}{RGB}{33,150,243}
\definecolor{blue-600}{RGB}{30,136,229}
\definecolor{blue-700}{RGB}{25,118,210}
\definecolor{blue-800}{RGB}{21,101,192}
\definecolor{blue-900}{RGB}{13,71,161}
\definecolor{blue-A100}{RGB}{130,177,255}
\definecolor{blue-A200}{RGB}{68,138,255}
\definecolor{blue-A400}{RGB}{41,121,255}
\definecolor{blue-A700}{RGB}{41,98,255}

% Light-blue
\definecolor{light-blue-50}{RGB}{225,245,254}
\definecolor{light-blue-100}{RGB}{179,229,252}
\definecolor{light-blue-200}{RGB}{129,212,250}
\definecolor{light-blue-300}{RGB}{79,195,247}
\definecolor{light-blue-400}{RGB}{41,182,246}
\definecolor{light-blue-500}{RGB}{3,169,244}
\definecolor{light-blue-600}{RGB}{3,155,229}
\definecolor{light-blue-700}{RGB}{2,136,209}
\definecolor{light-blue-800}{RGB}{2,119,189}
\definecolor{light-blue-900}{RGB}{1,87,155}
\definecolor{light-blue-A100}{RGB}{128,216,255}
\definecolor{light-blue-A200}{RGB}{64,196,255}
\definecolor{light-blue-A400}{RGB}{0,176,255}
\definecolor{light-blue-A700}{RGB}{0,145,234}

% Cyan
\definecolor{cyan-50}{RGB}{224,247,250}
\definecolor{cyan-100}{RGB}{178,235,242}
\definecolor{cyan-200}{RGB}{128,222,234}
\definecolor{cyan-300}{RGB}{77,208,225}
\definecolor{cyan-400}{RGB}{38,198,218}
\definecolor{cyan-500}{RGB}{0,188,212}
\definecolor{cyan-600}{RGB}{0,172,193}
\definecolor{cyan-700}{RGB}{0,151,167}
\definecolor{cyan-800}{RGB}{0,131,143}
\definecolor{cyan-900}{RGB}{0,96,100}
\definecolor{cyan-A100}{RGB}{132,255,255}
\definecolor{cyan-A200}{RGB}{24,255,255}
\definecolor{cyan-A400}{RGB}{0,229,255}
\definecolor{cyan-A700}{RGB}{0,184,212}

% Teal
\definecolor{teal-50}{RGB}{224,242,241}
\definecolor{teal-100}{RGB}{178,223,219}
\definecolor{teal-200}{RGB}{128,203,196}
\definecolor{teal-300}{RGB}{77,182,172}
\definecolor{teal-400}{RGB}{38,166,154}
\definecolor{teal-500}{RGB}{0,150,136}
\definecolor{teal-600}{RGB}{0,137,123}
\definecolor{teal-700}{RGB}{0,121,107}
\definecolor{teal-800}{RGB}{0,105,92}
\definecolor{teal-900}{RGB}{0,77,64}
\definecolor{teal-A100}{RGB}{167,255,235}
\definecolor{teal-A200}{RGB}{100,255,218}
\definecolor{teal-A400}{RGB}{29,233,182}
\definecolor{teal-A700}{RGB}{0,191,165}

% Green
\definecolor{green-50}{RGB}{232,245,233}
\definecolor{green-100}{RGB}{200,230,201}
\definecolor{green-200}{RGB}{165,214,167}
\definecolor{green-300}{RGB}{129,199,132}
\definecolor{green-400}{RGB}{102,187,106}
\definecolor{green-500}{RGB}{76,175,80}
\definecolor{green-600}{RGB}{67,160,71}
\definecolor{green-700}{RGB}{56,142,60}
\definecolor{green-800}{RGB}{46,125,50}
\definecolor{green-900}{RGB}{27,94,32}
\definecolor{green-A100}{RGB}{185,246,202}
\definecolor{green-A200}{RGB}{105,240,174}
\definecolor{green-A400}{RGB}{0,230,118}
\definecolor{green-A700}{RGB}{0,200,83}

% Light-green
\definecolor{light-green-50}{RGB}{241,248,233}
\definecolor{light-green-100}{RGB}{220,237,200}
\definecolor{light-green-200}{RGB}{197,225,165}
\definecolor{light-green-300}{RGB}{174,213,129}
\definecolor{light-green-400}{RGB}{156,204,101}
\definecolor{light-green-500}{RGB}{139,195,74}
\definecolor{light-green-600}{RGB}{124,179,66}
\definecolor{light-green-700}{RGB}{104,159,56}
\definecolor{light-green-800}{RGB}{85,139,47}
\definecolor{light-green-900}{RGB}{51,105,30}
\definecolor{light-green-A100}{RGB}{204,255,144}
\definecolor{light-green-A200}{RGB}{178,255,89}
\definecolor{light-green-A400}{RGB}{118,255,3}
\definecolor{light-green-A700}{RGB}{100,221,23}

% Lime
\definecolor{lime-50}{RGB}{249,251,231}
\definecolor{lime-100}{RGB}{240,244,195}
\definecolor{lime-200}{RGB}{230,238,156}
\definecolor{lime-300}{RGB}{220,231,117}
\definecolor{lime-400}{RGB}{212,225,87}
\definecolor{lime-500}{RGB}{205,220,57}
\definecolor{lime-600}{RGB}{192,202,51}
\definecolor{lime-700}{RGB}{175,180,43}
\definecolor{lime-800}{RGB}{158,157,36}
\definecolor{lime-900}{RGB}{130,119,23}
\definecolor{lime-A100}{RGB}{244,255,129}
\definecolor{lime-A200}{RGB}{238,255,65}
\definecolor{lime-A400}{RGB}{198,255,0}
\definecolor{lime-A700}{RGB}{174,234,0}

% Yellow
\definecolor{yellow-50}{RGB}{255,253,231}
\definecolor{yellow-100}{RGB}{255,249,196}
\definecolor{yellow-200}{RGB}{255,245,157}
\definecolor{yellow-300}{RGB}{255,241,118}
\definecolor{yellow-400}{RGB}{255,238,88}
\definecolor{yellow-500}{RGB}{255,235,59}
\definecolor{yellow-600}{RGB}{253,216,53}
\definecolor{yellow-700}{RGB}{251,192,45}
\definecolor{yellow-800}{RGB}{249,168,37}
\definecolor{yellow-900}{RGB}{245,127,23}
\definecolor{yellow-A100}{RGB}{255,255,141}
\definecolor{yellow-A200}{RGB}{255,255,0}
\definecolor{yellow-A400}{RGB}{255,234,0}
\definecolor{yellow-A700}{RGB}{255,214,0}

% Amber
\definecolor{amber-50}{RGB}{255,248,225}
\definecolor{amber-100}{RGB}{255,236,179}
\definecolor{amber-200}{RGB}{255,224,130}
\definecolor{amber-300}{RGB}{255,213,79}
\definecolor{amber-400}{RGB}{255,202,40}
\definecolor{amber-500}{RGB}{255,193,7}
\definecolor{amber-600}{RGB}{255,179,0}
\definecolor{amber-700}{RGB}{255,160,0}
\definecolor{amber-800}{RGB}{255,143,0}
\definecolor{amber-900}{RGB}{255,111,0}
\definecolor{amber-A100}{RGB}{255,229,127}
\definecolor{amber-A200}{RGB}{255,215,64}
\definecolor{amber-A400}{RGB}{255,196,0}
\definecolor{amber-A700}{RGB}{255,171,0}

% Orange
\definecolor{orange-50}{RGB}{255,243,224}
\definecolor{orange-100}{RGB}{255,224,178}
\definecolor{orange-200}{RGB}{255,204,128}
\definecolor{orange-300}{RGB}{255,183,77}
\definecolor{orange-400}{RGB}{255,167,38}
\definecolor{orange-500}{RGB}{255,152,0}
\definecolor{orange-600}{RGB}{251,140,0}
\definecolor{orange-700}{RGB}{245,124,0}
\definecolor{orange-800}{RGB}{239,108,0}
\definecolor{orange-900}{RGB}{230,81,0}
\definecolor{orange-A100}{RGB}{255,209,128}
\definecolor{orange-A200}{RGB}{255,171,64}
\definecolor{orange-A400}{RGB}{255,145,0}
\definecolor{orange-A700}{RGB}{255,109,0}

% Deep-orange
\definecolor{deep-orange-50}{RGB}{251,233,231}
\definecolor{deep-orange-100}{RGB}{255,204,188}
\definecolor{deep-orange-200}{RGB}{255,171,145}
\definecolor{deep-orange-300}{RGB}{255,138,101}
\definecolor{deep-orange-400}{RGB}{255,112,67}
\definecolor{deep-orange-500}{RGB}{255,87,34}
\definecolor{deep-orange-600}{RGB}{244,81,30}
\definecolor{deep-orange-700}{RGB}{230,74,25}
\definecolor{deep-orange-800}{RGB}{216,67,21}
\definecolor{deep-orange-900}{RGB}{191,54,12}
\definecolor{deep-orange-A100}{RGB}{255,158,128}
\definecolor{deep-orange-A200}{RGB}{255,110,64}
\definecolor{deep-orange-A400}{RGB}{255,61,0}
\definecolor{deep-orange-A700}{RGB}{221,44,0}

% Brown
\definecolor{brown-50}{RGB}{239,235,233}
\definecolor{brown-100}{RGB}{215,204,200}
\definecolor{brown-200}{RGB}{188,170,164}
\definecolor{brown-300}{RGB}{161,136,127}
\definecolor{brown-400}{RGB}{141,110,99}
\definecolor{brown-500}{RGB}{121,85,72}
\definecolor{brown-600}{RGB}{109,76,65}
\definecolor{brown-700}{RGB}{93,64,55}
\definecolor{brown-800}{RGB}{78,52,46}
\definecolor{brown-900}{RGB}{62,39,35}

% Grey
\definecolor{grey-50}{RGB}{250,250,250}
\definecolor{grey-100}{RGB}{245,245,245}
\definecolor{grey-200}{RGB}{238,238,238}
\definecolor{grey-300}{RGB}{224,224,224}
\definecolor{grey-400}{RGB}{189,189,189}
\definecolor{grey-500}{RGB}{158,158,158}
\definecolor{grey-600}{RGB}{117,117,117}
\definecolor{grey-700}{RGB}{97,97,97}
\definecolor{grey-800}{RGB}{66,66,66}
\definecolor{grey-900}{RGB}{33,33,33}

% Blue-grey
\definecolor{blue-grey-50}{RGB}{236,239,241}
\definecolor{blue-grey-100}{RGB}{207,216,220}
\definecolor{blue-grey-200}{RGB}{176,190,197}
\definecolor{blue-grey-300}{RGB}{144,164,174}
\definecolor{blue-grey-400}{RGB}{120,144,156}
\definecolor{blue-grey-500}{RGB}{96,125,139}
\definecolor{blue-grey-600}{RGB}{84,110,122}
\definecolor{blue-grey-700}{RGB}{69,90,100}
\definecolor{blue-grey-800}{RGB}{55,71,79}
\definecolor{blue-grey-900}{RGB}{38,50,56}

\definecolor{black}{RGB}{0,0,0}
\definecolor{white}{RGB}{255,255,255}

\usepackage{xeCJK}
\usepackage[papersize={185mm,260mm},hmargin=2.1cm,vmargin=1in]{geometry}
%%%%%%%%%%%%% 有需要可以自行调用字体
\setCJKmainfont
  [ItalicFont=KaiTi,
  BoldFont=SimHei]
  {SimSun}
\setCJKsansfont{SimHei}
\renewcommand{\heiti}
  {\CJKfontspec{SimHei}}
\renewcommand{\kaishu}
  {\CJKfontspec{KaiTi}}
\renewcommand{\youyuan}
  % {\CJKfontspec{YouYuan}}
% \renewcommand{\fangsong}
  % {\CJKfontspec{FangSong}}

\setmainfont{Times New Roman}
\setsansfont{Arial}
%%%%%%%%%%%%%%%%%%%%%%%%%%%
\usepackage{titletoc}
\usepackage{titlesec}
\titleformat{\section}[display]% 词条格式 (section)
{\zihao{4}\color{deep-orange-A700}\filright\kaishu\sf}
{}{.5em}{}[\color{orange-A700}\vspace*{-1.2em}\rule{\textwidth}{2pt}] %在节上显示*号

\definecolor{myblue}{RGB}{10,10,180} % 用于 subtitle
\titleformat{\subsection}[hang] % 小标题格式 (subsection)
{\normalsize\color{myblue}\heiti\filright} % 颜色和字体
{} % label
{0.5cm} % 小标题下方空白
{\vspace*{-0.2cm}\\} % 小标题前的命令
%[]  % 小标题下方文字结束后下方的命令

\titlespacing*{\section}
{0em}{1.5ex plus .1ex minus .2ex}{.5ex minus .1ex}
%\contentsmargin{0pt}
\titlecontents{part}[3.5em]{\vspace{1.25\baselineskip}\centering\heiti\zihao{3}}{\contentslabel{3.6em}}{\hspace*{-3.6em}}
        {\hfill}
\titlecontents{chapter}[5em]{\vspace{0.55\baselineskip}\fontsize{13.05pt}{16pt}\selectfont\heiti}{\contentslabel{4.5em}}{\hspace*{-4.5em}}
        {\hspace{.5em}}[\titlerule]
\titlecontents*{section}[0pc]
  {\addvspace{3pt}\zihao{-4}}
  {}
  {}
  {\,\textsuperscript{\color{blue-A700}[\thecontentspage]}}[\quad]

\titlecontents*{subsection}[1.8pc]
  {\zihao{5}}
  {\thecontentslabel. }
  {}
  {, \thecontentspage}
  [.---][.]
\titlespacing*{\subsection} {0pt}{1.25ex plus 1ex minus .2ex}{.5ex plus .2ex}

\usepackage{tikz}
\usepackage{multicol}


\newtheoremstyle{mystyle}{8pt}{8pt}{}{}{\bfseries}{}{\newline}
{\thmname{#1}\thmnumber{ #2}\thmnote{\bf #3}\indent}  % 新增:最后一个{}
\theoremstyle{mystyle}
\newtheorem{example}{\normalsize{例}}
\newtheorem{Thm}{\normalsize{{定理}}}
\newtheorem*{slove}{\normalsize{{【解】}}} % * 表示不编号!


\usepackage{indentfirst}
\usepackage[pdfstartview=FitH,
            CJKbookmarks=true,
            bookmarksnumbered=false,
            bookmarksopen=true,
            linktocpage=true,
            colorlinks, %注释掉此项则交叉引用为彩色边框(将colorlinks和pdfborder同时注释掉)
            pdfborder=001,   %注释掉此项则交叉引用为彩色边框
            linkcolor=blue-A700,
            anchorcolor=blue-A700,
            citecolor=blue-A700]{hyperref}
\setlength{\abovecaptionskip}{0pt}
\setlength{\belowcaptionskip}{3pt}

\newcommand\upref[1]{\textsuperscript{[\pageref{#1}]}}
\renewcommand\sectionmark[1]{%
\markright{\leftmark}}
\renewcommand{\textfraction}{0.05}
\renewcommand{\topfraction}{0.95}
\renewcommand{\bottomfraction}{0.65}
\renewcommand{\floatpagefraction}{0.60}
\makeatletter
\setlength{\@fptop}{5pt}
\setlength{\floatsep}{10pt \@plus3pt \@minus1pt}
\setlength{\intextsep}{12pt \@plus3pt \@minus2pt}
\setlength{\textfloatsep}{10pt \@plus3pt \@minus2pt}
\def\@makechapterhead#1{%
  \null\vskip3.8cm\@tempskipa = \glueexpr \CTEX@chapter@beforeskip \relax
  \ifodd \CTEX@chapter@fixbeforeskip
    \CTEX@fixbeforeskip
  \fi
  \vspace*{\@tempskipa}%
  {\normalfont \parindent \dimexpr \CTEX@chapter@indent \relax
   \CTEX@chapter@format\centering
   \ifnum \c@secnumdepth >\m@ne
     \if@mainmatter
       \ifodd \CTEX@chapter@numbering
         \CTEX@chaptername \CTEX@chapter@aftername
       \fi
     \fi
   \fi
   \par\vskip18pt%
   \CTEX@chapter@titleformat{#1}%
   \CTEX@chapter@aftertitle
   \nobreak
   \vfill\vskip \glueexpr \CTEX@chapter@afterskip \relax
   \newpage
  }}
\def\@makeschapterhead#1{%
  \null\vskip-2.8cm\@tempskipa = \glueexpr \CTEX@chapter@beforeskip \relax
  \ifodd \CTEX@chapter@fixbeforeskip
    \CTEX@fixbeforeskip
  \fi
  \vspace*{\@tempskipa}%
  {\normalfont \parindent \dimexpr \CTEX@chapter@indent \relax
   \CTEX@chapter@format
   \interlinepenalty\@M
   \CTEX@chapter@titleformat{#1}
   \CTEX@chapter@aftertitle
   \nobreak
   \null\vskip-2cm
   \vskip \glueexpr \CTEX@chapter@afterskip \relax
  }}
\makeatother

% 新增:以下命令为新加
\makeatletter
\def\HyRef@autopageref#1{%
 \hyperref[{#1}]{[\pageref*{#1}]}%
}
\makeatother

\renewcommand{\equationautorefname}{式}
\renewcommand{\figureautorefname}{图}
\renewcommand{\tableautorefname}{表}
\newcommand{\Exaautorefname}{例}
\newcommand{\exampleautorefname}{例}

\makeatletter
\def\thmt@amsthmlistbreakhack{%
  \leavevmode
  \vspace{-\baselineskip}%
  \par
  \everypar{\setbox\z@\lastbox\everypar{}}%
}
\makeatother

%\let\eqref\autoref
\newcommand{\exref}[1]{\autoref{#1}}
\renewcommand\upref[1]{\textsuperscript{\autopageref{#1}}}

%\AtBeginEnvironment{example}{\normalsize} % 调整 example 环境字号(标题也会受影响!不能用!),不能调整字体! 以下定义 exam 环境,以后一律使用 exam 环境!
\newenvironment{exam}[1]{\begin{example}[\;\; #1]\kaishu\hspace*{1.7em}}{\end{example}} % 调整字体字号

% 新增结束

% 编号在每个 section 内从 1 开始
\numberwithin{equation}{section}
\renewcommand\theequation{\arabic{equation}}
\numberwithin{example}{section}
\renewcommand\theexample{\arabic{example}}
\numberwithin{figure}{section}
\renewcommand\thefigure{\arabic{figure}}
\numberwithin{table}{section}
\renewcommand\thetable{\arabic{table}}
\numberwithin{chapter}{part}

\usepackage{etoolbox} % 新增
\setboolean{@twoside}{false} % 封面只占一面
\usepackage[capitalize,nameinlink,noabbrev]{cleveref} % 新增
\usepackage{enumitem}  % \begin{enumerate}[resume] 可以接着之前的编号
\usepackage{siunitx} % 单位

\begin{document}

\lstset{language=Matlab,
    basicstyle= \ttfamily, % 变量以外的符号
    breaklines=true,%?  
    keywordstyle=\color{blue},% 关键字的颜色
    morekeywords=[2]{1}, keywordstyle=[2]{\color{black}}, %? 
    identifierstyle=\color{black},%? 
    stringstyle=\color{string}, % 字符串的颜色
    commentstyle=\color{comment},% 注释的格式
    showstringspaces=false,% 不显示空格符
    numbers=left,
    numberstyle={\tiny \color{gray}}, % 行号大小
    numbersep=11pt, % 行号距离
    %backgroundcolor=\color{yellow}, % 背景颜色
    emph=[1]{for,end,break},emphstyle=[1]\color{blue} % 指定单词高量
    %morekeywords={smooth} % 增加 keywords
     % 删除 keywords
}

% Matlab 代码环境
\newcommand{\Matlab}{\begin{lstlisting}[language=Matlab, deletekeywords={rand, disp, error, nargin, flipud, fliplr, plot, axis, hold, roots, linspace, figure, title, sign, find, zeros, fzero, exp, format, sin, cos, tan, cot, asin, acos, atan, real, imag, sum, mean, diff, floor, ceil, sinh, cosh, tanh, round, ans, size, randn, eye, magic}]} % 设置 Matlab 颜色
\frontmatter % 开始罗马数字页码
\setcounter{tocdepth}{1}
\tableofcontents % 生成目录
\mainmatter % 开始阿拉伯数字页码

\part{创作中}
\chapter{创作中}
%====================================


%未完成: 单位正交阵, (转置等于逆矩阵)
%\entry{}{}\newpage
%\Entry{}{}\newpage

%============================
\chapter{修改审阅中}


\entry{本书编写规范}{Sample}\newpage



%%%%%%%%%%%%%    Phys Wiki      %%%%%%%%%%%%%%

%\part{数学}
%=======================================
%\chapter{数学拾遗}
%\entry{二项式定理}{BiNor}
%\entry{二项式定理(非整数幂)}{BiNorR}
%\entry{三角恒等式}{TriEqv}
%\entry{充分必要条件}{SufCnd}
%\entry{极坐标系}{Polar}
%\entry{柱坐标系}{Cylin}
%\entry{球坐标系}{Sph}
%\entry{球坐标与直角坐标的转换}{SphCar}
%\entry{圆锥曲线的极坐标方程}{Cone}
%\entry{椭圆的三种定义}{Elips3}
%\entry{双曲线的三种定义}{Hypb3}
%\entry{复变函数}{Cplx}
%\entry{指数函数(复数)}{CExp}
%\entry{三角函数(复数)}{CTrig}
%
%\chapter{一元微积分}
%%------------------------------------------------------------------------------
%\entry{微积分导航}{Calc}
%\entry{极限}{Lim}
%\entry{小角正弦极限}{LimArc}
%\entry{自然对数底}{E}
%\entry{切线与割线}{TanL}
%\entry{导数}{Der}
%\entry{求导法则}{DerRul}
%\entry{反函数求导}{InvDer}
%\entry{基本初等函数的导数}{FunDer}
%\entry{导数与函数极值}{DerMax}
%\entry{用极值点确定函数图像}{DerImg}
%\entry{一元函数的微分}{Diff}
%\entry{复合函数求导(链式法则)}{ChainR}
%\entry{泰勒展开}{Taylor}
%\entry{不定积分}{Int}
%\entry{积分表}{ITable}
%\entry{定积分}{DefInt}
%\entry{牛顿—莱布尼兹公式}{NLeib}
%\entry{换元积分法}{IntCV}
%\entry{分部积分法}{IntBP}
%\entry{常微分方程}{ODE}
%\entry{一阶线性微分方程}{ODE1}
%\entry{二阶常系数齐次微分方程}{Ode2}
%\entry{二阶常系数非齐次微分方程}{Ode2N}
%\entry{正交函数系}{Fbasis}
%\entry{傅里叶级数(三角)}{FSTri}
%\entry{傅里叶级数(指数)}{FSExp}
%\entry{偏导数}{ParDer}
%\entry{全微分}{TDiff}
%\entry{复合函数的偏导\ 链式法则}{PChain}
%\entry{全导数}{TotDer}
%
%\chapter{线性代数}
%%------------------------------------------------------------------------------
%\entry{线性代数导航}{Vector}
%\entry{几何矢量}{GVec}
%\entry{矢量点乘}{Dot}
%\entry{正交归一基底}{OrNrB}
%\entry{右手定则}{RHRul}
%\entry{矢量叉乘}{Cross}
%\entry{矢量叉乘分配律的几何证明}{CrossP}
%\entry{连续叉乘的化简}{TriCro}
%\entry{三矢量的混合积}{TriVM}
%\entry{平面旋转变换}{Rot2DT}
%\entry{线性变换}{LTrans}
%\entry{矩阵}{Mat}
%\entry{单位正交阵}{UOrM}
%\entry{平面旋转矩阵}{Rot2D}
%\entry{空间旋转矩阵}{Rot3D}
%\entry{行列式}{Deter}
%
%\chapter{多元微积分}
%%------------------------------------------------------------------------------
%\entry{矢量的导数\ 求导法则}{DerV}
%\entry{一元矢量函数的积分}{IntV}
%\entry{方向导数}{DerDir}
%\entry{重积分}{IntN}
%% 未完成 不同坐标系中的定积分
%\entry{矢量场}{Vfield}
%\entry{极坐标中单位矢量的偏导}{Dpol1}
%\entry{线积分}{IntL}
%\entry{梯度\ 梯度定理}{Grad}
%\entry{散度\ 散度定理}{Divgnc}
%
%
%\chapter{计算物理}
%%------------------------------------------------------------------------------
%\entry{计算物理导航}{NumPhy}
%\entry{Matlab 简介}{Matlab}
%\entry{Matlab 的变量与矩阵}{MatVar}
%\entry{Matlab 的判断与循环}{MIfFor}
%\entry{Matlab 的函数}{MatFun}
%\entry{Matlab 画图}{MatPlt}
%\entry{Matlab 的程序调试及其他功能}{MatOtr}
%\entry{二分法}{Bisec}
%\entry{多区间二分法}{MBisec}
%\entry{冒泡法}{Bubble}
%\entry{Nelder-Mead 算法}{NelMea}
%\entry{二项式定理(非整数)的数值验证}{BiNorM}
%\entry{弹簧振子受迫运动的简单数值计算}{SHOFN}
%\entry{天体运动的简单数值计算}{KPNum0}
%\entry{常微分方程(组)的数值解}{OdeNum}
%\entry{中点法解常微分方程(组)}{OdeMid}
%\entry{四阶龙格库塔法}{OdeRK4}
%
%% 未完成 Mathematica
%% 未完成 Wolfram Alpha
%
%%%%%%%%%%%%%%%%%%%%%%%%%%%%%%%
%
%
%\part{力学}
%%=======================================
%
%\chapter{质点}
%%------------------------------------------------------------------------------
%\entry{位置矢量\ 位移}{Disp}
%\entry{速度\ 加速度(一维)}{VnA1}
%\entry{速度\ 加速度}{VnA}
%\entry{匀速圆周运动的速度(几何法)}{CMVG}
%\entry{匀速圆周运动的速度(求导法)}{CMVD}
%\entry{匀速圆周运动的加速度(几何法)}{CMAG}
%\entry{匀速圆周运动的加速度(求导法)}{CMAD}
%\entry{匀加速运动}{ConstA}
%\entry{极坐标中的速度和加速度}{PolA}
%\entry{牛顿运动定律\ 惯性系}{New3}
%\entry{功\ 功率}{Fwork}
%\entry{动能\ 动能定理(单个质点)}{KELaw1}
%\entry{力场\ 势能}{V}
%\entry{机械能守恒(单个质点)}{ECnst}
%\entry{动量\ 动量定理(单个质点)}{PLaw1}
%\entry{角动量定理\ 角动量守恒(单个质点)}{AMLaw1}
%\entry{简谐振子}{SHO}
%\entry{受阻落体}{RFall}
%\entry{单摆}{Pend}
%\entry{傅科摆}{Fouclt}
%\entry{惯性力}{Iner}
%\entry{离心力}{Centri}
%\entry{科里奥利力}{Corio}
%\entry{地球表面的科里奥利力}{ErthCf}
%
%\chapter{质点系与刚体}
%%------------------------------------------------------------------------------
%
%% 未完成:用一个词条专门介绍一些物理量的定义,如密度,流密度
%
%\entry{质点系}{PSys}
%\entry{质心\ 质心系}{CM}
%\entry{刚体}{RigBd}
%\entry{二体系统}{TwoBD}
%\entry{二体碰撞}{TwoCld}
%\entry{质点系的动量}{SysP}
%\entry{动量定理\ 动量守恒}{PLaw}
%\entry{质点系的动能\ 柯尼西定理}{Konig}
%\entry{力矩}{Torque}
%\entry{刚体的静力平衡}{RBSt}
%\entry{角动量}{AngMom}
%\entry{角动量定理\ 角动量守恒}{AMLaw}
%\entry{刚体的绕轴转动\ 转动惯量}{RigRot}
%\entry{常见几何体的转动惯量}{ExMI}
%\entry{刚体的运动方程}{RBEM}
%\entry{浮力}{Buoy}
%
%\chapter{振动与波动}
%%------------------------------------------------------------------------------
%% 未完成:应该把简谐运动放到这里
%\entry{振动的指数形式}{VbExp}
%\entry{受阻简谐振子}{SHOf}
%\entry{简谐振子受迫运动}{SHOfF}
%\entry{平面波的复数表示}{CPWave}
%% 未完成 \entry{一维波动方程}
%% 未完成 边界条件 (两条密度不同的绳子)
%
%\chapter{天体运动与中心力场}
%\entry{万有引力\ 引力势能}{Gravty}
%\entry{开普勒三定律}{Keple}
%\entry{开普勒问题}{CelBd}
%\entry{反开普勒问题}{InvKep}
%% 散射, 定义微分截面 % d sigma/d Omega 真的就是面积元比对应的立体角元
%\entry{罗瑟福散射}{RuthSc}
%\entry{开普勒第一定律的证明}{Keple1}
%\entry{开普勒第二定律的证明}{Keple2}
%\entry{开普勒第三定律的证明}{Keple3}
%\entry{比耐公式}{Binet}
%
%% 未完成: 三体问题
%
%%=======================================
%
%\part{其他分册预览}
%
%\chapter{数学}
%\Entry{堆放排列组合}{StackC}\newpage
%\Entry{雅可比行列式}{JcbDet}\newpage
%\Entry{$\Gamma$ 函数}{Gamma}\newpage
%\Entry{高斯分布(正态分布)}{GausPD}\newpage
%\Entry{多维球体的体积}{NSphV}\newpage
%\Entry{柱坐标系中的拉普拉斯方程}{CylLap}\newpage
%\Entry{球坐标系中的梯度散度旋度及拉普拉斯算符}{SphNab}\newpage
%\Entry{勒让德多项式的生成函数}{Legen}\newpage
%\Entry{球谐函数}{SphHar}\newpage
%\Entry{贝赛尔函数}{Bessel}\newpage
%\Entry{球贝塞尔函数}{SphBsl}\newpage
%\Entry{傅里叶变换(指数)}{FTExp}\newpage
%\Entry{离散傅里叶变换}{DFT}\newpage
%\Entry{证明闭合曲面的法向量面积分为零}{CSI0}\newpage
%\Entry{平均值的不确定度}{MeanS}\newpage
%%------------------------------------------------------------------------------
%\chapter{理论力学}
%\Entry{经典力学笔记}{ClsMec}\newpage
%\Entry{拉格朗日方程}{Lagrng}
%\Entry{哈密顿原理}{HamPrn}
%
%%------------------------------------------------------------------------------
%\chapter{电动力学}
%\Entry{电流}{I}
%\Entry{电荷守恒\ 电流连续性方程}{ChgCsv}\newpage
%\Entry{LC 振荡电路}{LC}\newpage
%\Entry{比奥萨伐尔定律}{BioSav}\newpage
%\Entry{洛伦兹力}{Lorenz}\newpage
%\Entry{磁场的能量}{BEng}\newpage
%\Entry{磁通量的定义}{BFlux}\newpage
%\Entry{磁旋比 \  玻尔磁子}{BohMag}\newpage
%\Entry{安培力}{FAmp}\newpage
%\Entry{磁场中闭合电流的合力}{EBLoop}\newpage
%\Entry{闭合电流在磁场中的力矩}{EBTorq}\newpage
%\Entry{电磁场的动量守恒 \  动量流密度张量}{EBP}\newpage
%\Entry{电场的高斯定理}{EGauss}\newpage
%\Entry{法拉第电磁感应定律}{FaraEB}\newpage
%\Entry{电磁场的能量守恒\ 坡印廷矢量}{EBS}\newpage
%\Entry{非齐次亥姆霍兹方程\ 推迟势}{RetPot}\newpage
%\Entry{电场波动方程}{EWEq}\newpage
%\Entry{介质中的波动方程}{MedWF}\newpage
%\Entry{菲涅尔公式}{Fresnl}\newpage
%\Entry{盒中的电磁波}{EBBox}\newpage
%\Entry{拉格朗日电磁势}{EMLagP}\newpage
%
%
%%------------------------------------------------------------------------------
%\chapter{量子力学}
%% 考虑一下选取什么样的讲解顺序? 大概就是先介绍线性代数(包括离散和连续的矢量空间,狄拉%克符号),介绍量子力学的基本公设(本征方程,算符,时间演化三步法),德布罗意波,然后%推导平均值公式,位置动量表象的变换,束缚态,散射,等等
%%
%% 未完成 讲讲为什么定态波函数一定是实数函数? 实数波函数为什么会有动量平均值为零?
%% 未完成 x,p不确定原理,一般的不确定原理, 高斯波包时取等号.
%% 未完成 束缚态的一般性质: 节点数, 对称性 (偶势能的基态是偶函数), 简并性(一维情况不简并)
%
%\Entry{原子单位}{AU}\newpage
%\Entry{电子轨道与元素周期表}{Ptable}\newpage
%\Entry{玻尔原子模型}{BohrMd}\newpage
%\Entry{类氢原子的约化质量}{HRMass}\newpage
%\Entry{康普顿散射}{Comptn}\newpage
%\Entry{概率流密度}{PrbJ}\newpage
%\Entry{能量归一化}{EngNor}\newpage
%\Entry{氢原子基态的波函数}{HWF0}\newpage
%\Entry{算符对易与共同本征函数}{Commut}\newpage
%\Entry{算符的矩阵表示}{OpMat}\newpage
%\Entry{无限深势阱}{ISW}\newpage
%\Entry{有限深球势阱}{FiSph}\newpage
%\Entry{升降算符}{RLop}\newpage
%\Entry{简谐振子(升降算符)}{QSHOop}\newpage
%\Entry{简谐振子升降算符归一化}{RLNor}\newpage
%\Entry{简谐振子升降算符归一化}{QSHOnr}\newpage
%\Entry{简谐振子(级数)}{QSHOxn}\newpage
%\Entry{高斯波包}{GausWP}\newpage
%\Entry{轨道角动量}{QOrbAM}\newpage
%\Entry{轨道角动量升降算符归一化}{QLNorm}\newpage
%\Entry{自旋角动量}{Spin}\newpage
%\Entry{直积空间}{DirPro}\newpage
%\Entry{角动量加法}{AMAdd}\newpage
%\Entry{球坐标和柱坐标中的径向方程}{RadSE}\newpage
%\Entry{数值解薛定谔方程}{NSES}\newpage
%\Entry{氢原子的波函数}{HWF}\newpage
%\Entry{含时微扰理论}{TDPT}\newpage
%\Entry{几种含时微扰}{TDPEx}\newpage
%\Entry{含连续态的微扰理论}{PTCont}\newpage
%\Entry{量子散射\ 分波展开}{ParWav}\newpage
%\Entry{波恩近似(散射)}{BornSc}\newpage
%
%\Entry{三维简谐振子(球坐标)}{SHOSph}
%
%
%\chapter{热学与统计力学}
%\Entry{统计力学公式大全}{StatEq}\newpage
%\Entry{分子平均碰壁数}{AvgHit}\newpage
%\Entry{相空间}{PhSpace}\newpage
%\Entry{理想气体的状态密度(相空间)}{IdSDp}\newpage
%\Entry{理想气体单粒子能级密度}{IdED1}\newpage
%\Entry{理想气体(微正则系综法)}{IdNCE}\newpage
%\Entry{正则系宗法}{CEsb}\newpage
%\Entry{理想气体(正则系宗法)}{IdCE}\newpage
%\Entry{理想气体(巨正则系综法)}{IdMCE}\newpage
%\Entry{等间隔能级系统(正则系宗)}{EqCE}\newpage
%\Entry{巨正则系综法}{MCEsb}\newpage
%\Entry{量子气体(单能级巨正则系综法)}{QGs1ME}\newpage
%\Entry{量子气体(巨正则系宗)}{QGsME}\newpage
%
%\Entry{物理学常数定义}{Consts}\newpage
%
%\part{小时物理笔记}
%\Entry{电磁场角动量分解}{EMAMSp}\newpage
%\Entry{晶体衍射}{CrysDf}\newpage
%\Entry{Hartree-Fork 方法}{HarFor}\newpage
%\Entry{Lanczos 算法}{Lanc}\newpage
%%=======================================
%\entry{本书格式规范}{Sample}

\end{document}


