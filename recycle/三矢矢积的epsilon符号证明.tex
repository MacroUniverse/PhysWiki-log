% 用 epsilon 符号证明

把三种混合积都展开成分量的形式,进行对比即可.但这里介绍如何用求和符号证明.

叉乘可以用行列式表示\upref{Deter3} 进而用求和符号和 $\varepsilon$ 符号表示(见三阶行列式\upref{Deter3})
\begin{equation}
\vec A \cross \vec B = \sum_{i=1}^3 \sum_{j=1}^3 \sum_{k = 1}^3 \varepsilon_{ijk} {\uvec u}_i A_j B_k
\end{equation}
$\vec A \cross \vec B$ 的第 $i$ 个分量为
\begin{equation}
(\vec A \cross \vec B)_i = \sum_{j=1}^3 \sum_{k = 1}^3 {\varepsilon_{ijk}} A_j B_k
\end{equation}
所以
 \begin{equation}
\vec A \cross \vec B \vdot \vec C = \sum_{i=1}^3 \qty(C_i\sum_{j=1}^3 \sum_{k = 1}^3 \varepsilon_{ijk} A_j B_k)  = \sum_{i=1}^3 \sum_{j=1}^3 \sum_{k=1}^3 \varepsilon_{ijk} C_i A_j B_k
\end{equation} 
把右边的 $i,j,k$ 分别换成 $k,i,j$, 得
 \begin{equation}
\vec A \cross \vec B \vdot \vec C = \sum_{k=1}^3 \sum_{i=1}^3 \sum_{j=1}^3 \varepsilon_{kij} A_i B_j C_k
\end{equation}  
由 $\varepsilon$ 函数的性质,$\varepsilon_{kij} = \varepsilon _{ijk}$, 所以
 \begin{equation}
\vec A \cross \vec B \vdot \vec C = \sum_{i=1}^3 \sum_{j=1}^3 \sum_{k=1}^3 \varepsilon _{ijk}A_i B_j C_k}
\end{equation}   
这种求和符号的方式一眼看去比展开式还要复杂抽象,但却在矢量分析的一些复杂证明中很有用,在张量分析中更是扮演主要角色.
