% 用 epsilon 符号证明

把三种混合积都展开成分量的形式,进行对比即可.但这里介绍如何用求和符号证明.

叉乘可以用行列式表示\upref{Deter3} 进而用求和符号和 $\varepsilon$ 符号表示(见三阶行列式\upref{Deter3})
\begin{equation}
\vec A \cross \vec B = \sum\limits_{i = 1}^3 {\sum\limits_{j = 1}^3 {\sum\limits_{k = 1}^3 {{\varepsilon _{ijk}}{{\uvec u}_i}{A_j}{B_k}} } } 
\end{equation}
 $\vec A \cross \vec B$ 的第 $i$ 个分量为
 \begin{equation}
{\left( {\vec A \cross \vec B} \right)_i} = \sum\limits_{j = 1}^3 {\sum\limits_{k = 1}^3 {{\varepsilon _{ijk}}{A_j}{B_k}} } 
\end{equation}
所以
 \begin{equation}
\vec A \cross \vec B \vdot \vec C = \sum\limits_{i = 1}^3 {\left( {{C_i}\sum\limits_{j = 1}^3 {\sum\limits_{k = 1}^3 {{\varepsilon _{ijk}}{A_j}{B_k}} } } \right)}  = \sum\limits_{i = 1}^3 {\sum\limits_{j = 1}^3 {\sum\limits_{k = 1}^3 {{\varepsilon _{ijk}}{C_i}{A_j}{B_k}} } } 
\end{equation} 
把右边的 $i,j,k$ 分别换成 $k,i,j$, 得
 \begin{equation}
\vec A \cross \vec B \vdot \vec C = \sum\limits_{k = 1}^3 {\sum\limits_{i = 1}^3 {\sum\limits_{j = 1}^3 {{\varepsilon _{kij}}{A_i}{B_j}{C_k}} } } 
\end{equation}  
由 $\varepsilon$ 函数的性质,${\varepsilon _{kij}} = {\varepsilon _{ijk}}$, 所以
 \begin{equation}
\vec A \cross \vec B \vdot \vec C = \sum\limits_{i = 1}^3 {\sum\limits_{j = 1}^3 {\sum\limits_{k = 1}^3 {{\varepsilon _{ijk}}{A_i}{B_j}{C_k}} } }  
\end{equation}   
这种求和符号的方式一眼看去比展开式还要复杂抽象,但却在矢量分析的一些复杂证明中很有用,在张量分析中更是扮演主要角色.
