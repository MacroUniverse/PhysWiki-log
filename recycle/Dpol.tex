%极坐标中的矢量偏导

\pentry{矢量场\upref{Vfield}, 极坐标系中单位矢量的偏导\upref{Dpol1}}

在极坐标中必须注意的是, $\uvec r$ 与 $\uvec \theta $ 都是坐标的函数,所以一个矢量在求导时,并不一定是分别对其分量求导.例如,平面矢量场在极坐标下可以表示为
\begin{equation}\label{Dpol_eq1}
\vec v\left( {r,\theta } \right) = f\left( {r,\theta } \right)\,\uvec r + g\left( {r,\theta } \right)\,\uvec \theta 
\end{equation}
则 的两个偏导数为
\begin{equation}\label{Dpol_eq2}
\pdv{\vec v}{r} = \pdv{r} \left( {f\,\uvec r + g\,\uvec \theta } \right) = \pdv{f}{r}\,\uvec r + f\pdv{\uvec r}{r} + \pdv{g}{r}\,\uvec \theta  + \theta \pdv{\uvec \theta}{r}
\end{equation}
\begin{equation}\label{Dpol_eq3}
\pdv{\vec v}{\theta} = \pdv{\theta} \left( {f\,\uvec r + g\,\uvec \theta } \right) = \pdv{f}{\theta}\,\uvec r + f\pdv{\uvec r}{\theta} + \pdv{g}{\theta}\,\uvec \theta  + g\pdv{\uvec \theta}{\theta}
\end{equation}
根据极坐标系中单位矢量的偏导\upref{Dpol1}中的结论
\begin{equation}
\left\{ \begin{aligned}
\pdv{\uvec r}{r} &= 0\\
\pdv{\uvec r}{\theta} &= \,\uvec \theta 
\end{aligned} \right.
\qquad
\left\{ \begin{aligned}
\pdv{\uvec \theta}{r} &= 0\\
\pdv{\uvec \theta}{\theta} &=  - \,\uvec r
\end{aligned} \right.
\end{equation}
所以\autoref{Dpol_eq2} 和\autoref{Dpol_eq3} 可以化为
\begin{equation}
\pdv{\vec v}{r} = \pdv{f}{r}\,\uvec r + \pdv{g}{r}\,\uvec \theta 
\end{equation}
\begin{equation}\begin{aligned}
\pdv{\vec v}{\theta} &= \pdv{f}{\theta}\,\uvec r + f\pdv{\uvec r}{\theta} + \pdv{g}{\theta}\,\uvec \theta  + g\pdv{\uvec \theta}{\theta}\\
 &= \pdv{f}{\theta}\,\uvec r + f\,\uvec \theta + \pdv{g}{\theta}\,\uvec \theta  - g\,\uvec r\\
 &= \left( {\pdv{f}{\theta} - g} \right)\uvec r + \left( {f + \pdv{g}{\theta}} \right)\uvec \theta 
\end{aligned}\end{equation}
 
