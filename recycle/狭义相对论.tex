%\documentstyle[11pt,amssymb,epsf]{article}
\documentclass[11pt]{article}
\usepackage{amsmath,amssymb,color,graphics,epsfig,cite}
%\documentclass[12pt,prl,aps,superscriptaddress]{revtex}
%\documentclass[aps,twocolumn,12pt,prl,superscriptaddress,nobibnotes]{revtex4}
%\usepackage{amsmath,amssymb,epsf}

%%%%% change page size and line spacing %%%%
\textwidth=6.0in \hoffset=-.55in \textheight=9in \voffset=-.8in
\def\baselinestretch{1.4}
\usepackage{amsfonts}
\usepackage{CJK}
\usepackage{float}
%%%%%%%%%%%%%%%%%%%%%%%%%%%%%%%%%%%%%%%%%%%%
\def\td{{\rm total~derivative}}
%%%%%%%%%%%%%%%%%%%%%%%%%%%%%%%%%%%%%%%%%%%
\newcommand{\hoch}[1]{$\, ^{#1}$}

%%%%%%%%%%%%%%%%%%%%%%%%%%%%%%%%%%%%%%%%%%%%%%%%%%%%%%%%%%%%%%%%%%%%%%%%

%\makeatletter
%\@addtoreset{equation}{section}
%\makeatother
%\renewcommand{\theequation}{\thesection.\arabic{equation}}


\newcommand{\be}{\begin{equation}}
\newcommand{\ee}{\end{equation}}
\newcommand{\bea}{\setlength\arraycolsep{2pt} \begin{eqnarray}}
\newcommand{\eea}{\end{eqnarray}}
\newcommand{\nn}{\nonumber}

\def\cramp{\medmuskip = 2mu plus 1mu minus 2mu}
\def\cramper{\medmuskip = 2mu plus 1mu minus 1mu}
\def\crampest{\medmuskip = 1mu plus 1mu minus 1mu}
\def\uncramp{\medmuskip = 4mu plus 2mu minus 4mu}

\def\ft#1#2{{\textstyle{\frac{\scriptstyle #1}{\scriptstyle #2} } }}
\def\fft#1#2{{\frac{#1}{#2}}}
\def\CP{{{\mathbb C}{\mathbb P}}}
\def\0{{\sst{(0)}}}
\def\1{{\sst{(1)}}}
\def\2{{\sst{(2)}}}
\def\3{{\sst{(3)}}}
\def\4{{\sst{(4)}}}
\def\5{{\sst{(5)}}}
\def\6{{\sst{(6)}}}
\def\7{{\sst{(7)}}}
\def\8{{\sst{(8)}}}
\def\sst#1{{\scriptscriptstyle #1}}
\def\oneone{\rlap 1\mkern4mu{\rm l}}
\def\ep{{\epsilon}}
\def\del{{\partial}}
\def\ii{{\rm i}}

\def\cG{{{\cal G}}}
\def\th{{{f}}}
\def\cR{{{\cal R}}}
\def\cH{{{\cal H}}}
\def\im{{{\rm i\,}}}
\def\R{{\mathbb R}}


\thispagestyle{empty}
\numberwithin{equation}{section}
\begin{document}
\begin{CJK}{GBK}{song}
\section{时空与事件}
物理学中的事件概念实则现实事件的模型化,甚至可以不具备现实意义:一次爆炸,一声雷响也可以称为事件.所以在空间的一点和时间的一瞬间结合可以称为\textbf{事件(event)}.

如果两件事件同时同地发生,则称为\textbf{同一事件}.

在相对论中,时间与空间不在分离,而是结合成为\textbf{时空(spacetime)}作为物质运动的背景.由此,一件事件在时空中就是一个\textbf{时空点(spacetime point)}.

建立时空坐标系:以$O$为原点建立惯性坐标系$K\{t,x,y,z\}$, 设某一个事件为$p$发生在$t$时刻、$A$地,,则在时空中可表示为:
\be
p=(t,x_A,y_A,z_A)
\ee

拓展阅读:牛顿运动定律,惯性系
\section{相对性原理 伽利略变换}
\textbf{相对性原理(principle of relativity)}:所有惯性系平权,任何物理定律(不仅仅限于力学定律,也包括电磁学定律)在所有惯性系的数学表达形式都具有相同形式

设一列火车在地面上匀速运动,则火车也是惯性系,也可在火车上建立惯性坐标系,不妨设地面惯性坐标系为$K\{t,x,y,z\}$, 火车惯性坐标系为$K' \{t',x',y',z'\}$.则对同一事件$p$,不同坐标系测量可得到不同的结果
\bea
&&p=(t,x,y,z) \\
&&p=(t',x',y',z')
\eea
同一事件的两组坐标应当存在某种关系,称为坐标变换式.在牛顿力学框架下,其坐标变换式为\textbf{伽利略变换}:
\bea
&&t'=t \\
&&x'=x-vt \\
&&y'=y\\
&&z'=z
\eea
\section{光速不变原理}
真空中的光速沿任何方向,对任何惯性系都是$c$,并与光源的运动无关.
\section{闵氏时空 线元}
\section{洛伦兹变换}
结论:


在相对论中,由于光速不变原理,原来的坐标关系式伽利略变换不再适用.两个惯性系$K\{t,x,y,z\}$ 与$K' \{t',x',y',z'\}$的坐标变换式为洛伦兹变换:
\bea
&&x'=\gamma(x-vt)\\
&&y'=y\\
&&z'=z\\
&&t'=\gamma(t-\frac{v}{c^2}x)
\eea
其中$\gamma=\frac{1}{\sqrt{1-\frac{v^2}{c^2}}}$称为\textbf{洛伦兹因子}.


推导:\\

XXX

洛伦兹速度变换:
\bea
&&v_x'=\frac{v_x-u}{1-\frac{v_x u}{c^2}}\\
&&v_y'=\frac{v_y}{\gamma(1-\frac{v_x u}{c^2})}\\
&&v_z'=\frac{v_z}{\gamma(1-\frac{v_x u}{c^2})}
\eea
推导:\\
对洛伦兹变换进行微分运算可得:
\bea
&&dx'=\gamma(dx-v dt) \\
&&dy'=dy\\
&&dz'=dz\\
&&dt'=\gamma(dt-\frac{v}{c^2}dx)
\eea
所以可得:
\bea
&&v_x'=\frac{dx'}{dt'}=\frac{(dx-v dt)}{dt-\frac{v}{c^2}dx}=\frac{v_x-u}{1-\frac{v_x u}{c^2}} \\
&&v_y'=\frac{dy'}{dt'}=\frac{dy}{\gamma(dt-\frac{v}{c^2}dx)}=\frac{v_y}{\gamma(1-\frac{v_x u}{c^2})}\\
&&v_z'=\frac{dz'}{dt'}=\frac{dz}{\gamma(dt-\frac{v}{c^2}dx)}=\frac{v_z}{\gamma(1-\frac{v_x u}{c^2})}
\eea
\section{时空图}
\section{尺缩效应、动钟变慢}
时空图说明:\\
XXX\\
洛伦兹变换推导:


为了简化,设二维闵氏时空中有两个事件$p_1$、$p_2$,在$K$系(地面坐标系)测得的时空坐标为:
\bea
p_1(t_1.x_1)\\
p_2(t_2,x_2)
\eea
在$K'$系(火车坐标系,与$K$系的相对速度为$v$)中测得的时空坐标为:
\bea
p_1(t'_1,x'_1)\\
p_2(t'_2,x'_2)
\eea
\subsection{尺缩效应}
现在把事件具体化:\\
我们测量一个物体的长度,在地面上测量的时候,先用标准尺零刻度对准物体的一端,测出坐标值$x_1$,\textbf{同时地},用标准尺对准物体的另一端,读出坐标值$x_2$,相减得到物体的长度$l=x_2-x_1$.

所以两个事件(测量尺子的行为)\textbf{同时不同地},可抽象地写成:
\bea
&&\Delta t=t_2-t_1=0 \\
&&\Delta x =x_2-x_1=l
\eea
当采用洛伦兹变换把测量结果变换到$K'$系时,可得:
\bea
l'&&=x_2'-x_1' \\
&&=\gamma[x_2-x_1-v(t_2-t_1)]\\
&&=\gamma l \\
\Delta t'&&=t_2'-t_1' \\
&&=\gamma[t_2-t_1-\frac{v}{c^2}(x_2-x_1)]\\
&&=-\gamma\frac{v}{c^2}(x_2-x_1) \\
&&\not =0
\eea
其中$\gamma=\frac{1}{\sqrt{1-\frac{v^2}{c^2}}}$.

从上式可以看出,在$K$系看来,$K'$系的长度测量结果比自身参考系的长度测量结果要短,其原因为:

\textbf{在$K$系中同时事件在$K'$系中不是同时事件,所以并没有十分强的可比性}.
\subsection{钟慢效应}
我们现在测量一个时钟的快慢.先设所测时钟与标准钟走时率一致;在地面上测量的时候,先用标准钟零点对准时钟的零点(对钟),这个时刻设为$t_1$,过一段时间后,再用标准钟在\textbf{在同一地点(同地)}测量时钟的时刻$t_2$,相减得到时间差$\Delta t=t_2-t_1$

所以两个事件(测量时钟的行为)\textbf{同地不同时},可抽象地写成:
\bea
&&\Delta t=t_2-t_1 \\
&&\Delta x =x_2-x_1=0
\eea
当采用洛伦兹变换把测量结果变换到$K'$系时,可得:
\bea
\Delta t'&&=t_2'-t_1' \\
&&=\gamma[t_2-t_1-\frac{v}{c^2}(x_2-x_1)]\\
&&=\gamma \Delta t \\
\Delta x'&&=x_2'-x_1' \\
&&=\gamma[x_2-x_1-v(t_2-t_1)]\\
&&=-\gamma v(t_2-t_1) \\
&&\not =0
\eea
其中$\gamma=\frac{1}{\sqrt{1-\frac{v^2}{c^2}}}$.

从上式可以看出,在$K$系看来,$K'$系的时钟测量结果比自身参考系的时钟结果要长,称之为\textbf{钟慢效应}.其原因为:

\textbf{在$K$系中同地事件在$K'$系中不是同地事件,所以并没有十分强的可比性}.

\section{双生子佯谬}
\section{相对论力学}
在相对论中,物体的质量与其运动速度有关,其关系为:
\be
m=\frac{m_0}{\sqrt{1-\frac{v^2}{c^2}}}
\ee
推导:\\
XXX

\section{质能关系}
\subsection{动能}
在相对论中,若质点的静止质量为$m_0$,在速度为$v$时质量为$m$,则物体的动能为
\be
E_k=mc^2-m_0c^2
\ee

推导:
由相对论质量关系:
\be
m=\frac{m_0}{\sqrt{1-\frac{v^2}{c^2}}}
\ee
可得:
\be
m_0^2 c^2=m^2c^2-v^2m^2
\ee
对上式微分、移项可得:
\be
c^2dm=v^2dm+mvdv
\ee
由质点动能的定义可得:
\bea
dE_k&&=\vec{F}\cdot d \vec{s} \\
&&=\frac{d\vec{p}}{dt}\cdot \vec{v} dt\\
&&=d(m \vec{v})\cdot \vec{v} \\
&&=m \vec{v} \cdot d\vec{v}+v^2 dm \\
&&=mvdv+v^2dm\\
&&=c^2dm
\eea
当物体从速度为0加速运动,当速度为$v$时,其质量从$m_0$增至$m$,所以物体动能为
\be
E_k=\int^m_{m_0} c^2 dm=mc^2-m_0c^2
\ee
其中$m_0c^2$为物体的静质能,为物体的固有能量.

当$v\ll c$时,
\bea
E_k&&=mc^2-m_0c^2 \\
&&=\frac{m_0c^2}{\sqrt{1-\frac{v^2}{c^2}}}-m_0c^2 \\
&&\approx m_0c^2(1+\frac{v^2}{2c^2}+ \cdots)-m_0c^2 \\
&&=\frac{1}{2}m v^2
\eea
恰好回到了牛顿力学的动能形式,也从侧面证明了牛顿力学是相对论的低速近似.
\end{CJK}
\end{document}
