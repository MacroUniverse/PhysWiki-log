% 基本群
\pentry{同伦\upref{HomT1},群论}

%道路和道路积要配图.
%基本群的例子,需要配图.
%基点的改变,也需要配图.
本节我们将介绍拓扑空间上的一个重要的代数结构,这将会是我们研究同伦(或者说拓扑空间的连续变化)的有力工具.这个结构是一个群,被称为\textbf{基本群}.回忆一下,在点集拓扑中我们讨论了四类同胚不变量;基本群的地位是类似的,它是同伦不变量.

我们首先将简单讨论基本群的想法是怎么来的,从直观上定义一个道路运算开始,逐渐调整定义直到满足群的四个公理.

\subsection{基本群的构造过程}
\subsubsection{道路类的积}

一个自然的想法是,研究拓扑空间中的道路之间的运算,称这个运算为道路之间的积.这种运算首先要\textbf{满足封闭性},即两个道路的积还是一条道路.最简单的情况,就是两条道路首尾相连而成一条新的道路.因此,我们首先定义道路之间的积为首尾相连:

\begin{definition}{道路的积}\label{HomT3_def1}
给定拓扑空间$X$及其上的两条道路:$f, g:I\rightarrow X$,并设$f(1)=g(0)$,即$f$的终点是$g$的起点.定义$f$和$g$的积为$f*g=h$,其中\begin{equation}h=\leftgroup{f(2t),\quad &t\in[0, 1/2]\\g(2t-1),\quad &t\in[1/2, 1]}\end{equation}
显然,$h$也是一条道路,并且其起点是$f$的起点、终点是$g$的终点.
\end{definition}

如上定义的道路$f$和$g$的积,虽然满足封闭性,但是并\textbf{不满足结合性},这用一个简单的例子就可以说明:

\begin{exercise}{道路的积不满足结合性}
取通常的一维欧几里得空间$\mathbb{R}$.设$\mathbb{R}$上有三条道路,$f=t$,$g=t+1$和$h=t+2$.请验证,$(f*g)*h\not=f*(g*h)$.
\end{exercise}

简单来说,道路的积不满足结合律,是因为虽然$(f*g)*h$和$f*(g*h)$的轨迹是重合的,但是道路上的点走过这条轨迹的“速度”不一样,因此不被视作一条道路.如果我们能把$(f*g)*h$和$f*(g*h)$归入同一个等价类里,转而研究这种道路等价类上的运算,那么结合性就可以满足了.

为了定义这样的等价类,我们不使用一般的同伦,那样的范围太广了,以至于结构太简单;我们使用如下限制的同伦:

\begin{definition}{保起点和终点同伦}
如果两条道路$f, g$有相同的起点和终点,$f\overset{H}{\cong}g$,且对于任何$t\in [0, 1]$,都有$H(0, t)=f(0)=g(0)$和$H(1, t)=f(1)=g(1)$,那么我们称$f$和$g$\textbf{保起点和终点同伦},记为$f\dot{\cong}g$.
\end{definition}

\begin{definition}{道路类}
给定拓扑空间$X$.如果两条道路:$f, g:I\rightarrow X$满足$f(0)=g(0), f(1)=g(1)$(即有相同的起点和终点),并且$f\dot{\cong}g$,那么将它们归入同一个等价类.这样划分出的等价类,称为$X$上的\textbf{道路类(path class)},记为$[f]$,其中$f$是$[f]$中任意一条道路.
\end{definition}

显然,如果两条道路轨迹相同,那么它们一定是在同一个道路类里的,这就\textbf{解决了结合性的问题}.

\begin{definition}{道路类的积}
给定拓扑空间$X$和其上的两个道路类$[f]$和$[g]$.记$[f]$和$[g]$之间的积为$[f]*[g]=[f*g]$.为了方便,通常也记$[f]*[g]=[f][g]$.
\end{definition}

要注意的是,道路类的积可以省略运算符号$*$,但是道路的积不可以.因为道路本身还是一种映射,映射之间除了道路积,还有复合等运算;尽管道路之间通常没法复合,但是为了尽可能避免歧义,就不省略道路的积中的$*$了.

\subsubsection{基点和回路类}

道路类的运算仍然有缺陷,那就是并非任意道路类之间都可以作积,必须是首尾相连的道路才行.解决这个问题很简单,就是选定一个基点,只研究同时以这个基点为起点和终点的道路类就可以了.起点和终点重合的道路,称作\textbf{回路}或者\textbf{闭路(closed path)}.如果一条回路或者回路类以$x_0$为起点和终点,那么也称$x_0$是该回路或者回路类的基点.

两条回路如果保起点和终点同伦,那么也可以称它们是\textbf{保基点同伦}.

由道路连通分支的性质可知,选定基点以后,经过这个基点的回路都逃不出基点所在的道路连通分支.\textbf{因此同伦论中研究的通常是道路连通的空间}.

\begin{definition}{回路类的集合}
给定道路连通空间$X$及一个基点$x_0\in X$.记$\Omega(X, x_0)$为所有以$x_0$为基点的回路类的集合.
\end{definition}

\subsection{基本群的定义}

\begin{exercise}{基本群}\label{HomT3_exe1}
给定道路连通空间$X$及一个基点$x_0\in X$.给$\Omega(X, x_0)$赋予道路类的积作为运算.证明这个运算满足群的四条公理.

因此,$\Omega(X, x_0)$配合该运算构成一个群,记为$\pi_1(X, x_0)$,称为$X$上关于基点$x_0$的\textbf{基本群(fundamental group)}.
\end{exercise}

\autoref{HomT3_exe1} 的提示:封闭性和结合性,在构造道路类的积时已经满足了——道路类的积这个概念的引入就是为了满足这两条性质的.基本群的单位元是回路类$[e_{x_0}]$,其中$e_{x_0}: I\rightarrow\{x_0\}$,就是随着$t$变化却恒映射为$x_0$的回路;回路类$[f(t)]$的逆元为$[f(t)]^{-1}=[f(1-t)]$.

注意,严格的证明需要构造出$e_{x_0}$到$f*f^{-1}$的同伦$H$.

\subsection{基本群的例子}

\begin{example}{单连通空间}\label{HomT3_ex1}
考虑二维欧几里得空间$\mathbb{R}^2$.这个单连通空间上的所有道路之间都保起点和终点同伦,因此关于任何基点的回路都保基点同伦,换句话说,关于任何基点的回路类只有一个.因此,$\mathbb{R}^2$上关于任何基点的基本群都是平凡群(只含一个元素).

基本群是平凡群的空间,被称为\textbf{单连通空间(simply connected space)}.

任意维的欧几里得空间$\mathbb{R}^N$都是单连通的;这些空间中对应的球面$S^{N-1}$也都是单连通的.
\end{example}

\begin{example}{带一个孔洞的空间}
考虑$\mathbb{R}^2$上挖去一个点的空间$A_1=\mathbb{R}^2-\{(0,0)\}$,那么对于任何基点$x_0\in A_1$,基本群$\pi_1(A_1, x_0)=\mathbb{Z}$.其中正整数$n$对应的回路类,是顺时针绕$(0,0)$圈数为$n$的回路集合,$-n$对应的就是逆时针绕了$n$圈的回路类,$0$对应的就是没有绕$(0,0)$的回路类.当然,也可以反过来定义$n$对应逆时针绕转、$-n$对应顺时针绕转.

如果挖去的不是一个点,而是一个圆,那么基本群也是$\mathbb{Z}$.不管挖去的是一个点还是一个单连通的子集,我们都看成是挖去了一个孔洞.
\end{example}

\begin{example}{带多个孔洞的空间}
$\mathbb{R}^2$上挖去$2$个孔洞,那么考虑回路类时就要区分是绕转了哪个孔洞.这样一来,挖去两个孔洞后的空间的基本群就是自由群\upref{FreGrp}$<\{x_1, x_2\}>$.一般地,挖去$k$个孔洞以后的空间的基本群是自由群$<\{x_i|i=1, 2, \cdots, k\}>$.
\end{example}

\subsection{基点的选择}

在前面的论述中,我们总是说“任意基点”,而论述的内容和具体选择哪个基点无关.事实上,在道路连通空间中讨论基本群时,确实和基点无关.

\begin{theorem}{基点的选取不影响基本群的结构}
给定道路连通空间$X$和两个基点$x_0, y_0\in X$,则基本群$\pi_1(X, x_0)$和$\pi_1(X, y_0)$是同构的.
\end{theorem}

\textbf{证明}:

取$x_0$到$y_0$的任意道路$h$,则以$y_0$为基点的回路类$[g]$,可以对应到以$x_0$为基点的回路类$[h][g][h]^{-1}$.记这样的对应为$T_h: \{以y_0为基点的回路类\}\rightarrow\{以x_0为基点的回路类\}$,其中$T_h([g])=[h][g][h]^{-1}$.由同伦定义可知$T_h$是一个\textbf{双射}.

对于$y_0$为基点的回路类$[f]$和$[g]$,$$T_h([f][g])=[h][f][g][h^{-1}]=[h][f][h]^{-1}[h][g][h]^{-1}=T_h([f])T_h([g])$$
也就是说,回路积和映射可互换顺序,意味着$T_h$是一个\textbf{群同态}.

双射的群同态是\textbf{群同构}.

\textbf{证毕}.

既然道路连通空间中,基点的不同并不会导致基本群结构的不同,因此我们可以忽视基点的存在,而简单地称任何基点上构造的基本群都是同一个.在这种语境下,我们可以把道路连通空间的基本群记为$\pi_1(X)$.


