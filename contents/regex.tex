% 正则表达式

\pentry{文本文件与字符编码\upref{encode}}

在文本文件中搜索内容的时候, 有时候想要的是某种格式而不是某些具体的字符, 例如要搜索 “*月*日 消费 ** 元”, 找到匹配项后需要选中这几个字(以便进行替换等操作), 又或者仅选中 “元” 前面的数值(以便进行统计等). 理论上我们可以通过编程解决这个问题, 但更简单地, 可以用一种广为使用的表达式来达到同样的效果, 就是下面要介绍的\textbf{正则表达式(regular expresion)}.

正则表达式在许多软件中都被支持, 例如在常用的文本编辑器(如 VScode), 搜索软件(如 Fileseek), 和大部分编程语言(如 C++, python, Matlab)中都有很好的支持.

\subsubsection{字符匹配}
\begin{table}[ht]
\centering
\caption{字符匹配}\label{regex_tab1}
\begin{tabular}{|c|c|c|}
\hline
符号 & 说明 & 例子 \\
\hline
\verb|.| & 匹配单个任意字符, 包括空格回车等 & \verb|.at| 可以匹配 \verb|bat|, \verb|cat|, \verb|hat| 等 \\
\hline
\verb|[...]| & 匹配方括号中的任意一个字符. 如果要表示一个范围的字符可以用 \verb|-| 连接 & \verb|[bc]ase| 可以匹配 \verb|base| 和 \verb|case|; 又例如 \verb|[a-z0-9]| 可以匹配任意一个小写字母或数字 \\
\hline
\verb|[^...]| & 匹配任何除方括号中以外的字符 & 例如 \verb|[^b]ase| 不能匹配 \verb|base| 但可以匹配 \verb|case| \\
\hline
\verb|\w| & 等效于 \verb|[a-zA-Z_0-9]| &  \\
\hline
\verb|\W| & 等效于 \verb|[^a-zA-Z_0-9]| &  \\
\hline
\verb|\s| & 空白字符,等效于 \verb|[ \f\n\r\t\v]| (注意第一个字符是空格, 剩下的符号是不同功能的空格, 见下文) &  \\
\hline
\verb|\S| & 非空白字符,等效于 \verb|[^ \f\n\r\t\v]| & \\
\hline
\verb|\d| & 一个数字(digit), 等效于 \verb|[0-9]| & \\
\hline
\verb|\D| & 一个非数字, 等效于 \verb|[^0-9]| & \\
\hline
\end{tabular}
\end{table}

(未完成)
