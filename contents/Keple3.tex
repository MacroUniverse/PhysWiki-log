%开普勒第三定律的证明

\pentry{开普勒第一定律证明\upref{Keple1}, 开普勒第二定律证明\upref{Keple2}}

在开普勒第一定律证明\upref{Keple1}中,我们得出行星轨道的极坐标方程为
\begin{equation}\label{Keple3_eq1}
  r = \frac{p}{1 - e \cos \theta }
\end{equation}
其中 $p = h^2/(GM)$,  $h$ 是行星的角动量与质量之比, $G$ 是引力常数, $M$ 是中心天体的质量. $e$ 是圆锥曲线的离心率, 当 $0 \les e < 1$ 时, 轨道是椭圆, 取等号时, 轨道是圆的. 椭圆的半长轴为
\begin{equation}\label{Keple3_eq2}
a = \frac12 [r(0) + r(\pi)] = \frac{1}{1 - e^2}\frac{h^2}{GM}
\end{equation}
椭圆的半短轴为
\begin{equation}\label{Keple3_eq3}
b = a\sqrt {1 - e^2}  = \frac{1}{\sqrt {1 - e^2}}\frac{h^2}{GM}
\end{equation}
椭圆的面积为
\begin{equation}\label{Keple3_eq4}
S = \pi ab = \frac{\pi }{\sqrt{(1 - e^2)^3} }\frac{h^4}{(GM)^2}
\end{equation}
由“开普勒第二定律证明\upref{Keple2}” 中的结论, 单位时间扫过的面积为
\begin{equation}\label{Keple3_eq5}
\dv{S}{t} = \frac{L}{2m} = \frac{h}{2}
\end{equation}
其中 $L$ 是角动量, $m$ 是行星的质量. 所以行星的周期为
\begin{equation}\label{Keple3_eq6}
  T = \frac{S}{\dv*{S}{t}} = \frac{2\pi}{\sqrt{(1 - e^2)}^3}\frac{h^3}{(GM)^2}
\end{equation}
所以由\autoref{Keple3_eq2} 与\autoref{Keple3_eq6} 得 $T^2/a^3 = 4\pi ^2/(GM)$ 是一个常数, 但注意这个常数与中心天体的质量有关, 所以不同中心天体的比例常数不同. 这个比例系数不用记忆, 需要用时, 只需用高中的知识计算天体圆周运动时的特例即可. 圆作为特殊的椭圆, 其半长轴就是半径.
