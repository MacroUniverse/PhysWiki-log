% 投影算符

\pentry{内积\upref{InerPd}}

本词条只介绍有限维或可数维矢量空间的投影算符.

\begin{theorem}{正交分解}\label{projOp_the1}
令 $M$ 为 $N$ 维矢量空间 $X$ 的子空间. 那么任意 $u\in X$ 有唯一的分解
\begin{equation}\label{projOp_eq1}
u = v + w \qquad (v\in M, w\in M^\bot)
\end{equation}
其中 $M^\bot$ 是 $M$ 在 $X$ 中的正交补.
\end{theorem}

对每个子空间 $M\subseteq X$, 我们定义对应的\textbf{投影算符} $P$, 根据\autoref{projOp_eq1} 将每个 $u\in X$ 映射到 $v$.

\begin{theorem}{}
令投影算符将 $N$ 维空间 $X$ 中的矢量投影到其子空间 $A$ 中, $\qty{a_i}$ 是 $A$ 的一组正交归一基底. 那么投影算符可以表示为
\begin{equation}
P = \sum_i \ket{a_i}\bra{a_i}
\end{equation}
\end{theorem}
由该定理易证\autoref{projOp_the1}(将 $u$ 分别分解到 $M$ 和 $M^\bot$ 的正交归一基底上, 这个投影存在且唯一).



\subsection{投影算符是厄米算符}
投影算符 $P$ 是厄米算符(也叫自伴算符), 即对任意 $u, v\in X$ 满足 $\braket{u}{Pv} = \braket{Pu}{v}$.

投影算符有 $0$ 和 $1$ 两个本征值. 对应的本征矢分别是 $M$ 和 $M^\bot$ 空间中的所有矢量.

\subsubsection{证明}
对任意 $u, v\in X$, 有
\begin{equation}
\braket{u}{Pv} = \sum_i \\braket{u}{a_i}\braket{a_i}{v}
\end{equation}
\begin{equation}
\braket{Pu}{v} = \overline {\braket{v}{Pu}} = \sum_i \overline{\\braket{v}{a_i}}\overline{\braket{a_i}{u}} = \sum_i \\braket{u}{a_i}\braket{a_i}{v} = \braket{u}{Pv}
\end{equation}
证毕.
