% 有限深势阱
% 束缚态|能级|薛定谔方程|定态

\begin{issues}
\issueTODO
\end{issues}

% 束缚态的平均动量为零

% 这是区分束缚态(E<0) 和连续态(E>0)的最简单例子了.
% 当我们有势阱时, 都会有这种规律, 例如氢原子.

% 图未完成(量子力学简介里面似乎有图)

\pentry{定态薛定谔方程\upref{SchEq}}

\footnote{参考 Wikipedia \href{https://en.wikipedia.org/wiki/Finite_potential_well}{相关页面}.}本文使用原子单位. 一维定态薛定谔方程\autoref{SchEq_eq1}~\upref{SchEq}为
\begin{equation}
-\frac{1}{2m}\pdv[2]{x}\psi(x) + V(x) \psi(x) = E \psi(x)
\end{equation}
令势能函数为
\begin{equation}
V(x) = \begin{cases}
-V_0 \quad &(-L/2 \leqslant x \leqslant L/2)\\
0 \quad &(\text{其他})
\end{cases}
\end{equation}
该势能叫做\textbf{有限深势阱(finite square well)}.

有限深势阱既包含连续的本征态(散射态), 又可能包含有限个离散的束缚态. % 未完成:到底是可能还是一定?
所以有限深势阱是研究一维散射问题的一个简单模型.

\subsubsection{束缚态}
由于 $V(x)$ 是对称的, 波函数必定是奇函数或者偶函数(\autoref{SchEq_eq3}~\upref{SchEq}). 令
\begin{equation}
k = \sqrt{2mE} \qquad \kappa = \sqrt{2m(E + V_0)}
\end{equation}
第 1,3 区间的通解为 $E < V$
\begin{equation}
\psi(x) = C_1 \E^{kx} + C_2 \E^{-kx}
\end{equation}
为了让无穷远处波函可归一化, 所以
\begin{equation}
\psi_1 = A \E^{kx} \qquad \psi_3 = D\E^{-kx}
\end{equation}
第 2 区间的通解为 $E > V$
\begin{equation}
\psi_2(x) = B \cos(\kappa x) + C\sin(\kappa x)
\end{equation}

\subsubsection{奇波函数}
当波函数为奇函数时, 易得 $\psi(0) = 0$ 即 $B = 0$, 且 $A = -D$. 再考虑 $x = L/2$ 处波函数及一阶导数连续有
\begin{equation}\label{FSW_eq1}
\begin{aligned}
&C\sin(\kappa L/2) = D \exp(-kL/2)\\
&\kappa C \cos(\kappa L/2) = -kD \exp(-kL/2)
\end{aligned}
\end{equation}
其中可以把 $E$ 看成未知量, 决定 $k, \kappa$, $C,D$ 也是未知量. 两式相除得
\begin{equation}\label{FSW_eq2}
\tan(\kappa L/2) = -\kappa/k
\end{equation}
\addTODO{图未完成, \autoref{FSW_eq2} 和\autoref{FSW_eq3} 的图解}
这是一个超定方程, 可能存在解. 解出后再次代入\autoref{FSW_eq1} 可以求得比值 $D/C$.归一化即可确定 $C, D$.

\subsubsection{偶波函数}
当波函数为奇函数时, 易得 $\psi'(0) = 0$ 即 $C = 0$, 且 $A = D$. 与奇函数的情况同理得
\begin{equation}
\begin{aligned}
&B\cos(\kappa L/2) = D \exp(-kL/2)\\
&\kappa B \sin(\kappa L/2) = kD \exp(-kL/2)
\end{aligned}
\end{equation}
相除得
\begin{equation}\label{FSW_eq3}
\tan(\kappa L/2) = k/\kappa
\end{equation}

\addTODO{什么时候有束缚态什么时候没有?}
