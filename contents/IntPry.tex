% 多元函数积分和宇称
% 宇称|奇宇称|偶宇称|定积分

\pentry{定积分\upref{DefInt}}

如果一个 $N$ 维欧几里得空间中的函数 $f(\vec r) = f(x_1, x_2, \dots, x_N)$ 满足  $f(-\vec r) = f(\vec r)$, 我们就说它具有\bb{奇宇称}, 如果满足  $f(-\vec r) = -f(\vec r)$ 我们就说它有\bb{偶宇称}.

我们以下要说明的结论是: 在中心对称的定义域(即如果 $\vec r$ 在定义域中, $-\vec r$ 也在定义域中), 具有奇宇称的函数的定积分为 0, 具有偶宇称的函数的定积分可能不为 0.

对于 $N = 1, 2, 3$, 这是容易理解的, 例如一元函数 $\sin x$ 具有奇宇称, 所以在任意对称的区间 $[-a, a]$ 做定积分都为 0. 又例如二元函数 $x^3 + y^3$ 和三元函数 $sin(x + y + z)$ 也具有奇宇称. 证明的思想很简单, 做定积分时每个 $\vec r$ 处的 “微元”($\dd{x}\dd{y}\dd{z}$) 都会在 $-\vec r$ 处有一个函数值为相反数的 “微元”, 使两个微元的积分互相抵消.

\begin{exam}{极坐标中的函数的宇称}
极坐标中的函数
\begin{equation}\label{IntPry_eq1}
f(\vec r) = f(r, \theta) = r\sin(\theta)
\end{equation}
具有奇宇称(将 $\vec r$ 变为 $-\vec r$, 只需要将 $\theta$ 加上 $\pi$ 即可, 而 $\sin(\theta + \pi) = -\sin\theta$). 如果在一个圆环形区域上做定积分, 就有
\begin{equation}
\int_0^{2\pi} \int_b^a f(r, \theta) \dd{r} \dd{\theta} = 0
\end{equation}
\end{exam}

另一种情况如三维空间中的二维曲面上的积分, 如球谐函数满足(见\autoref{SphHar_eq8}\upref{SphHar})
\begin{equation}\label{IntPry_eq3}
Y_{l,m}(-\uvec r) = (-1)^l Y_{l,m}(\uvec r)
\end{equation}
也就是说当 $l$ 为奇数时球谐函数具有奇宇称, 偶数时具有偶宇称. 所以在对整个球面积分时, 前者必为零.

\subsection{宇称函数相乘}
如果两个函数 $f(\vec r)$ 和 $g(\vec r)$ 分别可能具有奇宇称或偶宇称, 它们相乘所得的函数 $h(\vec r) = f(\vec r) g(\vec r)$ 的宇称如何呢? 从定义不难证明, 如果二者宇称相同, 那么 $h(\vec r)$ 具有偶宇称, 如果一奇一偶, 则是奇宇称.

\begin{exer}{}
求球坐标系中函数 $f(\vec r) = r^{100}\sin(99\theta) Y_{l, m}(\uvec r)$ 的宇称.
\end{exer}

\begin{exer}{}
求单位球面上的积分(星号表示复共轭)
\begin{equation}
\int Y_{99, 77}^*(\uvec r) Y_{71, -30}(\uvec r) Y_{53, 20}(\uvec r) \dd{\Omega}
\end{equation}
\end{exer}
