% 全微分

\pentry{偏导数\upref{ParDer}}

以二元函数为例,在偏微分的几何意义中,若 $z = f(x,y)$ 在某点 $({x_0},{y_0})$ 附近的曲面光滑\footnote{光滑的数学定义是,各阶偏导数在区域内连续, 即没有断点.},那么如果考虑一个足够小的区域,可以把曲面近似为平面.设平面方程为
\begin{equation}
z = {c_0} + {c_x}(x - {x_0}) + {c_y}(y - {y_0})
\end{equation}
当 $x=x_0$ , $y=y_0$, 时显然有 ${c_0} = f({x_0},{y_0})$,求两个偏导,又有
\begin{equation}
{c_x} = \frac{{\partial f}}{{\partial x}} \qquad {c_y} = \frac{{\partial f}}{{\partial y}}
\end{equation}
令坐标增量为 $\Delta x \equiv x - {x_0}$ , $\Delta y \equiv y - {y_0}$,  $\Delta z \equiv z - {c_0}$,则平面方程变为
\begin{equation}
\Delta z = \frac{{\partial f}}{{\partial x}}\Delta x + \frac{{\partial f}}{{\partial x}}\Delta y
\end{equation}
令增量为无穷小,即
 \begin{equation}
\D z = \frac{{\partial f}}{{\partial x}}\D x + \frac{{\partial f}}{{\partial y}}\D y
\end{equation}
这就是\textbf{全微分}关系.全微分的意义是,从某一点开始向任意方向移动 $(\D x,\D y)$,函数的增量等于只向 $x$ 方向移动 $\D x$ 的增量加上只向 $y$ 方向移动 $\D y$ 的增量.类似地, $N$ 元函数的全微分关系为
\begin{equation}
\D z = \sum_{i = 1}^{N}\frac{{\partial f}}{{\partial {x_i}}}\D{x_i}
\end{equation}
事实上,偏微分也可以理解为是由该式定义的.

\subsection{全微分近似}
类比一元函数的微分近似\upref{diff} $\Delta y \approx \dv*{f}{x}\cdot \Delta x$, 若 $N$ 元函数各个变量的一阶偏导在一小块区域内变化不大,那么函数值的变化可近似为
\begin{equation}\label{TDiff_eq6}
\begin{aligned}
\Delta z &= f(x_1+\Delta x_1\ldots x_N + \Delta x_N) - f(x_1\ldots x_N) \\
&\approx \frac{{\partial f}}{{\partial {x_1}}}\Delta {x_1} +\ldots + \frac{{\partial f}}{{\partial {x_N}}}\Delta {x_N}
\end{aligned}\end{equation}

\begin{exam}{测量误差}
测量一个边长各不相同的长方体的体积,若三边的测量值和最大测量误差分别为 $a, \sigma_a, b, \sigma_b, c, \sigma_c$ (假设不确定度远小于边长),求体积的最大测量误差 $\sigma_V$ 及最大相对误差 $\sigma_V/V$.

类比“一元函数微分”中的\autoref{Diff_ex1}, 长方体的体积为 $V(a,b,c) = abc$, 由全微分近似得
\begin{equation}
{\sigma _V} \approx \frac{{\partial V}}{{\partial a}}{\sigma _a} + \frac{{\partial V}}{{\partial b}}{\sigma _b} + \frac{{\partial V}}{{\partial c}}{\sigma _c} = bc{\sigma _a} + ac{\sigma _b} + ab{\sigma _c}
\end{equation}
相对不确定度为
\begin{equation}
\frac{{{\sigma _V}}}{V} \approx \frac{{{\sigma _a}}}{a} + \frac{{{\sigma _b}}}{b} + \frac{{{\sigma _c}}}{c}
\end{equation}

\end{exam}