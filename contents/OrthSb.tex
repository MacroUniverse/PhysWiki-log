% 正交子空间
% 正交|子空间|内积|矢量空间|高等代数

\pentry{子空间\upref{SubSpc}, 内积\upref{InerPd}}

在一个定义了内积的矢量空间 $V$ 中, 若存在两个子空间 $X$ 和 $Y$ 使得 $X$ 空间中的任意矢量 $x$ 和 $Y$ 空间中的任意矢量 $y$ 都正交($\braket{x}{y} = 0$), 我们就说 $X$ 和 $Y$ 是正交的.

构造正交子空间的一种简单的方法是, 在 $V$ 中找到两组矢量 $x_1, \dots, x_m$ 和 $y_1, \dots, y_m$, 确保对任意 $x_i$ 和 $y_j$ 正交, 那么 $x_1, \dots, x_m$ 张成的子空间必定和 $y_1, \dots, y_m$ 张成的子空间正交.

\begin{example}{}
三维几何矢量空间中, 建立直角坐标系, 那么 $\uvec x$ 和 $\uvec y$ 张成的二维矢量空间(平面)与 $\uvec z$ 张成的一维矢量空间(直线)正交.
\end{example}

\begin{example}{}
虽然 $xy$ 平面和 $xz$ 平面是两个正交的平面, 但它们并不是两个正交子空间. 例如矢量 $\uvec x$ 是两个平面共同的矢量, 但 $\uvec x$ 和它本身不正交.
\end{example}

若两个正交子空间的维数分别为 $N_1$ 和 $N_2$, 它们之和等于母空间的维数 $N$, 那么就说它们是\textbf{互补(complementary)}的. 若分别在这两个空间中取一组基底, 那么将他们合并起来就得到了母空间中的一组基底.
