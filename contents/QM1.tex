% 量子力学的基本假设

% 未完成: 考虑将“量子力学” 词条中与基本假设无关的内容分到另一个词条中。

\pentry{量子力学\upref{QM0}, 厄米矩阵的本征值问题}

“量子力学\upref{QM0}” 中我们已经简单介绍了量子力学的基本假设。 这里我们来进行更详细的说明, 注意我们仍然只讨论做一维直线运动的单个微观粒子。 首先来做一个总结
\begin{itemize}
\item 粒子的状态由波函数描述。
\item 波函数按照薛定谔方程随时间变化。
\item 某时刻若对粒子测量一个物理量, 则先找到该物理量的本征函数 $\Psi_1, \Psi_2, \dots$ 以及对应的本征值, 把粒子在该时刻的波函数 $\Psi(x, t)$ 表示成这些本征函数的线性叠加 $\Psi = C_1 \Psi_1 + C_2 \Psi_2\dots$, 测得第 $i$ 个本征值的概率就是系数 $C_i$ 的模长平方。 本征值有可能是连续的或是离散的。
\item 测量完后, 粒子的波函数坍缩为第 $i$ 个本征函数。
\end{itemize}

\subsection{算符和本征问题}
之前提到, 位置的本征函数是一些无穷窄的函数, 叫做 $\delta$ 函数, 对应的位置本征值则是这些 $\delta$ 函数所在的位置。 动量的本征函数是一些平面波, 对应的动量本征值就是平面波的空间频率乘以一个常数。 然而我们并没有说明某个物理量的本征波函数是怎么得到的, 以下将进一步介绍。

在量子力学中, 每个物理量都可以对应一个\bb{算符}, 算符作用在波函数上可以得到一个新的波函数。 例如某时刻波函数为 $\sin x$, 求导算符 $\dv*{x}$ 作用在 $\sin x$ 上就得到一个新的波函数 $\cos x$。 又例如坐标 $x$ 也可以作为一个算符, 我们定义将其作用在任意波函数 $\Psi(x, t)$ 上, 就是将其相乘, 即 $x\Psi(x, t)$。 又例如任意函数 $f(x)$ 也可以是一个算符, 我们定义将其作用在 $\Psi(x, t)$ 上得 $f(x)\Psi(x, t)$。

在书写习惯上, 我们将某物理量 $Q$ 的算符用 $\Q \Omega$ 表示, 如位置的算符用 $\Q x$ 表示, 动量的算符用 $\Q p$ 表示。 当我们熟练以后, 为了书写简洁往往将 “$\hat{\phantom{x}}$” 符号省略。

要得到某个物理量的本征函数, 我们需要解本征方程
\begin{equation}
\hat Q \psi(x) = \lambda \psi(x)
\end{equation}
其中 $\lambda$ 是本征值, $\psi(x)$ 是本征函数, 二者都是未知的。 如果本征值是离散的, 我们就可以用整数角标来 $i$ (来区分不同的本征值和本征函数, 将它们分别记为 $\lambda_i$ 和 $\psi_i(x)$。 注意本征函数不含时间变量 $t$。

我们姑且认为, 量子力学的基本假设规定\footnote{严格来说这并不是基本假设的一部分, 但现阶段这么认为并没有大碍。 说实话, 本书并不打算深究这一点。 另外, 算符在不同的波函数\bb{表象}下具有不同的形式, 这里使用的是\bb{位置表象}。 表象的概念以后会学习, 现在先不用担心。}\bb{位置的算符} $\Q x$ 是 $x$, \bb{动量的算符} $\Q p$ 是微分算符 $-\I\hbar \pdv*{x}$。 其中 $\I$ 是虚数单位, $\hbar$ 是一个常数, 叫做\bb{约化普朗克常数}, 即普朗克常数 $h$ 除以 $2\pi$。

可以定性地验证位于坐标 $x_0$ 处的 $\delta$ 函数是 $\Q x$ 的本征矢, 且本征值为 $x_0$。 我们把位于原点处的 $\delta$ 函数记为 $\delta(x)$, % 图未完成
那么 $x_0$ 处的 $\delta$ 函数就是 $\delta (x - x_0)$。 将本征函数和本征值代入本征方程, 得
\begin{equation}
x \delta(x - x_0) = x_0 \delta(x - x_0)
\end{equation}
我们可以从函数图像上对该式做一个定性证明: 由于 $\delta(x - x_0)$ 只有在 $x_0$ 处一个无穷窄的区间不为零, 所以将 $\delta$ 函数乘以 $x$, 就相当于在这个不为零的无穷窄区间乘以 $x_0$。

动量的本征方程为
\begin{equation}
\Q p \psi(x) = -\I\hbar \pdv{x} \psi(x) = p \psi(x)
\end{equation}
代入即可证明本征函数为 $\psi(x) = \exp(\I k x)$, 对应的 动量本征值为 $p = \hbar k$。 

量子力学假设, 其他所有算符都可以通过位置和动量算符拼凑而成, 其形式与经典力学中对应物理量的形式相同。 例如, 经典力学中的动能为 $p^2/(2m)$, 那么量子力学中的动能算符(习惯上用大写字母 $T$ 表示)就是
\begin{equation}
\Q T = \frac{\Q p^2}{2m} = \frac{1}{2m}\qty(-\I \hbar\pdv{x}) \qty(-\I \hbar\pdv{x}) = -\frac{\hbar^2}{2m} \pdv[2]{x}
\end{equation}

要理解算符运算并不难, 比如这里的 $\Q p^2$ (也可以记为 $\Q p \Q p$, 即两个动量算符相乘)



