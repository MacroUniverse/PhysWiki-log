% 无穷的概念

\pentry{映射\upref{map}}

显然由于空集不含任何元素,$|\varnothing|=0$. 对两个集合求并集时,由于相同元素不重复计数,所以对于集合$A$和$B$,我们有不等式:$|A\cap B|\leqslant|A|+|B|$.

不同的集合有可能有一样多的元素,这个时候它们的基数相等.对于无穷集合,我们没法一个个数出集合中的元素,所以靠数元素数目是没法研究无穷集合的基数的.康托尔(Cantor)注意到,可以通过将两个集合的元素一一对应,从而比较集合的大小.能够建立双射的集合彼此大小一样,也就是说基数一样;对于集合$A$和$B$,如果$A$到$B$只能建立单射而无法建立满射,那么就认为$A$的基数严格小于$B$的基数;反之,如果$B$到$A$可以建立满射但不能建立双射,那么就认为$B$的基数严格大于$A$的基数.由 \textbf{Cantor-Bernstein 定理},这两个关于严格大于和小于的判断是等价的.

对于有限集合$A$,$|A|=n$,这里$n$是某个非负整数.显然,如果$A$和$B$都是有限集合,那么它们的基数之间的大小关系和对应的$n$的大小关系是一致的.

无限集合分为\textbf{可数集(countable set)}和\textbf{不可数集(uncountable set)}.对于有限集合,如果$A$是$B$的真子集,那么$|A|<|B|$严格成立;但是如果$A$是$B$的真子集且$A$是无穷集合,那么我们最多只能认为$|A|\leqslant|B|$.

可数集是指大小和整数集$\mathbb{Z}$相同的集合,也就是说,可以和全体整数进行一一对应\upref{map}. 比如,全体偶数的集合$2\mathbb{Z}$就是一个可数集,只需要令映射$f\rightarrow \mathbb{Z}=2\mathbb{Z}$满足对于任何$\mathbb{Z}$中的元素$n$,$f(n)=2n$就可以.由于全体整数的数量和全体非负整数的数量一样,可数集里的元素和非负整数也可以建立双射,这样我们就可以按照双射的连接“挨个数出”可数集里的元素,这就是可数集名称的来源.有理数集是可数的,但无理数集和实数集是不可数的.

可数集是最小的一类无穷集合,也就是说,任何无穷集合都可以满射到可数集上.

不可数集是指无法和可数集建立双射的无穷集合,它们都严格大于可数集.由 \textbf{Cantor 定理}可知,对于任何集合(不论有限无限)$A$,$|A|<|2^A|$\footnote{$2^A$指$A$的全体子集构成的集合,称作$A$的幂集.至于为什么用这个表示方法,请参考\upref{map}.}严格成立.我们把可数集的基数记为$\aleph_0$;如果集合$A$的基数为$\aleph_n$,那么定义$|2^A|=\aleph_{n+1}$. 实数集的基数被定义为$\aleph$,利用Cantor-Bernstein定理可以证明$\aleph=\aleph_1$. 

集合论中的基本难题,连续统假设,猜测在$\aleph_0$和$\aleph$之间没有别的无穷基数,即不存在一个集合,严格大于整数集而又严格小于实数集.很不幸的是,连续统假设对于目前公认的集合论公理系统(ZFC公理)是一个独立命题,即无法证明.

% 未完成: 例子: 有理数是可数的, 实数是不可数的. 例子: N 个有序整数的集合与整数存在一一对应, R^N 和 R 存在一一对应.
