% C++ 基础

Matlab 和 Python 等动态语言的缺点时运行较慢, 在高性能计算中广泛使用的只有两种语言即 C++ 和 Fortran. 虽然 Fortran 普遍被认为是一个过时的语言, 但在计算物理中, 许多人仍然在使用 Fortran, 一是因为以前遗留下的 Fortran 代码比较多, 二是一些老教授只会 Fortran.

一本在数值算法中很有名的书是 Numerical Recipes, 这本书第三版以前都使用 Fortran 或 C, 而第三版却只有 C++, 这也是本书选择介绍 C++ 而不是 Fortran 的原因之一. 本书将从 Numerical Recipes 中借鉴许多代码上的风格.

C++ 的特征实在多不胜数, 事实上无论是什么语言, 做计算物理的研究者大多会倾向于只选择一些最简单的语法来使用.

我们在这里列出本书推荐使用的 C++ 特性, 我们假设读者已经有少量的 C++ 基础.

\subsection{基础语法}
\begin{itemize}
\item 基本类型 (bool, char, int, long, long long, float, double)
\item 基本算符 (=, +, -, *, /, \%, ++, --, +=, -=, *=, /=)
\item scope (scope 内的名字在 scope 外没有定义, scope 内可以定义与 scope 外相同的名字并覆盖, 类的 destructor 会在 scope 结束时自动调用)
\item 判断(if, else if, else)
\item 循环 (for, whle)
\item 函数(函数名重载, 变量默认值, 算符函数)
\item const
\item 指针
\item 引用
\item 头文件机制 (相当于原地插入头文件中的代码)
\item 多文件编译
\item one definition rule
\item inline
\item stack 和 heap
\item typedef
\item 数组
\item 动态内存管理 new, delete
\end{itemize}

\subsection{标准库}
\begin{itemize}
\item cmath
\item complex
\item vector
\item string
\item iostream (cin, cout, << 算符, >> 算符)
\item fstream
\item string32
\end{itemize}

较高级的语法
\begin{itemize}
\item namespace
\item 编译时和运行时
\item constexpr
\item constexpr 函数
\item if constexpr
\item 函数模板
\item 类(constructor, destructor, operator=)
\item 类的继承
\item 类模板
\item 宏 (include, define, if, ifdef, ifndef, else, endif)
\end{itemize}
