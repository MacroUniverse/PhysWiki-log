% 首先介绍二阶和三阶行列式,因为最常用! 然后介绍一般行列式的计算法则及性质(必须是以后会用到的,否则不写!

行列式是线性代数中的一个重要工具,主要用于判断方阵中的所有列向量的线性无关\footnote{充分必要条件是所有行向量也线性无关}.%未完成: 证明
行列式运算的结果是一个数,若结果不为零,则线性无关,为零则线性相关.物理中经常出现的是二阶和三阶行列式,高阶行列式的计算较为复杂(见词条最后),不必记忆其算法,可通过数学软件\footnote{详见 Matlab,Mathematica 和 Wolfram Alpha 的计算方法.} %未完成: 链接
计算. 

\subsection{二阶行列式的定义}
\begin{equation}\label{Deter_eq1}
\begin{vmatrix}
a_{11} & a_{12} \\
a_{21} & a_{22}
\end{vmatrix} = a_{11}a_{22} - a_{12}a_{21}
\end{equation}

\subsection{三阶行列式的定义}
\begin{equation}\label{Deter_eq2}
\begin{vmatrix}
a_{11} & a_{12} & a_{13} \\
a_{21} & a_{22} & a_{23} \\
a_{31} & a_{32} & a_{33}
\end{vmatrix}
= \left\{ \begin{aligned}
&a_{11}a_{22}a_{33} + a_{12}a_{23}a_{31} + a_{13}a_{21}a_{32} \\
- &a_{11}a_{23}a_{32} - a_{12}a_{21}a_{33} - a_{13}a_{22}a_{31}
\end{aligned} \right.
\end{equation}

\subsection{几何理解}
$N$ 阶行列式是 $N$ 维空间中平行体的体积,平行体由矩阵的列矢量(或行矢量)定义. 例如,二阶行列式代表一个平行四边形的面积(二维体积),平行四边形的四个顶点坐标分别为 $(0,0)$,  $(a_{11},a_{21})$,  $(a_{12},a_{22})$ 和 $(a_{11}+a_{12}, a_{21}+a_{22})$. 用几何理解可以很形象地解释下面的性质 1,3,4.

\subsection{行列式的性质}
以下性质中,把“列”换成“行”同样成立,这是因为性质2.
\begin{enumerate}
\item 若矩阵的列矢量线性相关,行列式为零,否则不为零.
\item 矩阵转置后行列式的值不变. %未完成
\item 矩阵的任意一列乘以常数,行列式的值也要乘以该常数.
\item 把矩阵的第 $i$ 列叠加上“第 $j$ 列乘任意常数”,行列式的值不变.

\end{enumerate}
以上性质证明略\footnote{见同济大学的《线性代数》}.

\subsection{高阶行列式的定义}
$N$ 阶行列式($N$ 为正整数\footnote{一阶行列式定义为 $|a_{11}|=a_{11}$, 虽然几乎从不被使用})共有 $N!$ 项,每一项都是 $N$ 个矩阵元的乘积.这 $N$ 个矩阵元的行数和列数各不相同,我们既可以在每一项中按照行标来排序,也可以按照列标,我们选用前者.排序后,行列式展开后的任意一项可记为(先不考虑前面的 $\pm$ 号)
\begin{equation}\label{Deter_eq3}
\prod_{i=1}^N a_{i,P_n(i)} = 
a_{1,P_n(1)} \vdot a_{2,P_n(2)} \dots
\end{equation}
其中列标 ${P_n}(i)$ 是数列 $1,2 \dots N$ \textbf{置换}(用某种顺序排列)后的第 $i$ 个数,显然该数列共有 $N!$ 种不同的排列,这里用 $n$ 表示第 $n$ 种排列,也表示行列式展开的第 $n$ 项.

现在来考虑\autoref{Deter_eq3} 前面的 $\pm$ 号.这由 $P_n$ 的 \textbf{逆序数} 决定,若逆序数为偶数,则前面加正号,奇数则加负号.逆序数被定义为
\begin{equation}
\sum_{i=2}^N \text{满足}\, P_n(i) < P_n(j) \,\, (j<i) \,\text{的个数} 
\end{equation}
若根据 $P_n$ 对应的符号定义数列 $S_n$ (取值 $1$ 或 $-1$),则 $N$ 阶行列式的公式为
\begin{equation}
\begin{vmatrix}
a_{11} & \cdots & a_{1n} \\
\vdots & \ddots & \vdots \\
a_{n1} & \cdots & a_{nn}
\end{vmatrix}
= \sum_{n=1}^{N!} S_n \prod_{i=1}^N a_{i,P_n(i)}
\end{equation}


