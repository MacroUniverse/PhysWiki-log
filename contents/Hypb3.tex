% 双曲线的三种等效定义

双曲线的极坐标方程为($e>1$)
\begin{equation}
r = \frac{p}{{1 - e\cos \theta }}
\end{equation}
以相同的原点建立直角坐标系,$r = \sqrt {{x^2} + {y^2}}$,$r\cos \theta  = x$,得
\begin{equation}
\sqrt {{x^2} + {y^2}}  = p + ex
\end{equation}
两边平方且化简得
\begin{equation}
\frac{{{{({e^2} - 1)}^2}}}{{{p^2}}}{\left( {x + \frac{{ep}}{{{e^2} - 1}}} \right)^2} + \frac{{1 - {e^2}}}{{{p^2}}}{y^2} = 1
\end{equation}
对比直角坐标方程
\begin{equation}
\frac{{{x^2}}}{{{a^2}}} - \frac{{{y^2}}}{{{b^2}}} = 1
\end{equation}
得(直角坐标系沿 $x$ 轴移动了 $c$)
\begin{equation}
a = \frac{p}{{{e^2} - 1}} \quad  b = \frac{p}{{\sqrt {{e^2} - 1} }} \quad c = \frac{{ep}}{{{e^2} - 1}}
\end{equation}
\begin{equation}
c^2 = a^2 + b^2
\end{equation}
用 $a, b, c$ 表示 $e,p$ 有
\begin{equation}
e = \frac{c}{a} \qquad p = \frac{{{b^2}}}{a}
\end{equation}
由离心率的定义,双曲线的焦点到准线的距离为 $p/e=b^2/c$,准线的坐标为 $c-p/e = a^2/c$.由对称性,双曲线有两个焦点和两条准线,任意一个焦点到双曲线两支的任意一点比上该点到焦点同侧准线的距离都等于离心率.


\subsection{渐近线}
当 $x,y\to \infty$ 时1可以忽略不计,有 $y/x = \pm b/a$,渐近线与 $x$ 轴夹角为
\begin{equation}
\theta_0 = \arctan(b/a)
\end{equation}
两条渐近线到两个焦点的距离都为
\begin{equation}
c\sin\theta_0 = c\cross b/c = b
\end{equation}

\subsection{两个分支}
若取右焦点建立极坐标系,当 $\abs{\theta}<\theta_0$ 时 $r<0$,是双曲线的左支,当 $\abs{\theta}>\theta_0$ 时 $r>0$ 是双曲线的右支.若想用左焦点的极坐标表示右支,令 $\abs{\theta}<\theta_0$ 且取相反数即可.
\begin{equation}
r = \frac{p}{{e\cos \theta -1}} \qquad (\abs{\theta}<\theta_0)
\end{equation}










