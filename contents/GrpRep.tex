% 群的表示
% 表示|群论|置换群|线性空间|同态

\pentry{群作用\upref{Group3}}
\subsection{表示}

\textbf{表示(representation)}是数学中的重要部分.为了能够简洁明了地讨论数学对象的性质,我们有时候需要用一个易于理解的表达方法来描述这些对象.比如说,对于学龄前的小朋友,$1+1=2$的概念可能过于抽象,不利于理解;这个时候我们往往会使用“一个苹果加一个苹果等于两个苹果”来表示相同的概念,小朋友会更容易理解.

\textbf{表示论}常常是厚重的大部头作品.在本节中,我们仅初步介绍\textbf{群}的表示思想.

\subsection{群的表示}

最直观和易于讨论的群,莫过于置换群.事实上,当伽罗华(Galois)第一次提出群的概念的时候,并没有像我们现代理论那样高度抽象和严格;他主要都在讨论置换群的性质.遗憾的是,当年的数学家们都迷惑于伽罗华研究这种东西的意义何在.

我们使用置换群来尝试表示任意的群.

\begin{definition}{群在置换群上的表示}
设有群$G$和一个置换群$S_n$.如果存在\textbf{同态}$\phi: G\rightarrow S_n$,那么我们称$\phi$是群$G$在$S_n$上的一个\textbf{表示}.
\end{definition}

我们也可以使用线性空间来对群进行表示,利用线性空间的线性变换.

\begin{definition}{群在线性空间上的表示}
设有群$G$和一个线性空间$V$,记$V$上的全体可逆线性变换为$GL(V)$\footnote{注意$V$上全体可逆矩阵\autoref{Group_ex5}~\upref{Group} 也是这样的表达方式.矩阵可以看作是线性变换的表示.}.如果存在\textbf{同态}$\phi: G\rightarrow GL(V)$,那么我们称$\phi$是群$G$在$V$上的一个\textbf{表示}.
\end{definition}