% 地球表面的科里奥利力
\pentry{科里奥利力\upref{Corio}}

令延地轴向北的单位矢量为 $\uvec z$, 质点所在经线与赤道交点的单位矢量为 $\uvec x$, 则 $\uvec y = \uvec z\cross\uvec x$. 若质点运动的方向与所在经线的夹角为 $\phi$ (顺时针为正),运动平面的法向量与赤道平面的夹角为 $\theta$ (若把地球近似看做球形,则 $\theta$ 是质点所在纬度\footnote{但严格来说地球由于受离心力,赤道宽,两极窄.}).这样,运动平面内正北方向的单位矢量为 $\uvec z' = -\sin\theta \,\uvec x + \cos\theta\,\uvec z$, 正东方向的单位矢量为 $\uvec y$, 正上方为 $\uvec x' = \uvec y\cross\uvec z' = \sin\theta \,\uvec z + \cos\theta\,\uvec x$, 速度方向的单位矢量为
\begin{equation}
\uvec v = \sin\phi\,\uvec y + \cos\phi\,\uvec z' = -\cos\phi\sin\theta \,\uvec x + \sin\phi\,\uvec y+\cos\phi\cos\theta\,\uvec z
\end{equation}
现在可以计算科里奥利力
\begin{equation}
\vec F_{col} = 2mv\omega\,\uvec v\cross\uvec z = 2mv\omega (\cos\phi\sin\theta\,\uvec y + \sin\phi\,\uvec x)
\end{equation}
其向北,向东,向上的分量分别为
\begin{equation}
\vec F_{col}\vdot\uvec z' = -2mv\omega\sin\theta\sin\phi
\qquad
\vec F_{col}\vdot\uvec y = 2mv\omega\sin\theta\cos\phi
\end{equation}
\begin{equation}
\vec F_{col}\vdot\uvec x' = 2mv\omega\sin\phi\cos\theta
\end{equation}
可以证明水平分力可以表示为
\begin{equation}
\vec F_{col}^{\|} = (\vec F_{col}\vdot\uvec y)\uvec y + (\vec F_{col}\vdot\uvec z')\uvec z' = 2m\vec v\cross\vec\omega'
\end{equation}
其中 $\vec\omega' = \omega\sin\theta\,\uvec x'$. 可见科氏力的水平分量始终与速度垂直,且在地球的两极($\theta = pi/2$)处取最大值 $2mv\omega$, 在赤道处为 0.

要特别注意的是,地球表面的非惯性力除了科里奥利力外还有离心力,但离心力一般被地球的椭球形弥补%链接未完成
,可以不计.

% 举例: 北半球的火车
\begin{exam}{}
假设 30 吨重的高铁车厢在北纬 30 度以 300km/h 的速度行驶,其水平方向的科氏力大小为
\begin{equation}\begin{aligned}
F_{col} &= 2\cross30,000\mathrm{kg}\cross(300,000\mathrm{m}/3600\mathrm{s})\cross\frac{2\pi}{24h\cross3600s}\cross\sin\frac{\pi}{6} \\
&= 60.32\mathrm{N}
\end{aligned}\end{equation}
\end{exam}

% 未完成:一直解决不了 2mv\omega 中的 2
%\begin{exam}{傅科摆}
%假设小球以一定的初速度 $v_0$ 在水平面上\footnote{已包含了对离心力的修正}运动,
%\end{exam}
