% 二分法
% 二分法|Matlab|求零点

\textbf{二分法} 是数值计算中一种求连续一元函数零点的简单方法. 先来看一个显然的结论: 如果我们知道一个连续函数在某个开区间左端的值和右端的值(假设都不为 0)乘积小于 0(即正负号不同), 那么这个函数在该区间内必有至少一个根. 为了进一步缩小这个根所在的区间, 我们在区间中点求函数值, 如果中点处的函数值与区间左端的函数值同号(相乘大于 0), 则函数的根必然在区间中点和区间右端之间, 于是我们可以把区间左端移动到区间中点处, 再来求新区间的中点. 如果区间中点的函数值与区间右端的函数值同号, 同理我们也可以把区间右端移动到区间中点处得到新的区间. 如果区间中点的函数值恰好为 0, 我们便找到了一个根, 另一方面, 如果区间的长度小于我们对根的精度的要求, 那么我们就找到了一个根的近似值.

Matlab 中自带的 \lstinline|fzero| 函数如果按照以下格式使用, 大致可以认为是二分法
\begin{lstlisting}[language=matlabC]
>> f = @(x)x-1;
>>fzero(f, [0,2])
ans = 1
\end{lstlisting}
\lstinline|fzero| 的默认精度是 \lstinline|2.2e-16|. 注意要把一个函数作为其他函数的输入变量, 必须使用函数句柄. 下面我们给出一个二分法的 Matlab 程序.

\subsubsection{bisection.m}
\begin{lstlisting}[language=matlab]
% 二分法求函数的根
function x = bisection(f, int, err)
fl = f(int(1)); fr = f(int(2));
% 两端点是否为 0
if fl == 0
    x = int(1); return;
elseif fr == 0
    x = int(2); return;
end
% 两端点是否同号
if fl * fr > 0
    error('两端点同号');
end
% 主循环
while(int(2) - int(1) > err)
    mid = 0.5*(int(1) + int(2));
    fm = f(mid);
    if fm * fl > 0
        int(1) = mid;
    elseif fm * fr > 0
        int(2) = mid;
    else % fm = 0
        break;
    end
end
x = mid;
end
\end{lstlisting}

我们先来看函数的自变量, 与 \lstinline|fzero| 类似, 该函数的前两个输入变量分别是函数句柄和求根区间, 第三个变量是误差值, 当区间的长度小于 \lstinline|err| 时, 函数将输出区间中点作为输出. 函数的第 3-13 行做了一些必要的检查, 确保区间两端的函数值为异号. 第 15 行开始主循环, 每循环一次, 函数的区间长度减半, 直到区间中点处的函数值为 0 或区间长度小于 \lstinline|err| 时跳出循环, 最后返回区间中点的函数值.
