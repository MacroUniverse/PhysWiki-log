% 质点系的动能 柯尼西定理

\pentry{质点系\upref{PSys}, 质点系的动量\upref{SysP}}

\subsection{柯尼西定理}
某参考系中质点系的动能等于该参考系中其的质心的动能加上质心系中质点系的动能,即
\begin{equation}\label{Konig_eq1}
E_k = \frac12 Mv_c^2 + \frac12 \sum_i m_i v_{ci}^2 
\end{equation} 
其中 $M$ 是所有质点的质量和, $v_c$ 是质心系相对于当前参考系的运动速度, $m_i$ 是第 $i$ 个质点的质量, $v_{ci}$ 是第 $i$ 个质点在质心系中的速度.

\begin{exam}{圆环滚动的动能}
一个圆环在水平地面上延直线无摩擦地滚动, 其半径为 $R$, 质量为 $m$, 角速度为 $\omega$, 求地面参考系中圆环的动能.

圆环质心的速度大小为 $v_c = \omega R$, 圆环相对于圆心旋转的线速度大小处处为 $v_{ci} = \omega R$, 代入\autoref{Konig_eq1} 得动能为
\begin{equation}
E_k = \frac12 m\omega^2 R^2 + \frac12 \sum_i m_i \omega^2 R^2 = m\omega^2 R^2
\end{equation}
\end{exam}

\subsection{柯尼西定理的证明}
在当前参考系中,第 $i$ 个质点的运动速度为
\begin{equation}
\vec v_{i} = \vec v_c + \vec v_{ci}
\end{equation}
于是有
\begin{equation}
\ali{
E_k &= \frac12 \sum_i m_i \vec v_{i}^2
= \frac12 \sum_i m_i (\vec v_c + \vec v_{ci} )^2 \\
 &= \frac12 \sum_i m_i \vec v_{c}^2 + \frac12 \sum_i m_i \vec v_{ci}^2 + \sum_i m_i \vec v_c \vdot \vec v_{ci}
}\end{equation}
现在只需证明 $\sum\limits_i m_i \vec v_c \vdot \vec v_{ci} = 0$ 即可.考虑到
\begin{equation}
\sum_i m_i \vec v_c \vdot \vec v_{ci}  = \vec v_c \vdot \sum_i m_i \vec v_{ci}
\end{equation}
而质心系中的质点系动量为零(\autoref{CM_eq8}\upref{CM}), 所以
\begin{equation}
\sum_i m_i \vec v_{ci} = \vec 0
\end{equation}
证毕.

