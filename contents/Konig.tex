% 质点系的动能 柯尼西定理

\pentry{质点系\upref{PSys}, 质点系的动量\upref{SysP}}

\subsection{柯尼西定理}
某参考系中质点系的动能等于该参考系中其的质心的动能加上质心系中质点系的动能,即
\begin{equation}
{E_k} = \frac{1}{2}Mv_c^2 + \frac{1}{2}\sum_i m_i v_i^2 
\end{equation} 
其中 $M$ 是所有质点的质量和, $v_c$ 是质心系相对于当前参考系的运动速度, $m_i$ 是第 $i$ 个质点的质量, $v_i$ 是第 $i$ 个质点在质心系中的速度.

例如,在计算一个一边平移一边旋转的均匀圆盘,可以先计算它绕轴旋转时的动能,加上它不旋转只平动时的动能,就是它的总动能.

\subsection{证明}
在当前参考系中,第 $i$ 个质点的运动速度为
\begin{equation}
{\vec v_{0i}} = {\vec v_c} + {\vec v_i}
\end{equation}
于是有
\begin{equation}
\ali{
{E_k} &= \frac{1}{2}\sum_i m_i \vec v_{0i}^2
= \frac{1}{2}\sum_i m_i (\vec v_c + \vec v_i )^2 \\
 &= \frac{1}{2}\sum_i m_i \vec v_{0i}^2 + \frac{1}{2}\sum_i m_i \vec v_i^2 + \sum_i {{m_i}{{\vec v}_c} \vdot {{\vec v}_{0i}}} 
}\end{equation}
现在只需证明 $\sum\limits_i {{m_i}{{\vec v}_c} \vdot {{\vec v}_{0i}}}  = 0$ 即可.考虑到
\begin{equation}
\sum_i m_i \vec v_c \vdot \vec v_{0i}  = \vec v_c\sum_i m_i \vec v_{0i}
\end{equation}
而质心系中的质点系动量为零(\autoref{CM_eq8}\upref{CM}), 所以
\begin{equation}
\sum_i m_i \vec v_{0i} = \vec 0
\end{equation}
证毕.

