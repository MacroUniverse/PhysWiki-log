% 高斯消元法解线性方程组

\pentry{矩阵\upref{Mat}}

线性方程组(有 $x_1\dots x_n$ 这 $n$ 个未知量, 所以也叫 \bb{$\boldsymbol{n}$ 元一次方程组})
\begin{equation}\label{GAUSS_eq1}
\leftgroup{
&a_{1,1}x_1 + a_{1,2}x_2 + \dots + a_{1,n}x_n&=\quad &y_1\\
&a_{2,1}x_1 + a_{2,2}x_2 + \dots + a_{2,n}x_n&=\quad &y_2\\
&\qquad \qquad \dots  \qquad \dots \qquad  \dots\\
&a_{m,1}x_1 + a_{m,2}x_2 + \dots + a_{m,n}x_n&=\quad &y_m}\\
\end{equation}
可以写成矩阵和列矢量相乘的形式
\begin{equation}\label{GAUSS_eq2}
\mat A \vec x = \vec y
\end{equation}
其中 $\mat A$ 是维度 $m \times n$ 的矩阵,称为\bb{系数矩阵}.
\begin{equation}
\mat A=
\begin{pmatrix}
a_{1,1} &a_{1,2} &\cdots &a_{1,n} \\
a_{2,1} &a_{2,2} &\cdots &a_{2,n} \\
\vdots  &\vdots  &\ddots &\vdots  \\
a_{m,1} &a_{m,2} &\cdots &a_{m,n} \\
\end{pmatrix} 
\end{equation}
$\vec x$ 是 $n$ 维列矢量 $(x_1,x_2,\dots,x_m,\dots,x_n)\Tr$.
$\vec y$ 是 $m$ 维列矢量 $(y_1,y_2,\dots,y_m)\Tr$,称为\bb{常数列}. $\mat A$ 和 $\vec y$ 通常看做已知的,而 $\vec x$ 看做未知的, 即方程组待求的\bb{解}. 下面我们介绍一种解线性方程组的简单的方法, \bb{高斯消元法(Gauss elimination).} 先来看一个简单的例子.

\begin{exam}{}\label{GAUSS_ex1}
我们先回顾一下初中阶段如何解线性方程组, 例如
\begin{equation}\label{GAUSS_eq4}
\leftgroup{
&2x + 3y = 21\\
&5x - 2y = 5
}\end{equation}
一种方法是将第一条等式两边除以 2 再移项得到
\begin{equation}
x = - \frac32 y + \frac{21}{2}
\end{equation}
再代入第二个条式消去 $x$ 得
\begin{equation}
-\frac{19}{2} y + \frac{105}{2} = 5
\end{equation}
解得 $y = 5$, 再代入第一条等式得 $x = 3$.

另一种更方便的解法是, 将第一条等式(两边)乘以 $-5/2$, 加到第二条上消去 $x$ 得
\begin{equation}
-\frac{19}{2} y = -\frac{95}{2}
\end{equation}
解得 $y = 5$, 再代入第一条等式得 $x = 3$. 为什么可以这么做?  简单来说是因为如果两条等式都成立, 将它们两边相加得到的新的等式同样成立. 下面我们来详细讲解第二种方法.

为了书写简单, 我们可以用所谓的\bb{增广矩阵(augmented matrix)}来表示矩阵和常数列, 即把\autoref{GAUSS_eq4} 记为
\begin{equation}
\qty( \begin{array}{cc|c}
	2 & 3 & 21 \\
	5 & -2 & 5
	\end{array} 
)\end{equation}
同样把第一行乘以 $-5/2$, 加到第二行上得
\begin{equation}
\qty( \begin{array}{cc|c}
	2 & 3 & 21 \\
	0 & -19/2 & -95/2
	\end{array} 
)\end{equation}

如果这个方程有多个未知数, 且方程的数量和未知数相同(系数矩阵为方阵), 理想情况下我们可以用第一行消去所有 $i > 1$ 行的第一个系数, 再用第二行消去所有 $i > 2$ 的第二个系数, 以此类推, 最后得到一个三角形系数矩阵. 三角形系数矩阵的最后一行只有最后一个变量的系数不为零, 我们求出这个变量后, 再代入倒数第二行(只有两个未知量)求出另一个未知量, 最后就可以得到所有未知量的值.

注意这只是一种理想的情况, 如果在处理第 $i$ 行的时候发现 $a_{i,i} = 0$, 则该方法无法进行下去. 为此我们需要稍微复杂一些的方法, 也就是高斯消元法的一般步骤.
\end{exam}

\subsection{一般步骤}

我们将\autoref{GAUSS_eq1} 形式的方程组用增广矩阵表示为
\begin{equation}
\mat B=(\mat A ,\vec y)={
	\qty( \begin{array}{cccc|c}
	a_{1,1} &a_{1,2} &\cdots &a_{1,n} &y_1 \\
	a_{2,1} &a_{2,2} &\cdots &a_{2,n} &y_2 \\
	\vdots  &\vdots  &\ddots &\vdots  &\vdots \\
	a_{m,1} &a_{m,2} &\cdots &a_{m,n} &y_m \\
	\end{array} 
	)}
\end{equation}

定义以下三种矩阵(或增广矩阵)变换为\bb{初等行变换}. 初等行变换不改变方程组的解\footnote{这点我们留到以后证明}.% 未完成:讲线性方程组与矢量空间的时候说明初等行变换可以看做 y 空间的子空间的基底变换, 不影响矢量本身的映射, 所以解不变.
\begin{enumerate}
\item 对调矩阵中的第 $i$ 行与第 $j$ 行, 记作 $\vec r_i \leftrightarrow \vec r_j$

\item 将矩阵第 $i$ 行的所有元素乘以一个非零数 $k$, 记作 $\vec r_i \times k$

\item 把矩阵第 $i$ 行的所有元素乘以数 $k$ 后加到第 $j$ 行上, 记作 $\vec r_i + \vec r_j \times k$
\end{enumerate}

任何非零矩阵(或增广矩阵)经过有限次初等行变换后总可以使系数矩阵 $\mat A$ 转化为\bb{梯形矩阵(echelon form)}. 我们把梯形矩阵定义为满足该条件的矩阵: 第 $i$ 行的第一个非零系数\footnote{为了书写简单, 我们将每次变换后的系数矩阵元仍然用 $a_{i,j}$ 来表示, 虽然它的值可能已经发生变化. 若需要区分, 可以将第 $n$ 次变换后的矩阵元用 $a_{i,j}^{(n)}$ 表示.} $a_{i,q(i)}$ 的列标 $q(i)$ 总是大于第 $i-1$ 行的第一个非零元 $a_{i-1, q(i-1)}$ 的列标 $q(i-1)$. 与\autoref{GAUSS_ex1} 不同的是, 当系数矩阵经过行变换化为梯形矩阵后, 最后若干行有可能都为零, 最后一行也未必只有一个非零元.

高斯消元法的一般步骤如下:
\begin{itemize}
\item 先处理第 $i = 1$ 行, 如果 $a_{1,1} = 0$ 但某 $i' > 1$ 的行有 $a_{i', 1} \ne 0$, 就先进行行变换\footnote{如果 $a_{1,1} \ne 0$ 则不需要} $\vec r_i \leftrightarrow \vec r_{i'}$. 如果第一列全为 0, 我们就无视第  1 列, 从第 2 列重新开始, 以此类推. 记此时第 1 行第一个不为 0 的列标为 $q(1)$. 接下来做若干次行变换 $\vec r_{i'} + \vec r_1 \times k$ 使所有第 $i' > 1$ 行的 $a_{i', p(1)}$ 都为 0.

\item 依次处理第 $i = 2\dots m-1$ 行\footnote{如果在处理 $i = m-1$ 之前就得到了梯形矩阵, 则可提前终止.}. 要处理第 $i$ 行, 先令 $q(i) = q(i-1)+1$, 如果此时矩阵元 $a_{i, q(i)} = 0$, 但某 $i' > i$ 的行有 $a_{i', q(i)} \ne 0$, 就先进行行变换 $\vec r_i \leftrightarrow \vec r_{i'}$. 若不存在这样的 $i'$, 我们就改令 $q(i) = q(i-1) + 2$ 并重新开始该步骤, 以此类推. 接下来做若干次行变换 $\vec r_{i'} + \vec r_i \times k$ 使所有第 $i' > i$ 行的 $a_{i', p(i)}$ 都为 0.
\end{itemize}
完成后, 系数矩阵就会变为梯形矩阵.

\begin{exam}{解线性方程组}\label{GAUSS_ex2}
\begin{equation} % 11
\leftgroup{
&x_1  &+ &x_2 &- &x_3 &+ &x_4&=\quad &3\\
&2x_1 &+ &2x_2 &- &2x_3 &+ &x_4&=\quad &7\\
&x_1  &+ &x_2 & & &+ &2x_4&=\quad &3\\
&2x_1 &+ &2x_2 &- &x_3 &+ &5x_4&=\quad &4}
\end{equation}
解:

将该方程组写成增广矩阵形式
\begin{equation} % 12
\mat B={
	\qty( \begin{array}{cccc|c}
	1 &1 &-1 &1 &3 \\
	2 &2 &-2 &1 &7 \\
	1 &1 &0  &2 &3 \\
	2 &2 &-1 &5 &4 \\
	\end{array} 
	)}
\end{equation}
开始消元
\begin{equation} % 13
\begin{aligned}
\vec r_2 - 2 &\vec r_1 \\
\vec r_3 - 1 &\vec r_1 \\
\vec r_4 - 2 &\vec r_1 \\
\end{aligned}
\quad \Longrightarrow \quad
\mat B={
	\qty(\begin{array}{cccc|c}
	1 &1 &-1  &1   &3  \\
	0 &0 &0   &-1  &1  \\
	0 &0 &1   &1   &0  \\
	0 &0 &1   &3   &-2 \\
	\end{array} 
	)}
\end{equation}

发现此时,矩阵第二列自第二行以下全为零,所以需要依次向下一列寻找不为零的元素.继续消元
\begin{equation} % 14
\begin{aligned}
\vec r_2 \leftrightarrow &\vec r_4 \\
\vec r_3 - 1 &\vec r_2 \\
\vec r_4 - 0 &\vec r_2 \\
\end{aligned}
\quad \Longrightarrow \quad
\mat B={
	\qty(\begin{array}{cccc|c}
	1 &1 &-1  &1   &3  \\
	0 &0 &1   &3   &-2 \\
	0 &0 &0   &-2  &2  \\
	0 &0 &0   &-1  &1 \\
	\end{array} 
	)}
\end{equation}
\begin{equation} % 15
\vec r_4 - 0.5 \vec r_3
\quad \Longrightarrow \quad
\mat B={
	\qty(\begin{array}{cccc|c}
	1 &1 &-1  &1   &3  \\
	0 &0 &1   &3   &-2 \\
	0 &0 &0   &-2  &2  \\
	0 &0 &0   &0   &0  \\
	\end{array} 
	)}
\end{equation}
该方程组有无穷多组解
\begin{equation} % 16
\leftgroup{
&x_1 &=\quad &5-x_2\\
&x_3 &=\quad &1\\
&x_4 &=\quad &-1}
\qquad
\begin{aligned}
&x_2 := c \\
&\Longrightarrow \\
\end{aligned}
\qquad
\leftgroup{
&x_1 &=\quad &5-c\\
&x_2 &=\quad &c \\
&x_3 &=\quad &1\\
&x_4 &=\quad &-1}
\end{equation}
\end{exam}

% 未完成: 以下还没有修改

\subsection{解的分类}
% 未完成:如果有无穷多种解, 该如何表示它们呢? 什么是特解,什么是通解.

观察最终简化的每个具体问题的增广矩阵,很容易判断线性方程组解的情况(无解、有唯一解或有无穷多组解).若矩阵 $\mat B^{(m-1)}$ 满足 $m=n$ 且 $a_{k,k}\neq 0 \quad (k=1,2,\dots,n)$,则线性方程组 $\mat A \vec x = \vec y$ 有唯一解.利用\autoref{GAUSS_eq5} 就可以回代出这个解.



%%%%%%%%%%%%%


若系数矩阵 $\mat A$ 是一个阶梯形的矩阵,即线性方程组有如下形式:
\begin{equation}\label{GAUSS_eq17}
{
	\left[ \begin{matrix}
	a_{1,1} &a_{1,2} &a_{1,3} &\cdots &a_{1,n} \\
	        &a_{2,2} &a_{2,3} &\cdots &a_{2,n} \\
	        &        &a_{3,3} &\cdots &a_{3,n} \\
             &        &        &\ddots &\vdots  \\
	        &        &        &       &a_{n,n} \\
	\end{matrix} 
	\right ]}
{
	\left[ \begin{matrix}
	x_1 \\
	x_2 \\
     x_3 \\
     \vdots \\
	x_n \\
	\end{matrix} 
	\right ]}=
{
	\left[ \begin{matrix}
	y_1 \\
	y_2 \\
     y_3 \\
     \vdots \\
	y_n \\
	\end{matrix} 
	\right ]}
\end{equation}
当 $a_{k,k}\neq 0 \quad (k=1,2,\dots,n)$ 时,显然方程组具有唯一的一组解
\begin{equation}\label{GAUSS_eq5}
\leftgroup{
x_n &=\frac{y_n}{a_{n,n}} \\
x_k &=\frac{1}{a_{k,k}} \qty(y_k - \sum_{i=k+1}^n a_{k,i} x_i ) \quad (k=1,2,\dots,n-1) }\\
\end{equation}
特别的,若 $a_{k,k} = 1 \quad (k=1,2,\dots,n)$,则称之为线性方程组的\bb{最简形式}.

那么,对于\autoref{GAUSS_eq1} 这样一般形式的线性方程组,能否转化为\autoref{GAUSS_eq4} 这样的形式,进而求解呢?回答是肯定的. % 未完成:GAUSS_eq4 是三角形矩阵,未必所有方程组都能化为三角形,而是梯形.

