% 完结
\pentry{球坐标系的定义\upref{Sph}}

\subsection{结论}
根据球坐标的定义,可得两种坐标之间的变换关系
\begin{equation}\label{SphCar_eq1}
\left\{ \begin{aligned}
x &= r\sin \theta \cos \phi \\
y &= r\sin \theta \sin \phi \\
z &= r\cos \theta 
\end{aligned} \right.
\end{equation}
\begin{equation}\label{SphCar_eq2}
\left\{ \begin{aligned}
r &= \sqrt {x^2 + y^2 + z^2} \\
\theta  &= \arccos \frac{z}{\sqrt {{x^2} + {y^2} + {z^2}} }\\
\phi  &= \arctan \frac{y}{x}
\end{aligned} \right.
\end{equation}
以及两组基底之间的变换关系
\begin{equation}\label{SphCar_eq3}
\left\{ \begin{aligned}
\uvec r &= \sin\theta\cos\phi\,\uvec x + \sin\theta\sin\phi\,\uvec y + \cos\theta\,\uvec z\\
\uvec \theta  &= \cos\theta\cos\phi\,\uvec x + \cos\theta\sin\phi\,\uvec y - \sin\theta\,\uvec z\\
\uvec \phi  &=  - \sin\phi\,\uvec x + \cos\phi\,\uvec y
\end{aligned} \right.
\end{equation}
\begin{equation}\label{SphCar_eq4}
\left\{ \begin{aligned}
\uvec x &= \sin \theta \cos \phi \,\uvec r + \cos \theta \cos \phi \,\uvec \theta  - \sin \phi \,\uvec \phi \\
\uvec y &= \sin \theta \sin \phi \,\uvec r + \cos \theta \sin \phi \,\uvec \theta  + \cos \phi \,\uvec \phi \\
\uvec z &= \cos \theta \,\uvec r - \sin \theta \,\uvec \theta 
\end{aligned} \right.
\end{equation}
\subsection{推导}
把空间中一点 $P$ 的位矢 $r \,\uvec r$ 分解为垂直于 $xy$ 平面的分量 $z = r\cos \theta $ 和 $xy$ 平面的分量 $r\sin \theta $. 后者又可以进而分解成 $x$ 分量 和 $y$ 分量  $y = r\sin \theta \cos \phi$,  $y = r\sin \theta \sin \phi$, 这就得到了\autoref{SphCar_eq1}.

在直接坐标系中,显然有 $r = \sqrt {x^2 + y^2 + z^2}$, 代入\autoref{SphCar_eq1} 中的三条关系,就可以很容易解出\autoref{SphCar_eq2} 中的三条关系.

现在推导变换关系(\autoref{SphCar_eq3}).由于 $\uvec r,\uvec \theta ,\uvec \phi $ 都是关于 $\left( {r,\theta ,\phi } \right)$ 的函数,所以在考察一点 $\left( {r,\theta ,\phi } \right)$ 时, $\uvec r$ 的球坐标是 $\left( {1,\theta ,\phi } \right)$,  根据\autoref{SphCar_eq1} 变换到直角坐标为
\begin{equation}
(\sin \theta \cos \phi,\,\sin \theta \sin \phi,\,\cos \theta)
\end{equation}
写成矢量的形式,就是
 \begin{equation}
\uvec r = \sin \theta \cos \phi \,\uvec x + \sin \theta \sin \phi \,\uvec y + \cos \theta \,\uvec z
\end{equation}
至于\autoref{SphCar_eq3} 的第二条式子,在同一个球坐标 $(r,\theta ,\phi)$ 处, $\uvec \theta $ 的球坐标为 $\left( {1,\theta  + \pi /2,\phi } \right)$, 根据\autoref{SphCar_eq1} 变换到直角坐标再化简就得到直角坐标和对应的矢量形式为
\begin{equation}
\left( {\cos \theta \cos \phi ,\cos \theta \sin \phi , - \sin \theta } \right)
\end{equation}
 \begin{equation}
\uvec \theta  = \cos \theta \cos \phi \,\uvec x + \cos \theta \sin \phi \,\uvec y - \sin \theta \,\uvec z
\end{equation}
同理,在同一点 $\left( {r,\theta ,\phi } \right)$ 处, $\uvec \phi $ 的球坐标为 $\left( {1,\pi /2,\phi  + \pi /2} \right)$,  得到第三条式子.


下面推导变换\autoref{SphCar_eq4}. 由于已经知道了变换\autoref{SphCar_eq3}, 且直角坐标系和球坐标系中的基底都是单位正交基,所以直接把变换\autoref{SphCar_eq3} 中的系数写成 $3 \times 3$ 的矩阵形式,再转置 % 未完成: 矩阵转置
即可得到变换\autoref{SphCar_eq4} 中的系数矩阵.% 未完成: 正交基底变换,正交矩阵 等


























