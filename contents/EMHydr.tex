% 电磁场中的类氢原子

\footnote{本词条使用原子单位}假设原子核不动(无限核质量近似), 使用库仑规范得
\begin{equation}
\varphi = \frac{Z}{r}
\end{equation}
其中 $Z < 0$ 是核电荷数.

\autoref{QMEM_eq4}\upref{QMEM} 变为
\begin{equation}
H = H_0 + H_I = -\frac{1}{2m} \laplacian +  \frac{Z}{r} - \frac{\I}{m} \vec A \vdot \Nabla + \frac{1}{2m} \vec A^2
\end{equation}
其中含时哈密顿算符 $H_I$ 是后两项.

以下我们使用偶极近似, 则 $\vec A$ 只是时间的函数.

\subsection{速度规范}
 对库仑规范使用规范变换
\begin{equation}
\Psi^V(\vec r, t) =  \exp(\I \chi) \Psi(\vec r, t)
\end{equation}
\begin{equation}
\chi = \frac{1}{2m} \int^t \vec A^2(t') \dd{t'}
\end{equation}
就可以将 $\vec A^2$ 项消去
\begin{equation}
\I \pdv{t} \Psi^V = \qty[H_0 + \frac{1}{m} \vec A \vdot \vec p] \Psi^V
\end{equation}
这种规范叫做\bb{速度规范}.

\subsection{长度规范}

\begin{equation}
\Psi^L(\vec r, t) =  \exp(\I \chi) \Psi(\vec r, t)
\end{equation}
\begin{equation}
\chi = \vec A \vdot \vec r
\end{equation}
得薛定谔方程为
\begin{equation}
\I \pdv{t} \Psi^L = [H_0 + \vec{\mathcal{E}} \vdot \vec r] \Psi^L
\end{equation}
这种规范叫做\bb{长度规范}.
