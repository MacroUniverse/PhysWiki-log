\subsection{系数公式}

设函数的展开式为
\begin{equation}
f( x ) = \frac{{{a_0}}}{2} + \sum\limits_{n = 1}^{ + \infty } {{a_n}\cos nx \D x} + \sum\limits_{n = 1}^{ + \infty } {{b_n}\sin nx \D x} 
\end{equation}
可以利用三角函数系的正交性确定各项系数.例如,若要求${a_m}$ $[(m > 0)$, 则在等式两边乘以 $\cos mx$, 并积分,即
\begin{equation}\begin{split}
&\int_{ - \pi }^\pi {f( x )\cos mx \D x}\\ 
= &\int_{ - \pi }^\pi {(\frac{{{a_0}}}{2} + \sum\limits_{n = 1}^{ + \infty } {{a_n}\cos nx \D x} + \sum\limits_{n = 1}^{ + \infty } {{b_n}\sin nx \D x} )\cos mx \D x}
\end{split}\end{equation}
由于等号右边各项中,只有当 时积分不为零,上式变为
\begin{equation}
\int_{ - \pi }^\pi {f( x )\cos mx \D x} = \int_{ - \pi }^\pi {{a_m}\cos mx \vdot \cos mx \D x} = \pi {a_m}
\end{equation}
所以
\begin{equation}
{a_m} = \frac{1}{\pi }\int_{ - \pi }^\pi {f( x )\cos mx \D x}
\end{equation}
同理,有
\begin{equation}
{b_m} = \frac{1}{\pi }\int_{ - \pi }^\pi {f( x )\sin mx \D x}
\end{equation}
对于${a_0}$, 两边乘以1再积分
\begin{equation}\begin{aligned}
&\int_{-\pi}^\pi {f(x) \D x}\\  
= &\int_{-\pi}^\pi  (\frac{a_0}{2} + \sum\limits_{n = 1}^{+\infty} {a_n}\cos nx \D x  + \sum\limits_{n = 1}^{+\infty} {{b_n}\sin nx \D x}) \D x \\
= &\int_{ - \pi }^\pi  {\frac{{{a_0}}}{2} \D x}  = \pi {a_0}\\
\end{aligned}\end{equation}
即
\begin{equation}
{a_0} = \frac{1}{\pi }\int_{ - \pi }^\pi {f( x ) \D x} = \frac{1}{\pi }\int_{ - \pi }^\pi {f( x )\cos 0x \D x}
\end{equation}
故所有系数可以统一写成
\begin{equation}
{a_m} = \frac{1}{\pi }\int_{ - \pi }^\pi  {f( x )\cos mx \D x}
\end{equation}
\begin{equation}
{b_m} = \frac{1}{\pi }\int_{ - \pi }^\pi {f( x )\sin mx \D x}
\end{equation}

\subsection{拓展成$2l$为周期}
以上我们用的是以$2\pi $为周期的三角函数系,如果所求的$f( x )$以$2l$为周期,我们可以修改一下原始的三角函数系,得到
\begin{equation}
1,\,\sin \frac{\pi }{l}x,\,\cos \frac{\pi }{l}x,\,\sin \frac{{2\pi }}{l}x,\,\cos \frac{{2\pi }}{l}x,\,...\,\sin \frac{{n\pi }}{l}x,\,\cos \frac{{n\pi }}{l}x,\,...\,\,
\end{equation}
不难验证,以上函数系同样满足正交性与完备性.假设函数展开式为
\begin{equation}
f( x ) = \frac{{{a_0}}}{2} + \sum\limits_{n = 1}^\infty {{a_n}\cos (\frac{{n\pi }}{l}x) + } \sum\limits_{n = 1}^\infty {{b_n}\sin (\frac{{n\pi }}{l}x)}
\end{equation}
用同样的方法,可以得到系数表达式为
\begin{equation}
{a_n} = \frac{1}{l}\int_{ - l}^l {f( x )\cos (\frac{{n\pi }}{l}x)\D x}
\end{equation}
\begin{equation}
{b_n} = \frac{1}{l}\int_{ - l}^l {f( x )\sin (\frac{{n\pi }}{l}x)\D x}
\end{equation}