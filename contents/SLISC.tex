% SLISC 库简介
% C++|矩阵

相对于 fortran, 用 C++ 做数值计算的一个缺陷就是语言本身(或标准库)没有矩阵类型(以及高维矩阵类型).

% 放到 vector 中介绍
C++ 的数组只能在 stack 中使用, 而 stack 一般只有即 Mb, 一旦数组太大就会产生 stack overflow 错误. 第二是 stack 中的数据的大小都只能在编译时确定, 所以我们不能在运行是确定数组的长度(例如从文件中读取, 或通过运行时的计算得到).

之前介绍的 \verb|std::vector| 可以说是代替数组的最受欢迎的方法.

理论上我们可以用 \verb|vector| 套 \verb|vector| 的办法定义一个矩阵, 但是这样做数据在内存中是不连续的, 而且 memory allocation 是一个极其耗时的操作(最快的程序时先把所有需要的 memory 都 allocate 好再进行计算).

更原始(但是效率最高)的办法时干脆不定义矩阵库, 直接用 \verb|vector| 以及指针, 直接调用 LAPACK 函数进行线性代数运算. 但这样写出的代码难懂, 易错.

我们希望把矩阵进行一定程度的封装, 但又几乎不损失性能.

流行的矩阵库, 如 Eigen, 但是代码极其复杂, 错误信息不好懂, 修改起来非常困难.

Numerical Recipes 中自己定义了简单的矢量和矩阵库, 但是较为简单, 我们就在此基础上自己写一个矩阵库.

底层用 MKL 来实现, 速度非常快.

所有的函数都尽量使用指针 interface (因为是最兼容的!最笨的办法往往是最灵活的) 然后可以再封装一层更友好的 interface

slicing 会有少量的 overhead. 由用户自己权衡是使用 slicing 还是指针 interface.

