% 理想气体的状态密度(相空间)
% 理想气体|状态密度|球体体积|相空间

\begin{issues}
\issueDraft
\end{issues}

\pentry{$N$ 维球体的体积\upref{NSphV}}

\subsection{结论}
\begin{equation}
\Omega_0 = \frac{V^N}{N! h^3} \frac{(2\pi mE)^{3N/2}}{\Gamma(3N/2+1)}
\end{equation}
\begin{equation}
g(E) = \frac{V^N}{N! h^3} \frac{(2\pi m)^{3N/2}}{\Gamma(3N/2)} E^{3N/2 - 1}
\end{equation}

在 $N$ 个不相干粒子的相空间中, 能量小于 $E$ 的体积为
\begin{equation}
\Omega_0 = \frac{1}{N! h^3} \int\limits_{\sum p^2 \leqslant 2mE} \dd[3N]{q} \dd[3N]{p} = \frac{V^N}{N! h^3} \int\limits_{\sum p^2 \leqslant 2mE} \dd[3N]{p}
\end{equation}
其中积分 $\int_{\dots} \dd[3N]{p} $ 可以看做 $n=3N$ 维球体的体积, 半径为 $R = \sqrt{2mE}$. 

词条 $N$ 维球体的体积\upref{NSphV}
中的结论为 
\begin{equation}
V_n = \leftgroup{
&\frac{R^n}{\Gamma(n/2+1)} \pi^{(n - 1)/2} \ & &(\text{奇数} n)\\
&\frac{R^n}{\Gamma(n/2+1)} \pi^{n/2}  & &(\text{偶数} n)
} \ \approx \frac{n - 1}{2} = \frac{n}{2}
\end{equation}
(由于系统中粒子数 $N$ 非常多, 可以近似认为 $(n - 1)/2 = n/2$.  见热力学中的近似%未完成
). 代入 $n=3N$ 和 $R = \sqrt{2mE} $,   得, % 未完成
其中,
\begin{equation}
\int\limits_{\sum p^2 \leqslant 2mE} \dd[3]{p} = \frac{(2\pi mE)^{3N/2}}{\Gamma(3N/2+1)}
\end{equation}
$N$ 个不可区分粒子组成的理想气体, 能量小于 $E$ 的能级个数为
\begin{equation}
\Omega_0 = \frac{V^N}{N! h^3} \frac{(2\pi mE)^{3N/2}}{\Gamma(3N/2+1)}
\end{equation}
对能量求导得到状态密度为
\begin{equation}
g(E) = \dv{\Omega_0}{E} = \frac{V^N}{N! h^3} \frac{(2\pi m)^{3N/2}}{\Gamma(3N/2)} E^{3N/2 - 1}
\end{equation}
