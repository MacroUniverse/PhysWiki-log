%极限

先来看一个数列的例子.

\begin{exam}{}
我们都知道 $\pi$ 是一个无理数,所以 $\pi$ 的小数部分是无限多的.目前用计算机,已经可以将 $\pi$ 精确地计算到小数点后数亿位.然而在实际应用中,往往只用取前几位小数的近似即可.下面给出一个数列,定义第 $n$ 项是的前位小数近似(不考虑四舍五入),即
\begin{equation}
{a_0} = 3,\,\,{a_1} = 3.1,\,\,{a_2} = 3.14,\,{a_3} = 3.141,\,\dots.
\end{equation}
\end{exam}

这个数列显而易见的性质,就是当 $n$ 趋于无穷时,$a_n$ 趋(近)于 $\pi$. 无穷通常用符号 $\infty$ 来表示(像“8”横过来写).我们把这类过程叫做极限.以上这种情况,用极限符号表示,就是
\begin{equation}
\mathop {\lim }\limits_{n \to \infty } {a_n} = \pi 
\end{equation}
这里 $\lim$ 是极限(limit)的意思,下方用箭头表示某个量变化的趋势.$\mathop {\lim }\limits_{n \to \infty }$ 在这里相当于一个“操作”,叫算符(operator).$a_n$ 是其作用的对象(相当于函数的自变量),算符的“因变量”就是一个数( $a_n$ 的极限值).所以不要误以为这条式子是说当 $n \to \infty$ 时,$a_n=\pi$ ($a_n$ 是有理数,$\pi$ 是无理数,等式恒不成立),而要理解成 $a_n$ 经过算符 $\lim_{n \to \infty }$ 的作用以后,得出其极限是 $\pi$. 类比函数 $\sin x = y$,并不是说 $x=y$, 而是说 $x$ 经过正弦函数作用后等于 $y$. 

所以从概念上来说,极限中的 \textbf{趋于} 和 \textbf{等于} 是不同的.趋于更强调变化的过程.趋于的意思可以粗略理解为
\begin{itemize}
\item 越来越接近,但不一定相等
\item (在不相等的情况下)只有更近,没有最近
\end{itemize}

对极限来说,第2点成立是非常必要的.但是怎样能说明 “没有最近”呢?可以看出,当 $n$ 越大,$a_n$ 越接近 $\pi$, 它们的 “距离”,可以用 $\abs{a_n - \pi}$ 来表示.也就是说,对任何一个 $a_n$, 如果所对应的距离 $\abs{a_n - \pi } \ne 0$, 总能找到一个更大的数 $m>n$, 使 $\abs{a_m - \pi} < \abs{a_n - \pi}$ (更近),并且要求之后的所有项都能满足这一条件.只有这样,才能从数学上说明上面两个意思.这就是极限思想的精髓.根据这个思想,下面可以写出数列的极限的定义.这个定义无需硬记,如果理解了上面的描述,就觉得它理所当然了.

% 未完成: 数学定义的格式!
对于任意给定的(无论它有多么小),总存在 $N$, 当 $n>N$ 时,就有 $\abs{a_n - A} < \delta $ ($A$为常数) 成立,那么数列 $a_n$ 的极限就是 $A$. 

在命题中,通常把 “任意” 用 “ $\forall$” (any) 表示,把 “存在” 用 “$\exists $” (exist)表示.即
% 未完成: 数学定义的格式!
对 $\forall \delta$, $\exists N$, 当 $n>N$ 时,有 $\abs{a_n - A} < \delta$. 

由于以上讨论中 lim 作用的对象是数列,那么箭头右边只能是 $\infty$ (准确来说应该是正无穷 $+\infty$, 但是由于数列的项一般是正的,所以正号省略了).

把定义套用到上面的例一中, 如果要求 $\abs{a_n - \pi} < 10^{ - 3}$ (给定 $\delta  = {10^{ - 3}}$),只要令 $N=3$ (当然也可以令 $N=4, N=5$, 等) 就可以保证第项后面所有的项都满足要求.一般地如果给定 $\delta  = b \times 10^{ - q}  (b > 1)$, 就令 $N = q$, 第 $N$ 项以后的项就满足要求.这就从定义直接证明了 $\lim_{n \to \infty } a_n = \pi$. 

% 未完成: 感觉这个词条太罗嗦了,不符合本书 “数学从简” 的宗旨! 这不是数学书,不要照抄高数! 像倒立的A,E符号根本没必要存在,后面根本用不上! 极限主要是为了导数和积分的定义,那就快点进入函数的极限! 想象那些是必须的哪些用不上.