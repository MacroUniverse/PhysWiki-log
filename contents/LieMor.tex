% 李代数的同态与同构
% Lie|Lie algebra|homomorphism|isomorphism|理想|ideal

\pentry{李代数的子代数、理想与商代数\upref{LieSub}}


\subsection{同态与同构}

和其它代数结构一样,李代数之间可以建立同态与同构映射,其基本思想依然是保持运算结构.

\begin{definition}{李代数的同态}
给定李代数$\mathfrak{g}$和$\mathfrak{h}$和\textbf{线性映射}$f:\mathfrak{g}\to\mathfrak{h}$,如果对于任意$\bvec{g}_1, \bvec{g}_2$还满足$f([\bvec{g}_1, \bvec{g}_2])=[f(\bvec{g}_1), f(\bvec{g}_2)]$,那么称$f$是$\mathfrak{g}$到$\mathfrak{h}$的一个\textbf{同态(homomorphism)}.

记$\opn{ker}f =\{g\in\mathfrak{g}|f(g)=0\in\mathfrak{h}\}$,称为$f$的\textbf{核(kernel)}.
\end{definition}

\begin{definition}{李代数的同构}
双射的李代数同态,称为\textbf{同构(isomorphism)}.
\end{definition}

一个李代数$\mathfrak{g}$到自身的同态称为$\mathfrak{g}$上的一个\textbf{自同态(endomorphism)}.$\mathfrak{g}$上的全体自同态构成的集合,配合映射的复合作为二元运算,就可以构成一个\textbf{半群},记为$\opn{End}\mathfrak{g}$.它只是一个半群,因为有一些自同态不是双射\footnote{比如把所有元素都映射到$0$上的映射.},它们就没有逆映射.

$\mathfrak{g}$到自身的同构,称为$\mathfrak{g}$上的一个\textbf{自同构(automorphism)},全体自同构的集合配合映射的符合可以构成一个\textbf{群},记为$\opn{Aut}\mathfrak{g}$.

同样,李代数的也有相应的同态基本定理:

\begin{definition}{李代数的同态基本定理}
令$f:\mathfrak{g}\to \mathfrak{h}$为李代数同态,则有
\begin{itemize}
\item $\mathfrak{g}/\opn{ker} f\cong \mathfrak{h}$;
\item 如果$\mathfrak{g}_0\supseteq \opn{ker}f$是$\mathfrak{g}$的理想,那么$f(\mathfrak{g}_0)$是$\mathfrak{h}$的理想.
\end{itemize}
\end{definition}


\subsection{李代数的表示}

同态可以用于把李代数的性质转嫁到容易处理的结构上去,方便进行讨论和计算.这样的同态映射就叫做“表示”.

\begin{definition}{}\label{LieMor_def1}
设$\mathfrak{g}$是域$\mathbb{F}$上的一个李代数,$V$为$\mathbb{F}$上的一个线性空间,$\opn{gl}V$是$V$上全体线性变换的集合构成的李代数\footnote{即用线性变换的复合和相加作为运算构成一个环,然后再用从结合代数中导出李代数的方法导出一个李代数.当选定了$V$的一组基时,还可以把每个线性变换表示成一个矩阵.}.如果$\rho:\mathfrak{g}\to \opn{gl}V$是一个同态,那么称$\rho$是$\mathfrak{g}$的一个\textbf{以}$V$\textbf{为表示空间}的\textbf{线性表示(linear representation)}.

$\opn{dim} V$被称为该表示的\textbf{维数(dimension)};$\opn{ker}\rho$被称为该表示的\textbf{核(kernel)}.
\end{definition}

\autoref{LieMor_def1} 中$\mathfrak{g}$的每个元素都被$\rho$映射到了$V$的一个线性变换上,所以我们也可以把表示看成是李代数对表示空间的作用,这是一种群作用.



\begin{example}{伴随表示}
设$\mathfrak{g}$是域$\mathbb{F}$上的一个李代数,$\bvec{x}\in\mathfrak{g}$,定义映射$\opn{ad}_\bvec{x} :\mathfrak{g}\to\mathfrak{g}$为:对于任意$\bvec{y}\in\mathfrak{g}$,有$\opn{ad}_\bvec{x}(\bvec{y})=[\bvec{x}, \bvec{y}]$.这样,$\opn{ad}_\bvec{x}$就可以看成是李代数对自身的一个线性变换.

令$\rho:\mathfrak{g}\to\opn{gl}\mathfrak{g}$,其中$\rho(\bvec{x})=\opn{ad}_\bvec{x}$.称$\rho$是$\mathfrak{g}$上的一个\textbf{伴随表示(adjoint representation)}.

由定义容易知道,$\opn{ker}\rho=C(\mathfrak{g})$.
\end{example}










