%矢量的导数 求导法则

%未完成:在下面介绍矢量的高阶导数
\pentry{矢量的加减,矢量的数乘%未完成:引用
,矢量函数%未完成:引用
,导数\upref{Der}}

\subsection{矢量的导数}

若矢量 $\vec v$ 只是一个标量 $t$ 的函数,记为 $\vec v(t)$, 则 $\vec v$ 对 $t$ 的导数可记为以下的一种(括号可选择省略)
\begin{equation}
\frac{\D\vec v}{\D t} \qquad \frac{\D}{\D t}\vec v \qquad \dot{\vec v}
\end{equation}
其定义为(类比导数\upref{Der}中导数的代数定义)
\begin{equation}
\frac{{\D\vec v}}{{\D t}} = \mathop {\lim }\limits_{\Delta t \to 0} \frac{{\vec v\left( {t + \Delta t} \right) - \vec v\left( t \right)}}{{\Delta t}}
\end{equation}
唯一与标量函数的导数不同的是, 这里的减法是矢量相减,结果还是矢量.除以 $\Delta t$ 相当于矢量的数乘 $1/\Delta t$, 结果也是矢量. 所以 $\D\vec v / \D t$ 也是一个关于 $t$ 的矢量函数.

直角坐标系中对单变量矢量函数求导就是对矢量的各个分量分别求导(见下文“求导法则”).
\eentry{速度和加速度(矢量)\upref{VnA}, 匀速圆周运动的速度\upref{CMVG} 和加速度\upref{CMAG}(几何法)}

\subsection{矢量的偏导数}
与标量函数的偏导类似, 对一个多元的矢量函数 $\vec v(x_1, x_2\ldots x_N)$, 如果把其他自变量都看做常数而对 $x_i$ 求导, 那么就得到矢量函数关于 $x_i$ 的\textbf{的偏导数}.
\begin{equation}
\pdv{\vec v}{x_i} = \lim_{\Delta x_i \to 0} \frac{\vec v(x_1 \ldots x_i+\Delta x_i\ldots x_N) -  \vec v(x_1 \ldots x_i\ldots x_N)}{\Delta x_i}
\end{equation}

直角坐标系中对多变量矢量函数求偏导就是对矢量的各个分量分别求偏导(见下文“求导法则”).

\subsection{矢量的求导法则}
与标量函数一样,由定义不难证明矢量函数求导也是线性算符( $c_i$ 为常数)\footnote{以下法则虽然以导数为例, 但对偏导也同样适用.}
\begin{equation}
\frac{\D}{{\D t}}[{c_1}{\vec v_1}(t) + {c_2}{\vec v_2}(t) + \dots] = {c_1}\dv{\vec v_1}{t} + c_2\dv{\vec v_2}{t} \dots
\end{equation}

直角坐标中,矢量函数可以看做三个分量上的标量函数且矢量基底不变,所以由上式可得矢量求导就是对每个标量函数求导.
\begin{equation}
\frac{{\D\vec v}}{{\D t}} = \frac{\D}{{\D t}}[{v_x}(t)\uvec x] + \frac{\D}{{\D t}}[{v_y(t)}\uvec y] + \frac{\D}{{\D t}}[{v_z(t)}\uvec z]
= {\dot v_x}(t)\uvec x + {\dot v_y(t)}\uvec y + {\dot v_z(t)}\uvec z
\end{equation}
要特别注意该式成立的条件是三个基底不随 $t$ 改变,这在其他坐标系中并不成立, 例如“极坐标中单位矢量的偏导\upref{Dpol1}”.

\eentry{匀速圆周运动的速度\upref{CMVD} 和加速度\upref{CMAD}(求导法)}

矢量数乘,点乘或叉乘的求导在形式上都与标量函数的情况类似.
\begin{equation}
\frac{\D}{{\D t}}[f(t)\vec v(t)] = \frac{{\D f}}{{\D t}}\vec v + f\frac{{\D\vec v}}{{\D t}}
\end{equation}
\begin{equation}\label{DerV_eq5}
\frac{\D}{{\D t}}[\vec u(t)\vdot\vec v(t)] = \frac{{\D\vec u}}{{\D t}}\vdot\vec v + \vec u\vdot\frac{{\D\vec v}}{{\D t}}
\end{equation}
\begin{equation}
\frac{\D}{{\D t}}[\vec u(t) \cross \vec v(t)] = \frac{{\D\vec u}}{{\D t}} \cross \vec v + \vec u \cross \frac{{\D\vec v}}{{\D t}}
\end{equation}
由定义出发,不难证明以上三式,这里以\autoref{DerV_eq5} 为例进行证明. 根据点乘定义以及标量函数的求导法则\upref{DerRul}有
\begin{equation}\begin{aligned}
\dv{}{t} (\vec u \vdot \vec v) &= \dv{}{t} (u_x v_x + u_y v_y + u_z v_z)\\
&= \left(\dv{u_x}{t} v_x +
 u_x \dv{v_x}{t}\right) + \left(\dv{u_y}{t} v_y + u_y \dv{v_y}{t}\right) + \left(\dv{u_z}{t} v_z   + u_z \dv{v_z}{t}\right) \\
&= \left(\dv{u_x}{t} v_x + \dv{u_y}{t} v_y + \dv{u_z}{t} v_z\right) + \left(u_x \dv{v_x}{t} + u_y \dv{v_y}{t} + u_z \dv{v_z}{t}\right) \\
&= \dv{\vec u}{t}\vdot \vec v + \vec u\vdot \dv{\vec v}{t}
\end{aligned}\end{equation}

\eentry{动量定理\upref{PLaw},角动量定理(单个质点)\upref{AMLaw1}}

\subsection{矢量的高阶导数和偏导}
与标量函数的高阶导数类似, 对某个矢量连续求 $N$ 次导数, 就得到该函数的 $N$ 阶导数. 上面在求圆周运动的加速度时, 事实上我们已经计算了位置矢量的导数(速度)的导数, 即位置矢量关于时间的二阶导导数.

与标量函数的偏导\upref{ParDer}类似, 多元矢量函数的高阶导数也要声明各阶导数是对哪个变量进行的


