% 磁旋比 玻尔磁子

%未完成
如果一个带电刚体,质量密度和电荷密度成正比,当它绕轴转动时,角动量为
\begin{equation}
\bvec L = I\bvec \omega  = \bvec \omega \int r^2 \rho_m \dd{V}
\end{equation}
磁矩定义为
\begin{equation}
\bvec \mu  = Ia \uvec\omega = \int \frac{\dd{Q}}{2\pi/\omega}  \pi r^2 \uvec \omega
= \frac12 \bvec \omega \int r^2\rho_e \dd{V} 
\end{equation}
其中 $r$ 为质量元到转轴的距离.两式比较,得
\begin{equation}
\bvec \mu  = \frac{q}{2m} \bvec L
\end{equation}
但对基本粒子(例如电子)的实验中,发现上式还需要一个修正因子
\begin{equation}
\bvec \mu  = g\frac{q}{2m}\bvec L
\end{equation}
$g$ 一般就叫做 $g$ 因子.定义磁旋比 $\gamma  = gq/(2m)$

对于粒子的自旋, $L = \hbar \sqrt{l(l + 1)} $. 所以 $\mu = \sqrt{l (l + 1)} \hbar gq/(2m)$. 
对于电子,实验测得 $g_e = 2.0023193043617(15) \approx 2$ 
\begin{equation}
\bvec \mu  = g_e \frac{e\hbar}{2 m_e} \sqrt{\frac12 \qty(1 + \frac12)}  = \frac{\sqrt 3}{2} g_e \mu_B \approx \sqrt 3 \mu_B
\end{equation}
其中 $\mu_B$ 为玻尔磁子,定义为
\begin{equation}
\mu_B = \frac{e\hbar}{2 m_e}
\end{equation}

