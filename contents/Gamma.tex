% gamma 函数
% 微积分|定积分|Γ 函数| Gamma 函数

\pentry{定积分\upref{DefInt} }

\subsection{结论}
当 $x$ 取实数且 $x>-1$ 时,可以定义连续的阶乘函数为
\begin{equation}
x! \equiv \Gamma (x + 1) = \int_0^{+\infty} t^x \E^{-t} \dd{t}
\end{equation}
递推关系仍为
\begin{equation}\label{Gamma_eq2}
x!=x(x-1)! \qquad (x>0)
\end{equation}
且 $(-1/2)!=\sqrt{\pi}$,  $0! = 1$.

\subsection{推导}

首先定义 $\Gamma$ \bb{(Gamma)} 函数为
\begin{equation}
\Gamma(x) = \int_0^{+\infty} t^{x-1} \E^{-t} \dd{t}
\end{equation}
当 $x \les 0$ 时该积分在 $x=0$ 处不收敛,以下仅讨论 $x$ 为正实数的情况\footnote{事实上,自变量为负实数(非整数)时,$\Gamma$ 函数有另一种定义,这里不讨论.}.

我们现在验证当 $x$ 取正整数时,新定义的阶乘 $x! = \Gamma(x+1)$ 与原来的定义 $x! = x(x-1)\dots 1$ 相同.首先
\begin{equation}\label{Gamma_eq4}
0! = \Gamma(1) = \int_0^{+\infty} \E^{-t} \dd{t} = 0 - (-1) = 1
\end{equation}

使用分部积分法\upref{IntBP},令 $t^x$ 为“求导项”, $\E^{-t}$ 为积分项,可得递推公式\footnote{该证明仅对 $x>0$ 适用, 这样才有 $0^x \E^{-0} = 0$, 使第三个等号成立.}(\autoref{Gamma_eq2})
\begin{equation}\label{Gamma_eq5}
\begin{aligned}
x! &= \Gamma(x+1) = \int_0^{+\infty} t^x \E^{-t} \dd{t} =  - \eval{t^x \E^{-t}}_{t=0}^{t=+\infty} + \int_0^{+\infty} x t^{x-1} \E^{-t} \dd{t} \\
&= x\int_0^{+\infty} t^{x-1} \E^{-t} \dd{t} = x\Gamma (x) = x(x-1)!
\end{aligned} \end{equation} 
由递推\autoref{Gamma_eq5} 和初值\autoref{Gamma_eq4}, 对任意正整数 $n$ 有
\begin{equation}
n! = n(n-1)! = n(n-1)(n-2)!... = n(n-1)...1
\end{equation}

再来看半整数的阶乘,我们讨论范围内的最小半整数的阶乘为 
\begin{equation}
\qty(-\frac12) ! = \int_0^{+\infty} \frac{\E^{-x}}{\sqrt x}\dd{x} = \sqrt{\pi}
\end{equation}
该积分可以用换元法令 $x = t^2$ 变为高斯积分\upref{GsInt}
\begin{equation}
\qty(-\frac 12)! = 2\int_0^{+\infty} \E^{-t^2} \dd{t} = \int_{-\infty}^{+\infty} \E^{-t^2} \dd{t}
\end{equation}
进行计算, 结果为 $\sqrt{\pi}$.

对任意大于零的半整数 $n/2$,有
\begin{equation}
\frac{n}{2}! = \frac{n}{2} \qty(\frac{n}{2}-1)! = \frac{n}{2} \qty(\frac{n}{2}-1) \dots \frac12 \qty(-\frac12) ! = \frac{n}{2} \qty(\frac{n}{2}-1) \dots \frac12 \sqrt{\pi}
\end{equation}














