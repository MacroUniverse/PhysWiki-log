% 代数矢量

\pentry{正交归一基底\upref{OrNrB}, 复数\upref{CplxNo}}

我们可以把 $N$ 个有序的数排成一行或一列, 分别称为\bb{代数行矢量}或\bb{代数列矢量}, 简称\bb{行矢量}或\bb{列矢量}, 本书中同样用加粗正体字母表示。 若一个代数矢量中所有的数都是实数, 且 $N \les 3$%未完成: 斜线的小于等于号是什么命令?
时, 我们可以把它看做是一个几何矢量在一组(往往是正交归一的)基底上的展开\autoref{GVec_eq5}。 这样就可以很自然地定义这种代数矢量的加法和数乘分别为\autoref{GVec_eq8} 和\autoref{GVec_eq9}。 然而当 $N > 3$ 或矢量中含有复数时, 我们定义加法和数乘分别为
\begin{equation}
\vec u + \vec v = \sum_{i = 1}^N u_i + v_i
\end{equation}
\begin{equation}
\lambda \vec v = \sum_{i = 1}^N \lambda v_i
\end{equation}
注意 $N$ 维行矢量只能与 $N$ 维行矢量相加, 列矢量也一样。 有了这两个定义, 我们就可以类似几何矢量得到其他重要的概念, 如\bb{共线}, \bb{线性组合}, \bb{线性相关}, \bb{线性无关}, \bb{基底}, 和\bb{坐标}。 代数矢量也可以组成许多不同的矢量空间, 例如所有的 $N$ 维的实数行矢量组成一个 $N$ 维矢量空间, 所有的 $N$ 维复数列矢量组成一个 $N$ 维矢量空间, 等等。 为了便于理解, 我们仍然可以想象代数矢量是某种抽象的个体, 不依赖任何基底而存在, 而代数矢量只是这些个体在某组基底上的展开系数(线性组合的系数)。

\begin{exam}{用代数矢量表示多项式}
作为上述“抽象个体” 的一个例子, 我们可以把以所有小于 $N$ 阶的多项式
\begin{equation}
P(x) = \sum_{n = 0}^{N-1} c_n x^n
\end{equation}
看做一个 $N$ 维矢量空间, 矢量的加法就是多项式相加, 数乘就是多项式与常数相乘。 一组简单的基底是
\begin{equation}
\vec \beta_n = x^n \qquad (n = 0, \dots, N-1)
\end{equation}
显然这组基底是线性无关的, 且空间中所有的多项式都可以在该基底上展开。 任意 $N$ 维代数矢量 $(c_0, \dots, c_{N-1})$ 可以看做是这些一个多项式的坐标, 且有一一对应的关系。 代数矢量的加法和数乘对应多项式的加法和数乘。
\end{exam}

\subsection{点乘}
由于代数矢量没有一般的几何意义, 我们定义同一空间中两代数矢量的点乘为
\begin{equation}
\vec u \vdot \vec v = \sum_{i = 1}^N u_i^* v_i
\end{equation}
其中 $u_i^*$ 是 $u_i$ 的复共轭(\autoref{CplxNo}\upref{CplxNo_eq6}), 所以对于实数矢量, 点乘化简为
\begin{equation}
\vec u \vdot \vec v = \sum_{i = 1}^N u_i v_i
\end{equation}
现在我们可以定义任意代数矢量 $\vec v$ 的\bb{模长} 为(注意模长都是非负的实数)
\begin{equation}
\abs{v} = \sqrt{\vec v \vdot \vec v} = \qty(\sum_i \abs{v_i}^2)^{1/2}
\end{equation}
且定义若两个代数矢量点乘为 0, 则它们互相\bb{正交}。
% 未完成, 在矩阵中提一下, \vec u \vdot \vec v = \vec u\Her \vec v, 即矩阵乘法。

现在我们可以类比几何矢量定义代数矢量的\bb{正交归一基底}, 事实上在 “正交归一基底\upref{OrNrB}” 中, 我们并没有要求所有的矢量都是几何矢量, 而只要存在” 点乘”, “ 模长” 和“ 正交” 的概念。

我们知道两个几何矢量的正交(垂直)并不取决于基底的选取, 那么代数矢量是否也有同样的性质呢? 答案是肯定的, 但证明我们留到以后。% 未完成: 证明, 但要先证明酋矩阵的性质, 即如果所有的行正交归一, 那么所有的列也正交归一。 由 AA^T = 1, A^T = A^{-1}, 所以 A^T A = 1. 所以又涉及到逆矩阵的性质。
所以, 代数矢量的点乘也可以看做是两个不取决于基底的抽象矢量见的一种性质。




