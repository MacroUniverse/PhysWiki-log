% 法拉第电磁感应定律
% 法拉第|电磁感应|感生电动势|楞次定律

\begin{issues}
\issueAbstract
\issueTODO
\end{issues}

\pentry{磁通量\upref{BFlux}}

\subsection{电磁感应定律的积分形式}

我们在高中物理学过, 闭合线圈产生的感生电动势% 未完成:链接
等于线圈内磁通量随时间的变化率.方向由楞次定律\upref{Lenz}决定. 即
\begin{equation}\label{FaraEB_eq1}
\varepsilon  =  -\dv{\Phi}{t} =  - \dv{t} \int \bvec B \vdot \dd{\bvec a} =  - \int \dv{\bvec B}{t} \vdot \dd{\bvec a}
\end{equation} 
这里的 $\bvec a$ 表示面积, 积分的曲面是以线圈为边界的任意曲面. 另一方面,感生电动势是由感生电场产生的. 
\begin{equation}\label{FaraEB_eq2}
\varepsilon  = \oint \bvec E \vdot \dd{\bvec r}
\end{equation}
这里的路径积分是沿着线圈进行的. 规定线圈正方向以后, \autoref{FaraEB_eq1} 中曲面的正方向由右手定则\upref{RHRul}决定.

根据麦克斯韦方程组, 电场产生的原因有两种, 一种是电荷产生电场(电场的高斯定理\upref{EGauss}), 另一种是变化的磁场产生电场(法拉第电磁感应). 前者在\autoref{FaraEB_eq2} 中的环路积分为零, 对电动势没有贡献. 所以\autoref{FaraEB_eq2} 中的 $\bvec E$ 既可以只包含感生电场, 也可以是总电场. 我们一般理解为总电场.

对比上面两式,得
\begin{equation}
\oint \bvec E \vdot \dd{\bvec r}  =  - \int \pdv{\bvec B}{t} \vdot \dd{\bvec a} 
\end{equation} 
如果我们假设感生电场只与电场的分布和变化率有关,则这个公式对空间中任何假想中的回路都成立,而不需要有真正的线圈存在.注意上式中的磁场是空间中的所有磁场.

\subsection{电磁感应定律的微分形式}
\pentry{斯托克斯定理\upref{Stokes}}
斯托克斯定理告诉我们, 若对任意闭合回路, 一个矢量场对曲面正方向的面积分等于另一个场在曲面边界线正方向的线积分, 那么前者是后者的旋度.
\addTODO{“斯托克斯定理” 并没有提到这个逆向的结论}
应用到上式, 可得电场的旋度为
\begin{equation}
\curl \bvec E =  - \pdv{\bvec B}{t}
\end{equation} 
