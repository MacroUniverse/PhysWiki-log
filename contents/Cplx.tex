%9:20-9;30
% 未完成: 太简略了,解析是什么意思? 怎么求导? 为什么各个方向导数都是一样的?
\pentry{复数}%未完成:引用
复变函数是自变量和因变量都在复数域内取值的函数,通常表示为
\begin{equation}
w = f\left( z \right)
\end{equation}
若把自变量用 $z$ 的实部 $x$ 和虚部 $y$ 表示,因变量表示成实部函数 $u\left( {x,y} \right)$ 和虚部函数 $v\left( {x,y} \right)$ 两个函数相加,则复变函数记为
\begin{equation}
w = u\left( {x,y} \right) + \I v\left( {x,y} \right)
\end{equation}
例如,复数范围内的指数函数%未完成:引用
被定义为
\begin{equation}
w = {\E^z} = {\E^x}\cos y + \I {\E^x}\sin y
\end{equation}
由于复变函数的图像%未完成:引用
比较复杂,没有必要记忆图像,只需要知道一些基本的性质即可.

\subsection{与实变函数的“兼容性”}
复变函数中很多函数与我们原来我们学过的函数同名,只是自变量的范围从实数拓展到了复数.例如以指数函数,三角函数,对数函数,指数函数,等.这些新函数的定义必须要与原来的函数“兼容”,即当自变量被限制在实数范围内取值时,这些函数与原来的函数相同.

例如,当复数范围内的指数函数%未完成:引用
 $w = {\E^z} = {\E^x}\cos y + \I {\E^x}\sin y$ 的自变量只在实数范围取值(即 $y = 0$) 时,该函数变为我们原来所熟悉的 $\E^x$. 
又如,复数范围内正弦函数%未完成:引用
被定义为
\begin{equation}
\sin z = \sin \left( {x + iy} \right) = \sin x\cosh y + \I\cos x\sinh y
\end{equation}
其中 $\sinh $ 和 $\cosh $ 是双曲正弦和双曲余弦函数.当 $y = 0$ 时,该函数变为 $\sin x$ 
所以从这个意义上来说,与实变函数同名的复变函数只是把函数的定义域扩大了.

\subsection{复变函数的导数}
由于复变函数相当于两个实数自变量和两个实数因变量的函数,一般情况下求导变得非常复杂.但如果复变函数在某个域上解析%未完成:引用
,那么可以在该域上进行求导,得到唯一的导数.对于复数域初等基本函数,求导的结果也和实数域的求导一样.















