%平面波的复数表示

\pentry{振动的指数形式\upref{VbExp}}

用复数表示波函数,往往可以化简书写和计算,所以许多教材都使用复数表示波动或振动.复变函数在物理学中的一个最广泛的应用就是通过欧拉公式%(链接未完成)
把三角函数写成指数函数的形式.
\begin{equation}%未完成: 标签
\E^{\I x} = \cos x + \I\sin x 
\end{equation} 
在经典力学和电磁学中,我们默认用复变函数的实部表示某个波动的物理量.所以波函数 $A\cos (kx - \omega t)$ 可以表示为
\begin{equation}
A \E^{\I( kx - \omega t )}
\end{equation}
注意虚部无物理意义.又例如向 $\vec k$ 方向传播的平面波 $A\cos( \vec k\vdot\vec r - \omega t )$ 可表示为
\begin{equation}
A \E^{\I( \vec k \vdot \vec x - \omega t )}
\end{equation}

用指数函数表示波动的一个优势在于可以用两个指数函数相乘将其相位叠加.例如
\begin{equation}
A \E^{\I( \vec k \vdot \vec r - \omega t + \phi_0 )} = A \E^{\I \phi_0} \E^{\I\vec k \vdot \vec r} \E^{ - \I\omega t}
\end{equation}
可以定义复振幅 $\tilde A = A \E^{\I \phi_0}$ 用于同时表示振幅和初相位.另外 $\E^{\I\vec k \vdot \vec r}$ 只是空间坐标的函数,不妨叫空间因子, $\E^{ - \I\omega t}$ 叫时间因子.要注意的是, $\E^{\I( \vec k \vdot \vec r - \omega t )}$ 和 $\E^{\I(  - \vec k \vdot \vec r + \omega t )}$ 的实部一样,表示同样的波函数,但习惯上总把时间因子写为 $\E^{ - \I\omega t}$ 的形式.












