% 自旋角动量
% 自旋|角动量|对易关系

\begin{issues}
\issueDraft
\end{issues}

\pentry{轨道角动量\upref{QOrbAM}, 张量积空间\upref{DirPro}}

\addTODO{简单介绍斯特恩-格拉赫实验}

自旋是量子力学中的基本粒子特有的性质, 描述粒子的波函数不包含自旋的信息, 自旋处于单独的有限维希尔伯特空间中, 和波函数的空间做张量积以后用于描述粒子的状态.

\begin{enumerate}
\item 自旋角动量三个分量算符 $S_x, S_y, S_z$ 的互相对易关系以及自旋模长平方算符 $S^2$ 的对易关系 %(已经不想写了)
\item 与轨道角动量同理,存在一组本征态 $\ket{s,m}$ 

( $s = 0, 1/2, 1, 3/2\dots$, $m = -s, -s+1\dots ,s-1, s$ 但是每种粒子都有固有的 $s$ ) 满足
\begin{equation}
S^2\ket{s, m} = \hbar^2 s(s+1)\ket{s, m}  \quad \text{和} \quad
S_z\ket{s, m} = \hbar m\ket{s, m}
\end{equation}

\item 存在升降算符 $S_\pm = S_x \pm \I S_y$, 且(根号项是归一化系数)
\begin{equation}
S_\pm \ket{s,m} = \hbar \sqrt{s(s + 1) - m(m \pm 1)} \ket{s, m+1} 
\end{equation}

\item 对于 $s = 1/2$ 的粒子,一共有 2 个本征态, 分别是 $\ket{1/2, 1/2}$,  $\ket{1/2, -1/2}$. 它们的角动量模长平方都是 $3\hbar^2/4$, 角动量 $z$ 分量都是 $\hbar/2$. 以这两个本征态为基底,令第一个为 $\chi_+ =(1, 0)\Tr$, 第二个为 $\chi_- = (0, 1)\Tr$. 可以得出角动量平方算符的矩阵为
\begin{equation}
\mat S^2 = \frac{3\hbar^2}{4} \pmat{1&0\\0&1} \qquad
\mat S_z = \frac{\hbar}{2} \pmat{1&0\\0&-1}
\end{equation}
根据 $S_+ \chi_- = \hbar \chi_+$ 和 $S_- \chi_+ = \hbar \chi_-$,   得到
\begin{equation}
S_x = \frac{\hbar}{2}\pmat{0&1\\1& 0} \qquad
S_y = \frac{\hbar}{2}\pmat{0&-\I\\ \I& 0}
\end{equation}
然后, 定义泡利矩阵.  
\begin{equation}
\sigma_x = \pmat{0&1\\1& 0} \qquad
\sigma_y = \pmat{0&-\I\\ \I& 0} \qquad
\sigma_z = \pmat{1&0\\ 0&-1}
\end{equation}
其实, 根据对易关系直接就可以得到泡利矩阵.
% 未完成
\end{enumerate}
