% 动量表象的薛定谔方程

\pentry{薛定谔方程\upref{TDSE}}

本文使用原子单位制\upref{AU}. 一维薛定谔方程为
\begin{equation}\label{moTDSE_eq1}
-\frac{1}{2m} \pdv[2]{x}\psi(x, t) + V(x,t)\psi(x,t) = \I \pdv{t} \psi(x,t)
\end{equation}
先来看一维的情况: 可以证明, 如果 $V(x,t)$ 关于 $x$ 可以被泰勒展开, 那么动量表象的薛定谔方程为
\begin{equation}
-\frac{k^2}{2m}\varphi(k, t) + V\qty(\I \pdv{k})\varphi(k, t) = \I \pdv{t} \varphi(k,t)
\end{equation}

证明: 把
\begin{equation}
\psi(x,t) = \frac{1}{\sqrt{2\pi}}\int_{-\infty}^{+\infty} \varphi(k,t)\E^{\I kx}\dd{x}
\end{equation}
代入\autoref{moTDSE_eq1}, 再利用\autoref{FTExp_eq8}~\upref{FTExp} 即可. 证毕.

从另一种方式来理解,表象无关的薛定谔方程为
\begin{equation}
\Q H\ket{\psi} = \I \pdv{t}\ket{\psi}
\end{equation}
\begin{equation}
\Q H = \frac{\Q p^2}{2m} + V(\Q x)
\end{equation}
而动量表象中 $\Q p = p$, $\Q x  = \I \pdv*{k}$, 代入即可.

\addTODO{例子}

类似地, 动量表象得三维薛定谔方程为
\begin{equation}
-\frac{\bvec k^2}{2m}\varphi(\bvec k, t) + V\qty(\I \grad_k)\varphi(\bvec k, t) = \I \pdv{t} \varphi(\bvec k,t)
\end{equation}
