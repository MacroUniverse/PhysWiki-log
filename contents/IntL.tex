\pentry{功\ 功率\upref{Fwork}, 定积分\upref{DefInt}}

在“功\ 功率\upref{Fwork}”中,我们大致了解了线积分 $\int_{C_{ab}} {\vec F(\vec r) \vdot \D\vec r} $ 的意义.下面讨论如何在直角坐标系中具体计算线积分.为书写方便,以下省略积分路径 $C_{ab}$. 

将被积曲线的参数方程可以表示为\footnote{注意这里的 $t$ 不一定代表时间,可以是任意参数,甚至可以是 $x,y,z$ 中的一个.} $x(t),y(t),z(t)$, 则曲线上任意一点都唯一对应一个 $t$ 值.则根据微分关系,当 $t$ 增加 $\D t$ 时,曲线上的一小段微位移矢量 $\D \vec r = (\D x,\D y,\D z)$ 中
\begin{equation}\label{IntL_eq1}
\D x = x'(t)\D t \qquad \D y = y'(t)\D t \qquad \D z = z'(t)\D t
\end{equation}
这样,对曲线上任意一点(对应参数 $t$),$\vec F$ 可表示成 $t$ 的矢量函数 $\vec F(t) = \vec F[x(t),y(t),z(t)]$.  $\vec F$ 的三个分量\footnote{为了书写简洁,这里定义 $x_1\equiv x, x_2\equiv y,x_3\equiv z$.} 则表示为关于 $t$ 的单变量标量函数
\begin{equation}
{F_{x_i}}(t) = {F_{x_i}}[x(t),y(t),z(t)] \quad (i = 1,2,3)
\end{equation}

下面将三维空间的线积分转换为三个一元定积分
\begin{equation}\begin{aligned}
\int {\vec F(\vec r) \vdot \D \vec r}  &= \lim \limits_{n \to \infty } \sum\limits_{i = 1}^n {\vec F({{\vec r}_i}) \vdot \Delta {{\vec r}_i}} \\
&= \lim_{n \to \infty } \sum_{i = 1}^n {{F_x}({{\vec r}_i})\Delta {x_i} + \lim_{n \to \infty } \sum_{i = 1}^n {F_y}({{\vec r}_i})\Delta{y_i} + \lim_{n \to \infty } \sum_{i = 1}^n {F_z}({{\vec r}_i})\Delta{z_i}} \\
&= \int {{F_x}(\vec r)\D x}  + \int {{F_y}(\vec r)\D y}  + \int {{F_z}(\vec r)\D z} 
\end{aligned}\end{equation} 
设积分路径 ${C_{ab}}$ 的起点对应 $t = a$, 终点对应 $t = b$. 结合\autoref{IntL_eq1}, 上面每一项积分可以表示为 
\begin{equation}\label{IntL_eq4}
\int {{F_{x_i}}(\vec r) \D x}  = \int_a^b {{F_{x_i}}[x(t),y(t),z(t)] x_i'(t)\D t} \quad (i=1,2,3)
\end{equation} 
计算这三个关于 $t$ 的定积分再相加,就可以得出线积分结果.

\begin{exam}{计算力场对质点的做功}\label{IntL_ex1}
令力场为 $\vec F = \alpha r \,\uvec r$, 一质点从原点出发,沿轨迹 $(x-a)^2 + y^2 = a^2$ 的上半部分移动到 $(a,0)$,求力对质点做的功. 若起点终点不变,轨迹改为延 $x$ 轴,结果又如何?

我们先来建立运动轨迹的参数方程.由于运动是一个圆,我们可以使用圆的参数方程.把角度作为参数 $t, t\in [0,\pi]$.
\begin{equation}
\leftgroup{
x(t) &= a(1-\cos t)\\
y(t) &= a \sin t
}
\qquad 
\leftgroup{
x'(t) &= a \sin t\\
y'(t) &= a \cos t
}
\end{equation}
把力场在直角坐标系中表示为 $\vec F(x,y) = \alpha (x\,\uvec x + y\,\uvec y)$, 两个分量分别为 $F_x = \alpha x, F_y = \alpha y$. 由\autoref{IntL_eq4} $(i=1,2)$, 力场对质点做功等于两个定积分之和
\begin{equation}
W = \int \vec F \vdot \D \vec r =\int_0^\pi \alpha a(1-\cos t) \cdot a \sin t \D t + \int_0^\pi \alpha a \sin t \cdot a \cos t \D t
\end{equation}
注意到第一个积分中的第二项恰好是第二个积分的相反数,所以上式变为
\begin{equation}
\int_0^\pi a^2 \sin t \D t = \frac{\alpha a^2}{2}
\end{equation}

现在来计算延 $x$ 轴的直线轨迹运动的情况.由于轨迹上处处都有 $y=0$, $F_y = 0$,积分只有 $F_x$ 一项. 另外 $x$ 本身就可以作为轨道参数,即 $x(t) = t, y(t) = 0, x\in [0,a]$. 代入\autoref{IntL_eq4} 得做功为
\begin{equation}
W = \int_0^a \alpha x \D x = \frac{\alpha a^2}{2}
\end{equation}
\end{exam}

在上例中, 我们发现对于给定的矢量场, 即使路径不同,当起点和终点相同时, 线积分的结果也相同(虽然我们只计算了两条路径, 但这个结论是正确的). 具有这样性质的矢量场叫做\textbf{保守场},并总存在一个势能函数. 

\rentry{梯度\ 梯度定理\upref{Grad}}



