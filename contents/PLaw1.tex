% 动量 动量定理

\pentry{牛顿第二定律\upref{New3}, 矢量的导数\upref{DerV}}

令质点质量为 $m$,速度为 $\vec v$,定义其\textbf{动量}为
\begin{equation}
\vec p = m\vec v
\end{equation}
注意动量是矢量,与速度(矢量)的方向相同,且取决于坐标系.

现在把动量和速度都看做时间的函数. 等式两边求导,速度对时间的导数等于加速度 $\vec a$
\begin{equation}
\frac{\D\vec p}{\D t} = m\frac{\D\vec v}{{\D t}} = m\vec a
\end{equation}
根据牛顿第二定律,$m\vec a$ 等于质点所受合外力 $\vec F$ (注意力和加速度也都是时间的函数),所以
\begin{equation}
\frac{\D\vec p}{\D t} = \vec F
\end{equation}
这就是\textbf{动量定理}的微分形式,即动量的变化率等合外于力.也可以写成微分形式
\begin{equation}\label{PLaw_eq4}
\D\vec p = \vec F\D t
\end{equation}
即微小时间内的动量变化等于力乘以微小时间.

现在用定积分\upref{DefInt} 中的微元思想考虑动量从时刻 $t_1$ 到 $t_2$ 的总变化,我们可以把这段时间划分为 $N$ 段微小时间,第 $i$ 段所在的时刻记为 $t_i$,每小段时间内 $\vec F$ 可认为是恒力 $\vec F(t_i)$
\begin{equation}
\vec p(t_2)-\vec p(t_1) = \sum_{i=1}^{N} \Delta\vec p_i= \sum_{i=1}^{N} \vec F(t_i) \Delta t_i
\end{equation}
当 $N\to\infty, \Delta t\to 0$ 时该式可以用定积分(矢量函数)% 未完成
表示\footnote{通常省略以上的推导而直接表达为“\autoref{PLaw_eq4} 两边定积分得到\autoref{PLaw_eq6}”}
\begin{equation}\label{PLaw_eq6}
\vec p(t_2)-\vec p(t_1) = \int_{t_1}^{t_2}\vec F(t) \D t
\end{equation}
这是\textbf{动量定理}的积分形式.特殊地,对于恒力 $\vec F$,右边的积分等于 $(t_2-t_1)\vec F$.