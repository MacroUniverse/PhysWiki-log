% 电磁场中的单粒子薛定谔方程
% 电磁场|薛定谔方程|哈密顿算符|标势|矢势|广义动量

\pentry{电磁场标势和矢势\upref{EMPot}, 量子力学基本假设\upref{QMPos}}

\footnote{本词条使用原子单位}电动力学中,电磁场中单个粒子的哈密顿量为
\begin{equation}\label{QMEM_eq1}
H = \frac{1}{2m} (\bvec p - q\bvec A)^2 + q\varphi
\end{equation}
其中 $\varphi$ 和 $\bvec A$ 分别是电磁场的标势和矢势,$\bvec p$ 是广义动量,
\begin{equation}
\bvec p = q\bvec A + m \dot{\bvec r}
\end{equation}
这个公式适用于任意规范.

转换为量子力学中的算符为
\begin{equation}\label{QMEM_eq2}
\ali{
H &= \frac{\bvec p^2}{2m} - \frac{q}{2m} (\bvec A \vdot \bvec p + \bvec p \vdot \bvec A)
+ \frac{q^2}{2m} \bvec A^2 + q \varphi\\
&= -\frac{1}{2m} \laplacian + \I \frac{q}{2m} (\bvec A \vdot \Nabla + \Nabla \vdot \bvec A) + \frac{q^2}{2m} \bvec A^2 + q\varphi
}\end{equation}
注意其中 $\bvec p = -\I\Nabla$ 代表的是\textbf{广义动量}而不是 $m\bvec v$, $\Nabla \vdot \bvec A$ 是指先把波函数乘以矢势再取散度而不是直接对 $\bvec A$ 取散度(想想量子力学中算符相乘的定义).

电磁场的规范变换(\autoref{Gauge_eq3}~\upref{EMPot})导致波函数发生相位变化
\begin{equation}\label{QMEM_eq3}
\Psi'(\bvec r, t) = \exp[\I q\chi(\bvec r, t)] \Psi(\bvec r, t)
\end{equation}
其中 $\chi$ 是\autoref{Gauge_eq3}~\upref{EMPot} 中的任意标量函数 $\lambda$.

在库仑规范下,\autoref{QMEM_eq2} 变为
\begin{equation}\label{QMEM_eq4}
H = -\frac{1}{2m} \laplacian + \I \frac{q}{m} \bvec A \vdot \Nabla + \frac{q^2}{2m} \bvec A^2 + q\varphi
\end{equation}
这是因为
\begin{equation}
\div (\bvec A \Psi) = (\div \bvec A) \Psi + \bvec A \vdot (\grad \Psi) = \bvec A \vdot (\grad \Psi)
\end{equation}

