% 电磁场中的单粒子薛定谔方程

\footnote{本词条使用原子单位}电动力学中,电磁场中单个粒子的哈密顿量为
\begin{equation}
H = \frac{1}{2m} (\vec p - q\vec A)^2 + q\varphi
\end{equation}
其中 $\varphi$ 和 $\vec A$ 分别是电磁场的标势和矢势.

转换为量子力学中的算符为
\begin{equation}\label{QMEM_eq2}
\ali{
H &= \frac{\vec p^2}{2m} - \frac{q}{2m} (\vec A \vdot \vec p + \vec p \vdot \vec A)
+ \frac{q^2}{2m} \vec A^2 + q \varphi\\
&= -\frac{1}{2m} \laplacian + \I \frac{q}{2m} (\vec A \vdot \Nabla + \Nabla \vdot \vec A) + \frac{q^2}{2m} \vec A^2 + q\varphi
}\end{equation}
注意其中 $\vec p = -\I\Nabla$ 代表的是\bb{广义动量}而不是 $m\vec v$, $\Nabla \vdot \vec A$ 是指先把波函数乘以矢势再取散度而不是直接对 $\vec A$ 取散度.

电磁场的规范变换导致波函数发生相位变化
\begin{equation}\label{QMEM_eq3}
\Psi'(\vec r, t) = \exp[\I q\chi(\vec r, t)] \Psi(\vec r, t)
\end{equation}
其中 $\chi$ 是\autoref{EMPot_eq3}\upref{EMPot} 中的任意标量函数.

在库仑规范下,\autoref{QMEM_eq2} 变为
\begin{equation}\label{QMEM_eq4}
H = -\frac{1}{2m} \laplacian + \I \frac{q}{m} \vec A \vdot \Nabla + \frac{q^2}{2m} \vec A^2 + q\varphi
\end{equation}
这是因为
\begin{equation}
\div (\vec A \Psi) = (\div \vec A) \Psi + \vec A \vdot (\grad \Psi) = \vec A \vdot (\grad \Psi)
\end{equation}

