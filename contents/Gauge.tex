% 规范变换
% 库仑规范|洛伦兹规范|标量势能|矢量势能|麦克斯韦方程组

\pentry{电磁场标势和矢势\upref{EMPot}}

虽然标势和矢势可以唯一确定电磁场,但是同一个电磁场却可以由不同的标势和矢势得到.

首先给矢势(\autoref{EMPot_eq2})加上某个标量函数的梯度,磁场不变.
\begin{equation}
\bvec B' = \curl \bvec A' = \curl (\bvec A + \grad \lambda) = \curl \bvec A = \bvec B
\end{equation}
但如果 $\lambda$ 随时间变化,\autoref{EMPot_eq2}\upref{EMPot} 中的电场会改变.我们只需把标势减去 $\lambda$ 关于时间的导数即可使电场不变
\begin{equation}
\bvec E' = -\grad \varphi' - \pdv{\bvec A'}{t} = -\grad \qty(\varphi - \pdv{\lambda}{t}) - \pdv{t} (\bvec A + \grad \lambda) = \bvec E
\end{equation}
所以标势和矢势的\textbf{规范变换}为
\begin{equation}\label{Gauge_eq3}
\leftgroup{
&\bvec A' = \bvec A + \grad \lambda\\
&\varphi' = \varphi - \pdv{\lambda}{t}
}\end{equation}
任何产生相同电磁场的标势矢势都可以通过规范变换联系起来.

\subsection{库仑规范}
当我们选择 $\lambda$ 使得 $\div \bvec A = 0$ 时,就得到了\textbf{库仑规范}. 库仑规范下, 标势和矢势的麦克斯韦方程组(\autoref{EMPot_eq4}\upref{EMPot} \autoref{EMPot_eq5}\upref{EMPot})化简为
\begin{equation}\label{Gauge_eq4}
\laplacian \varphi = -\frac{\rho}{\epsilon_0}
\end{equation}
\begin{equation}
\laplacian \bvec A - \mu_0\epsilon_0 \pdv[2]{\bvec A}{t} = -\mu_0\bvec J + \mu_0\epsilon_0 \grad\qty(\pdv{\varphi}{t})
\end{equation}
其中\autoref{Gauge_eq4} 的形式与静止电荷分布的泊松方程%链接未完成
形式一样,但同样适用于变化的电荷.这看起来似乎是瞬时作用,但由于标势和矢势都只是数学上的量而不是物理上存在的量,所以是完全正确的.

当空间中没有电荷只有电磁场时,以上两式进一步化简为
\begin{equation}
\laplacian \varphi = 0
\end{equation}
\begin{equation}
\laplacian \bvec A - \mu_0\epsilon_0 \pdv[2]{\bvec A}{t} = \mu_0\epsilon_0 \grad\qty(\pdv{\varphi}{t})
\end{equation}
若假设无穷远处没有电荷电流也没有电场磁场, 那么无穷远处的标势矢势需满足
\begin{equation}
\laplacian \varphi = 0 \qquad
\grad \varphi + \pdv{\bvec A}{t} = 0
\end{equation}
以及 $\bvec A$ 为调和场. 任何满足 $\laplacian \lambda = 0$ 的规范变换都能满足这些条件. 可见库仑规范本身并不能唯一确定标势矢势,还需要一定的边界条件.

令标势的边界条件为\footnote{见 Griffiths}无穷远处 $\varphi = 0$, 有
\begin{equation}
\varphi(\bvec r, t) = \frac{1}{4\pi\epsilon_0} \int \frac{\rho(\bvec r', t)}{\abs{\bvec r - \bvec r'}} \dd[3]{r'}
\end{equation}

如果空间中没有电荷和电流,那么 $\varphi$ 处处为零.由此得到一个常见的结论是
\begin{equation}
\bvec E = - \pdv{\bvec A}{t}
\end{equation}
矢势的波动方程也变得非常简单
\begin{equation}
\laplacian \bvec A - \frac{1}{c^2} \pdv[2]{\bvec A}{t} = 0
\end{equation}
该方程的通解(先不要求 $\div \bvec A = 0$)就是由任意极化方向\footnote{这里指 $\bvec A$ 的方向} $\uvec \varepsilon$ 和传播方向的光速平面波
\begin{equation}
\uvec \varepsilon \cos(\bvec k \vdot \bvec r - \omega t + \phi) \qquad (\omega = ck)
\end{equation}
的线性组合. 再加上 $\div \bvec A = 0$ 条件,得 $\uvec \varepsilon \vdot \bvec k = 0$, 即极化方向与传播方向垂直.

\subsection{洛伦兹规范}
如果令
\begin{equation}
\div \bvec A = -\mu_0 \epsilon_0 \pdv{\varphi}{t}
\end{equation}
那么标势和矢势就符合\textbf{洛伦兹规范}. 

麦克斯韦方程组(\autoref{EMPot_eq4}\upref{EMPot} \autoref{EMPot_eq5}\upref{EMPot})将变为十分对称的形式
\begin{equation}\label{Gauge_eq14}
\laplacian \varphi - \mu_0\epsilon_0 \pdv[2]{\varphi}{t} = -\frac{\rho}{\epsilon_0}
\end{equation}
\begin{equation}\label{Gauge_eq15}
\laplacian \bvec A - \mu_0\epsilon_0 \pdv[2]{\bvec A}{t} = -\mu_0 \bvec J
\end{equation}
