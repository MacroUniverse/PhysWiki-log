% 匀加速运动

\pentry{速度\ 加速度\upref{VnA}}
若在一段时间内, 质点的加速度矢量不随时间变化(常矢量) $\bvec a$, 那么我们说质点做\bb{匀加速运动}. 由“ 速度\ 加速度\upref{VnA}” 中的\autoref{VnA_eq7} 和\autoref{VnA_eq8}, 速度和位移函数分别为
\begin{equation}\label{ConstA_eq1}
\bvec v(t) = \bvec v_0 + \int_{t_0}^{t} \bvec a \dd{t} = \bvec v_0 + \bvec a \cdot (t-t_0)
\end{equation}
\begin{equation}\label{ConstA_eq2}
\bvec r(t) = \bvec r_0 + \int_{t_0}^{t} \bvec v(t) \dd{t} = \bvec r_0 + \bvec v_0\cdot (t-t_0) + \frac12 \bvec a\cdot (t-t_0)^2
\end{equation}

% 速度 加速度 词条中应该说明积分两次得到位移-时间函数

\subsection{自由落体运动}
一个最简单的匀加速运动是\bb{自由落体运动}. 自由落体运动是初速度 $\bvec v_0 = 0$, 竖直向下加速度为重力加速度恒为 $g$ 的匀加速直线运动. 其中 $g\approx 9.8\Si{m/s^2}$ 是\bb{重力加速度}, 也可以用常矢量 $\bvec g$ 表示.代入\autoref{ConstA_eq1} 和\autoref{ConstA_eq2} 得
\begin{equation}\label{ConstA_eq3}
\bvec v(t) = \bvec g \cdot (t-t_0)
\end{equation}
\begin{equation}\label{ConstA_eq4}
\bvec r(t) = \bvec r_0 + \frac12 \bvec g \cdot (t-t_0)^2
\end{equation}

\subsection{抛体运动}
作为一个稍复杂的情况, 抛体运动是加速度为 $\bvec g$, 初速度为 $\bvec v_0$ 的匀加速运动. 将 $\bvec a = \bvec g$ 代入\autoref{ConstA_eq1} 和\autoref{ConstA_eq2} 得
\begin{equation}\label{ConstA_eq5}
\bvec v(t) = \bvec v_0 + \bvec g \cdot (t-t_0)
\end{equation}
\begin{equation}\label{ConstA_eq6}
\bvec r(t) = \bvec r_0 + \bvec v_0\cdot (t-t_0) + \frac12 \bvec g\cdot (t-t_0)^2
\end{equation}
对比\autoref{ConstA_eq4} 和\autoref{ConstA_eq6} 可以发现抛体运动就是自由落体运动与匀速直线运动的矢量叠加. 所以如果我们在一个相对于当前参考系以 $\bvec v_0$ 运动的参考系中观察抛体运动, 就会是自由落体运动.

