% SLISC 密矩阵的切割
% C++|密矩阵|切割|view|SLISC

\begin{issues}
\issueDraft
\end{issues}

\pentry{SLISC 的矢量和矩阵\upref{SliMat}}

如果我们想把矩阵的一部分(例如一行)输入到某个函数中怎么办呢? 如果我们不使用任何容器而是用指针, 函数的输入输出也使用指针, 那么可以通过 \verb|step| 参数实现:
\begin{lstlisting}[language=cpp]
Doub sum(Doub_I *x, Long_I N, Long_I step)
{
    Doub s = 0;
    for (Long i = 0; i < N; ++i)
        s += x[step*i];
    return s;
}
\end{lstlisting}

\begin{lstlisting}[language=cpp]
Long N0 = 10, N1 = 10; row = 3;
Doub *a; // 列主序矩阵
new Doub a[N0*N1];
for (Long i = 0; i < N0*N1; ++i)
    a[i] = i;
// 对第 3 行求和
cout << "sum a(3,all) = " << sum(a+3, N1, N0) << endl;
// 对第 3 列求和
cout << "sum a(all,3) = " << sum(a+3*N0, N0, 1) << endl;
// 对第 3 列的前 3 个元求和
cout << "sum(0-2,3) = " << sum(a+3*N0, 3, 1) << endl;
delete a[];
\end{lstlisting}
再看一个例子, 如何对子矩阵求和呢? 我们需要一个参数叫 leading dimension, 该参数表示列标(行主序时:行标)增加 1 时单索引增加的数量.
\begin{lstlisting}[language=cpp]
// N0, N1 时子矩阵的尺寸, lda 是原矩阵的行数
Doub sum(Doub_I *a, Long_I N0, Long_I N1, Long_I lda)
{
    Doub s = 0;
    for (Long j = 0; i < N1; ++i) {
        for (Long i = 0; i < N0; ++i)
            s += a[i];
        a += lda;
    }
    return s;
}
\end{lstlisting}
使用方法:
\begin{lstlisting}[language=cpp]
Long N0 = 10, N1 = 10;
Doub *a;
new Doub a[N0*N1]; // 列主序
for (Long i = 0; i < N0*N1; ++i)
    a[i] = i;
// 对 3-5 行, 4-6 列的子矩阵求和
cout << "sum(3-5,4-6) = " << sum(a+4*N0+3, 3, 3, N0) << endl;
\end{lstlisting}
可见直接使用裸指针给函数传递数组是非常灵活的, 在 C 语言中这也是很常见的做法, 见 BLAS\upref{BLAS} 和 LAPACK. 然而这么做的缺点是容易出错, 例如索引超出分配的内存, 或者忘记用 \verb|delete| 释放内存. 这样的错误调试起来会非常困难.

\subsection{view 类型}
在 SLISC 库中, 函数的输入或输出除了各种容器外, 也可以是 view 类型的一种. 例如上面的 \verb|sum| 函数也可以定义为
\begin{lstlisting}[language=cpp]
Doub sum(DvecDoub_I x)
{
    Doub s = x[0];
    for (Long i = 1; i < x.size(); ++i)
        s += x[x.step()*i];
    return s;
}
\end{lstlisting}
其中 \verb|DvecDoub| 是 view 类型的一种, 它包含一个指针 \verb|Doub *m_p|, 一个长度 \verb|Long *m_N| 以及一个步长 \verb|Long *m_step|. 它生成和毁灭时不会分配或释放内存, 它只用于表示另一个容器的一部分数据, 所以叫做 view, 也叫 slicing.

SLISC 提供一系列开头为 \verb|cut| 的函数, 可以从容器中截取一部分然后返回 view 类型的一种. 例如以上 \verb|sum| 函数可以这样使用
\begin{lstlisting}[language=cpp]
CmatDoub a(10, 10);
// 对第 3 行求和
cout << "sum row 3 = " << sum(cut1(a, 3)) << endl;
\end{lstlisting}
其中 \verb|cut1(a, 3)| 从矩阵 \verb|a| 中切下第 3 行(\verb|1| 表示第 1 个维度, 即 “行”), 然后返回一个临时的 \verb|const DvecDoub &| 类型传给 \verb|sum| 函数.

类似地, \verb|cut0(a, 3)| 从矩阵 \verb|a| 中切下第 3 列(\verb|0| 表示第 0 个维度, 即 “列”), 然后返回一个临时的 \verb|const SvecDoub &| 类型. \verb|SvecDoub| 是一个密矢量的 view, 没有 \verb|m_step| 成员, 所有元素在内存中都是相邻的. 在 view 类型命名中, \verb|S| 代表 smooth, \verb|D| 代表 dash.

同理, 如果要切割一个子矩阵, 用 \verb|cut(a, i, N0, j, N1)| 可以切出一个 \verb|DcmatDoub| 类型, 它的成员包括一个起始指针 \verb|Doub *m_p|, 子矩阵尺寸 \verb|Long *m_N0, *m_N1|, 总长度 \verb|Long m_N|, 以及 leading dimension \verb|m_lda|.

view 类型和指针一样, 也分为 low level const 和 high level const, \verb|const SvecDoub| 代表 high level const, 也就是它一旦初始化后数据成员就都是 const, 其中 \verb|m_p| 是 high level, 即\verb|const Doub *|. view 的 low level const 则有专门的类型如 \verb|SvecDoub_c|, 它的成员可以随意改变, 但所代表的数据不能被修改.

在函数参数中, \verb|SvecDoub_I| 的定义是 \verb|const SvecDoub_c &| 用于输入; \verb|SvecDoub_O, SvecDoub_IO| 的定义是 \verb|const SvecDoub &|, 用于输出.

\addTODO{具体介绍一下所有类型}
