% 平行轴定理与垂直轴定理

\pentry{转动惯量\upref{RigRot}}

\subsection{平行轴定理}
若我们已知刚体关于一个通过其质心的轴(称为\bb{质心轴})的转动惯量为 $I_0$, 那么我们可以通过平行轴定理简单地求出刚体关于另一个与质心轴平行的轴的转动惯量 $I$, 而无需重新算一次定积分. 令两个轴之间的距离为 $R$, 刚体质量为 $M$, 则计算公式为% 图未完成
\begin{equation}\label{MIthm_eq1}
I = I_0 + MR^2
\end{equation}

要证明该式, 我们把刚体看做质点系, 令质心轴到质点 $m_i$ 的垂直矢量为 $\vec r_i$, 平行轴到质心轴的垂直矢量为 $\vec R$, 则刚体关于平行轴的转动惯量为
\begin{equation}
I = \sum_i m_i (\vec R + \vec r_i)^2 = R^2\sum_i m_i + \sum_i m_i r_i^2 + 2\vec R \vdot \sum_i m_i \vec r_i
\end{equation}
由于质心轴经过刚体的质心, 上式最后一项中的求和为零(\autoref{CM_eq7}\upref{CM}), 而右边第二项恰好是 $I_0$, 右边第一项中 $\sum_i m_i = M$, 立即可得\autoref{MIthm_eq1}. 证毕.

% 未完成: 例子, 细棒的中心轴和平行轴.

\subsection{垂直轴定理}

若我们要求一个刚体薄片关于一条与其垂直的轴(称为\bb{垂直轴})的转动惯量 $I$, 我们可以在薄片上取两个互相垂直且与垂直轴相交的轴并分别计算薄片关于这两条轴的转动惯量 $I_x$ 和 $I_y$. 这样就有
\begin{equation}
I = I_x + I_y
\end{equation}

要证明该式, 我们建立空间直角坐标系, 令垂直轴与 $z$ 轴重合, 另外两条轴分别与 $x$ 轴和 $y$ 轴重合. 把刚体看做质点系, 令质点 $m_i$ 的坐标为 $(x_i, y_i, 0)$
\begin{equation}
I = \sum_i m_i (x_i^2 + y_i^2) = \sum_i m_i x_i^2 + \sum_i m_i y_i^2 = I_x + I_y
\end{equation}
证毕.