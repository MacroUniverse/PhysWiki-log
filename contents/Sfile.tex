% SLISC 的文件读写

\begin{issues}
\issueDraft
\end{issues}

\verb|Bool file_exist(Str_I fname, Bool_I case_sens = true)| 判断指定文件是否存在

\verb|Bool dir_exist(Str_I path)| 判断路径是否存在

\verb|Long file_size(Str_I fname)| 获取文件大小(字节数), \verb|fname| 是文件名.

\verb|Str path2dir(Str_I fname)| 从 “路径+文件” 字符串中提取 “路径” 部分

\verb|void mkdir(Str_I path)| 创建路径

\verb|void rmdir(Str_I path)| 删除路径

\verb|void ensure_dir(Str_I dir_or_file)| 如果路径不存在, 创建该路径(可以任意多层)

\verb|void file_list(vecStr_O fnames, Str_I path, Bool_I append = false)| 列出指定路径中所有文件, \verb|append = true| 则 \verb|fnames| 原有的元素不被清空.

\verb|void file_list_full(vecStr_O fnames, Str_I path, Bool_I append = false)| 和 \verb|file_list()| 相同, 只是文件名包含路径.

\verb|void folder_list(vecStr_O folders, Str_I path, Bool_I append = false)| 列出指定路径中所有文件夹

\verb|void folder_list_full(vecStr_O folders, Str_I path, Bool_I append = false)| 和 \verb|folder_list()| 相同, 但包含路径.

\verb|void file_list_r(vecStr_O fnames, Str_I path, Bool_I append = false)| 列出指定路径中所有文件, 包括所有子文件夹中的.

\verb|void file_ext(vecStr_O fnames_ext, vecStr_I fnames, Str_I ext, Bool_I keep_ext = true, Bool_I append = false)| 在文件列表中选出所有具有某个拓展名的文件

\verb|void file_list_ext(vecStr_O fnames, Str_I path, Str_I ext, Bool_I keep_ext = true, Bool_I append = false)| 在某个路径中寻找所有具有某个拓展名的文件

\verb|void file_copy(Str_I fname_out, Str_I fname_in, Bool_I replace = false)| 复制文件 \verb|replace| 用于替换目标的同名文件, 否则会一直提示, 需要手动删除程序才能继续.

\verb|void file_copy(Str_I fname_out, Str_I fname_in, Str_IO buffer, Bool_I replace = false)| 用户提供缓存 \verb|buffer| 的 \verb|file_copy|, 提高速度.

\verb|void file_move(Str_I fname_out, Str_I fname_in, Bool_I replace = false)| 移动文件.

\verb|void file_move(Str_I fname_out, Str_I fname_in, Str_IO buffer, Bool_I replace = false)| 带用户缓存的版本.

\verb|void open_bin(ofstream &fout, Str_I fname)| 打开二进制文件写入

\verb|void open_bin(ifstream &fin, Str_I fname)| 打开二进制文件读取

\verb|void write(const Char *data, Long_I Nbyte, Str_I fname)| 把字符串写入二进制文件

\verb|Long read(Char *data, Long_I Nbyte, Str_I fname)| 读取二进制文件到字符串, 返回实际上读取到的字节数.

\verb|void write(Str_I str, Str_I fname)| 把字符串写入二进制文件

\verb|void write_vec_str(vecStr_I vec_str, Str_I fname)| 把每个字符串写入文件中的一列

\verb|void read_vec_str(vecStr_O vec_str, Str_I fname)|

\verb|void read(Str_O str, Str_I fname)| 读取二进制文件到字符串

\verb|void write(ofstream &fout, Char_I s)| 把一个 \verb|Char| 写入二进制文件, 也可以是 \verb|Int|, \verb|Llong|, \verb|Doub|, \verb|Comp|, \verb|Str|.

\verb|void read(ifstream &fin, Char_I s)| 和 \verb|write| 同理.

\verb|void read(CmatInt_O mat, Str_I file, Long_I skip_lines = 0)| 从一个文件中读取密矩阵, 矩阵会被 \verb|.resize| 成需要的形状. \verb|CmatInt| 也可以是 \verb|CmatLlong| 或 \verb|CmatDoub|

\verb|void read(VecDoub_O v, Str_I file, Long_I skip_lines = 0)| 从文件中读取单列矢量.

\verb|void last_modified(Str_O yymmddhhmmss, Str_I fname)| 获取文件的最后修改时间.

\verb|void set_buff(ofstream &fout, Str_IO buffer)| 缓存越大, 速度越快.

\verb|Bool little_endian()| 操作系统是否是 \verb|little_endian| 的.

\verb|void change_endian(Char *data, Long_I elm_size, Long_I Nelm)| 把一个字符串中每个长度为 \verb|elm_size| 的单元前后翻折.
