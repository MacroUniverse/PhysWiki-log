%正交归一基底

\pentry{矢量点乘\upref{Dot}}

我们已经知道了矢量基底的概念, 如果一组矢量基底中的每个矢量模长都为 $1$ 且每两个矢量都正交, 则我们把这组基底称为\bb{正交归一基} 或\bb{单位正交基}. 若把这组正交归一基记为 $\uvec x_1,\uvec x_2\dots\uvec x_n$, 则正交归一可以用点乘表示为
\begin{equation}
\uvec x_i \vdot \uvec x_j = \delta _{ij}
\end{equation}
其中 $\delta_{ij}$ 是\bb{克罗内克 $\delta$ 函数(Kronecker delta function)}, 定义为
\begin{equation}\label{OrNrB_eq2}
\delta_{ij} = \leftgroup{&1\qquad (i = j)\\ &0\qquad (i \ne j)}
\end{equation}

任意矢量在单位正交基上的展开
 \begin{equation}
\vec v = \sum_{i = 1}^n (\vec v\vdot\uvec x_i)\,\uvec x_i = \sum_{i = 1}^n v_i \,\uvec x_i
\end{equation}
最常见的例子就是几何矢量在直角坐标系的 $\uvec x, \uvec y, \uvec z$ 三个单位正交矢量上的展开.
 \begin{equation}
\vec v = (\vec v \vdot \uvec x)\,\uvec x + (\vec v \vdot \uvec y)\,\uvec y + (\vec v \vdot \uvec z)\,\uvec z = v_x \,\uvec x + v_y \,\uvec y + v_z \,\uvec z
\end{equation} 

\subsection{证明}
由于任何矢量都可以表示成基底 $\uvec x_1 \dots \uvec x_n$ 的线性组合,设
\begin{equation}\label{OrNrB_eq5}
\vec v = \sum_{i = 1}^n c_i \uvec x_i
\end{equation} 
用 $\uvec x_k$ 乘以等式两边,得
\begin{equation}
\vec v \vdot \uvec x_k = \sum_{i = 1}^n  c_i \uvec x_i \vdot\uvec x_k = \sum_{i = 1}^n c_i \delta_{ik}  = c_k
\end{equation}
所以\autoref{OrNrB_eq5} 中的系数有唯一确定的值 $c_k = \vec v \vdot\uvec x_k$. 证毕.






