%8 min
%未完成
\textbf{正交归一基}(也叫\textbf{单位正交基}).
这里先讨论几何矢量或者代数矢量,但结论也适用于其他矢量.

如果所讨论的任意的矢量都可以用一组矢量 $\uvec x_1,\uvec x_2\dots\uvec x_n$ 的线性组合表示,且定义了点乘运算,使得 ${\uvec x_i}\vdot{\uvec x_j} = {\delta _{ij}}$ ( 是克罗内克 $\delta$ 函数),那么 ${\uvec x_1},{\uvec x_2}\dots{\uvec x_n}$ 就是正交归一基,因为 $\uvec x_i$ 的模长 $\sqrt{\uvec x_i \vdot \uvec x_i} = \sqrt{\delta _{ij}}= 1$, 且任意两个不同的矢量正交.

任意矢量在单位正交基上的展开
 \begin{equation}
\vec v = \sum\limits_{i = 1}^n (\vec v\vdot\uvec x_i)\,\uvec x_i = \sum\limits_{i = 1}^n v_i \,\uvec x_i
\end{equation}
最常见的例子就是几何矢量在直角坐标系的 $\uvec x$, $\uvec y$, $\uvec z$ 三个单位正交矢量上的展开.
 \begin{equation}
\vec v = (\vec v \vdot \uvec x)\,\uvec x + (\vec v \vdot \uvec y)\,\uvec y + (\vec v \vdot \uvec z)\,\uvec z = v_x \,\uvec x + v_y \,\uvec y + v_z \,\uvec z
\end{equation} 

\subsection{证明}
由于任何矢量都可以表示成基底 ${\uvec x_1}\dots{\uvec x_n}$ 的线性组合,设
\begin{equation}\label{OrNrB_eq1}
\vec v = \sum\limits_{i = 1}^n {{c_i}{{\,\uvec x}_i}} 
\end{equation} 
用 ${\uvec x_k}$ 乘以等式两边,得 $\vec v{\vdot\uvec x_k} = \sum\limits_{i = 1}^n {{c_i}{{\uvec x}_i}{{\vdot\uvec x}_k}}  = \sum\limits_{i = 1}^n {{c_i}{\delta _{ik}}}  = {c_k}$. 所以\autoref{OrNrB_eq1} 中的系数有唯一确定的值 ${c_k} = \vec v{\vdot\uvec x_k}$. 证毕.






