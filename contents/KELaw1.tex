% 动能 动能定理(单个质点)

\pentry{功\ 功率\upref{Fwork},牛顿第二定律\upref{New3}}%未完成

令质点的质量为 $m$,速度为 $\vec v$,则质点的\textbf{动能}定义为
\begin{equation}
E_k = \frac 12 m\vec v^2 = \frac 12 mv^2
\end{equation}

质点的\textbf{动能定理}是,\textbf{一段时间内质点动能的变化等于合外力对质点做的功}.从变化率(即时间导数)的角度来看,动能定理也可以表述为\textbf{质点的动能变化率等于合外力对质点的功率}.

\subsection{推导}
力对质点做功的功率\upref{Fwork}为
\begin{equation}\label{KELaw1_eq2}
P = \dv{W}{t} =  \vec F\vdot \dv{\vec r}{t} = \vec F\vdot\vec v
\end{equation}
再来看动能的变化率
\begin{equation}
\dv{t} E_k = \frac 12 m\dv{t} (\vec v\vdot\vec v)
\end{equation}
由矢量点乘的求导\upref{DerV}\autoref{DerV_eq5},$\dv*{\vec v\vdot\vec v}{t} = 2\vec v\vdot \dv*{\vec v}{t} = 2\vec v\vdot\vec a$,上式变为
\begin{equation}\label{KELaw1_eq4}
\dv{t} E_k = m\vec a\vdot\vec v = \vec F\vdot\vec v
\end{equation}
最后一步使用了牛顿第二定律.%未完成
注意\autoref{KELaw1_eq2} 与\autoref{KELaw1_eq4} 相等,所以动能变化率等于合外力的功率.