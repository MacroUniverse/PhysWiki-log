% 比耐公式

\pentry{开普勒第一定律证明\upref{Keple1}}

若质量为 $m$ 的质点受有心力 $f\left( r \right)$ (规定引力为负,斥力为正)时,若运动轨道用极坐标方程表示为 $r\left( \theta  \right)$. 且令 $u\left( \theta  \right) = 1/{r\left( \theta  \right)}$, 且把 $f\left( r \right)$ 中的 $r$ 替换成 $u$, 令其为
 $f\left( r \right) = f(1/u) = F\left( u \right)$. 则有以下微分方程.
 \begin{equation}
{h^2}{u^2}\left( {\dv[2]{u}{\theta} + u} \right) = -\frac{F\left( u \right)}{m}
\end{equation} 
这方程叫做\textbf{比耐公式(Binet Equation)}.其中 $h = {r^2} \dv*{\theta}{t}$ 是质点对力心的角动量比质量,对一个质点的一条轨道来说,是一个常数.

\subsection{推导}
在开普勒第一定律证明中,式(14)%未完成
列出了 $u$ (行星到中心天体的距离的倒数)与 $\theta $ (行星围绕中心天体转过的角度)的微分方程. 
\begin{equation}\label{Binet_eq1}
{h^2}{u^2}\left( {\dv[2]{u}{\theta} + u} \right) =  GM{u^2}
\end{equation} 
注意等号右边是行星受到中心天体的万有引力.一个质点受到有心力%未完成
 $F\left( u \right)$, 只有两个可能的方向,现规定向外为正,向内为负.所以\autoref{Binet_eq1} 变为
\begin{equation}\label{Binet_eq2}
{h^2}{u^2}\left( {\dv[2]{u}{\theta} + u} \right) = -\frac{F(u)}{m}
\end{equation} 
事实上,当 $F\left( u \right)$ 是其他函数时,例如大小与 $r$ 成正比(橡皮绳),经过同样的推导,也可以得到\autoref{Binet_eq2}.所以\autoref{Binet_eq2} 描述了一个质点受到有心力时的运动情况,通过解这个方程,就可以求出质点的运动轨迹.














