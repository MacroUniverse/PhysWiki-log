% 狄拉克 delta 函数
% 微积分|定积分|狄拉克 delta 函数|\delta 函数

\pentry{定积分\upref{DefInt}}
在物理中我们经常会遇到一些模型, 如质点和点电荷等, 这类模型使用了极限的思想(如令体积趋于无穷小). 如果考察质点的密度或点电荷的电荷密度, 将得到无穷大, 然而将其密度(电荷密度)在空间中积分却又能得到有限的质量与电荷. 为了描述这样的密度(电荷密度)分布, 我们引入狄拉克 $\delta$ 函数.

\subsection{一维情况}

\begin{figure}[ht]
\centering
\includegraphics[width=11cm]{./figures/Delta1.pdf}
\caption{$\delta(x - x_0)$ 的几个例子} \label{Delta_fig1}
\end{figure}

我们来考虑一个函数(\autoref{Delta_fig1} 左)
\begin{equation}
f(x) =
\begin{cases}
h & \qty( \abs{x - x_0} \leqslant \frac{1}{2h} )\\
0 & \qty( \abs{x - x_0} > \frac{1}{2h} )
\end{cases}
\end{equation}
其中 $h, x_0$ 是常数. 由函数图像易得函数曲线下面的面积为 $\int_{-\infty}^{+\infty} f(x) \dd{x} = 1$. 现在我们令 $h \to 0$, 长方形的高将趋于无穷大, 宽将趋于零, 而定积分结果不变.

我们定义\textbf{狄拉克 $\delta$ 函数}\footnote{严格来说, 这并不是一个数学上的函数, 但在物理中却经常出现.} $\delta(x)$ 为满足以下两个性质的函数\footnote{事实上,$\delta(x)$ 并不是数学中一个严格意义上的函数, 而是被称为广义函数.}
\begin{equation}\label{Delta_eq2}
\delta(x) =
\begin{cases}
+\infty & (x = 0)\\
0 & (x \ne 0)
\end{cases}
\end{equation}
\begin{equation}\label{Delta_eq3}
\int_{-\infty}^{+\infty} \delta(x) \dd{x} = 1
\end{equation}

这样, 上面的 $f(x)$ 就可以表示为 $\delta(x - x_0)$. 当然, 我们还可以选取其他含有参数的 $f(x)$ 来逼近 $\delta$ 函数, 如\autoref{Delta_fig1} 中的另外两种函数.

\begin{example}{高斯波包}
我们来
\end{example}

\subsection{性质}
对任意连续函数 $f(x)$, 有
\begin{equation}
\int_{-\infty}^{+\infty} f(x) \delta(x - x_0) \dd{x}= f(x_0)
\end{equation}
要证明这个性质, 我们可以把积分上下限变为 $x_0 \pm \epsilon$, 这样并不会改变积分结果, 因为在区间 $[x_0+\epsilon, +\infty)$ 和 $(-\infty, x_0 - \epsilon]$ 中, $\delta(x-x_0) = 0$. 然后我们令 $\epsilon\to 0$, 这样 $f(x)$ 在 $[x_0 - \epsilon, x_0 + \epsilon]$ 区间内的值就趋近于常数 $f(x_0)$. 所以有
\begin{equation}\ali{
\int_{-\infty}^{+\infty} f(x) \delta(x - x_0) \dd{x} &= \int_{x_0-\epsilon}^{x_0+\epsilon} f(x) \delta(x - x_0)\dd{x} \\
&= f(x_0)\int_{x_0-\epsilon}^{x_0+\epsilon} \delta(x - x_0)\dd{x} = f(x_0)
}\end{equation}
证毕.

$\delta$ 函数的另一个性质是
\begin{equation}
\delta(ax) = \frac{1}{\abs{a}} \delta(x)
\end{equation}
我们不妨从几何上来证明这个性质, 与 $\delta(x)$ 相比较, $\delta(ax)$ 的图像在 $x$ 方向缩小了 $\abs{a}$ 倍(当 $a < 0$ 时, 还另需要关于 $y$ 轴翻转), 所以 $\delta(ax)$ 下的面积是 $1/\abs{a}$(注意翻转并不影响面积), 故 $\abs{a}\delta(ax)$ 下的面积是 $1$, 且满足\autoref{Delta_eq2} 和\autoref{Delta_eq3}, 所以有 $\abs{a}\delta(ax) = \delta(x)$. 证毕.

\subsection{三维情况}
(未完成)
