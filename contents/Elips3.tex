% 椭圆的三种定义

\pentry{圆锥曲线的极坐标方程\upref{Cone}}

\subsection{第二种定义}
我们前面已经见过椭圆在极坐标中的定义, 但从椭圆的极坐标公式难以看出椭圆的对称性,这里用相同的定义推导直角坐标的表达式. 我们不妨先以一个焦点为原点定义直角坐标系, 且令 $x$ 轴指向另一个焦点, 则\autoref{Cone_eq1}\upref{Cone} 变为
\begin{equation}
\frac{{\sqrt {{x^2} + {y^2}} }}{{x + h}} = e
\end{equation}
其中 $h$ 为 $y$ 轴到准线的距离. 两边平方并整理得
\begin{equation}\label{Elips3_eq2}
(1 - {e^2}){\left( {x - \frac{{{e^2}h}}{{1 - {e^2}}}} \right)^2} + {y^2} = \frac{{{e^2}{h^2}}}{{1 - {e^2}}}
\end{equation}
由此可见,如果我们把椭圆左移 ${e^2}h/(1 - {e^2})$,椭圆将具有
\begin{equation}\label{Elips3_eq3}
\frac{{{x^2}}}{{{a^2}}} + \frac{{{y^2}}}{{{b^2}}} = 1
\end{equation}
的形式. 其中 $a$ 为\textbf{半长轴}, $b$ 为\textbf{半短轴}.这就是椭圆的第二种定义, 即把单位圆沿两个垂直方向分别均匀拉长 $a$ 和 $b$.下面来看系数的关系.首先定义椭圆的焦距为焦点到椭圆中心的距离(即以上左移的距离)为
\begin{equation}
c = \frac{{{e^2}h}}{{1 - {e^2}}}
\end{equation}
\autoref{Elips3_eq2} 和\autoref{Elips3_eq3}对比系数得
\begin{equation}
a = \frac{{eh}}{{1 - {e^2}}} \qquad b = \frac{{eh}}{{\sqrt {1 - {e^2}} }}
\end{equation}
不难证明
\begin{equation}
{a^2} = {b^2} + {c^2}
\end{equation}
以及
\begin{equation}
e = \frac{c}{a} \qquad h = \frac{{{b^2}}}{c}
\end{equation}

\subsection{第三种定义}
椭圆的第三种定义是, 椭圆上任意一点到两焦点的距离之和等于 $2a$. 现在我们来证明前两种定义下的椭圆满足这个条件. 由直角坐标方程可知对称性,可在椭圆的两边做两条准线,令椭圆上任意一点到两焦点的距离分别为 $r_1$, $r_2$,到两准线的距离分别为 $d_1$, $d_2$,则有
\begin{equation}
e = \frac{{{r_1}}}{{{d_1}}} = \frac{{{r_2}}}{{{d_2}}} = \frac{{{r_1} + {r_2}}}{{{d_1} + {d_2}}}
\end{equation}
所以
\begin{equation}
{r_1} + {r_2} = e({d_1}{\text{ + }}{d_2}) = 2e(c + h) = 2\frac{c}{a}\left( {c + \frac{{{b^2}}}{c}} \right) = 2a
\end{equation}
证毕.
