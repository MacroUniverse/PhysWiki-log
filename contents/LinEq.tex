% 线性方程组

% 未完成:这个之前应该有更低级的词条

\pentry{矩阵与矢量空间} %未完成:矩阵代表矢量空间之间的映射,一一映射,

线性方程组可以记为
\begin{equation}\label{LinEq_eq1}
\mat A \vec x = \vec y
\end{equation}
其中 $\mat A$ 是 $M \times N$ 的矩阵, $\vec x$ 是 $N$ 维列矢量, $\vec y$ 是 $M$ 维列矢量. $\mat A$ 和 $\vec y$ 是已知的, $\vec x$ 是未知的, 即方程组的解.

从矢量空间的角度来看, $\vec x$ 是 $N$ 维矢量空间(以下称为 $x$ 空间)中一个矢量关于某组基底的坐标, $\vec y$ 是 $M$ 维矢量空间(以下称为 $y$ 空间)中一个矢量关于某组基底的坐标. 矩阵 $\mat A$ 可以将 $x$ 空间中的任意矢量映射到 $y$ 空间的对应矢量.

我们知道 $\mat A$ 的第 $i$ 列代表的矢量就是 $x$ 空间中的第 $i$ 个基底映射到 $y$ 空间的对应矢量. 我们把这 $N$ 个矢量记为 $\{\vec \alpha_i\}$. 下面我们根据 $\mat A$ 的不同性质来分类讨论方程组的解.

我们知道矩阵的秩等于线性无关的行数, 也等于线性无关的列数%未完成
, 将其记为 $R$.

\subsection{满秩方阵($R = M = N$)}
这是最简单的情况, 由于 $\{\vec \alpha_i\}$ 两两线性无关, 它们可以作为 $y$ 空间的一组基底, 与 $x$ 空间的基底一一对应. 那么这个映射既是单射%未完成:为什么?
又是满射.%未完成:为什么?
对于 $y$ 空间的任意矢量 $\vec y$, $x$ 空间都存在唯一的解 $\vec x$. 特殊地,当 $\vec y = \vec 0$ 时(即方程是\bb{齐次}的),唯一解就是 $\vec x = \vec 0$.

\subsection{$R = M < N$}

当 $\mat A$ 的秩等于 $M$ 且小于 $N$ 时, 映射变为从 $N$ 维空间到 $M$ 维空间 . 虽然任意的 $\vec x$ 仍然映射到唯一的 $\vec y$, 但任意的 $\vec y$ 却对应无穷多个 $\vec x$. 

% 举例未完成:三维矢量投影到二维矢量

当方程是齐次的时候, $\vec y = \vec 0$ 对应的所有 $\vec x$ 组成 $x$ 空间中的 $N- M$ 维的子空间, 我们把它叫做\bb{零空间}. 这种情况下,我们希望能解出零空间的一组($N - M$ 个)基底,使得这组基底的任意线性组合都是齐次方程的解.

对于非齐次方程, 我们可以先求对应的齐次方程组的零空间的一组基底,再求出非齐次方程的任意一个解(\bb{特解}), 那么非齐次方程组的\bb{解集}(所有解的集合)就等于零空间中的所有矢量与特解相加. 注意非齐次方程的解集并不构成一个矢量空间, 因为它不包含零矢量($\vec x = \vec 0$ 总是对应 $\vec y = \vec 0$, 所以不可能是非齐次方程组的解),解集中若干矢量的线性组合也不一定仍然属于解集.

\subsection{$R < M$}
当 $R < M$ 时, $\{\vec \alpha_i\}$ 中只有 $R$ 个线性无关的矢量, 在 $y$ 空间中张成% 未完成
一个 $R$ 维子空间. 如果 $\vec y$ 落在这个子空间中(我们可以通过 $\vec y$ 是否与 $\{\vec \alpha_i\}$ 线性无关来判断), 方程组就存在解, 如果落在子空间外, 方程组就无解.

% 未完成
