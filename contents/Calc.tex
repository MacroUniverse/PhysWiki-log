%微积分导航

从物理学巨人牛顿发明了微积分以来\footnote{一般认为牛顿和莱布尼兹都分别在十七世纪中独立地发明了微积分,然而他们都声称对方窃取了自己的成果,并为此争执了一生},微积分就在物理学的各个方面被大量使用.高中的物理教学有意避开了使用微积分,但从本科的学习开始,微积分与物理将形影不离.不夸张地说,不懂微积分,就几乎不懂物理.微积分最核心的内容就是极限,求导/微分,积分,常/偏微分方程和无穷级数.

\subsection{极限}
极限\upref{Lim}的概念是微积分的基础,大致可以理解为“某个表达式在某个量为无穷小或无穷大时所趋近的值”, 例如 $1/x$ 在 $x\to\infty$ 时的极限为零, $(1+x)/(2+x)$ 在 $x\to 0$ 时的极限为 $1/2$. 

\subsection{导数}
理解极限了以后,导数\upref{Der} 便是一个首要的应用.事实上高中物理的许多物理量都使用了导数的概念,只是没有提出“导数”这个词.例如(瞬时)速度的定义就是 $\Delta\vec s/\Delta t$ 在 $\Delta t \to 0$ (趋近于0)时的极限,而这恰好是导数的定义,即速度% 词条未完成
是位置矢量(关于时间的函数) $\vec r(t)$ 对时间的导数.同理,加速度% 词条未完成
矢量是速度矢量(关于时间的函数)对时间求导.又例如,高中对匀速圆周运动的向心加速度的推导过程中就运用了几何微元法,在微小时间 $\Delta t$ 内计算圆周运动速度矢量的微小变化\upref{CMAG}. 学习了矢量求导\upref{DerV} 以后,就不必再使用这种不成熟的“几何微元法”而直接按照矢量求导法\upref{CMAD}则即可严谨而轻易地得出向心加速度的公式 $\vec a = -\omega^2 \vec R$, 甚至可以计算非匀速圆周运动乃至任意变速曲线运动的加速度.

\subsection{积分}
高中物理中,位移 $\vec s$ 等于速度 $\vec v$ 乘以时间 $t$, 功 $W$ 等于力 $F$ 乘以位移 $x$ 等概念都已经耳熟能详.然而如果速度随时间变化或者力随位置变化时,就不能用简单的乘法来计算这些问题.这时一个基本的思想就是把时间或位移分成许多小份,每份中的速度或力都近似为恒定不变,然后再把所有小份的位移或做功加起来即可.这时用极限的思想,求出当这些小份为无穷小(或者说分成无穷多份)时求和的极限,就得到了总位移和总功.这个过程叫做定积分\upref{DefInt},以上例子中变化的量(速度,密度,力).在求定积分时,我们需要先求出一个原函数,而巧妙的地方在于,

\subsection{微分方程}
大量的物理定律和问题都是通过微分方程(组)来描述的.最简单的微分方程是线性常微分微分方程,%链接未完成
是函数 $y(x)$ 及不同阶导 $y'(x)$,  $y''(x)$ 以及自变量 $x$ 组成的等式.例如力学中著名的弹簧振子\upref{SHO}(又称简谐振子)模型就是通过二阶线性常微分方程(二阶代表方程中出现的最高阶导数为2).

中,结合牛顿第二定律\upref{New3} 和胡克定律得到 $ma = F = -kx$ 其中位移 $x$ 可看做关于时间的函数 $x(t)$, 是未知函数(微分方程的解),加速度是时间的二阶导数 $a(t) = x''(t)$.所以微分方程为 

本章介绍本书物理部分需要使用的微积分内容,
并以帮助读者理解为主而不求严谨详尽.对于一些定理如自然对数底极限和非整数幂的二项式定理,本章只给出数值验证的方法.



% 这个词条要写详细! 用最易懂的方式说明主线中这些东西都是做什么用的,让大家没开始学高数就对高数有一个形象的理解! 例如,定积分是做什么的, 例如求一个不均匀绳子的质量, 例如求汽车变速运动的路程! 这谁都可以理解. 又例如, 微分方程有什么用, 解决一些积分无法解决的问题, 例如受阻力的落体运动! 这也是任何一个高中生都能理解的! 只有这样, 读者才不会对那些陌生的名词望而生畏!