% 理想气体单粒子能级密度
% 理想气体|能级密度|相空间|量子力学|量子态

\subsection{相空间法}
\begin{equation}
\Omega_0 = \frac{1}{h^3}\int\limits_{\sum {p^2}  \les 2mE} \dd[3]{q} \dd[3]{p}
 = \frac{V}{h^3}\frac43 \pi {p^3}
 = \frac{V}{h^3}\frac43 \pi (2m\varepsilon)^{3/2}
\end{equation}
\begin{equation}
a(\varepsilon) = \dv{\Omega_0}{\varepsilon} = \frac{2\pi V(2m)^{3/2}}{h^3} \varepsilon^{1/2}
\end{equation}

\subsection{量子力学法}
由盒中粒子%链接未完成
得, 单粒子的能级为
\begin{equation}
\varepsilon = \frac{\hbar ^2}{2m} \qty[\qty(\frac{\pi n_x}{L_x})^2 + \qty(\frac{\pi n_y}{L_y})^2 + \qty(\frac{\pi n_z}{L_z})^2] = \frac{\hbar ^2}{2m} (k_x^2 + k_y^2 + k_z^2)
\end{equation}
在 $k$ 空间中, 每个能级所占的体积为
\begin{equation}
V_1 = \frac{\pi^3}{L_x L_y L_z} = \frac{\pi^3}{V}
\end{equation}
$K$ 空间中, 能量小于 $E$ 的量子态数为(注意 $n$ 为正值, 所以只求一个卦限的体积, 加 $1/8$ 系数)
\begin{equation}
\Omega_0 = \left. \frac18\cdot \frac{\hbar^2}{2m}\frac43 \pi k^3 \middle/ \frac{\pi^2}{V} \right. = \frac{V}{h^3}\frac43 \pi(2m\varepsilon)^{3/2}
\end{equation}
\begin{equation}
a(\varepsilon) = \dv{\Omega_0}{\varepsilon} = \frac{2\pi V(2m)^{3/2}}{h^3} \varepsilon^{1/2}
\end{equation}

