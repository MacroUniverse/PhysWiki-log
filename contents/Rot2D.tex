%平面旋转矩阵

\pentry{单位正交阵\upref{UOrM}}

平面旋转变换\upref{Rot2DT}属于线性变换,可以用矩阵 $\mat R_2$ 表示.虽然我们可以直接把变换写成矩阵乘以列矢量的形式,但这里我们用另一种方法推导一次,更能帮助理解和记忆. 由矩阵乘法的分配律,
\begin{equation}
\mat R_2\left( {{c_1}{{\vec v}_1} + {c_2}{{\vec v}_2}} \right) = {c_1}\mat R_2 {\vec v_1} + {c_2}\mat R_2 {\vec v_2} 
\end{equation}
已知单位矢量 $\uvec x$, $\uvec y$逆时针旋转 $\theta$ 为
\begin{equation}
\mat R_2\pmat{1\\0} = \pmat{\cos \theta \\ \sin \theta}
\qquad
\mat R_2\pmat{0\\1} = \pmat{-\sin\theta \\ \cos\theta}
\end{equation}
而任何矢量 $\pmat{x_1\\x_2}$ 都可以表示成 $\uvec x$, $\uvec y$ 的线性组合 ${x_1}\pmat{1\\0} + {x_2}\pmat{0\\1}$, 所以
\begin{equation}\begin{split}
\mat R_2\left[x_1 \pmat{1\\0} + x_2 \pmat{0\\1} \right] 
&= x_1 \mat R_2\pmat{1\\0} + x_2 \mat R_2\pmat{0\\1}\\
&= x_1\pmat{\cos \theta \\ \sin \theta} 
  + x_2 \pmat{-\sin\theta \\ \cos\theta} \\
&= \pmat{{\cos \theta }&{ - \sin \theta }\\{\sin \theta }&{\cos \theta }}
\pmat{{x_1}\\{x_2}}
\end{split}\end{equation}
所以旋转矩阵为
\begin{equation}
\mat R_2 = \begin{pmatrix}
{\cos\theta }&{ - \sin\theta }\\
{\sin\theta }&{\cos\theta }
\end{pmatrix}
\end{equation}
这与平面旋转变换\upref{Rot2DT}得出的结果一致.

\subsection{逆矩阵}
我们既可以使用平面旋转变换\upref{Rot2DT}中求逆变换的方法把 $\theta$ 变为 $-\theta$ 再化简求出 $\mat R_2$ 的逆矩阵,也可以通过解方程求逆矩阵. 但最方便的是,由于 $\mat R_2$ 是一个单位正交阵\upref{UOrM}, 我们只需要把矩阵转置即可得到逆矩阵.
\begin{equation}
\mat R_2^{-1} = \mat R_2\Tr = \begin{pmatrix}
{\cos\theta }&{\sin\theta }\\
{-\sin\theta }&{\cos\theta }
\end{pmatrix}
\end{equation}

