% 矩阵与矢量空间

\pentry{矩阵\upref{Mat}}

我们已经知道 $M\times N$ 的矩阵可以表示一个 $N$ 维列矢量的线性组合, 得到一个 $M$ 维列矢量.
\begin{equation}
\vec y = \mat A \vec x
\end{equation}
我们可以把这个 $\vec x$ 看做任意一个 $N$ 维矢量空间(以下称为 $x$ 空间)中某矢量关于某组基底 $\{\vec x_i\}$ 的坐标, 而把 $\vec y$ 看做任意一个 $M$ 维矢量空间(以下称为 $y$ 空间)中某矢量关于某组基底 $\{\vec y_i\}$ 的坐标.

这样, 我们就通过矩阵 $\mat A$ 建立了从 $x$ 空间到 $y$ 空间的一个\bb{映射}. 显然, $x$ 空间的任意矢量 $\vec x$, 都可以映射到 $y$ 空间中唯一矢量 $\vec y$.

特殊地, 当矩阵 $\mat A$ 为方阵时, 矩阵 $\mat A$ 可以用于表示 $x$ 空间到自身的\bb{自映射}, 即 $\vec x$ 和 $\vec y$ 都是 $x$ 空间中的矢量, 但 $\{\vec x_i\}$ 和 $\{\vec y_i\}$ 仍然可以是两组不同的基底.

由矩阵与列矢量乘法的性质\autoref{Mat_eq17} 可知 $x$ 空间中若干个矢量的任意线性组合的映射等于这些矢量先分别映射再做同样的线性组合. 我们把这样的映射叫做\bb{线性映射}, 矩阵代表的映射都是线性映射.

