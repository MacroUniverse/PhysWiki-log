\pentry{匀速圆周运动的加速度\upref{CMAG}\upref{CMAD},惯性力\upref{Iner}}

令参考系 $abc$ 和 $xyz$ 的 $c$ 轴和 $z$ 轴式中重合.其中 $xyz$ 是惯性系, $abc$ 以恒定的角速度 $\omega$ 绕 $z$ 轴逆时针转动.求 $abc$ 系中一个质量为 $m$ 的静止质点所受的惯性力(离心力).

令质点的坐标 $(a,b,c)$ 离 $c$ 轴的距离为 $r = \sqrt {{b^2} + {c^2}}$, 对应的径向矢量为 $\vec r = (a,b,0)\Tr$. 在 $xyz$ 系中,质点做匀速圆周运动,相对于 $xyz$ 系的加速度(绝对加速度)(用 $abc$ 系的坐标表示)为
\begin{equation}
{\vec a_{xyz}} =  - {\omega ^2}\vec r =  - {\omega ^2}\begin{pmatrix} a\\b\\0 \end{pmatrix}
\end{equation}
质点相对于 $abc$ 系静止,相对加速度为零
\begin{equation}
{\vec a_{abc}} = \vec 0
\end{equation}
所以由惯性力\upref{Iner} 中的结论,惯性力为
\begin{equation}
\vec f = m\left( {{{\vec a}_{abc}} - {{\vec a}_{xyz}}} \right) = m{\omega ^2}\begin{pmatrix} a\\b\\0 \end{pmatrix}
\end{equation}
注意离心力向外,与直觉相符.

注意这个结论只适用于质点相对于 $abc$ 系静止的情况,若有相对运动,则惯性力除了离心力,还会有一项科里奥利力\upref{Corio}.