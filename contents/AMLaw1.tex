% 角动量 角动量定理 角动量守恒(单个质点)
% 角动量|角动量定理|角动量守恒|力矩|牛顿第二定律

\pentry{牛顿第二定律\upref{New3},力矩\upref{Torque}}

\subsection{质点的角动量}
一个质点的质量为 $m$, 某时刻速度为 $\bvec v$, 则其动量为 $\bvec p = m\bvec v$. 在三维空间中建立坐标系, 原点为 $O$, $O$ 点到质点的位置矢量为 $\bvec r$. 定义该质点关于 $O$ 点的\textbf{角动量}为
\begin{equation}
\bvec L = \bvec r \cross \bvec p = m\bvec r\cross \bvec v
\end{equation}
由叉乘的几何定义\upref{Cross} 可知,当速度与位矢平行时角动量为 $\bvec 0$,垂直时角动量模长为距离和动量模长的积 $L = rp$.

\subsection{角动量定理}
令质点在某时刻受到的力矩\upref{Torque}为 $\bvec \tau$,可以证明
\begin{equation}
\dv{\bvec L}{t} = \bvec \tau
\end{equation} 
这就是(单个质点的)\textbf{角动量定理}.

特殊地,若质点受到的力矩为零,则 $ \dv*{\bvec L}{t} = \bvec 0$,即角动量不随时间变化.这个现象叫做(单个质点的)\textbf{角动量守恒}.由力矩的定义,$\bvec \tau = \bvec r \cross \bvec F$,可见以下两种情况下力矩为零,角动量守恒.
\begin{enumerate}
\item 质点受合力 $\bvec F= \bvec 0$,即质点静止或做匀速直线运动.
\item $\bvec F$ 与 $\bvec r$ 同向,即质点只受关于 $O$ 点的\textbf{有心力}.
\end{enumerate}

\subsubsection{证明}
我们来证明单个质点的角动量定理. 令质点的速度为 $\bvec v = \dv*{\bvec r}{t}$,加速度为 $\bvec a = \dv*{\bvec v}{t}$, 叉乘的求导法则(\autoref{DerV_eq8}\upref{DerV}) 与标量乘法求导类似, 牛顿第二定律\upref{New3}为 $\bvec F = m\bvec a$,两个同方向矢量叉乘\upref{Cross}为零,
\begin{equation}
\ali{
\dv{\bvec L}{t} &= \dv{( \bvec r \cross \bvec p )}{t} = m\dv{(\bvec r \cross \bvec v)}{t}
= m \qty( \dv{\bvec r}{t} \cross \bvec v + \bvec r \cross \dv{\bvec v}{t} )\\
&= m(\bvec v \cross \bvec v + \bvec r \cross \bvec a) = \bvec r \cross (m\bvec a)\\
&= \bvec r \cross \bvec F = \bvec \tau
} \end{equation}
证毕.
