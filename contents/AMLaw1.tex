%角动量定理 角动量守恒(单个质点)

\pentry{角动量\upref{AngMom},牛顿第二定律\upref{New3},力矩\upref{Torque}}
一个质点的质量为 $m$,某时刻速度为 $\vec v$. 则其动量为 $\vec p = m\vec v$.在三维空间中指定一点 $O$ 为参考点,$O$ 点到质点的矢量为 $\vec r$.根据定义,该质点关于 $O$ 点的角动量为 $\vec L = \vec r \cross \vec p$.这个质点在该时刻受到的力矩%未完成:引用
为 $\vec M$,可以证明
\begin{equation}
\dv{\vec L}{t} = \vec M
\end{equation} 
这就是(单个质点的)\bb{角动量定理}.

特殊地,若质点受到的力矩为零,则 $ \dv*{\vec L}{t} = \vec 0$,即角动量不随时间变化.这个现象叫做(单个质点的)\bb{角动量守恒}.由力矩的定义,$\vec M = \vec r \cross \vec F$,可见以下两种情况下力矩为零,角动量守恒.
\begin{enumerate}
\item 质点受合力 $\vec F= \vec 0$,即质点静止或做匀速直线运动.
\item $\vec F$ 与 $\vec r$ 同向,即质点只受关于 $O$ 点的\bb{有心力}.
\end{enumerate}

\subsection{单个质点的角动量定理证明}

质点的速度为 $\vec v = \dv*{\vec r}{t}$,加速度为 $\vec a = \dv*{\vec v}{t}$,
叉乘的求导法则% 未完成,引用矢量求导中的对应公式
与标量乘法求导类似,
牛顿第二定律为%未完成:引用
$\vec F = m\vec a$,两个同方向矢量叉乘%未完成:引用
 为零,
\begin{equation}
\ali{
\dv{\vec L}{t} &= \dv{( \vec r \cross \vec p )}{t} = m\dv{(\vec r \cross \vec v)}{t}
= m \qty( \dv{\vec r}{t} \cross \vec v + \vec r \cross \dv{\vec v}{t} )\\
&= m(\vec v \cross \vec v + \vec r \cross \vec a) = \vec r \cross (m\vec a)\\
&= \vec r \cross \vec F = \vec M
} \end{equation}
 
