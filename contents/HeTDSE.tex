% 氦原子数值解 TDSE
\pentry{广义球谐函数\upref{GenYlm}}

使用角向基底为两个电子总轨道角动量和 $z$ 分量的本征态\upref{AMAdd}, 即广义球谐函数 $\mathcal{Y}_{l_1,l_2}^{L,M}(\uvec r_1, \uvec r_2)$.

总波函数所在的空间可以看做角向空间和径向空间的张量积空间. 现在有了角向空间的基底, 总波函数就可以在该基底上展开
\begin{equation}
\ket{\Psi} = \sum_\lambda \ket{R_\lambda}\ket{\mathcal{Y_\lambda}}
\end{equation}
其中 $\lambda$ 是将所有的 $(l_1,l_2,L,M)$ 组合排序后的序号. $\ket{R_\lambda}$ 是二维径向波函数
\begin{equation}
\ket{R_\lambda} = \frac{1}{r_1 r_2} \psi_\lambda(r_1, r_2)
\end{equation}

由张量积空间\upref{DirPro}中的结论, 我们可以求哈密顿算符 $H$ 关于角向基底的“矩阵元” $H_{\lambda, \lambda'}$, 每个矩阵元是径向空间中的一个算符. 列出薛定谔方程的“矩阵形式”, 就得到了一组 couple 的径向波函数的薛定谔方程
\begin{equation}
\sum_{\lambda'} H_{\lambda, \lambda'} \ket{R_{\lambda'}} = \I \pdv{t} \ket{R_{\lambda'}}
\end{equation}

总哈密顿可以表示为
\begin{equation}
H = H_1 + H_2 + V_{12} + V_F
\end{equation}
其中 $H_i \ \ (i = 1, 2)$ 是单个电子的哈密顿算符, 对应对角矩阵.
\begin{equation}
H_i = K_{ri} + \frac{L_i^2}{2m r_i^2} - \frac{2}{r_i}
\end{equation}
其中第二项是对角矩阵是因为 $\mathcal Y$ 基底是 $L_i^2$ 的本征函数.

存在 coupling 的项自然是两电子之间的库仑势能 $V_{12}$
\begin{equation}
\ali{
V_{12} = \frac{1}{\abs{\vec r_2 - \vec r_1}} &= 4\pi \sum_{l=0}^{\infty} \frac{1}{2l+1} \frac{r_<^l}{r_>^{l+1}} \sum_{m = -l}^l Y_{l,m}^*(\uvec r_1) Y_{l,m}(\uvec r_2)\\
&= 4\pi \sum_{l=0}^{\infty} \frac{(-1)^l}{\sqrt{2l + 1}} \frac{r_<^l}{r_>^{l+1}} \mathcal{Y}_{l,l}^{0,0} (\uvec r_1, \uvec r_2)
}\end{equation}
这就相当于将张量积空间中的矢量拆到了角向空间基底的各个子空间中. 第二步使用了\autoref{SphCup_eq9}\upref{SphCup}, 即
\begin{equation}
\mathcal{Y}_{l,l}^{0,0} (\uvec r_1, \uvec r_2) = \frac{(-1)^l}{\sqrt{2l+1}} \sum_{m = -l}^l Y_{lm}^*(\uvec r_1) Y_{lm}(\uvec r_2)
\end{equation}

要计算矩阵元 $\mel{\mathcal Y_\lambda}{V_{12}}{\mathcal Y_{\lambda'}}$, 就要对六个球谐函数的乘积的线性组合做两次角向积分. 使用\autoref{SphCup_eq1}\upref{SphCup} 可以将上式表示为六个 CG 系数相乘的线性组合(二重求和).
\begin{equation}\label{HeTDSE_eq10}
\ali{
&\quad\mel{\mathcal{Y}_{l_1 l_2}^{LM}}{Y_{lm}^*(\uvec r_1)Y_{lm}(\uvec r_2)}{\mathcal{Y}_{l_1' l_2'}^{L'M'}}\\
&= \sum_{m_1',m_2'}\sum_{m_1,m_2} \bmat{l_1 & l_2 & L\\ m_1 & m_2 & M}\bmat{l_1' & l_2' & L'\\ m_1' & m_2' & M'} \times\\
&\qquad  \iint Y_{l_1m_1}^*(\uvec r_1)Y_{l_2m_2}^*(\uvec r_2)Y_{l m}^*(\uvec r_1)Y_{l m}(\uvec r_2)Y_{l_1' m_1'}(\uvec r_1)Y_{l_2' m_2'}(\uvec r_2) \dd{\Omega_1}\dd{\Omega_2}\\
&=\sum_{m_1',m_2'}\sum_{m_1,m_2} (-1)^{m_1 + m_2 + m} \bmat{l_1 & l_2 & L\\ m_1 & m_2 & M}\bmat{l_1' & l_2' & L'\\ m_1' & m_2' & M'} \times\\
&\qquad \int Y_{l_1, -m_1}(\uvec r_1)Y_{l,-m}(\uvec r_1)Y_{l_1' m_1'}(\uvec r_1)  \dd{\Omega_1} \int Y_{l_2, -m_2}(\uvec r_2)Y_{l m}(\uvec r_2)Y_{l_2' m_2'}(\uvec r_2) \dd{\Omega_2}\\
&=\frac{2l+1}{4\pi}\sqrt{(2l_1+1)(2l_1'+1)(2l_2+1)(2l_2'+1)} \sum_{m_1',m_2'}\sum_{m_1,m_2}  (-1)^{m_1 + m_2 + m} \times\\
&\qquad\pmat{l_1 & l & l_1'\\ 0 & 0 & 0}\pmat{l_1 & l & l_1'\\ -m_1 & -m & m_1'}\pmat{l_2 & l & l_2'\\ 0 & 0 & 0}\pmat{l_2 & l & l_2'\\ -m_2 & m & m_2'}\times\\
&\qquad\bmat{l_1 & l_2 & L\\ m_1 & m_2 & M}\bmat{l_1' & l_2' & L'\\ m_1' & m_2' & M'}
}\end{equation}
现在试图对 $m$  求和并使用选择定理(3j 符号第二行之和为 0), 得(注意 $m$ 的范围)该求和中最多只有一项不为 0, 所以
\begin{equation}\label{HeTDSE_eq9}
\ali{
& \mel{\mathcal{Y}_{l_1 l_2}^{LM}}{\mathcal{Y}_{l,l}^{0,0}}{\mathcal{Y}_{l_1' l_2'}^{L'M'}}
=\\
& \delta_{M,M'} (-1)^l \frac{\sqrt{2l+1}}{4\pi} \sqrt{(2l_1+1)(2l_1'+1)(2l_2+1)(2l_2'+1)}\times\\
&\sum_{m_1',m_2'}\sum_{m_1,m_2} (-1)^{m_1' + m_2} \bmat{l_1 & l_2 & L\\ m_1 & m_2 & M}\bmat{l_1' & l_2' & L'\\ m_1' & m_2' & M} \times\\
&\pmat{l_1 & l & l_1'\\ 0 & 0 & 0}\pmat{l_1 & l & l_1'\\ -m_1 & m_1-m_1' & m_1'}\pmat{l_2 & l & l_2'\\ 0 & 0 & 0}\pmat{l_2 & l & l_2'\\ -m_2 & m_2-m_2' & m_2'}
}\end{equation}
其中对 $m_1, m_2$ 的求和要求
\begin{equation}
m_1 + m_2 = m_1' + m_2' = M \qquad
\abs{m_1'-m_1} \les l \qquad
\abs{m_2'-m_2} \les l
\end{equation}

由于 CG 系数都是实数, 这必定是一个实数矩阵. 选择定则如下
\begin{itemize}
\item 根据总角动量守恒($1/r_{12}$ 算符与 $L, L_z$ 算符对易), $L \ne L'$ 或 $M \ne M'$ 时, 矩阵元为零.
\item 根据 parity CG coefficients, 当 $l_1' + l + l_1$ 或 $l_2' + l + l_2$ 为奇数时, 上式等于 0.
\item 根据 $\Pi \mathcal{Y} = (-1)^{l_1 + l_2} \mathcal{Y}$, 考虑到 $1/r_{12}$ 具有偶宇称, 所以 $l_1 + l_2$ 与 $l_1' + l_2'$ 的奇偶性必须不同.
\end{itemize}

爱华的论文中将\autoref{HeTDSE_eq9} 用 9j 符号表示, 但仅限于 $M \ne M'$. 然而经过数值验证, 我发现应该可以将爱华的公式拓展为
\begin{equation}
\ali{
\mel{\mathcal{Y}_{l_1 l_2}^{LM}}{\mathcal{Y}_{ll}^{00}}{\mathcal{Y}_{l_1' l_2'}^{L'M'}}
= &\frac{2l+1}{4\pi} \delta_{L,L'}\delta_{M,M'} \sqrt{(2l_1'+1)(2l_2'+1)(2L+1)}\times\\
&\bmat{l & l_1' & l_1\\ 0 & 0 & 0}
\bmat{l & l_2' & l_2\\ 0 & 0 & 0}
\Bmat{l & l_1' & l_1\\ l & l_2' & l_2\\ 0 & L & L}
}\end{equation}
注意当 $M = M'$ 时, 结果与 $M$ 无关.

\subsection{波函数演化}
传播同样使用 split operator 
\begin{equation}
\exp(-\I H\Delta t) = \exp(-\I H_0\frac{\Delta t}{2})\exp(-\I H_{int}\Delta t) \exp(-\I H_0\frac{\Delta t}{2}) + \order{\Delta t^3}
\end{equation}
只是这里的 $H_{int} = V_{12} + V_F$ 是所有 $\mel{\mathcal{Y}}{\ }{\mathcal{Y}}$ 作用后为非对角(算符)矩阵的项, 即电子的相互作用和电场对每个电子的作用.

$H_0 = H_1 + H_2$ 由于 $H_1$ 和 $H_2$ 对易, 我们可以将它们独立传播
\begin{equation}
\exp(-\I H_0 \frac{\Delta t}{2}) = \exp(-\I H_1 \frac{\Delta t}{2}) \exp(-\I H_2 \frac{\Delta t}{2})
\end{equation}
也就是对每个 partial wave 的二维网格, 用类似氢原子的方法传播每一列在传播每一行.

在传播 $\exp(-\I H_{int}\Delta t)$ 的时候, 爱华的做法是进一步 split 成三个 operators
\begin{equation}
\exp(-\I H_{int}\Delta t) = \exp(-\I V_F\frac{\Delta t}{2})   \exp(-\I V_{12} \Delta t) \exp(-\I V_F\frac{\Delta t}{2})
\end{equation}
所以总哈密顿一共是 split 成 5 个算符.

\subsection{束缚态}
氦原子的初始态用的是 singlet 自旋的的基态, 通过虚时间演化得到, 占用了 $L = 0, M = 0$ 的所有 partial waves.
