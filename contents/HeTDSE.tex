% 氦原子数值解 TDSE
\pentry{广义球谐函数\upref{GenYlm}}

使用角向基底为两个电子总轨道角动量和 $z$ 分量的本征态\upref{AMAdd}, 即广义球谐函数 $\mathcal{Y}_{l_1,l_2}^{L,M}(\uvec r_1, \uvec r_2)$.

总波函数所在的空间可以看做角向空间和径向空间的张量积空间. 现在有了角向空间的基底, 总波函数就可以在该基底上展开
\begin{equation}
\ket{\Psi} = \sum_\lambda \ket{R_\lambda}\ket{\mathcal{Y_\lambda}}
\end{equation}
其中 $\lambda$ 是将所有的 $(l_1,l_2,L,M)$ 组合排序后的序号. $\ket{R_\lambda}$ 是二维径向波函数
\begin{equation}
\ket{R_\lambda} = \frac{1}{r_1 r_2} \psi_\lambda(r_1, r_2)
\end{equation}

由张量积空间\upref{DirPro}中的结论, 我们可以求哈密顿算符 $H$ 关于角向基底的“矩阵元” $H_{\lambda, \lambda'}$, 每个矩阵元是径向空间中的一个算符. 列出薛定谔方程的“矩阵形式”, 就得到了一组 couple 的径向波函数的薛定谔方程
\begin{equation}
\sum_{\lambda'} H_{\lambda, \lambda'} \ket{R_{\lambda'}} = \I \pdv{t} \ket{R_{\lambda'}}
\end{equation}

总哈密顿可以表示为
\begin{equation}
H = H_1 + H_2 + V_{12} + V_{int}
\end{equation}
其中 $H_i \ \ (i = 1, 2)$ 是单个电子的哈密顿算符, 对应对角矩阵.
\begin{equation}
H_i = K_{ri} + \frac{L_i^2}{2m r_i^2} - \frac{2}{r_i}
\end{equation}
其中第二项是对角矩阵是因为 $\mathcal Y$ 基底是 $L_i^2$ 的本征函数.

存在 coupling 的项自然是两电子之间的库仑势能 $V_{12}$
\begin{equation}
\ali{
V_{12} = \frac{1}{\abs{\vec r_2 - \vec r_1}} &= 4\pi \sum_{l=0}^{\infty} \frac{1}{2l+1} \frac{r_<^l}{r_>^{l+1}} \sum_{m = -l}^l Y_{l,m}^*(\uvec r_1) Y_{l,m}(\uvec r_2)\\
&= 4\pi \sum_{l=0}^{\infty} \frac{(-1)^l}{\sqrt{2l + 1}} \frac{r_<^l}{r_>^{l+1}} \mathcal{Y}_{l,l}^{0,0} (\uvec r_1, \uvec r_2)
}\end{equation}
这就相当于将张量积空间中的矢量拆到了角向空间基底的各个子空间中. 第二步使用了
\begin{equation}
\bmat{l & l & 0 \\ -m & m & 0} = \frac{(-1)^{l+m}}{\sqrt{2l+1}}
\end{equation}
即
\begin{equation}
\mathcal{Y}_{l,l}^{0,0} (\uvec r_1, \uvec r_2) = \frac{(-1)^l}{\sqrt{2l+1}} \sum_{m = -l}^l Y_{lm}^*(\uvec r_1) Y_{lm}(\uvec r_2)
\end{equation}

要计算矩阵元 $\mel{\mathcal Y_\lambda}{V_{12}}{\mathcal Y_{\lambda'}}$, 就要对六个球谐函数的乘积的线性组合做两次角向积分. 使用\autoref{SphCup_eq1}\upref{SphCup} 可以将上式表示为六个 CG 系数相乘的线性组合(二重求和).

\begin{equation}\label{HeTDSE_eq10}
\ali{
&\quad\mel{y_{l_1 l_2}^{LM}}{Y_{lm}^*(\uvec r_1)Y_{lm}(\uvec r_2)}{y_{l_1' l_2'}^{L'M'}}\\
&= \sum_{m_1',m_2'}\sum_{m_1,m_2} \bmat{l_1 & l_2 & L\\ m_1 & m_2 & M}\bmat{l_1' & l_2' & L'\\ m_1' & m_2' & M'} \times\\
&\qquad  \iint Y_{l_1m_1}^*(\uvec r_1)Y_{l_2m_2}^*(\uvec r_2)Y_{l m}^*(\uvec r_1)Y_{l m}(\uvec r_2)Y_{l_1' m_1'}(\uvec r_1)Y_{l_2' m_2'}(\uvec r_2) \dd{\Omega_1}\dd{\Omega_2}\\
&=\sum_{m_1',m_2'}\sum_{m_1,m_2} (-1)^{m_1 + m_2 + m} \bmat{l_1 & l_2 & L\\ m_1 & m_2 & M}\bmat{l_1' & l_2' & L'\\ m_1' & m_2' & M'} \times\\
&\qquad \int Y_{l_1, -m_1}(\uvec r_1)Y_{l,-m}(\uvec r_1)Y_{l_1' m_1'}(\uvec r_1)  \dd{\Omega_1} \int Y_{l_2, -m_2}(\uvec r_2)Y_{l m}(\uvec r_2)Y_{l_2' m_2'}(\uvec r_2) \dd{\Omega_2}\\
&=\frac{2l+1}{4\pi} \sum_{m_1',m_2'}\sum_{m_1,m_2}  (-1)^{m_1 + m_2 + m} \sqrt{(2l_1+1)(2l_1'+1)(2l_2+1)(2l_2'+1)} \times\\
&\qquad\pmat{l_1 & l & l_1'\\ 0 & 0 & 0}\pmat{l_1 & l & l_1'\\ -m_1 & -m & m_1'}\pmat{l_2 & l & l_2'\\ 0 & 0 & 0}\pmat{l_2 & l & l_2'\\ -m_2 & m & m_2'}\times\\
&\qquad\bmat{l_1 & l_2 & L\\ m_1 & m_2 & M}\bmat{l_1' & l_2' & L'\\ m_1' & m_2' & M'}
}\end{equation}

根据总角动量守恒($1/r_{12}$ 算符与 $L, L_z$ 算符对易), $L \ne L'$ 或 $M \ne M'$ 时, 矩阵元为零.

根据 parity CG coefficients, 当 $l_1' + l + l_1$ 或 $l_2' + l + l_2$ 为奇数时, 上式等于 0.

根据 $\Pi y = (-1)^{l_1 + l_2} y$, 考虑到 $1/r_{12}$ 具有偶宇称, 所以 $l_1 + l_2$ 与 $l_1' + l_2'$ 的奇偶性必须不同.

由于 CG 系数都是实数, 这必定是一个实数矩阵.

由于$Y_{l,-m}^*(\uvec r_1)Y_{l,-m}(\uvec r_2) = [Y_{lm}^*(\uvec r_1)Y_{lm}(\uvec r_2)]^*$, 矩阵 $\mel{y_{l_1 l_2}^{LM}}{Y_{lm}^*(\uvec r_1)Y_{lm}(\uvec r_2)}{y_{l_1' l_2'}^{L'M'}}$ 中将 $m$ 变为 $-m$ 会使矩阵取厄米共轭,在这里就是转置. 所以这是一个对称实矩阵, 且我们只需要求 $m \ges 0$ 的项即可.

爱华的论文中将\autoref{HeTDSE_eq10} 用 9j 符号表示, 但仅限于 $M \ne M'$, 经过数值验证, 我发现应该可以将爱华的公式拓展为
\begin{equation}
\ali{
\mel{y_{l_1 l_2}^{LM}}{y_{ll}^{00}}{y_{l_1' l_2'}^{L'M'}}
= &\frac{2l+1}{4\pi} \delta_{L,L'}\delta_{M,M'} \sqrt{(2l_1'+1)(2l_2'+1)(2L+1)}\times\\
&\bmat{l & l_1' & l_1\\ 0 & 0 & 0}
\bmat{l & l_2' & l_2\\ 0 & 0 & 0}
\pmat{l & l_1' & l_1\\ l & l_2' & l_2\\ 0 & L & L}
}\end{equation}
注意当 $M = M'$ 时, 结果与 $M$ 无关.
