% 梯度 梯度定理

\pentry{方向导数\upref{DerDir}}

在方向导数\upref{DerDir}中, 我们推出方向导数为
\begin{equation}\label{Grad_eq1}
\pdv{f}{n} = \grad f \vdot \uvec n
\end{equation}
 
其中 $\grad f$ 就叫标量函数 $f$ 的\bb{梯度}\footnote{这里假设 $f$ 在某区域内处处光滑,即所有一阶偏导数处处连续. 这个性质也叫\bb{可微}.}. 要注意当且仅当 Del 算符 $\grad$ 作用在标量函数(即因变量是一个数而不是矢量)上时, 可以称其为\bb{梯度算符}. 这里的 $f$ 叫做\bb{势函数}.对于 $N$ 维直角坐标系中的 $N$ 元函数 $f(x_1,x_2\dots x_N)$, 其梯度是一个\bb{矢量}函数
\begin{equation}
\grad f = \sum_{i=1}^{N}\pdv{f}{x_i}\uvec x_i
\end{equation}
其中所有的 $\uvec x_i$ 组成直角坐标系的正交归一基\upref{OrNrB}, 现在来看全微分\upref{TDiff} 关系
\begin{equation}\label{Grad_eq3}
\dd{f} = \sum_{i = 1}^N \pdv{f}{x_i} \dd{x_i}
\end{equation}
若定义微位移矢量为
\begin{equation}
\dd{\vec r} = \sum_{i=1}^{N} \dd{x_i} \uvec x_i
\end{equation}
\autoref{Grad_eq3} 可用势函数的梯度和微位移矢量的点乘表示
\begin{equation}\label{Grad_eq5}
\dd{f} = \grad f \vdot \dd{\vec r}
\end{equation}
由点乘的几何定义\upref{Dot}可知,从某点出发,若微位移 $\dd{\vec r}$ 的大小不变,那么当其方向与梯度方向相同时函数增量 $\dd{f}$ 最大;二者方向垂直时,函数增量为零;二者夹角为 $\theta$ 时,函数增量等于最大值乘以 $\cos \theta$. 所以梯度矢量的方向是函数 $f$ 增加最快的方向, 梯度的大小等于该方向的方向导数.注意\autoref{Grad_eq5} 和\autoref{Grad_eq1} 的关系可以类比一元函数的导数\upref{Der}和微分\upref{Diff} 的关系, 当函数可微时, 二者等效.

现在我们也可以把全微分近似(“ 全微分\upref{TDiff}” \autoref{TDiff_eq6}) 记为矢量的形式
\begin{equation}\label{Grad_eq6}
\Delta f \approx \grad f \vdot \Delta \vec r
\end{equation}

\subsection{用梯度计算曲线(面)的法向量}
先以 $xy$ 平面的曲线为例,任意曲线可以用函数 $f(x,y)$ 的\bb{等值线}来表示,即 $f(x,y) = C$( $C$ 为常数).若从曲线上的某点出发,沿曲线的切线方向取一个微位移 $\dd{\vec r} = \dd{x}\uvec x + \dd{y} \uvec y$,由于 $(x+\dd{x}, y+\dd{y})$ 仍然在等值线上,函数增量 $\dd{f} = 0$. 代入\autoref{Grad_eq5} 得
\begin{equation}
\grad f \vdot \dd{\vec r} = 0
\end{equation}
即 $f(x,y)$ 的梯度与 $\dd{\vec r}$ 垂直. 所以 $\grad f(x,y)$ 必定是 $(x,y)$ 点所在等值线的法向量,且指向函数值 $C$ 更大的等值线(因为函数值在梯度方向增加最快).
% 未完成:图,画若干条等值线,其中一条上画一个法向量

\subsection{极坐标,柱坐标和球坐标中的梯度算符}
\pentry{极坐标系\upref{Polar}, 柱坐标系\upref{Cylin}, 球坐标系\upref{Sph}}

我们先写出极坐标中函数 $f(r,\theta)$ 的全微分为
\begin{equation}\label{Grad_eq8}
\dd{f} = \pdv{f}{r} \dd{r} + \pdv{f}{\theta} \dd{\theta}
\end{equation}
再写出极坐标中的微位移为% 未完成:哪里提到过这个?
\begin{equation}\label{Grad_eq9}
\dd{r} = \dd{r} \uvec r + r \dd{\theta} \uvec \theta
\end{equation}
所以为了满足梯度的定义\autoref{Grad_eq5}, 我们可以把\autoref{Grad_eq8} 第二项的 $\pdv*{f}{\theta}$ 除以 $r$, 第二个因子 $\dd{\theta}$ 乘以 $r$, 分别作为梯度和微位移的两个分量.
\begin{equation}\label{Grad_eq10}
\dd{f} = \pdv{f}{r} \dd{r} + \frac 1r \pdv{f}{\theta}  r\dd{\theta}
\end{equation}
对比\autoref{Grad_eq5}, \autoref{Grad_eq9} 和\autoref{Grad_eq10} 可以得出\bb{极坐标中的梯度算符}为
\begin{equation}
\grad = \uvec r \pdv{r} + \uvec \theta \frac 1r \pdv{\theta}
\end{equation}

同理, 柱坐标中的微位移与函数 $f(r,\theta, z)$ 的全微分可以分别表示为
\begin{equation}
\dd{r} = \dd{r} \uvec r + r\dd{\theta} \uvec \theta + \dd{z} \uvec z
\end{equation}
\begin{equation}
\dd{f} = \pdv{f}{r} \dd{r} + \frac 1r \pdv{f}{\theta}  r\dd{\theta} + \pdv{f}{z} \dd{z}
\end{equation}
所以柱\bb{坐标中的梯度算符}为
\begin{equation}
\grad = \uvec r \pdv{r} + \uvec \theta \frac 1r \pdv{\theta} + \uvec z \pdv{z}
\end{equation}

再来看球坐标, 球坐标中的微位移与 $f(r,\theta,\phi)$ 的全微分可以分别表示为



\subsection{梯度定理}
\pentry{线积分\upref{IntL}, 牛顿—莱布尼兹公式\upref{NLeib}}

\bb{梯度定理} :一个标量函数的梯度延任何路径从起点 $\vec r_i$ 到终点 $\vec r_f$(角标 $i$ 表示 initial, $f$ 表示 final)  线积分的结果等于该函数在末位置的函数值减去初位置的函数值.可以用下式表示
\begin{equation}\label{Grad_eq15}
\int_{\vec r_i}^{\vec r_f} \grad f(\vec r) \vdot \dd{\vec r} = f(\vec r_f) - f(\vec r_i)
\end{equation}
梯度定理可以看做是牛顿—莱布尼兹公式\upref{NLeib}
\begin{equation}\label{Grad_eq16}
\int_a^b f'(x) \dd{x}  = f(b) - f(a)
\end{equation}
的拓展, 即把一元函数拓展为多元函数,把导函数拓展为梯度函数(在一维情况下,\autoref{Grad_eq15} 变为\autoref{Grad_eq16}).所以前者的证明也可以类比后者的证明.

\subsubsection{梯度定理的证明}

我们先把\autoref{Grad_eq8} 路径分为许多首尾相接的小段曲线, 则整段曲线的线积分等于所有小曲线的线积分之和.假设曲线处处光滑, 如果每段小曲线都足够短,就可以把它们近似看做线段,且梯度值在上面近似为常矢量.令第 $i$ 小段的起点和终点分别为 $\vec r_{i0}, \vec r_{i1}$, 则第 $i$ 段的线积分可近似为
\begin{equation}
\int_{\vec r_{i0}}^{\vec r_{i1}} \grad f(\vec r) \vdot \dd{\vec r} \approx  \grad f(\vec r_i) \vdot \Delta \vec r_i
\end{equation}
再利用全微分近似( \autoref{Grad_eq6}),上式等于
\begin{equation}
\int_{\vec r_{i0}}^{\vec r_{i1}} \grad f(\vec r) \vdot \dd{\vec r} \approx f(\vec r_{i1}) - f(\vec r_{i0})
\end{equation}
将所有小段的线积分求和得到总的线积分得( 注意 $\vec r_{i1} = \vec r_{(i+1)0}$) 
\begin{equation}\ali{
\int_{C} \grad f(\vec r) \vdot \dd{\vec r}
&= \sum_{i=1}^n \int_{\vec r_{i0}}^{\vec r_{i1}} \grad f(\vec r) \vdot \dd{\vec r}\\
&\approx  \sum_{i=1}^n [f(\vec r_{i1}) - f(\vec r_{i0})] 
= f(\vec r_f) - f(\vec r_i)
}\end{equation}
最后取极限 $n\to \infty$, 可使上式精确成立.证毕.

\subsection{由梯度求势函数}
我们通常把上面的标量函数 $f(\vec r)$ 叫做\bb{势函数}, 其地位相当于牛顿—莱布尼兹公式中的原函数. 在这个类比中, 既然“对原函数求导” 对应“对势函数求梯度”, 那么不定积分对应的“通过梯度函数求势函数”又该如何实现呢?

以二维的情况为例, 我们可以先指定势函数在某点 $\vec r_0 (x_0,y_0)$ 的值, 然后根据\autoref{Grad_eq8}, 要求势函数任意一点 $\vec r (x,y)$ 的值, 只需从 $\vec r_0$ 点出发由任意路径线积分到点 $\vec r$ 即可得到势函数 $f(\vec r)$.
\begin{equation}\label{Grad_eq13}
f(\vec r) =  f(\vec r_0) + \int_{\vec r_0}^{\vec r} \grad f(\vec r) \vdot \dd{\vec r} 
\end{equation}
计算该线积分一般选取一种简单的路径:即先延从 $\vec r_0(x_0,y_0)$ 到 $\vec r_1(x, y_0)$ 的水平线段, 再延从 $\vec r_1(x, y_0)$ 到 $\vec r(x,y)$ 的竖直线段(当然也可以取中间点为 $(x_0,y)$). 若把 $\grad f$ 的两个分量 $\pdv*{f}{x}, \pdv*{f}{y}$ 简写为 $f_x(x,y), f_y(x,y)$, 分关于 $x$ 和 $y$ 的不定积分记为 $F_x(x,y), F_y(x,y)$, 延两个线段的线积分\upref{IntL}(分别把 $x$ 和 $y$ 作为线积分的参数)分别为
\begin{equation}
 \int_{\vec r_0}^{\vec r_1} \grad f(\vec r) \vdot \dd{\vec r} = \int_{x_0}^{x} f_x(x,y_0) \dd{x} + 0 = F_x(x,y_0) - F_x(x_0,y_0)
\end{equation}
\begin{equation}
 \int_{\vec r_1}^{\vec r} \grad f(\vec r) \vdot \dd{\vec r} = \int_{y_0}^{y} f_y(x,y) \dd{y} + 0 = F_y(x,y) - F_y(x,y_0)
\end{equation}
代回\autoref{Grad_eq13} 得势函数为
\begin{equation}\label{Grad_eq23}\ali{
f(x,y) &= f(x_0,y_0) + F_x(x,y_0) - F_x(x_0,y_0) + F_y(x,y) - F_y(x,y_0) \\
&= F_y(x,y) - F_y(x,y_0) + F_x(x,y_0) + C
}\end{equation}
其中 $C$ 为待定常数.

\eentry{势能\upref{V}}

