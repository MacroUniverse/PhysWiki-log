% 量子简谐振子(级数法)
% 量子力学|微分方程|能级|厄密多项式|Hermite|薛定谔方程

\pentry{简谐振子(升降算符)\upref{QSHOop}}

\subsection{结论}

量子力学中, 简谐振子的能级为
\begin{equation}
E_n = \qty(\frac12 + n)\hbar \omega 
\end{equation}
波函数为
\begin{equation}
\psi_n (x) = \frac{1}{\sqrt{2^n n!}} \qty(\frac{\alpha^2}{\pi })^{1/4} H_n(u) \E^{-u^2/2}
\end{equation}
其中
\begin{equation}
\alpha \equiv \sqrt{\frac{m\omega}{\hbar }} \qquad
u \equiv \alpha x
\end{equation}
$H_n(u)$ 叫做 Hermite 多项式
\begin{equation}
H_n(u) \equiv (- 1)^n \E^{u^2} \dv[n]{u} \qty(\E^{-u^2})
\end{equation}
前 6 阶 Hermite 多项式分别为
\begin{equation}
\begin{array}{l}
H_0(u) = 1\\
H_1(u) = 2u\\
H_2(u) = 4u^2 - 2
\end{array}
\qquad
\begin{array}{l}
H_3(u) = 8u^3 - 12u\\
H_4(u) = 16u^4 - 48u^2 + 12\\
H_5(u) = 32u^5 - 160u^3 + 120u
\end{array}
\end{equation}
前4个波函数分别为(注意函数的奇偶性与角标的奇偶性相同)
\begin{equation}\ali{
\psi_0(x) &= \qty(\frac{\alpha^2}{\pi})^{1/4} \E^{-u^2/2} &
\psi_2(x) &= \qty(\frac{\alpha^2}{\pi })^{1/4} \frac{1}{\sqrt 2 } (2u^2 - 1)\E^{-u^2/2}\\
\psi_1(x) &= \qty(\frac{\alpha ^2}{\pi })^{1/4} \sqrt2u \E^{-u^2/2} \quad &
\psi_3(x) &= \qty(\frac{\alpha ^2}{\pi })^{1/4} \frac{1}{\sqrt 3} u(2u^2 - 3)\E^{-u^2/2}
}\end{equation}

\subsection{推导}%未完成

薛定谔方程为
\begin{equation}
-\frac{\hbar^2}{2m} \dv[2]{\psi}{x} + \frac12 m \omega^2 x^2\psi  = E\psi
\end{equation}