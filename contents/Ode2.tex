二阶常系数齐次微分方程形式如下
\begin{equation}
ay'' + by' + c = 0
\end{equation}
注意到指数函数 $y = C{\E^{rx}}$ 第 $n$ 阶导数为 ${r^n}{\E^{rx}}$, 不妨尝试把指数函数代入方程,得
\begin{equation}
\left( {a{r^2} + br + c} \right){\E^{rx}} = 0
\end{equation}
由于 ${\E^{rx}} \ne 0$, 必有 $a{r^2} + br + c = 0$. 把这个二次函数叫做\textbf{特征方程},解特征方程,就可以得到方程的解.下面分 $4$ 种情况来讨论

\begin{enumerate}
\item 有两个不同的实根 ${r_1}$,  ${r_1}$ ( ${b^2} - 4ac > 0$),方程的通解为
 \begin{equation}
y = {C_1}{\E^{{r_1}x}} + {C_2}{\E^{{r_2}x}}
\end{equation}
\item 有一个重根 $r$ ( ${b^2} - 4ac = 0$ ),方程的通解为
 \begin{equation}
y = {C_1}{\E^{rx}} + {C_2}x{\E^{rx}}
\end{equation}
\item 有两个纯虚数根( ${b = 0,\,\,{b^2} - 4ac < 0}$), 方程的通解为(见弹簧振子,%未完成:引用,词条
 LC振荡电路\upref{LCCircuit})
\begin{equation}
y = {C_1}\cos \left( {{\omega _0}x} \right) + {C_2}\sin \left( {{\omega _0}x} \right)
\end{equation}
或 
\begin{equation}
y = {C_1}\cos \left( {{\omega _0}x + {C_2}} \right)
\end{equation} 
其中 ${\omega _0} = \sqrt {{c}/{a}}$. 

\item 有两个复数根 $r \pm \I\omega $ (${b \ne 0,\,\,{b^2} - 4ac < 0}$), 方程的通解为(见受阻弹簧振子%未完成:引用,词条
,无源LRC电路)%未完成:引用,词条
\begin{equation}
y = {\E^{rx}}[ {{C_1}\cos \left( {\omega x} \right) + {C_2}\sin \left( {\omega x} \right)} ]
\end{equation} 
或 
\begin{equation}
y = {C_1}{\E^{rx}}\cos \left( {\omega x + {C_2}} \right)
\end{equation} 
其中 $r =  - \frac{b}{{2a}}$,  $\omega  = \frac{1}{{2a}}\sqrt {4ac - {b^2}} $

若令 ${\omega _0} = \sqrt {\frac{c}{a}} $,  $\gamma  = \frac{b}{{2\sqrt {ac} }}$,  则 $r =  - {\omega _0}\gamma $,  $\omega  = {\omega _0}\sqrt {1 - {\gamma ^2}} $. 满足 ${r^2} + {\omega ^2} = \omega _0^2$. 
\end{enumerate}


















 
