% Matlab 的变量与矩阵

\pentry{Matlab 简介\upref{Matlab}}

\subsection{变量与矩阵}
\textbf{变量(variable)}可用于储存数据并通过变量名获取变量值.变量名可以由多个字母,数字和下划线组成. 注意变量名区分大小写,且首字符只能是字母. 用等号可以对变量赋值, 被赋值的变量放在等号左边, 等号右边表达式的运算结果会储存在被赋值的变量中, 直到再次被赋值.
\begin{Command}
>> a = 1.2/3.4 + (5.6+7.8)*9 -1 \\
a = 119.9529 \\
>> b = atan(a + 1) \\
b = 1.5625
\end{Command}
如果新的变量第一次被赋值,它会自动出现在 Workspace 窗口中.注意 Workspace 中的一个特殊的变量 \texttt{ans},如果命令的输出结果没有赋值给变量,就会自动赋值给 \texttt{ans}.注意一般不要对 \texttt{ans} 赋值. 另外两个特殊的变量是 \texttt{pi} (圆周率)和 \texttt{i} (虚数单位), 一般也不要对他们赋值. 自然对数底没有对应的变量, 若要使用自然对数底, 用 \texttt{exp(1)} 即可.

另外,如果在命令后面加分号(semicolon)“\,\texttt{;}”,则命令执行后不输出结果.也可以用分号把多个命令写到一行.
\begin{Command}
>> 1 + 1; a = ans\^{}2 \\
a = 4
\end{Command}
用 Editor 编写程序时,每个命令后面都需要加分号, 需要在 Command Window 输出时, 用 \texttt{disp} 函数.
\begin{Command}
>> disp({\color{string}'something'}); disp(10);\\
something\\
10
\end{Command}

\texttt{clear} 命令可以清空 Workspace 中的所有变量,用 \texttt{clear <var1>,<var2> ...} 清除指定的变量(\texttt{<var1>,<var2>}是变量名).用 \texttt{clc} 命令可以清空 Command Window (按上箭头仍然可以查看历史命令).

本书只涉及到 3 种\textbf{变量类型(class)}:\textbf{双精度(double)},\textbf{字符(char)} 和 \textbf{逻辑(logical).}

\subsection{双精度变量}

双精度变量用于储存数值,有效数字约为 16 位(如果是复数,实部和虚部各 16 位),取值范围约为 $10^{-308}$ 到 $10^{308}$. 如无变量类型声明,所有命令中出现的常数及储存数值的变量都为 \texttt{double}.

Matlab 中的所有变量都可以理解为\textbf{矩阵},单值变量(\texttt{标量},scalar)可以理解为 \texttt{1×1} 的矩阵,只有一行或一列的矩阵叫做\textbf{行矢量(row vector)}和\textbf{列矢量(column vector)}.一些简单的矩阵操作如下
\begin{Command}
>> a = [1,2,3] \\
a = 1\ \ 2\ \ 3
\end{Command}
用方括号创建矩阵,用逗号分隔每行的矩阵元,行矢量中逗号可省略
\begin{Command}
>> a = [1 2 3] \\
a = 1\ \ 2\ \ 3
\end{Command}
用分号分隔行
\begin{Command}
>> b = [1;2;3] \\
b = \par
1 \par
2 \par
3 \\
>> c = [1 2 3; 2 3 4; 3 4 5]\\
c = \par
1\ \ 2\ \ 3 \par
2\ \ 3\ \ 4 \par
3\ \ 4\ \ 5
\end{Command}
方括号还可以用来合并矩阵(注意矩阵尺寸必须合适)
\begin{Command}
>> d = [a;a] \\
d = \par
1\ \ 2\ \ 3 \par
1\ \ 2\ \ 3 \\
>> e = [a a] \\
e = \par
1\ \ 2\ \ 3\ \ 1\ \ 2\ \ 3
\end{Command}
用 \texttt{size} 函数获取矩阵尺寸,用 \texttt{numel} 函数获取矩阵元个数
\begin{Command}
>> size(d) \\
ans = 2\ \ 3 \\
>> size(d,1) \\
ans = 2 \\
>> numel(d) \\
ans = 6
\end{Command}
用 \texttt{zeros} 函数生成全零矩阵
\begin{Command}
>> zeros(2,3) \\
ans = \par
0\ \ 0\ \ 0 \par
0\ \ 0\ \ 0
\end{Command}
用 \texttt{zeros([2,3])} 和 \texttt{zeros(size(d))} 结果也相同.用 \texttt{ones} 可以生成全 1 矩阵,也可以乘以任意常数
\begin{Command}
>> ones(2,3)*5 \\
ans = \par
5\ \ 5\ \ 5 \par
5\ \ 5\ \ 5
\end{Command}
用 \texttt{eye(N)} 生成 $N\times N$ 的单位矩阵,用 \texttt{rand(M,N)} 生成随机矩阵,矩阵元从 0 到 1 均匀分布.用 \texttt{M:step:N} 生成等差数列(行矢量),例如
\begin{Command}
>> 1:2:10 \\
ans = 1\ \ 3\ \ 5\ \ 7\ \ 9 \\
>> 0:pi/3:pi*2 \\
ans = 0\ \ 1.0472\ \ 2.0944\ \ 3.1416\ \ 4.1888\ \ 5.2360\ \ 6.2832 \\
>> 10:-2:1 \\
ans = 10\ \ 8\ \ 6\ \ 4\ \ 2
\end{Command}
如果只用一个冒号, 那么间隔默认为 1
\begin{Command}
>> 1:3 \\
ans = 1\ \ 2\ \ 3
\end{Command}
用 \texttt{linspace(x1,x2,Nx)} 生成指定首项尾项和项数的等差数列(行矢量)
\begin{Command}
>> linspace(0,pi,4) \\
ans = 0\ \ 1.0472\ \ 1.2566\ \ 2.0944\ \ 3.1416
\end{Command}

下面介绍矩阵运算.同规格的尺寸可以进行 \texttt{+} 和 \texttt{-} 运算,矩阵和标量也可以, 结果是把每个矩阵元加(减)标量;矩阵乘法\texttt{*} 既可以把常数与矩阵相乘,也可以进行数学上的矩阵乘法;矩阵的幂“\texttt{\^{}}”相当于矩阵与自己多次相乘;“\texttt{/}”可以把矩阵除以一个常数.
\begin{Command}
>> a = [1 2; 3 4]; b = [1 -1; 2 -2]; \\
>> a + b \\
ans = \par
2\ \ 1 \par
5\ \ 2\\
>> a * b \\
ans = \par
5\ \ -5 \par
11\ \ -11
\end{Command}
若要两个尺寸相同,可进行\textbf{逐个元素运算},如
\begin{Command}
>> a .* b \\
ans = \par
1\ \ -2 \par
6\ \ -8\\
>> a.\^{}2 \\
ans = \par
1\ \ 4 \par
9\ \ 16 \\
>> a ./ b \\
ans = \par
1.0000\ \ -2.0000 \par
1.5000\ \ -2.0000
\end{Command}
单引号“\texttt{'}”可以使实数矩阵转置,或使复矩阵取厄米共轭.若需要对复矩阵转置,用“\texttt{.'}”即可.
\begin{Command}
>> c = a + 1i*b \\
c = \par
1.0000 + 1.0000i\ \ 2.0000 - 1.0000i \par
3.0000 + 2.0000i\ \ 4.0000 - 2.0000i \\
>> c' \\
ans = \par
1.0000 - 1.0000i\ \ 3.0000 - 2.0000i \par
2.0000 + 1.0000i\ \ 4.0000 + 2.0000i \\
>> c.' \\
ans = \par
1.0000 + 1.0000i\ \ 3.0000 + 2.0000i \par
2.0000 - 1.0000i\ \ 4.0000 - 2.0000i
\end{Command}
使用 \texttt{fliplr} 和 \texttt{flipud} 函数可以分别把矩阵左右翻转和上下翻转, 例如
\begin{Command}
>> fliplr(a)\\
ans = \par
2\ \ 1\par
4\ \ 3\\
>> flipud(a)\\
ans = \par
3\ \ 4\par
1\ \ 2
\end{Command}

Matlab 自带的数学函数一般支持矩阵自变量,结果是该函数对每个矩阵元分别运算
\begin{Command}
>> cos(0:pi/4:pi)\\
ans = \par
1.0000\ \ 0.7071\ \ 0.0000\ \ -0.7071\ \ -1.0000
\end{Command}
用矢量运算可以使代码简短易懂,且提高计算效率.

用 \texttt{sum} 函数可以分别求矩阵每列或每行的和. 当矩阵为行矢量或列矢量时, \texttt{sum} 对所有矩阵元求和. 其他情况下, \texttt{sum} 默认对每列求和, 若想对每行求和, 可以在第二个输入变量中输入 \texttt{2} (行是矩阵的第二个维度). 例如
\begin{Command}
>> a = [1,2; 3,4];\\
>> sum(a)\\
ans = 4\ \ 6\\
>> sum(a,2)\\
ans =\par 3\par 7
\end{Command}
用 \texttt{mean} 函数可以求平均值, 使用格式与 \texttt{sum} 相同.

\subsection{矩阵索引}

\textbf{矩阵索引} 用于表示矩阵部分矩阵元,例如
\begin{Command}
>> a = [1 2 3; 4 5 6; 7 8 9]; \\
ans = \par
1\ \ 2\ \ 3 \par
4\ \ 5\ \ 6 \par
7\ \ 8\ \ 9 \\
>> a(1,2) \\
ans = 2 \\
>> a(2:3,1) \\
ans = \par
4 \par
7 \\
>> a(2:3,1:2) \\
ans = \par
4\ \ 5 \par
7\ \ 8 \\
>> a([2,3],[1,2]) \\
ans = \par
4\ \ 5 \par
7\ \ 8 \\
>> a(:,2) \\
ans = \par
2 \par
5 \par
8 \\
>> a(1:end-1,2:3) \\
ans = \par
2\ \ 3 \par
5\ \ 6
\end{Command}
其中 \texttt{end} 表示某维度的最大索引. 以上的格式可以归结为“在小括号中用逗号把行矢量隔开”.注意索引不仅可以用来取值,还可以放在等号左边赋值.
\begin{Command}
>> b = a; b(1:3) = a(2:4) \\
b = \par
4\ \ 2\ \ 3 \par
7\ \ 5\ \ 6 \par
2\ \ 8\ \ 9
\end{Command}
要求左边的矩阵元个数等于右边.唯一的例外是当右边为标量
\begin{Command}
b(1:3) = 0 \\
b = \par
0\ \ 2\ \ 3 \par
0\ \ 5\ \ 6 \par
0\ \ 8\ \ 9 
\end{Command}
我们还可以用单个索引
\begin{Command}
>> a(2:5)\\
ans = 4\ \ 7\ \ 2\ \ 5 \\
>> a(7:end)\\
ans = 3\ \ 6\ \ 9
\end{Command}
注意单个索引的顺序是先增加第一个维度(行标),再增加第二个维度(列标). 虽然上面都是以双精度矩阵为例, 但这些索引方法适用于任何数据类型的矩阵.

\subsection{字符变量}

字符型变量一般用于控制行输出结果或对生成的图片进行标注.把 \texttt{N} 个字符放在一对单引号内,可生成 \texttt{1×N} 的字符类型数组.
\begin{Command}
>> str1 = {\color{string}'\!这是一个字符串'}; str2 = {\color{string}'this is a string'};\\
>> [str1, {\color{string}','}, str2] \\
ans = '\!这是一个字符串, this is a string'\\
>> numel(str) \\
ans = 24
\end{Command}
把双精度变类型变为字符串可以用 \texttt{num2str} 函数(注意 \texttt{2} 的英文读音与“to”相同, \texttt{num} 代表“number”, \texttt{str} 代表字符串“string”),通常用于与其他字符矩阵合并,如
\begin{Command}
>> number = 3; str = [{\color{string}'The number is'}, num2str(number), {\color{string}'.'}] \\
str = 'The number is 3.'
\end{Command}
若要在字符串中加入英文单引号,可用两个英文单引号表示.

\subsection{逻辑变量}

逻辑变量只能具有 \texttt{0} 或 \texttt{1} 两个值,分别代表\textbf{假(false)}和\textbf{真(true)}.最常用的地方是判断语句 \texttt{if} 的后面以及获取矩阵元.逻辑算符有
\begin{Command}
>,>=(大于等于),<,<=,==(等于)
\end{Command}
可用于比较双精度数组,返回逻辑型数组
\begin{Command}
>> L = 1 + 1 > 3 \\
L = {\color{blue}\underline{logical}} 0
\end{Command}
在控制行中, 输出的 \texttt{\color{blue}\underline{logical}} 与 \texttt{0} 各占一行, 这里为了节约空间将其写成一行. 逻辑“与”,“或”,“非”算符分别为(仅用于逻辑标量)\texttt{\&\&,||,\texttilde}. 当算符两边都为真时,与运算才能为真, 若至少有一边为真,或运算就为真. 非运算用于把真假互换. 例如
\begin{Command}
>> 1 > 0 \&\& 2 > 1 \\
ans = {\color{blue}\underline{logical}} 1\\
>> 1 > 0 || 2 < 1 \\
ans = {\color{blue}\underline{logical}} 1\\
>> \~ {}(1 > 0)\\
ans = {\color{blue}\underline{logical}} 0
\end{Command}
注意由于非运算的优先级比 \texttt{>} 运算要高, 所以最后一条命令必须要加括号.

在需要的时候, 双精度变量可以自动转换为逻辑变量, 规则是只有双精度的 0 转换为逻辑 0, 其他双精度值一律转换为逻辑 1. 如
\begin{Command}
>> 1.3 \&\& -0.8 \\
ans = {\color{blue}\underline{logical}} 1
\end{Command}

除了上文中介绍的索引方法外, 我们还可以用相同大小逻辑数组索引任意矩阵. 注意逻辑索引的输出结果是一个列矢量, 以下为了节约空间我们将输出结果改为行矢量.
\begin{Command}
>> a = [1 2 3; 4 5 6; 7 8 9]; \\
>> mark = logical([1 0 0; 0 0 1; 1 0 0]); a(mark) \\
ans = 1\ \ 7\ \ 6
\end{Command}
逻辑索引常见的例子如
\begin{Command}
>> a(a-3 <= 1) \\
ans = 1\ \ 4\ \ 2\ \ 3
\end{Command}
注意逻辑索引中不能使用双精度类型代替逻辑类型.

\texttt{find} 函数可以用于寻找逻辑矩阵中所有值为 1 的矩阵元的位置. 如果只用一个输出变量(或不提供输出变量), 函数返回单索引, 如果提供两个输出变量, 函数返回行标和列标索引. 注意输出变量仍然是列矢量, 为了节约空间以下记为行矢量. 例如
\begin{Command}
>> a = [1, 0; 0, 1]\\
a =\par
1\ \ 0\par
0\ \ 1\\
>> find(a)\\
ans = 1\ \ 4\\
>> [c, r] = find(a)\\
c = 1\ \ 2\\
r = 1\ \ 2
\end{Command}
注意以上的双精度矩阵 \texttt{a} 被自动转换为逻辑矩阵. 

\texttt{any} 函数用于判断逻辑矩阵中的每一列是否存在任何值为“真” 的矩阵元, 若有, 则返回真, 否则返回假. \texttt{all} 函数用于判断逻辑矩阵中的每一列是否所有矩阵元都为“真”, 若是, 返回真, 否则返回假. 这两个函数的使用格式与 \texttt{sum} 和 \texttt{mean} 类似, 当第二个输入变量为 \texttt{2} 时, 对每行进行操作.
\begin{Command}
>> a = [1 6 7; 2 7 1; 3 8 9];\\
>> all(a > 5)\\
ans = 1×3 {\color{blue}\underline{logical}} array\par
0\ \ 1\ \ 0\\
>> any(a > 7, 2)\\
ans = 3×1 {\color{blue}\underline{logical}} array\par
0\par 0\par 1\\
>> all(a(:) < 10)\\
ans = {\color{blue}\underline{logical}} 1
\end{Command}

