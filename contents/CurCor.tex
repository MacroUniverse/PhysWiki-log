% 正交曲线坐标系

\subsection{柱坐标系}
我们先来分析柱坐标系, 位置矢量 $\uvec r$ 在直角坐标系中展开为
\begin{equation}\label{CurCor_eq1}
\vec r(r, \theta, z) = r\cos\theta \uvec x + r\sin\theta \uvec y + z\uvec z
\end{equation}
柱坐标系中三个单位矢量 $\uvec r, \uvec \theta, \uvec z$ 的方向被定义为每个坐标增加时 $\vec r$ 增加的方向, 即以下偏导数的方向
\begin{equation}\label{CurCor_eq2}
\leftgroup{
\pdv*{\vec r}{r} &= \cos\theta \uvec x + \sin\theta \uvec y\\
\pdv*{\vec r}{\theta} &= -r\sin\theta \uvec x + r \cos\theta \uvec y\\
\pdv*{\vec r}{z} &= \uvec z
}\end{equation}
将这三个矢量归一化, 就得到三个单位矢量
\begin{equation}\label{CurCor_eq3}
\leftgroup{
\uvec r &= \cos\theta \uvec x + \sin\theta \uvec y\\
\uvec \theta &= -\sin\theta \uvec x + \cos\theta \uvec y\\
\uvec z &= \uvec z
}\end{equation}

可见柱坐标系和直角坐标系中的 $\uvec z$ 相同, 而 $\uvec r, \uvec \theta$ 分别是 $\uvec x, \uvec y$ 绕 $z$ 轴逆时针旋转 $\theta$ 角所得. 所以尽管柱坐标系中的三个单位矢量的方向取决于坐标, 但它们始终两两垂直.

我们把这样的坐标系叫做\bb{正交曲线坐标系}, 或简称为\bb{曲线坐标系}. 我们熟知的直角坐标系显然就是一个正交曲线坐标系, 稍后我们会看到球坐标系也是正交曲线坐标系.

现在我们可以将\autoref{CurCor_eq1} 和\autoref{CurCor_eq2} 用柱坐标中的三个单位矢量来表示.
\begin{gather}
\vec r = r\uvec r + z\uvec z\\
\pdv{\vec r}{r} = \uvec r \qquad \pdv{\vec r}{\theta} = r\uvec \theta \qquad \pdv{\vec r}{z} = \uvec z\label{CurCor_eq5}
\end{gather}
与极坐标的情况(%极坐标单位矢量偏导公式的链接 未完成
)类似, 将\autoref{CurCor_eq3} 对 $\theta$ 求偏导可以得到单位矢量的偏导(事实上极坐标系完全看以看做柱坐标系 $z = 0$ 时的情况)
\begin{equation}
\pdv{\uvec r}{\theta} = \uvec \theta \qquad
\pdv{\uvec \theta}{\theta} = -\uvec r \qquad
\pdv{\uvec z}{\theta} = \vec 0
\end{equation}
另外将\autoref{CurCor_eq5} 代入矢量全微分与偏导的关系, 一段微小位移可记为%未完成:链接
\begin{equation}
\dd{\vec r} = \dd{r}\uvec r + r\dd{\theta} \uvec \theta + \dd{z} \uvec z
\end{equation}

\subsection{球坐标系}
% 未完成




