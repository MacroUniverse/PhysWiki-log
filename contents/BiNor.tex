% 二项式定理
% 多项式|二项式定理|排列组合

\pentry{组合\upref{combin}}

二项式展开公式为
\begin{equation}\label{BiNor_eq1}
(a + b)^n = \sum_{i = 0}^n  \pmat{n\\i} a^i b^{n - i} \quad (n \text{为正整数})
\end{equation}
其中表示\textbf{组合(combination)}, 定义为 % 链接未完成
\begin{equation}
\pmat{n\\i} = \frac{n(n - 1)\dots (n - i + 1)}{i!} = \frac{n!}{i!(n - i)!}
\end{equation}

\subsection{推导}\label{BiNor_sub5}
$\mathfrak{g}$ 若展开多项式的时候先不合并同类项(每项前面的系数都是 1 )则
\begin{itemize}
\item $(a + b)^0 = 1$ 有 1 项
\item $(a + b)^1 = a + b$ 有 2 项
\item $(a + b)^2 = aa + ab + ba + bb$ 有 4 项
\item $(a + b)^3 = aaa + aab + aba + abb + baa + bab + bba + bbb$ 有 8 项
\item $(a + b)^n$ 有 $2^n$ 项(若不合并相同项)
\end{itemize}

这就相当于用 $a$ 和 $b$ 填满 $n$ 个有序的位置,每个位置都可以取 $a$ 或 $b$, 共有 $2^n$ 种排列,每种排列就是一项,所以共有 $2^n$ 项.

下面把 $2^n$ 项中的相同项进行合并,把其中出现了 $i$ 个 $a$ 及 $n-i$ 个 $b$ 的项都记为 $a^i b^{n-i}$, 那么共有 $\pmat{n\\i}$ 个这样的项.把它们相加得 $\pmat{n\\i} a^i b^{n-i}$. 所以
\begin{equation}
(a + b)^n = \sum_{i = 0}^n  \pmat{n\\i} a^i b^{n - i}
\end{equation}
证毕.
