% 一维势能束缚态数值解(试射法)

\subsection{ODE 函数}
TISE 化为一阶微分方程组用于 ODE 解算器(如 Matlab 的 ode45)
\begin{equation}
\leftgroup{
\psi'(x) &= v(x)\\
v'(x) &= 2m[V(x) - E]\psi(x)
}\end{equation}
若某个小区间 $V(x)$ 看做常数, 解为
\begin{equation}
\psi(x) = \leftgroup{
&\sin(kx)& \quad &(E > V)\\
&\exp(\kappa x) && (E < V)
}\end{equation}
其中
\begin{equation}
k = \sqrt{2m(E - V)} \qquad
\kappa = \sqrt{2m(V - E)}
\end{equation}

\subsection{初始条件}
所以 ODE  初始条件选择
\begin{equation}
(\psi(x_0), \psi'(x_0)) = \leftgroup{
&(0, 1)& \quad &(E > V(x_0))\\
&(\epsilon, \kappa\epsilon) && (E < V(x_0))
}\end{equation}
其中 $\epsilon$ 是一个很小的数. 第二种情况中, 解析初始条件应该是 $(0, 0)$, 但显然不能用这个.

从两端射完并 match 以后, $\psi(x)$ 有可能具有很大的值, 可以归一化使其最大值等于 1. 归一化以后再检验两端是否满足 $\psi(x_0) \ll 1$(取决于精度).

\subsection{中点匹配}
理论上, 我们可以用所谓的“甩尾法”,从波函数的一端射到另一端, 判断另一端是否为零. 但从数值误差来说,从指数末端出射误差远远大于从指数末端入射. 所以更好的办法是从两端 $x_L, x_R$ 分别入射后在某个中点 $x_M$ 处匹配.

简单的匹配可以采用对数导数(log derivative) 来计算匹配误差, 即
\begin{equation}
\text{err}(E) = \psi'_L(x_M)/\psi_L(x_M) - \psi'_R(x_M)/\psi_R(x_M)
\end{equation}
用多区间二分法即可得到定态能量.

但是这么做有两个缺点, 一是 $\psi(x_M)$ 为零或很小的时候, err 有可能不稳定, 二是用多区间二分法解 $\text{err}(E) = 0$ 有可能出现不收敛的解, 因为该函数存在断点(想象 $\sin(k x_M)$ 的 log derivative 随 k 变化的情况).

为了解决这个问题, 我们可以用“瞬时相位” 来构建误差函数(即 $\psi(x) = \cos(kx)$ 中的 $kx$, 用 $\psi, \psi'$ 来表示)
\begin{equation}
\theta_i = \text{atan2}(-\psi'_i, k\psi_i) \qquad (i = 1, 2 \quad \theta_i \in [-\pi, \pi])
\end{equation}
\begin{equation}
\theta = \theta_2 - \theta_1 \qquad (\theta \in [-2\pi, 2\pi])
\end{equation}
构建误差函数 $\text{err}(\theta)$ 满足一些条件: 1. $\theta = N\pi$ 为零点, 2. 零点两侧异号, 3. 函数连续, 4.周期为 $2\pi$. 一个显然的误差函数是满足这些条件的三角波.

现在再用多区间二分法就可以解出所有正确的束缚态能量了.
