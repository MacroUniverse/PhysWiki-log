% 极坐标中的速度和加速度

\pentry{速度\ 加速度\upref{VnA}, 极坐标中单位矢量的偏导\upref{DPol1}}

若已知某点的极坐标关于时间的函数 $r(t)$ 和 $\theta (t)$, 求该点的速度和加速度.

极坐标中的位置矢量可以用 $\vec r = r\uvec r$ 表示, 注意其中径向单位矢量可以看做复合函数 $\uvec r[\theta(t)]$.根据定义, 速度是位矢的一阶导数, 在力学中经常在变量上面加一点表示对时间的一阶导数, 两点表示二阶导数, 根据矢量的求导法则%未完成: 矢量和常数相乘的求导
\begin{equation}
\vec v = \dot r \uvec r + r \dv{\uvec r}{t}
\end{equation}
由链式法则%未完成: 矢量函数的链式法则
和\autoref{DPol1_eq1}\upref{DPol1}, 上式中
\begin{equation}
\dv{\uvec r}{t} = \dv{\uvec r}{\theta} \dot \theta = \dot \theta \uvec \theta
\end{equation}
所以极坐标中的速度为
\begin{equation}
\vec v = \dot r \uvec r + r \dot \theta \uvec \theta
\end{equation}
这是符合直觉的, \bb{径向}速度等于位矢模长的导数, 而\bb{角向}速度等于位矢模长乘以角速度.

我们再来计算加速度, 用同样的方法对速度求一阶导数得
\begin{equation}\ali{
\vec a &= \ddot r \uvec r + \dot r \dv{\uvec r}{t} + \dot r \dot \theta \uvec \theta + r\ddot \theta \uvec \theta + r\dot \theta \dv{\uvec \theta}{t}\\
&= \ddot r \uvec r + \dot r \dot \theta \uvec \theta + \dot r \dot \theta \uvec \theta + r\ddot \theta \uvec \theta - r{\dot \theta }^2 \uvec r\\
&= (\ddot r - r{\dot \theta}^2) \uvec r + (2\dot r\dot \theta + r\ddot\theta)\uvec \theta
}\end{equation}
这个结论并不是那么显而易见. 我们将加速度的径向和角向分量分别记为 $a_r$ 和 $a_\theta$, 其中 $a_\theta$ 还可以记为另一种更紧凑形式即
\begin{equation}
a_\theta = \frac{1}{r}\dv{t} \qty(r^2\dv{\theta}{t} )
\end{equation}