% 一元矢量函数的积分

% 介绍单变量的, 以及体积分(例如流体的动量)
\pentry{矢量的导数\upref{DerV}}% 未完成

\subsection{单变量不定积分}
令 $\vec f(t)$ 为只有一个自变量的矢量函数, 则与标量函数类似, 定义其不定积分为求导的逆运算. 也就是说, 若能找到 $\vec F(t)$, 使得 $\vec F(t)$ 对 $t$ 求导就是 $\vec f(t)$, 那么 $\vec F(t) + \vec C$ ($\vec C$ 为任意常矢量) 就是定积分的结果, 都是 $\vec f(t)$ 的原函数.

在直角坐标系中, 我们已经知道对矢量函数 $\vec F(t)$ 求导就是对它的每个分量函数分别求导, 即
\begin{equation}\label{IntV_eq1}
\vec F'(t) = \vec f(t)
\end{equation}
\begin{equation}\label{IntV_eq2}
F'_x(t) = f_x(t) \qquad F'_y(t) = f_y(t) \qquad F'_z(t) = f_z(t)
\end{equation}
考虑到标量函数的不定积分是标量函数求导的逆运算, 所以对 $\vec f(t)$ 不定积分, 只需对它的各个分量分别进行不定积分即可. 注意每个分量函数在不定积分后都会出现一个待定常数, 三个分量中的待定常数相加就得到一个待定常矢量 $\vec C$.
\begin{equation}\begin{aligned}
\int \vec f(t) \D t &= \uvec x \int f_x(t) \D t + \uvec y \int f_y(t) \D t + \uvec z \int f_z(t) \D t\\
&= [F_x(t)+C_x]\uvec x + [F_y(t)+C_y]\uvec y + [F_z(t)+C_z]\uvec z\\
&= \vec F(t) + \vec C
\end{aligned}\end{equation}
根据\autoref{IntV_eq1} \autoref{IntV_eq2}, 显然有 $[\vec F(t) + \vec C]' = \vec f(t)$.

\subsection{单变量定积分}
类比一元标量函数定积分\upref{DefInt}的定义, 要计算一元矢量函数 $\vec f(t)$ 从 $t_1$ 到 $t_2$ 的定积分, 就先把区间 $[t_1, t_2]$ 分为 $N$ 个小区间, 长度分别为 $\Delta t_i$, 且令 $t_i$ 为第 $i$ 个区间内的任意一点. 当我们取极限令所有区间长度 $\Delta t_i$ 都趋近于 $0$ (这时 $N\to\infty$) 时, 如果以下极限存在, 得到的矢量就是定积分的结果.
\begin{equation}
\int_{t_1}^{t_2} \vec f(t) \D t = \lim_{\Delta t_i \to 0} \sum_{i = 0}^N \vec f(t_i) \Delta t_i
\end{equation}
唯一与标量函数的定积分不同的是, 这里的求和是矢量求和. 但在直角坐标系中, 我们可以把上式对矢量的求和表示成对各个分量分别求和, 而每个分量的极限就是一个标量定积分. 
\begin{equation}\begin{aligned}
\int_{t_1}^{t_2} \vec f(t) \D t &= \uvec x\lim_{\Delta t_i \to 0} \sum_{i = 0}^N f_x(t_i) \Delta t_i
+ \uvec y\lim_{\Delta t_i \to 0} \sum_{i = 0}^N f_y(t_i) \Delta t_i
+ \uvec y\lim_{\Delta t_i \to 0} \sum_{i = 0}^N f_z(t_i) \Delta t_i\\
&= \uvec x\int_{t_1}^{t_2} f_x(t) \D t + \uvec y\int_{t_1}^{t_2} f_y(t) \D t + \uvec z\int_{t_1}^{t_2} f_z(t) \D t
\end{aligned}\end{equation}
所以 $\vec f(t)$ 的定积分就是把直角坐标的各个分量分别进行定积分.现在对三个定积分分别运用牛顿—莱布尼兹公式\upref{NLeib}, $\vec f(t)$ 的原函数为 $\vec F(t)$, 各分量的原函数为 $F_x(t), F_y(t), F_z(t)$, 则上式等于
\begin{equation}\begin{aligned}\label{IntV_eq6}
\int_{t_1}^{t_2} \vec f(t) \D t &= \uvec x [F_x(t_2) - F_x(t_1)] + \uvec y [F_y(t_2) - F_y(t_1)] + \uvec z [F_z(t_2) - F_z(t_1)]\\
&= \vec F(t_2) - \vec F(t_1)
\end{aligned}\end{equation}
这就是矢量函数的牛顿—莱布尼兹公式.

\begin{exam}{加速度,速度和位移的积分关系}\label{IntV_ex1}
由于质点的速度—时间函数 $\vec v(t)$ 是位移—时间函数 $\vec r(t)$ 的导函数, 后者就是前者的原函数. 所以根据牛顿—莱布尼兹公式\autoref{IntV_eq6} 有
\begin{equation}
\vec r(t) - \vec r(t_0) = \int_{t_1}^{t_2} \vec v(t) \D t
\end{equation}
即
\begin{equation}
\vec r(t) = \vec r(t_0) + \int_{t_1}^{t_2} \vec v(t) \D t
\end{equation}
这是一维情况\autoref{VnA1_eq5}\upref{VnA1} 的拓展.

同理, 由于质点的加速度函数 $\vec a(t)$ 是速度函数 $\vec v(t)$ 的导函数, 后者可以通过前者定积分得到
\begin{equation}
\vec v(t) = \vec v(t_0) + \int_{t_1}^{t_2} \vec a(t) \D t
\end{equation}
\end{exam}

\eentry{匀加速运动\upref{ConstA}}