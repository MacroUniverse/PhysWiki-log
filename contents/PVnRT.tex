% 理想气体状态方程

\pentry{分子撞击对容器壁的压强\upref{MolPre}}

理想气体是热力学中一个理想化的模型(可以类比自由落体\upref{ConstA}, 简谐振子等\upref{SHO}), 用于探索密闭容器中一定量气体的压强, 体积和温度之间的关系. 该模型假设密闭容器中有一定量的气体(由原子或分子构成, 以下统称为分子), 假设气体非常稀薄, 分子之间几乎不发生相互作用(例如碰撞), 且与容器壁之间的碰撞是完全弹性的且在瞬间完成的, 假设每个分子所受重力忽略不计.

理想气体状态方程最开始由实验得到, 这是因为在日常环境下, 大部分气体与理想气体符合较好. 其最常见的形式为
\begin{equation}\label{PVnRT_eq1}
PV = nRT
\end{equation}
其中 $P$ 是气体的压强(处处相等), $V$ 是容器的体积, $n$ 是气体分子的摩尔数, $R$ 是\textbf{理想气体常数}, $T$ 是热力学温度(单位为开尔文).

记 $k_B$ 为\textbf{玻尔兹曼常数}, $N_A$ 为\textbf{阿伏伽德罗常数}(一摩尔粒子中的粒子数).
\begin{equation}
k_B \approx 1.38064852 \times 10^{-23} \,\Si{m^2 kg / (K s^2)}
\end{equation}
\begin{equation}
N_A \approx 6.02214086 \times 10^{23} \,\Si{/mol}
\end{equation}
容器中总分子数为
\begin{equation}
N = n N_A
\end{equation}
理想气体常数定义为
\begin{equation}\label{PVnRT_eq2}
R = k_B N_A \approx 8.314 \,\Si{J / (mol · K)}
\end{equation}
所以状态方程(\autoref{PVnRT_eq1}) 也可以记为
\begin{equation}\label{PVnRT_eq4}
PV = N k_B T
\end{equation}

从微观角度, \bb{热力学温度}\upref{tmp} 可以由理想气体分子平均动能定义\footnote{注意这里的动能是平动动能, 我们暂时不讨论气体分子的转动.}
\begin{equation}\label{PVnRT_eq3}
\bar E_k = \frac32 k_B T
\end{equation}

\subsection{由经典力学推导}

我们在 “分子撞击对容器壁的压强\upref{MolPre}” 中已经详细推导了压强和分子速度的关系, 以下重复的部分将简略带过. 注意分子密度趋近于 0 的假设导致分子之间没有作用力, 每个分子的运动都是可以看作是独立的.

假设长方体容器的 $x, y, z$ 三个方向的边长分别为 $a, b, c$, 则体积为  $V = abc$. 考虑一个初始延任意方向运动的分子, 与容器壁发生完全弹性碰撞, 它在 $x$ 方向的周期(即运动一个来回所需的时间)为 $2a/v_x$, 每个周期带给 $x = a$ 容器壁的冲量为 $2m v_x$, 该容器壁面积为 $bc$, 所以受到该粒子的平均压强 $P$ 为冲量除以周期除以面积 $mv_x^2/V$, 即
\begin{equation}
P_i V = mv_{x,i}^2
\end{equation}
注意这里我们用角标 $i$ 来表示第 $i$ 个分子. 如果有 $N$ 个分子, 质量都为 $m$, 那么
\begin{equation}
P V = \sum P_i V = 2 N \qty(\frac12 m \overline {v_x^2}) = 2 N \bar E_{kx}
\end{equation}
注意 $\overline {v_x^2}$ 和 $\bar v_x^2$ 是不同的. 前者是先对每个分子的速度取平方再平均. 而后者是先计算速度的平均值在平方. 等式右边等于 $2N$ 乘以所有分子在 $x$ 方向的平均动能. 由于我们假设分子运动各向同性(即不会出现某些方向的分子运动较快), 所以平均的总动能等于单方向平均动能的 3 倍(注意这里我们假设是 3 维空间, 如果是 $N_d$ 维空间, 就是 $N_d$ 倍)
\begin{equation}
\bar E_k = \frac{1}{2} m (\overline {v_x^2} + \overline {v_y^2} + \overline {v_z^2}) = \frac{3}{2} m \overline {v_i^2}
\end{equation}
所以有(参考\autoref{MolPre_eq4}\upref{MolPre})
\begin{equation}
P V = \frac23 N \bar E_k
\end{equation}
这说明压强和体积的乘积等于分子总平动动能的 $2/3$. 用\autoref{PVnRT_eq3} 消去 $E_k$, 就得到了\autoref{PVnRT_eq4} 形式的理想气体状态方程.
