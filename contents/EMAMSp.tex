%电磁场角动量分解

电磁场的动量为
\begin{equation}
\vec p = \epsilon_0 \int \dd{V} \vec E \cross \vec B
\end{equation}
角动量为
\begin{equation}
\vec J = \vec r \cross \vec p = \epsilon_0 \int \dd{V} \vec r \cross \vec E \cross (\curl \vec A)
\end{equation}
现在假设电磁场只在一定范围内不为零, 且体积分的边界处场强为零. 假设该范围内没有净电荷与电流, 则
\begin{equation}\ali{
\vec r \cross \vec E \cross (\curl \vec A) &= \vec r \cross [\grad (\vec E \vdot \vec A) - \vec A(\vec E \vdot \vec\nabla )]_{\partial A} \\
&= [(\vec r \cross \vec\nabla )(\vec E \vdot \vec A) - (\vec E \vdot \vec\nabla )(\vec r \cross \vec A)]_{\partial A}
}\end{equation}
其中转微分算符 $[\,]_{\partial A}$ 的作用是先把方括号内的 $\vec\nabla$ 作为普通矢量进行计算, 再把展开结果中每一项的偏微分作用在 $A$ 的分量上. 上式第一项为 $\sum\limits_i {E_i} (\vec r \cross \vec\nabla ) A_i$, 第二项为
\begin{equation}
-[(\vec E \vdot \vec\nabla )(\vec r \cross \vec A)]_{\partial A} =  - (\vec E \vdot \vec\nabla )(\vec r \cross \vec A) + [(\vec E \vdot \vec\nabla )(\vec r \cross \vec A)]_{\partial r}
\end{equation}
其中第二项为 $[(\vec E \vdot \vec\nabla )(\vec r \cross \vec A)]_{\partial r} = [(\vec E \vdot \vec\nabla )\vec r] \cross \vec A = \vec E \cross \vec A$, 第一项中
\begin{equation}
\begin{aligned}
(\vec E \vdot \vec\nabla )(\vec r \cross \vec A) &= [(\vec E \vdot \vec\nabla )(\vec r \cross \vec A)]_{\partial ErA} - [(\vec E \vdot \vec\nabla )(\vec r \cross \vec A)]_{\partial E}\\
& = [(\vec E \vdot \vec\nabla )(\vec r \cross \vec A)]_{\partial ErA}
\end{aligned}
\end{equation}
这是因为 $[(\vec E \vdot \vec\nabla )(\vec r \cross \vec A)]_{\partial E} = (\div\vec E)(\vec r \cross \vec A) = 0$.  综上,
\begin{equation}
\vec J = \epsilon_0 \int \dd{V} \sum_i  E_i (\vec r \cross \vec\nabla ) A_i + \epsilon_0 \int \dd{V} \vec E \cross \vec A + \epsilon_0 \int \dd{V} [(\vec E \vdot \vec\nabla )(\vec r \cross \vec A)]_{\partial ErA}
\end{equation}
现在证明最后一项为0. 以 $x$ 分量为例,
\begin{equation}
\begin{aligned}
\uvec x \int \dd{V} [(\vec E \vdot \vec\nabla )(\vec r \cross \vec A)]_{\partial ErA}  &= \int \dd{V} \div [\vec E(\uvec x \vdot \vec r \cross \vec A) ] \\
&= \oint \dd{\vec s} \vec E(\uvec x \vdot \vec r \cross \vec A)  = 0
\end{aligned}
\end{equation}
最后一步是因为边界处场强为零. 现在我们可以看出角动量由两部分组成
\begin{equation}
\vec J = \vec L + \vec S \qquad
\vec L = \epsilon_0 \int \dd{V} \sum_i E_i (\vec r \cross \vec\nabla ) A_i \qquad
\vec S = \epsilon_0 \int \dd{V} \vec E \cross \vec A
\end{equation}
其中 $\vec L$ 是轨道角动量, $\vec S$ 是自旋角动量.
