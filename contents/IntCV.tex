% 换元积分法

% 其实我非常想说明两种换元积分法其实是一种,第一种如果把 x 看成 x(u),d(x(u))=x'(u)du,而 x'(u)*u'(x)=1. 就好了. 但是数学从简原则,略去吧
\pentry{不定积分\upref{Int}}
\subsection{第一类换元积分法}
由复合函数的求导法则%未完成:链接
,若令 $F'\left( x \right) = f\left( x \right)$, 则
\begin{equation}
\frac{\D}{\D x}	F[u(x)] = f[u(x)]u'(x)
\end{equation}
由于求导的逆运算是积分,有
\begin{equation}\label{IntCV_eq2}
\int f[u(x)]u'(x) \D x  = F[u(x)] + C
\end{equation}
所以如果某个积分可以看成 $\int f[u(x)]u'(x)\D x$ 的形式,且 $F(x)$ 较容易求出,即可根据\autoref{IntCV_eq2} 写出结果.这种方法叫做\textbf{第一类换元积分法}.这类换元积分法的技巧就在于如何看出被积函数的的结构是 $\int f[u(x)]u'(x)\D x$, 只有多练习才能熟能生巧. 

\begin{exam}{}
计算
\[\int {a\sin \left( {ax + b} \right)} \D x\]
令 $f\left( x \right) = \sin \left( x \right)$, $u\left( x \right) = ax + b$, 则上式刚好是 $\int f[u(x)]u'(x)\D x$ 的形式.从基本初等函数积分表 %未完成
已知 $\sin x$ 的一个原函数是 $F(x) = -\cos x$, 那么答案就是
\begin{equation}
F[u(x)] + C =  - \cos \left( {ax + b} \right) + C
\end{equation}
\end{exam}

总结到更一般的情况,根据换元积分法,若已知 $\int {f\left( x \right)\D x}  = F\left( x \right) + C$, 则对于任意常数 $a$ 和 $b$, 必有 $\int {a\,f\left( {ax + b} \right)\D x}  = F\left( {ax + b} \right)$. 根据积分的基本性质,%未完成
两边同除 $a$, 得
\begin{equation}
\int {f(ax + b) \D x}  = \frac{1}{a}F\left( {ax + b} \right)
\end{equation}
该式使用频率很高,需要熟练掌握.类似例子还有
\[ \int \frac{1}{ax+b}\D x = \frac{1}{a}\ln (ax + b)+C  \qquad
\int \E^{ax + b}\D x = \frac{1}{a}{\E^{ax + b}}+C \]
等等.

\subsection{积分变量替换}

换元积分法的过程在形式上可以记为 (见微分\upref{Diff})
\begin{equation}\label{IntCV_eq1}
\begin{split}
\int f[u(x)]u'(x)\D x &= \int f[u(x)]\D [u(x)] = \int f(u)\D u = F(u)+C\\
&= F[u(x)]+C
\end{split}\end{equation}
该式把积分变量由 $x$ 换成了 $u$, 故称为\textbf{换元积分法}.

\subsection{第二类换元积分法}
第二类换元积分法从某种意义上和第一类换元积分相反.若要对一个函数积分,先把它的自变量看做另一个变量的函数,再逆向使用\autoref{IntCV_eq1}, 即可化简积分.
\begin{equation}\label{IntCV_eq6}
\int f(x)\D x = \int f[x(t)]\D [x(t)] = \int f[x(t)]x'(t)\D t
\end{equation}
这个积分看似复杂了,但是如果 $x\left( t \right)$ 选取适当,反而可以使计算化简.

\begin{exam}{}
计算
\[\int \frac{\D x}{\sqrt{1-x^2}}\]
显然 $x \in \left( { - 1,1} \right)$, 选取 $x(t)=\sin t$. 替换后的定义域为 $t \in ( -\pi/2,\pi/2)$, 函数单调递增 \footnote{注意任何积分换元法中的两个变量必须有一一对应的关系,即相互的函数关系在定义域内都为单调.}.上面积分变为
\begin{equation}
\int \frac{\D x}{\sqrt{1-x^2}}  = \int \frac{\D\sin t}{\sqrt{1-\sin^2 t}} = \int \frac{\cos t\D t}{\cos t}  = \int {1\D t}  = t + C = \arcsin x + C
\end{equation}
验证: 根据反函数求导法则\upref{InvDer}
\begin{equation}
\arcsin'x = \frac{1}{{\cos \left( {\arcsin x} \right)}} = \frac{1}{{\sqrt {1 - {x^2}} }}
\end{equation}
\end{exam}




