% 换元积分法

% 其实我非常想说明两种换元积分法其实是一种,第一种如果把 x 看成 x(u),d(x(u))=x'(u)du,而 x'(u)*u'(x)=1. 就好了. 但是数学从简原则,略去吧
\pentry{不定积分\upref{Int}}
\subsection{第一类换元积分法}
由复合函数的求导法则%未完成:链接
,若令 $F'(x) = f(x)$, 则
\begin{equation}
\dv{x}	F[u(x)] = f[u(x)]u'(x)
\end{equation}
由于求导的逆运算是积分,有
\begin{equation}\label{IntCV_eq2}
\int f[u(x)]u'(x) \dd{x}  = F[u(x)] + C
\end{equation}
所以如果某个积分可以看成 $\int f[u(x)]u'(x) \dd{x}$ 的形式,且 $F(x)$ 较容易求出,即可根据\autoref{IntCV_eq2} 写出结果.这种方法叫做\textbf{第一类换元积分法}.这类换元积分法的技巧就在于如何看出被积函数的的结构是 $\int f[u(x)]u'(x) \dd{x}$, 只有多练习才能熟能生巧. 

\begin{exam}{}
计算
\begin{equation}
\int a\sin(ax + b) \dd{x}
\end{equation}
令 $f(x) = \sin(x)$, $u(x) = ax + b$, 则上式刚好是 $\int f[u(x)]u'(x) \dd{x}$ 的形式.从基本初等函数积分表 %未完成
已知 $\sin x$ 的一个原函数是 $F(x) = -\cos x$, 那么答案就是
\begin{equation}
F[u(x)] + C =  - \cos(ax + b) + C
\end{equation}
\end{exam}

总结到更一般的情况,根据换元积分法,若已知 $\int f(x) \dd{x}  = F(x) + C$, 则对于任意常数 $a$ 和 $b$, 必有 $\int a\,f(ax + b) \dd{x} = F(ax + b)$. 根据积分的基本性质,%未完成
两边同除 $a$, 得
\begin{equation}
\int f(ax + b) \dd{x} = \frac{1}{a} F(ax + b)
\end{equation}
该式使用频率很高,需要熟练掌握.类似例子还有
\begin{equation}
\int \frac{1}{ax+b} \dd{x} = \frac{1}{a}\ln (ax + b)+C  \qquad
\int \E^{ax + b} \dd{x} = \frac{1}{a} \E^{ax + b}+C
\end{equation}
等等.

\subsection{积分变量替换}

换元积分法的过程在形式上可以记为 (见微分\upref{Diff})
\begin{equation}\label{IntCV_eq1}
\ali{
\int f[u(x)]u'(x) \dd{x} &= \int f[u(x)] \dd{[u(x)]} = \int f(u) \dd{u} = F(u)+C\\
&= F[u(x)]+C
}\end{equation}
该式把积分变量由 $x$ 换成了 $u$, 故称为\textbf{换元积分法}.

\subsection{第二类换元积分法}
第二类换元积分法从某种意义上和第一类换元积分相反.若要对一个函数积分,先把它的自变量看做另一个变量的函数,再逆向使用\autoref{IntCV_eq1}, 即可化简积分.
\begin{equation}\label{IntCV_eq6}
\int f(x) \dd{x} = \int f[x(t)] \dd{[x(t)]} = \int f[x(t)]x'(t) \dd{t}
\end{equation}
这个积分看似复杂了,但是如果 $x(t)$ 选取适当,反而可以使计算化简.

\begin{exam}{}
计算
\begin{equation}
\int \frac{\dd{x}}{\sqrt{1-x^2}}
\end{equation}
显然 $x \in ( - 1,1)$, 选取 $x(t)=\sin t$. 替换后的定义域为 $t \in ( -\pi/2,\pi/2)$, 函数单调递增 \footnote{注意任何积分换元法中的两个变量必须有一一对应的关系,即相互的函数关系在定义域内都为单调.}.上面积分变为
\begin{equation}
\int \frac{\dd{x}}{\sqrt{1-x^2}}  = \int \frac{\dd{\sin t}}{\sqrt{1-\sin^2 t}} = \int \frac{\cos t\dd{t}}{\cos t}  = \int 1\dd{t}  = t + C = \arcsin x + C
\end{equation}
验证: 根据反函数求导法则\upref{InvDer}
\begin{equation}
\arcsin'x = \frac{1}{\cos(\arcsin x)} = \frac{1}{\sqrt {1 - {x^2}} }
\end{equation}
\end{exam}




