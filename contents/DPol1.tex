% 极坐标系中单位矢量的偏导

\pentry{极坐标系\upref{Polar}, 矢量的偏导\upref{DerV}}
与直角坐标系不同的是, 极坐标系中的 $\uvec r$ 与 $\uvec \theta $ 都是坐标的函数,即 $\uvec r = \uvec r (r,\theta)$, $\uvec \theta  = \uvec \theta (r,\theta)$,它们对坐标的偏导如下
\begin{equation}\label{DPol1_eq1}
\leftgroup{
\pdv{\uvec r}{r} &= 0\\
\pdv{\uvec r}{\theta} &= \uvec \theta }
\qquad
\leftgroup{
\pdv{\uvec \theta}{r} &= 0\\
\pdv{\uvec \theta}{\theta} &=  - \uvec r }
\end{equation}
这是容易理解的,若一个单位矢量绕着它的起点逆时针转动,那么它的终点的速度的方向必然是它本身逆时针旋转 90 度的方向, 而大小等于矢量模长乘以角速度.
\subsection{证明}
如果令极轴方向的单位矢量为 $\uvec x$, 令其逆时针旋转 $\pi/2$ 的矢量为 $\uvec y$, 则
\begin{gather}
\uvec r = \cos \theta \,\uvec x + \sin \theta \,\uvec y\\
\uvec \theta  = \cos (\theta +\pi/2)\,\uvec x + \sin (\theta +\pi/2)\,\uvec y
=  - \sin \theta \,\uvec x + \cos \theta \,\uvec y
\end{gather} 
所以
\begin{equation}
\leftgroup{
\pdv{\uvec r}{r} &= 0\\
\pdv{\uvec r}{\theta} &=  - \sin \theta \,\uvec x + \cos \theta \uvec y = \,\uvec \theta }
\quad
\leftgroup{
\pdv{\uvec \theta}{r} &= 0\\
\pdv{\uvec \theta}{\theta} &=  - \cos \theta \,\uvec x - \sin \theta \,\uvec y =  - \uvec r
}\end{equation}  
事实上,由于 $\uvec r$ 与 $\uvec \theta $ 都只是 $\theta$ 的函数,也可以把偏导符号改成导数符号
 \begin{equation}\label{DPol1_eq5}
\dv{\uvec \theta}{\theta} =  - \uvec r
\qquad
\dv{\uvec r}{\theta} = \uvec \theta 
\end{equation}
 
 
 
 
 
 
 
 
 
 
 
 
 
 
 
 
 
 
 
 
 
 
 
 
 
 