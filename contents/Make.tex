% Makefile 笔记

% 摘自我的 GitHub/Notes/Programming/Fortran/makefile_笔记2.md

\begin{itemize}
\item 要系统地学习 make, 参考资料是 \link{GNU Make 文档}{https://www.gnu.org/software/make/manual/}, 提供网页版和 pdf 下载.
\item make 程序广义上是声明如何更新一系列文件, 以及他们的 dependencies. 当某个文件改变后, 任何依赖于它的文件都需要通过指定的规则 (rule) 来更新
\item 突然发现, rule 的 “:” 前面不可以有空格 ?
\item rule 的第一行用于声明什么文件取决于什么文件, 剩下的行声明用什么命令去更新.
\item “:” 后面的文件会按照顺序更新, 而且只会更新一次, 若以如果更新第二个文件的时候第一个文件被删掉也不会出错
\item “:” 后面可以有 phony target, 比如说 \lstinline|make clean| 中的 \lstinline|clean|
\item \lstinline|rm| 命令后面记得加 \lstinline|-f| 选项, 否则如果文件不存在就会出错导致 make 就会失败
\item 可以根据所有 rule 中的 target 和 dependency 画一个树状图, 如果某个点的文件不存在, 那么就会先运行生成它的 rule, 如果图中的任何一点更新了, 从这点到顶点的所有文件都要更新. 具体的规则是, 如果一个 target 的任何一个 dependency 比它要新, 那么 target 就要重新编译.
\item “goal :” 可以声明 make 的终极目标, 如
\begin{lstlisting}
goal: file1 file2 ...
	command1
	command2
\end{lstlisting}
\item 如果没有 goal 的话, default goal 就是第一个开头不为 “.” 的 target. 剩下的 rule 的顺序应该可以随意
任何一个被依赖的文件改变了, 或者它们依赖的文件改变了, 就会执行 command. make 并不知道 command 的含义. 只是把它传给 shell 来执行.
\item implicit rules 大概就是可以仅声明 “A: B”, 由自定义的命令或者 make 默认的命令来生成 “A”. implicit rule 中的 dependency 是至少有这些 dependency, 而不是只能有这些 dependency.
\item “MAKEFLAGS = -r” 大概是用来取消默认 implicit rules, 包括所有后缀名识别
\item 老的 suffix rule 和 implicit rules 的功能差不多, 现在已经过时了, 应该用 implicit rule. 一个例子如
\begin{lstlisting}
  .f90.o:
	gfortran -c $<
\end{lstlisting}
其中 \lstinline|$<| 是 auto variable 中的一个 (见 10.5.3 Automatic Variables), 在执行的时候被替换成 \lstinline|:| 右边的第一个 dependency. 现在如果有
\begin{lstlisting}
file1.o: file1.f90 file2.o file3.o
\end{lstlisting}
那么应该会执行 \lstinline|gfortran -c file1.f90|. 另外, 如果 “file2.o” 或 “file3.o” 被更新了, 这条命令应该也会再执行一次.
\item \lstinline|$^| 列出所有的 prerequisites (“:” 右边的内容)
\item \lstinline|$(shell ...)| 可以执行 shell 命令, 如 \lstinline|$(shell echo *.f90)| 可以在当前位置列出所有 “.f90” 文件.
\item \lstinline|$@| 大概就是 target file (如果 “:” 左边只有一个文件的话)
\end{itemize}
