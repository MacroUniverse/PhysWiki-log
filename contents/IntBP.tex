% 分部积分法
\pentry{不定积分\upref{Int}, 牛顿莱布尼兹公式\upref{NLeib}}

\subsection{结论}
\begin{equation}\label{IntBP_eq1}
\int {F(x)g(x)\D x}  = F(x)G(x) - \int {f(x)G(x)\D x}
\end{equation}
\begin{equation}
\int_a^b {F(x)g(x)\D x}  = \left[ {F(x)G(x)} \right]_a^b - \int_a^b {f(x)G(x)\D x}
\end{equation}
\begin{equation}
\int {f(x)g(x)\D x}  = f(x){g^{[1]}}(x) - {f^{(1)}}(x){g^{[2]}}(x) + \int {{f^{(2)}}(x){g^{[2]}}(x)\D x}
\end{equation}

\subsection{推导}
令$f(x) = F'(x)$,  $g(x) = G'(x)$, 根据乘法的求导公式% 链接未完成
\begin{equation}
[F(x)G(x)]' = f(x)G(x) + F(x)g(x)
\end{equation}
即
\begin{equation}\label{IntBP_eq2}
F(x)g(x) = [F(x)G(x)]' - f(x)G(x)
\end{equation}
两边不定积分(积分常数可任取)得
\begin{equation}\label{IntBP_eq6}
\int {F(x)g(x)\D x}  = F(x)G(x) - \int {f(x)G(x)\D x}
\end{equation}
所以如果被积函数等于两个函数的乘积,则可选择其中一个($F$ )为“求导项”进行求导,另一个( $g$)为“积分项”进行不定积分(积分常数可任取),然后代入该式即可.

若要计算定积分,既可以先计算不定积分然后使用牛顿-莱布尼兹公式,也可以直接对\autoref{IntBP_eq2} 进行定积分得
\begin{equation}\label{IntBP_eq4}
\int_a^b {F(x)g(x)\D x}  = \left[ {F(x)G(x)} \right]_a^b - \int_a^b {f(x)G(x)\D x}
\end{equation}

\begin{exam}{求 $x{\E^{ - x}}$ 的不定积分和从 $0$ 到 $+\infty$ 的定积分}
令 $x$ 项为“求导项”,导数为1, ${\E^{ - x}}$ 为“积分项”,积分为 $- {\E^{ - x}}$.代入\autoref{IntBP_eq6} 得
\begin{equation}
\int {x{\E^{ - x}}\D x}  = x( - {\E^{ - x}}) - \int {1 \cross ( - {\E^{ - x}})\D x}  =  - x{\E^{ - x}} - {\E^{ - x}} + C
\end{equation}
如果直接计算定积分,把“求导项”和“积分项”直接代入\autoref{IntBP_eq4} 得
\begin{equation}
\int_0^{ + \infty } {x{\E^{ - x}}\D x}  = \left[ {x( - {\E^{ - x}})} \right]_0^{ + \infty } - \int_0^{ + \infty } {1 \cross ( - {\E^{ - x}})\D x}  = 0 - \left[ {{\E^{ - x}}} \right]_0^{ + \infty } = 1
\end{equation}
\end{exam}

\subsection{连续分布积分}
由于 $f(x)$ 的 $n$ 次导数可以记为 ${f^{(n)}}(x)$,不妨把 $g(x)$ 的 $n$ 次不定积分( $n$ 个积分常数任取)记为\footnote{这是我自己发明的符号} ${g^{[n]}}(x)$.则分部积分\autoref{IntBP_eq6} 可记为
\begin{equation}
\int {f(x)g(x)\D x}  = f(x){g^{[1]}}(x) - \int {{f^{(1)}}(x){g^{[1]}}(x)\D x}
\end{equation}
再对第二项利用分部积分,仍然将 ${f^{(1)}}$ 作为“求导项”, ${g^{[1]}}$ 作为“积分项”,得
\begin{equation}
\int {f(x)g(x)\D x}  = f(x){g^{[1]}}(x) - {f^{(1)}}(x){g^{[2]}}(x) + \int {{f^{(2)}}(x){g^{[2]}}(x)\D x}
\end{equation}
再把 $f^{(2)}$ 作为“求导项”, $g^{[2]}$ 作为“积分项”,分布积分得
\begin{equation}
\int {f(x)g(x)\D x}  = f(x){g^{[1]}}(x) - {f^{(1)}}(x){g^{[2]}}(x) + {f^{(2)}}(x){g^{[3]}}(x) - \int {{f^{(3)}}(x){g^{[3]}}(x)\D x}
\end{equation}
可以发现若要使用 $N$ 次分部积分,第 $i\le N$ 项等于第 $i-1$ 项中的“求导项”求导,“积分项”积分,再取相反数,最后不定积分中只需把“求导项”额外求一次导即可. 

% 未完成: 例二: 用连续分布积分来做 例1