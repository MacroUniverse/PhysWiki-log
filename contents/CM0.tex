% 经典力学
\pentry{物理理论\upref{MecThe}}

\subsection{经典力学}
经典力学讲的是若干(宏观低速)的粒子(质点)受若干力以后的运动情况. 牛顿三定律可以作为经典力学的公设,可以将其想象成一个懂经典力学的计算机,只需输入初始时刻所有粒子的状态(位置和速度/动量),以及每个粒子的受力/势能函数,就可以得到接下来每个粒子的运动方式(位置关于时间的函数). 我们在研究某个问题时,往往把讨论范围内的粒子叫做系统.

\subsection{刚体和流体}
质点的模型忽略了物体的实际大小,如果要考虑一个大小不可忽略的物体的转动,以及若干个大小不可忽略的物体之间的相互作用该怎么办?事实上,这些物体也是由更小的原子(注:经典力学不能准确描述原子的结构和运动,要使用量子力学,但我们这里讨论的是组成物体后的宏观运动,所以即使用经典力学来描述原子得到的宏观运动也是正确的),原子之间的作用力使它们相对位置很难发生变化,例如可以想象每个原子和与它相邻的原子都以弹簧相连. 这样,我们就可以用经典力学来描述有形状有质量分布的物体了. 在力学中我们经常使用一种理想模型叫做刚体,即假设这些弹簧的劲度系数为无限大,或者把他们看成不可伸缩的杆或棒. 所以如果我们近似认为一个物体无限硬,忽略其自身的形变,我们就可以用刚体的模型来描述它. 另一种物质的形态,例如水,同样可以用理想化的力学模型来描述,我们把它叫做流体,流体比较复杂,我们先不讲.

\subsection{力学的难点}
经典力学中最简单的问题是,已知初始时刻若干粒子的运动状态,以及接下来每一个粒子的受力[u]关于时间[/u]的函数. 根据牛顿第二定律,可以求出每个粒子任意时刻的加速度(求曲线下面的面积,这叫做定积分),进而求出速度(同样是定积分),和位置关于时间的函数.

但我们几乎从来不会遇到这么理想化的条件. 即使在单摆这样简单的模型中,我们也不可能事先就知道粒子受力关于时间的变化情况. 因为受力分析得到的是力\bb{关于位置}的变化. 而粒子的位置变化方式又取决于受力. 这似乎有些循环论证的味道,乍看起来,解出这个问题似乎是不可能的.
(未完成:提一提微分方程,以及把时间分成小块的思想)

\subsection{分析力学}
分析力学的数学形式比牛顿定律复杂,但和牛顿力学是等效的,即任何分析力学计算的问题用牛顿定律同样能计算并得到完全相同的结果,反之亦然. 分析力学常见的形式有两种,一种是拉格朗日力学,使用拉格朗日方程,另一种叫哈密顿力学,使用哈密顿方程. 量子力学就是在哈密顿力学的基础上建立起来的,其重要性可见一斑.(未完成:提一下位置和动量的重要性,广义坐标和广义动量)

相比与牛顿力学,分析力学的优势主要有点,一是当计算的系统越来越复杂的时候(例如蒸汽机等复杂的机械结构)分析力学一般能更快捷地列出微分方程组,而牛顿力学所需要的受力分析会变得非常复杂. 另一方面,分析力学能使人站在更高的高度看问题(例如可以分析出系统中的守恒量).

为什么复杂的数学形式反而在一些情况下使列方程变得简单呢?如果把牛顿定律比作计算机,那么分析力学就相当于一个更智能的计算机,对于系统中的某些物体,无需告诉计算机它的受力只需告诉计算机它运动的约束即可. 约束简单来说就是怎样的运动是不可能的:例如单摆中的质点就“不可能”沿着绳的方向运动,两个咬合的齿轮“不可能”一个转动一个不转. 用约束条件代替受力分析,在适当的时候可以大大提高列方程的效率.

\subsection{狭义相对论}
同样讨论质点组成的系统受力后的运动情况. 牛顿三定律只适用于宏观低速弱引力场条件,如果粒子的速度相对光速不可忽略,那就需要使用“更精确”的牛顿第二定律,且惯性参考系切换也变得更复杂(洛伦兹变换). 当速度越低,狭义相对的计算结果越接近牛顿力学. 相对论同样并不适用微观.

\subsection{广义相对论}
加上强引力场和非惯性系.

\subsection{势能曲线}
在更高级的物理理论中我们往往不讨论力,而是势能. 一维运动(直线运动)中如果受力只是关于位置的函数(如弹簧振子),那么这个力(可以称为力场)就叫保守力(高维的情况下有更复杂的定义). 对于保守力,我们可以计算出每个位置的势能(势能函数/势能曲线). 一维直线运动情况下,坐标为 $x$,势能函数可以记为 $V(x)$. 某个位置力的大小就是曲线的斜率,方向就是曲线下降的方向. 粒子沿受力方向运动,动能增加,势能减小,总能量/机械能(动能加势能)不变.

作为一个比喻,想象高低不平的“直”轨道上的小车,我们可以让轨道的高度为 $h(x) = V(x)/(mg)$(高度与势能成正比,$x$ 处小车具有重力势能 $V(x)$),这样具有一定总能量 $E$ 的小车就会在轨道上运动. $V(x) > E$ 的位置小车无法达到,会原路返回. 如果小车开始时在某 $E \leq V(x)$ 的区间运动,那么它将一直在该区间往返运动(忽略摩擦),某点的速度 $v(x) = \sqrt{2[E - V(x)]/m}$(推导一下). 注意这只是一种比喻,我们讨论的是沿直线运动的粒子(没有重力,只有势能 $V(x)$,无论它是什么力). $h(x)$ 越小,$g$ 越大,二者就越接近(运动更接近直线). 如何从图中看出粒子在各个可能位置的运动情况?先画代表总能量的横线,横线与势能曲线的交点就是拐弯的地方,横线的高度减掉某点势能曲线的高度就是动能 $mv^2/2$.

\subsection{简谐振子}
(什么是简谐运动,势能曲线如何?频率/周期与什么有关?)

\subsection{单摆}
为什么小角的单摆和弹簧振子的势能曲线类似(圆的底部与抛物线很接近)?导致质点类似简谐振动. 为什么需要小角?(上文的 $h(x)$ 越小,二者越接近)

\subsection{量子力学中的常见势能曲线}
量子力学中的公式不会出现力,只会出现势能. 常见的势能(三角势垒见下文,方势垒,方势阱,边界处斜率无穷大怎么办?类比碰撞时的冲力. 初始动能(总能量),大于,小于最大势能的时候分别会沿原方向运动,反弹. 更理想的情况:无限深势阱(小球在两面墙之间无限反弹),delta 势垒/势阱(受到一个微小扰动,相当于无限窄的方势垒/势阱),对经典粒子运动没有任何影响(物理上不存在,但是作为模型有计算简单的优点).

\subsection{动量}
动量描述了了物体运动的物体的惯性. 惯性可以简单理解为让一个运动的粒子停下来有多困难. 这个困难程度可以用力乘以时间表示. 例如一个在光滑水平面有动量为 $mv$ 的箱子,要使它停下来,人就要沿运动反方向以恒力 $F$ 推箱子, 若一段时间 $t$ 后箱子停了下来,那么有 $mv = Ft$. 同理,如果要使箱子从静止达到动量 $mv$, 就需要以恒力 $F$ 作用在箱子上一段时间 $t$, 使 $mv = Ft$. 所以动量是力在时间上的累加. 相比动量,能量是力在空间上的累加. 同样是箱子的例子, 如果想让静止的箱子达到动能 $mv^2/2$, 就需要用恒力 $F$ 推箱子, 在力的方向移动距离 $s$,使 $Fs = mv^2/2$.

\subsection{刚体}
如果我们忽略一个物体的形变, 假设它为无限硬, 就可以称之为刚体. 刚体可以看做由许多质点组成,我们假设这些质点之间有某种相互作用力(例如想象他们之间以不可伸缩的轻杆相连),使刚体内任意两个质点之间的距离总保持不变.

\subsection{刚体的旋转}
刚体绕某个固定轴的转动与质点的直线运动存在许多相似之处. 旋转的角度对应(可以类比)质点直线运动的位移,角速度对应质点的速度,角加速度对应质点的加速度. 力矩对应力,力矩与角加速度的关系与力与加速度的关系相同. 力矩乘以旋转角度等于做功(力乘以位移等于做功),力矩在时间上的累加等于角动量(力在时间上的累加等于动量). 转动惯量乘以角速度等于角动量(质量乘以速度等于动量).

\subsection{角动量}

\subsection{万有引力}

\subsection{天体运动}
