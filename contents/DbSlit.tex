% 双缝干涉
% 波动性|叠加|干涉条纹|双缝干涉|球面波|双曲面

解释了光的波动性.

光通过狭缝之后, 会变为球面波. 两个球面波叠加, 就得到了干涉条纹.

叠加就是数值相加.

严格来说, 需要 $\abs{l_1 - l_2} = n\lambda$ 而满足这个条件的点组成许多双曲面, 双曲面与屏幕相交的曲线就是干涉条纹的位置. 当两个点的距离远小于它们到屏幕的距离时, 干涉条纹就几乎是等距离的直线.
