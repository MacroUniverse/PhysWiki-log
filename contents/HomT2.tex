% 可缩空间
%同伦|homotopy|映射|形变|收缩|连续|连续变换|拓扑|topology

\pentry{同伦\upref{HomT1}}
%未完待续

\subsection{收缩核映射}

首先我们要引入三个依次增强的概念.第一个是\textbf{保核收缩映射},就是将拓扑空间映射到自身的子集(称为核)上,同时还保证核的元素都是映射的不动点;第二个是\textbf{形变收缩映射},它是一种保核收缩映射,但是多了一个条件,要求拓扑空间中各点都连续地移动到核中;第三个是\textbf{强形变收缩映射},它是一种形变收缩映射,同样多了一个条件,要求移动过程中时时刻刻保持核中的元素是不动点.

下面,我们严格描述这三个概念.

\begin{definition}{保核收缩映射}
设拓扑空间 $X$ 及其子集 $A$.如果存在一个\textbf{连续}映射 $f:X\rightarrow X$,使得 $f(X)=A$,并且 $\forall a\in A, f(a)=a$,那么称 $f$ 是 $X$ 到 $A$ 的一个\textbf{保核收缩映射},同时称 $A$ 是 $f$ 的\textbf{收缩核}.保核收缩映射也可以简称\textbf{保核收缩}.
\end{definition}

\begin{definition}{形变收缩映射}
设拓扑空间 $X$ 及其子集 $A$.如果 $f$ 是 $X$ 到 $A$ 的一个保核收缩,且 $f\cong i_X$,其中 $i_X$ 是 $X\rightarrow X$ 的恒等映射,那么称 $f$ 是 $X$ 到 $A$ 的一个\textbf{形变收缩映射},也可简称\textbf{形变收缩}.称 $A$ 是 $X$ 的\textbf{形变收缩核}.
\end{definition}

从\textbf{映射的同伦和空间的同伦}\upref{HomT1}词条的\autoref{HomT1_exe1} 可知,同伦意味着映射的每个点都画出一条道路,或者说连续移动.因此,如果 $X$ 有一个到 $A$ 的形变收缩,那么 $X$ 中的所有点都可以同时、连续地移动到 $A$ 当中.

形变收缩是比保核收缩更强的概念,这就意味着存在某个映射,它是保核收缩,但不是形变收缩.

\begin{example}{}
设 $X$ 是由平面上的一个开圆盘 $A$ 和一个与 $A$ 不相交的开圆环 $B$ 取并集而得的.简单粗暴地设映射 $f:X\rightarrow X$ 满足 $\forall a\in A, b\in B$,存在一个 $a_0\in A$,使得 $f(a)=a, f(b)=a_0$.那么 $f$ 是一个保核收缩,但不是形变收缩——很明显,$B$ 中的点无法通过一条道路连续地移动到 $a_0\in A$ 上.
\end{example}

\begin{definition}{强形变收缩映射}\label{HomT2_def1}
设拓扑空间 $X$ 及其子集 $A$.如果 $f$ 是 $X$ 到 $A$ 的一个形变收缩,$H:X\times I\rightarrow X$ 是从 $i_X$ 到 $f$ 的同伦,且满足 $\forall t\in I, a\in A$,有 $H(a, t)=a$,那么称 $f$ 是 $X$ 到 $A$ 的一个\textbf{强形变收缩映射},简称\textbf{强形变收缩}.称 $A$ 是 $X$ 的强形变收缩核.
\end{definition}

强形变收缩要求在 $X$ 的各点连续移动到 $A$ 中时,$A$ 中的点本身保持不动.

从定义可以看出来,形变收缩还意味着 $X$ 和 $A$ 是同伦的空间.这启发我们对同伦的认识:同伦的空间是能够通过连续变化而变成彼此的.和同胚不同的是,同伦只要求了连续变化,并不要求双射,这就使得我们可以将多个点映射到同一个点,甚至将拓扑空间降维.弦论中,形变收缩的性质对弦的世界环性质有重要意义.

\subsection{可缩空间}
可缩空间这一名称很直白,就是指可以被连续压缩为一个点的空间.

\begin{definition}{零伦和可缩空间}
设拓扑空间 $X$ 及其上一点 $x_0$.如果存在一个从 $X$ 到 $x_0$ 的形变收缩 $f$,那么称 $X$ 是一个\textbf{可缩空间},称 $f$ 是一个\textbf{零伦}的映射.
\end{definition}

\begin{example}{锥空间}
锥空间 $\widetilde{C}X$ 的定义见\textbf{商拓扑}\upref{Topo7}的“锥空间”小节.对于任意的拓扑空间 $X$,其锥空间 $\widetilde{C}X$ 都是可缩的.事实上,我们可以直接写出这个收缩 $H:CX\times I\rightarrow\overline{(x, 0)}$ 如下:

$\forall x\in X, t_n\in I, $ 有 $H(\overline{(x, t_1)}, t_2)=\overline{(x, t_1t_2)}$.

也可以想象成,锥空间逐渐向顶点压缩.
\end{example}


