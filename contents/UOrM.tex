% 单位正交阵

\pentry{正交归一基底\upref{OrNrB}, 矩阵\upref{Mat}}

若一个实数方阵的每一列的模长都等于 $1$, 且任意两列都正交, 那么这个矩阵就是一个\bb{单位正交阵}. 若 $\mat U$ 为单位正交阵, 则其矩阵元满足
\begin{equation}
\sum_k U_{ki} U_{kj} = \delta_{ij}
\end{equation}
其中 $\delta_{ij}$ 是克罗内克 $\delta$ 函数. 所以若把 $N$ 阶单位正交阵的每一列看做 $N$ 维空间中的一个单位矢量的直角坐标, 那么这些单位矢量就组成该空间的一组正交归一基底.

\subsection{行矢量的性质}
单位正交阵的一个性质是其所有的行也同样满足正交归一的条件
\begin{equation}
\sum_k U_{ik} U_{jk} = \delta_{ij}
\end{equation}
所以我们也可以把定义中的所有“列”字换为“行”字.

现在我们以二维的单位正交阵 $\mat U$ 为例来证明. 我们把它的两列分别记为单位矢量 $\uvec u$ 和 $\uvec v$, 则它的矩阵元可以用 $\uvec u$, $\uvec v$ 和正交归一基底 $\uvec x = (1, 0)\Tr$, $\uvec y = (0, 1)\Tr$ 的点乘来表示
\begin{equation}\label{UOrM_eq3}
\mat U = \pmat{\uvec u \vdot \uvec x & \uvec v \vdot \uvec x \\
\uvec u \vdot \uvec y & \uvec v \vdot \uvec y}
\end{equation}
由于 $\uvec u$, $\uvec v$ 和 $\uvec x$, $\uvec y$ 一样也是正交归一基底, 我们可以把矩阵 $\mat U$ 的两行分别看做 $\uvec x$, $\uvec y$ 在 $\uvec u$, $\uvec v$ 基底上的坐标, 所以 $\mat U$ 的两行也分别正交归一. 证毕.

\subsection{逆矩阵与转置矩阵}
单位正交阵的另一个性质就是其逆矩阵等于其转置矩阵.还是以上面的 $\mat U$ 为例来证明, 我们先来看线性变换
\begin{equation}
\pmat{x\\y} = \pmat{\uvec u \vdot \uvec x & \uvec v \vdot \uvec x \\
\uvec u \vdot \uvec y & \uvec v \vdot \uvec y} \pmat{u\\v}
\end{equation}
事实上, 该变换就是把一个矢量以 $\uvec u$ 和 $\uvec v$ 作为基底的坐标 $(u, v)$ 变换为同一个矢量以 $\uvec x$ 和 $\uvec y$ 作为基底的坐标 $(x, y)$, 即把矢量 $u\uvec u + v\uvec v$ 分别在 $\uvec x$ 和 $\uvec y$ 上投影.
\begin{equation}
\leftgroup{x &= (u\uvec u + v\uvec v)\vdot\uvec x\\
y &= (u\uvec u + v\uvec v)\vdot\uvec y}
\end{equation}
显然, 该变换的逆变换为
\begin{equation}
\leftgroup{u &= (x\uvec x + y\uvec y)\vdot\uvec u\\
v &= (x\uvec x + y\uvec y)\vdot\uvec v}
\end{equation}
该变换所对应的矩阵就是 $\mat M$ 的逆矩阵
\begin{equation}
\mat M^{-1} = \pmat{\uvec x \vdot \uvec u & \uvec y \vdot \uvec u \\
\uvec x \vdot \uvec v & \uvec y \vdot \uvec v}
\end{equation}
对比上式和\autoref{UOrM_eq3} 可以发现 $\mat M$ 与 $\mat M^{-1}$ 互为转置矩阵. 证毕.




