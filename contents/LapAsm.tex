% 拉普拉斯方法

\pentry{极值\upref{DerMax}, 积分换元\upref{IntCV}, 高斯积分\upref{GsInt}, 渐近展开\upref{Asympt},  Gamma函数\upref{Gamma}}

在分析数学中,拉普拉斯方法是一种计算含参数积分的渐近展开式的办法. 所考察的积分一般具有如下形式:
\begin{equation}\label{LapAsm_eq1}
I(t)=\int_a^b \phi(x)\E^{t f(x)}dx,
\end{equation}
其中$\lambda$是正的实参数. 要考察当$t\to+\infty$时积分的行为. 拉普拉斯方法背后的想法很简单: 如果函数$f$在某点处达到极大值, 那么当$t$很大时, 只有极大值点附近的贡献才比较可观, 其余部分相比起来都要小得多.

\subsection{Watson 引理}
拉普拉斯方法基于 Watson 引理. 它本身也是很有用的.

\begin{lemma}{Watson 引理}
设$\alpha>0$, 函数$\phi(x)$在区间$[0,b]$上连续, 而且$x\to0$时有渐近展开
$$
\phi(x)\simeq c_0+c_1x+c_2x^2+...
$$
则当$t\to+\infty$时, 含参数的积分
$$
I(t)=\int_0^b \phi(x)\E^{-t x^\alpha}dx
$$
有渐近展开式
$$
I(t)\simeq\sum_{n=0}^\infty \frac{c_n}{\alpha}\Gamma\left(\frac{n+1}{\alpha}\right)t^{-(n+1)/\alpha}.
$$
\end{lemma}

证明是直接的计算. 固定一个$n$. 按照渐近展开的定义, 存在$\delta>0$使得$0\leq x<\delta$时$\phi(x)-(c_0+...+c_nx^n)=O(x^{n+1})$. 远离$0$的$x$对积分的贡献是可以忽略的:
$$
\left|\int_\delta^b\phi(x)\E^{-tx^\alpha}dx\right|
\leq\max_{x\in[0,b]}|\phi(x)|\E^{-t\delta^\alpha}.
$$
而在区间$[0,\delta]$上, 通过换元可得
$$
\begin{aligned}
\int_0^\delta x^k\E^{-tx^\alpha}dx
&=\frac{t^{-(k+1)/\alpha}}{\alpha}\int_0^{t\delta^\alpha}y^{-(k+1)/\alpha-1}\E^{-y}dy\\
&=\frac{t^{-(k+1)/\alpha}}{\alpha}\Gamma\left(\frac{k+1}{\alpha}\right)+O(\E^{-t\delta^\alpha/2}).
\end{aligned}
$$
于是就有
$$
\begin{aligned}
I(t)
&=\int_0^\delta\phi(x)\E^{-tx^\alpha}dx+O(\E^{-t\delta^\alpha})\\
&=\sum_{k=0}^n\frac{c_n}{\alpha}\Gamma\left(\frac{k+1}{\alpha}\right)t^{-(k+1)/\alpha}+O(t^{-(n+2)/\alpha}).
\end{aligned}
$$
这就是所求的结果.

\subsection{拉普拉斯方法}
拉普拉斯方法是将 Watson 引理用于\autoref{LapAsm_eq1} 得到的. 作出如下假设:

\begin{enumerate}
\item 函数$f,\phi$都是光滑函数.

\item 函数$f(x)$在积分区间内部仅有一个严格的极大值点$x_0\in(a,b)$, 而且$x_0$是整个区间上的最大值点.

\item \autoref{LapAsm_eq1} 对于$t=t_0$绝对收敛.
\end{enumerate}

于是$f'(x_0)=0$. 不妨设$f''(x_0)<0$. 从泰勒公式知道, 有一$\delta>0$使得在区间$(x_0-\delta,x_0+\delta)$上, 
$$
f(x)=f(x_0)+\frac{1}{2}f''(x_0)(x-x_0)^2+(x-x_0)^3h(x),
$$
其中函数$h$有界. 当$t$很大时, 区间$(x_0-\delta,x_0+\delta)$之外的积分的贡献比$t$的任何负幂衰减得都快:根据假定 2,当$|x-x_0|>\delta$时有一$c>0$使得$f(x)<f(x_0)-c$, 从而区间$(x_0-\delta,x_0+\delta)$之外的积分估计为
$$
\begin{aligned}
\left|\int_{|x-x_0|>\delta}\phi(x)\E^{tf(x)}dx\right|
&\leq \E^{-c(t-t_0)}\int_a^b|\phi(x)|\E^{t_0f(x)}dx\\
&=O(\E^{-ct}).
\end{aligned}
$$
而在区间$(x_0-\delta,x_0+\delta)$上, 可作如下换元: $y^2=f(x_0)-f(x)$, 即
$$
y=\sqrt{\frac{-f''(x_0)}{2}}(x-x_0)\left(1-\frac{2}{f''(x_0)}(x-x_0)h(x)\right)^{1/2}.
$$
命$\beta_{\pm}=\pm\sqrt{f(x_0)-f(x_0\pm\delta)}$, 又设有泰勒展开
\begin{equation}\label{LapAsm_eq2}
\psi(y):=\phi(x(y))x'(y)\simeq c_0+c_1y+...,y\to 0
\end{equation}
(这里显然有$c_0=\phi(x_0)\sqrt{-2/f''(x_0)}$), 则
$$
\begin{aligned}
\int_{x_0-\delta}^{x_0+\delta}\phi(x)\E^{tf(x)}dx
&=\E^{tf(x_0)}\int_{\beta_{-}}^{\beta_+}\psi(y)\E^{-ty^2}dy\\
&\simeq\E^{tf(x_0)}\int_{0}^{c}[\psi(y)+\psi(-y)]\E^{-ty^2}dy.
\end{aligned}
$$
应用 Watson 引理, 最后终于得到\autoref{LapAsm_eq1} 的渐近展开:
\begin{equation}\label{LapAsm_eq3}
I(t)\simeq \E^{tf(x_0)}\sum_{n=0}^\infty c_{2n}\Gamma\left(n+\frac{1}{2}\right)t^{-n-1/2},
\end{equation}
其中$c_n$由\autoref{LapAsm_eq2} 给出. 特别地, 由于$\Gamma(1/2)=\sqrt{\pi}$, \autoref{LapAsm_eq1} 渐近公式的首项是
$$
I(t)=\E^{tf(x_0)}\left(\phi(x_0)\sqrt{-\frac{2\pi}{tf''(x_0)}}+O(t^{-3/2})\right).
$$

\subsection{斯特林公式}
拉普拉斯方法最基本的应用就是导出$\Gamma$函数的\textbf{斯特林公式(Stirling formula)}. 按照定义,
$$
\Gamma(t+1)
=\int_0^\infty \E^{-x}x^tdx
=t^{t+1}\int_0^\infty \E^{-t(x-\log x)}dx.
$$
最后一步用到换元$x\to tx$. 最后这个积分刚好符合上一小节所要求的三条假设:这里$\phi(x)\equiv1$,$f(x)=-(x-\log x)$. 于是$x=1$是$f(x)$唯一的极大值点,同时也是唯一的最大值点.代入\autoref{LapAsm_eq3}, 就得到渐近公式
$$
\Gamma(t+1)\simeq\sqrt{2\pi t}\left(\frac{t}{\E}\right)^t.
$$
\autoref{LapAsm_eq3} 还给出了更精细的渐近级数:
$$
  \Gamma(t+1)
  =\sqrt{2\pi t}\left({t\over \E}\right)^t
  \left(
   1
   +{1\over12t}
   +{1\over288t^2}
   -{139\over51840t^3}
   -{571\over2488320t^4}
   + \cdots
  \right).
$$
毫无疑问这是一个可以对大的$n$来近似计算$n!$的公式.
