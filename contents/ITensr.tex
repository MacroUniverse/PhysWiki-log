% 转动惯量张量
% 刚体|惯量张量|坐标系基底变换
% 所有连接未完成

刚体的角动量等于\bb{惯性张量} $\mat I$ 乘以瞬时角速度矢量 $\vec \omega$
\begin{equation}
\vec L = \mat I \vec \omega
\end{equation}
其中\bb{惯性张量} $\mat I$ 是一个 3 维方阵, 其阵元一般记为
\begin{equation}
\ten I = \begin{pmatrix}
I_{xx}& I_{xy}& I_{xz} \\
I_{yx}& I_{yy}& I_{yz} \\
I_{zx}& I_{zy}& I_{zz}
\end{pmatrix}
\end{equation}
如果将 $x, y, z$ 分别记为 $x_1, x_2, x_3$, 则 $\mat I$ 的矩阵元为
\begin{equation}
I_{ij} = \delta_{i, j} \int r^2 \rho(\vec r)\dd{V} - \int x_i x_j \rho(\vec r)\dd{V} \qquad (i, j = 1, 2, 3)
\end{equation}

\subsection{推导}
\begin{equation}
\vec L = \sum_i m_i \vec r_i \vec v_i = \sum_i m_i \vec r_i \cross (\vec \omega \cross \vec r_i) = \sum_i m_i r_i^2 \vec \omega - \sum_i m_i (\vec \omega \vdot \vec r_i) \vec r_i
\end{equation}
写成分量的形式, 并将求和表示为密度 $\rho$ 的积分得
\begin{equation}
\pmat{L_x\\ L_y\\ L_z} = \int \rho r^2 \pmat{\omega_x\\ \omega_y\\ \omega_z} \dd{V} - \int \rho
\begin{pmatrix}
xx & xy & xz\\
yx & yy & yz\\
zx & zy & zz
\end{pmatrix}
\pmat{\omega_x\\ \omega_y\\ \omega_z} \dd{V}
\end{equation}

\subsection{坐标系变换}
注意惯性张量矩阵与坐标系的选取有关, 我们先建立一个与刚体相对静止的参考系叫做 body frame,% body frame 该怎么翻译……
一般我们选择 body frame 的原点在刚体质心处, 但刚体做任意转动时, body frame 并不是一个惯性系, 所以我们还有选一个\bb{实验室参考系(lab frame)}. 令 body frame 的矢量到 lab frame 的基底变换矩阵为 $R$, 记 body frame 和 lab frame 中的惯性张量分别为 $\mat I_0$ 和 $\mat I$, 则 lab frame 中的惯性张量等于
\begin{equation}
\mat I = \mat R \mat I_0 \mat R\Tr
\end{equation}

% 例子未完成, 一根细杆与转轴成一定倾角转动(计算杆需要提供的力矩)
% 例子未完成, 球体的惯性张量是对角的
% 例子未完成, 一个长方体的惯性张量
