% 质心 质心系

\pentry{体积分}%未完成

\subsection{质心的定义}

对质点系,令第 $i$ 个质点质量为 $m_i$,位置为 ${\vec r_i}$,总质量为 $M = \sum\limits_i {{m_i}}$, 则该质点系的\textbf{质心}定义为
\begin{equation}\label{CM_eq1}
{\vec r_c} = \frac{1}{M}\sum\limits_i {{m_i}{{\vec r}_i}} 
\end{equation}
对连续质量分布,令密度关于位置的函数为 $\rho (\vec r)$,总质量为密度的体积分 %(链接未完成)
\begin{equation}
M = \int {\rho (\vec r)\D V} 
\end{equation}
质心定义为
\begin{equation}
{\vec r_c} = \frac{1}{M}\int {\vec r\rho (\vec r)\D V}
\end{equation}
以下的讨论都对质点系进行,连续质量分布可看做由许多体积微元组成,也可看做质点系.

\subsection{质心的唯一性}
既然质心的定义取决于参考系(因为 $\vec r_i$ 取决于参考系),那么不同参考系中计算出的质心是否是空间中的同一点呢?我们只需要证明,在 $A$ 坐标系中得到的质心 $\vec r_{Ac}$ 与 $B$ 坐标系中得到的质心 $\vec r_{Bc}$ 满足关系
\begin{equation}\label{CM_eq4}
{\vec r_{Ac}} = {\vec r_{AB}} + {\vec r_{Bc}}
\end{equation}
首先根据定义
\begin{equation}
{\vec r_{Ac}} = \frac{1}{M}\sum\limits_i {{m_i}{{\vec r}_{Ai}}} \qquad {\vec r_{Bc}} = \frac{1}{M}\sum\limits_i {{m_i}{{\vec r}_{Bi}}} 
\end{equation}
由位矢的坐标系变换%(未完成: 确保有介绍)
,${\vec r_{Ai}} = {\vec r_{AB}} + {\vec r_{Bi}}$, 所以
\begin{equation}
{\vec r_{Ac}} = \frac{1}{M}\sum\limits_i {{m_i}({{\vec r}_{AB}} + {{\vec r}_{Bi}})}  = {\vec r_{AB}} + \frac{1}{M}\sum\limits_i {{m_i}{{\vec r}_{Bi}}}  = {\vec r_{AB}} + {\vec r_{Bc}}
\end{equation}
 
\subsection{质心系}
定义质点系的\textbf{质心系}为原点固定在质心上且没有转动的参考系(平动参考系).%链接未完成: 平动是相对的, 转动是绝对的.
根据质心的唯一性(\autoref{CM_eq4}),在质心系中计算质心(\autoref{CM_eq1}) 仍然落在原点,即
\begin{equation}\label{CM_eq7}
\sum\limits_i {{m_i}{{\vec r}_{ci}}} = \vec 0
\end{equation}
其中 $\vec r_{ci}$ 是质心系中质点 $i$ 的位矢.

注意质心系并不一定是惯性系,只有当合外力为零质心做匀速直线运动,质心系才是惯性系.在非惯性系中,每个质点受惯性力.

\subsection{质心系中总动量}
把\autoref{CM_eq7} 两边对时间求导,得
\begin{equation}\label{CM_eq8}
\sum\limits_i {{m_i}{{\vec v}_{ci}}} = \vec 0
\end{equation}
注意到等式左边恰好为质心系中质点系的总动量,所以我们得到质心系的一个重要特点,\textbf{质心系中总动量为零}.


 
