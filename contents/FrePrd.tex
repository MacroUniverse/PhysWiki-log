% 群的自由积
\pentry{自由群\upref{FreGrp}}

将自由群的概念推广,即可得到两个群之间的自由积的概念.

\subsection{自由积的构造}

给定两个群$G$和$H$,取集合$G\cup H$上的自由群$F(G\cup H)$,则$F(G\cup H)$的元素形如$x_1x_2\cdots x_k$的有限长字符串,其中$k$是某个正整数,各$x_i$都是$G\cup H$的元素.

在$F(G\cup H)$上定义一个等价关系:如果字符串$g_1g_2\cdots g_k$中各$g_i\in G$,那么令$g_1g_2\cdots g_k\sim g_1\cdot g_2\cdot\cdots\cdot g_k$,即把该字符串等同于各字母在群$G$中运算的结果;同样地把字母都是$H$中元素的字符串等同于这些元素在群$H$中的运算结果;把$G$和$H$的单位元等同于空词.比如说,在整数加法群$\mathbb{Z}$中,把字符串$123$等同于数字$1+2+3$所代表的字符串,即只有一个字母$6$的字符串.

这样一来,商群$F(G\cup H)/\sim$中的字符串就形如$g_1h_1g_2h_2\cdots g_kh_k$、$g_1h_1g_2h_2\cdots g_k$、$h_1g_1h_2g_2\cdots h_kg_k$或$h_1g_1h_2g_2\cdots h_k$的字符串,或者简单来说,有限长的$g$和$h$的交替字符串,其中$g, g_i\in G$,$h, h_i\in H$.

称商群$F(G\cup H)/\sim$为群$G$和群$H$的\textbf{自由积(free product)},记为$G*H$.

\subsection{共合积}
\begin{definition}{共合积}
设有三个拓扑空间$F, G, H$,且有同态$\phi:F\rightarrow G$和$\varphi:F\rightarrow H$,那么可以定义$G$和$H$关于$F$的共合积$G*_FH$如下:

$G*_FH$是$G*H$的商空间,$G*H/\sim$,其中等价关系为:$\forall f\in F, \phi(f)\sim\varphi(f)$.
\end{definition}

简单来说,$G*_FH$的元素依然是$H$和$G$中元素交替排列的字符串,但是在给定同态$\phi:F\rightarrow G$和$\varphi:F\rightarrow H$时,把所有$\phi(f)\sim\varphi(f)$都看成等价元素.


