% 物理量和单位转换
% 量纲|物理量|单位|转换

\pentry{几何矢量\upref{GVec}}

在物理中我们首先要区分两类变量, 一种是只有数值没有单位, 例如“甲的身高是乙的 $1.27$ 倍”. 另一种既有数值也有单位, 例如“甲的身高是 $1.8\Si{m}$”. 我们把前者叫做一个\textbf{数}, 把后者叫做\textbf{物理量}. 我们可以把物理量和一维几何矢量的部分性质做类比: 一个一维几何矢量本身也不能用一个数描述, 而是需要先选取一个矢量基底, 然后用一个数(即坐标)乘以这个基底表示这个矢量. 例如若甲选择的基底是乙的基底的两倍, 那么甲的坐标数值将会是乙的一半.

用公式表示, 假设某个一维矢量是 $\bvec v$, 两种不同的基底分别是 $\bvec u_1$ 和 $\bvec u_2$, 则
\begin{equation}\label{Units_eq1}
\bvec v = x_1 \bvec u_1 = x_2 \bvec u_2
\end{equation}
其中 $x_1$ 和 $x_2$ 是数学量. 现在, 如果我们已知 $\bvec u_1$ 和 $\bvec u_2$ 的关系, 例如
\begin{equation}\label{Units_eq2}
\bvec u_1 = c \bvec u_2
\end{equation}
其中 $c$ 也是一个数学量. 我们就可以直接将这个关系代入\autoref{Units_eq1} 中的 $\bvec u_1$, 得
\begin{equation}
x_1 (c \bvec u_2) = x_2 \bvec u_2
\end{equation}
现在等式两边都使用同样的基底, 于是他们的坐标也必定相同, 即
\begin{equation}
x_2 = c x_1
\end{equation}

同样, 若基底间的关系是
\begin{equation}
\bvec u_2 = b\bvec u_1
\end{equation}
显然有 $b = 1/c$, 代入\autoref{Units_eq1} 中的 $\bvec u_2$ 得
\begin{equation}
x_1 = b x_2
\end{equation}

\begin{example}{长度}\label{Units_ex1}
我们来考虑一个表示长度的物理量 $L$. 在我们确定单位(类比矢量基底)以前, 它不能使用任何数表示. 现在规定单位(即基底) 为 $l_1 = 1\Si{cm}$, $l_2 = 1\Si{m}$, 且有
\begin{equation}
l_2 = 100 l_1
\end{equation}
若令 $L = 2\Si{m}$, 则
\begin{equation}
L = 2 l_2 = 2 (100 l_1) = 200 l_1 = 200 \Si{cm}
\end{equation}
或者记为
\begin{equation}
L = 2 \Si{m} = 2 (100 \Si{cm}) = 200 \Si{cm}
\end{equation}
若令 $L = 5\Si{cm}$, 则
\begin{equation}
L = 5 l_1 = 5 \qty(\frac{1}{100} l_2) = 0.05 \Si{m}
\end{equation}
或者记为
\begin{equation}
L = 5 \Si{cm} = 5 \qty(\frac{1}{100} \Si{m}) = 0.05 \Si{m}
\end{equation}
\end{example}

与一维矢量不同的是, 两个相同的物理量可以相除并得到一个无量纲的数, 且这个数不依赖于单位制的选取. 这么做的时候, 可以类比为将两个矢量的模长相除. 例如
\begin{equation}
\frac{200\Si{cm}}{1\Si{m}} = \frac{2\Si{m}}{1000\Si{mm}} = \frac{2\Si{m}}{1\Si{m}} = 2
\end{equation}
所以, 另一种单位转换的办法, 是先凑出一个等于 1 的无量纲分数, 例如 $1\Si{m}/(1000\Si{mm}) = 1$, 使得分母是原单位, 分子是新单位, 然后将它乘以需要转换的物理量. 例如
\begin{equation}
2000\Si{mm} = 2000\Si{mm} \times \frac{1\Si{m}}{1000\Si{mm}} = 2\Si{m}
\end{equation}
这与\autoref{Units_ex1} 中的方法是完全等效的.
