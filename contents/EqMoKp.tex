% 开普勒问题的运动方程

\pentry{开普勒问题\upref{CelBd}, 中心力场问题\upref{CenFrc}}

若已知轨道形状, 我们来计算质点在轨道上的位置如何关于时间变化. 由\autoref{CenFrc_eq8}~\upref{CenFrc}得
\begin{equation}
t = \sqrt{\frac{m}{2}} \int_{r_0}^r \frac{\dd{r'}}{\sqrt{E - k/r' - L^2/(2mr'^2)}}
\end{equation}
该式对任何圆锥曲线轨道都适用, 其中 $r_0$ 是轨道离中心最近的一点, 令质点经过该点时 $t= 0$. 把这个积分的结果 $t(r)$ 取反函数, 就可以得到 $r(t)$. 同理, 有
\begin{equation}
\dd{t} = \frac{mr^2}{L}\dd{\theta}
\end{equation}
将\autoref{Cone_eq5}~\upref{Cone}代入, 积分得
\begin{equation}
t = \frac{L^3}{mk^2} \int_{\theta_0}^\theta \frac{\dd{\theta'}}{(1 - e\cos \theta')^2 }
\end{equation}

对\textbf{抛物线}($e = 1$), 有
\begin{equation}
t = \frac{L^3}{2mk^2} \qty(\tan\frac{\theta}{2} +  \frac{1}{3}\tan^3 \frac{\theta}{2})
\end{equation}

对于\textbf{椭圆}($e < 1$), 可以用一个参数\textbf{偏近点角(eccentric anomaly)} $\psi$ 来代替 $\theta$ 会更方便. $\psi$ 的定义为
\begin{equation}\label{EqMoKp_eq1}
r = a(1-e\cos\psi)
\end{equation}
其中 $a$ 是半长轴. 当 $\theta$ 从 $0$ 变化到 $2\pi$ 时, $\psi$ 也从 $0$ 变化到 $2\pi$, 只是速度不一样.

\begin{equation}
t = \sqrt{\frac{ma^3}{-k}} (\psi - e \sin\psi)
\end{equation}
该式被称为\textbf{开普勒方程(Kepler's equation)}, 开普勒第二定律也可以由该式验证.

对于\textbf{双曲线}($k<0$), 偏近点角使用\autoref{EqMoKp_eq2} 定义, 可取任意实数, 使得
\begin{equation}\label{EqMoKp_eq2} % 已经验算正确
r = a(e\cosh\xi - 1)
\end{equation}
\begin{equation}
t = \sqrt{\frac{ma^3}{-k}} (e\sinh\xi - \xi)
\end{equation}


\addTODO{双曲线 $k>0$ 情况是否也相同?}

\subsection{推导}
\addTODO{……}
