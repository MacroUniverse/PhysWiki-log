% Python 字符串处理

\pentry{Python 简介\upref{Python}}

\subsection{字符串处理}
字符串的处理个人认为在网络数据挖掘中使用非常广泛,比如网络爬虫. 对于字符串 \verb|'python String function'|, 生成字符串变量
\begin{lstlisting}[language=python]
str='python String function'
\end{lstlisting}
字符串长度获取:\verb|len(str)|
\begin{lstlisting}[language=python]
print('%s length=%d' % (str,len(str)))
\end{lstlisting}


\subsubsection{字母处理}
全部大写:\verb|str.upper()|

全部小写:\verb|str.lower()|

大小写互换:\verb|str.swapcase()|

首字母大写,其余小写:\verb|str.capitalize()|

首字母大写:\verb|str.title()|

\begin{lstlisting}[language=python]
str='python String function'
print('%s lower=%s' % (str,str.lower()))
print('%s upper=%s' % (str,str.upper()))
print('%s swapcase=%s' % (str,str.swapcase()))
print('%s capitalize=%s' % (str,str.capitalize()))
print('%s title=%s' % (str,str.title()))
输出:
python String function lower=python string function
python String function upper=PYTHON STRING FUNCTION
python String function swapcase=PYTHON sTRING FUNCTION
python String function capitalize=Python string function
python String function title=Python String Function
\end{lstlisting}
\subsubsection{格式化相关}
获取固定长度,右对齐,左边不够用空格补齐:\verb|str.ljust(width)|

获取固定长度,左对齐,右边不够用空格补齐:\verb|str.ljust(width)|

获取固定长度,中间对齐,两边不够用空格补齐:\verb|str.ljust(width)|

获取固定长度,右对齐,左边不足用0补齐

\begin{lstlisting}[language=python]
str='python String'
print('%s ljust=%s' % (str,str.ljust(20)))
print('%s rjust=%s' % (str,str.rjust(20)))
print('%s center=%s' % (str,str.center(20)))
print('%s zfill=%s' % (str,str.zfill(20)))
输出:
python String ljust=python String       
python String rjust=       python String
python String center=   python String    
python String zfill=0000000python String

\end{lstlisting}

\subsubsection{字符串搜索相关}
搜索指定字符串,没有返回-1:\verb|str.find('t')|

指定起始位置搜索:\verb|str.find('t',start)|

指定起始及结束位置搜索:\verb|str.find('t',start,end)|

从右边开始查找:\verb|str.find('t')|

搜索到多少个指定字符串:\verb|str.count('t')|

\begin{lstlisting}[language=python]
str='python String function'
print('%s find nono=%d' % (str,str.find('nono')))
print('%s find t=%d' % (str,str.find('t')))
print('%s find t from %d=%d' % (str,1,str.find('t',1)))
print('%s find t from %d to %d=%d' % (str,1,2,str.find('t',1,2)))
print('%s rfind t=%d' % (str,str.rfind('t')))
print('%s count t=%d' % (str,str.count('t')))
输出:
python String function find nono=-1
python String function find t=2
python String function find t from 1=2
python String function find t from 1 to 2=-1
python String function rfind t=18
python String function count t=3
\end{lstlisting}

\subsubsection{字符串替换相关}

替换old为new:\verb|str.replace('old','new')|

替换指定次数的old为new:\verb|str.replace('old','new',maxReplaceTimes)|


\subsubsection{字符串去空格及去指定字符}

去两边空格:\verb|str.strip()|

去左空格:\verb|str.lstrip()|

去右空格:\verb|str.rstrip()|

去两边字符串:\verb|str.strip('d')|,相应的也有\verb|lstrip,rstrip|

按指定字符分割字符串为数组:\verb|str.split(' ')|

\begin{lstlisting}[language=python]
str=' python String function '
print('%s strip=%s' % (str,str.strip()))
str='python String function'
print('%s strip=%s' % (str,str.strip('d')))
\end{lstlisting}



\subsubsection{翻转字符串}
\begin{lstlisting}[language=python]
sStr1 = 'abcdefg'
sStr1 = sStr1[::-1]
print(sStr1)
输出
gfedcba
\end{lstlisting}

\subsubsection{连接字符串}
\begin{lstlisting}[language=python]
delimiter = ','
mylist = ['Brazil', 'Russia', 'India', 'China']
print (delimiter.join(mylist))
print('abc'+'def')
输出
Brazil,Russia,India,China
abcdef
\end{lstlisting}

\subsubsection{截取字符串}
\begin{lstlisting}[language=python]
str='0123456789'
print(str[0:3]) #截取第一位到第三位的字符
print(str[:]) #截取字符串的全部字符
print(str[6:]) #截取第七个字符到结尾
print(str[:-3]) #截取从头开始到倒数第三个字符之前
print(str[2]) #截取第三个字符
print(str[-1]) #截取倒数第一个字符
print(str[::-1]) #创造一个与原字符串顺序相反的字符串
print(str[-3:-1]) #截取倒数第三位与倒数第一位之前的字符
print(str[-3:]) #截取倒数第三位到结尾
输出:
012
0123456789
6789
0123456
2
9
9876543210
78
789
\end{lstlisting}


关于更多高效的字符串处理方法,感兴趣的读者可以查阅正则表达式的使用方法.
