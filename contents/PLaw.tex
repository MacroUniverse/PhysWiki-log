% 动量定理 动量守恒
% 动量守恒|动量定理|合外力|质点系
\pentry{质点系的动量\upref{SysMom}}

以下我们将根据牛顿定律推导出系统的\textbf{动量定理}, 即系统总动量的变化率等于合外力
\begin{equation}
\dot {\bvec p} = \bvec F
\end{equation}
由此可以得出系统所受合外力时系统总动量不随时间变化, 即\textbf{动量守恒}.

任何系统都可以看做质点系,质点系中第 $i$ 个质点可能受到系统内力 $\bvec F_i^{in}$ 或系统外力 $\bvec F_i^{out}$. 由单个质点的动量定理\upref{PLaw1},
\begin{equation}
\dv{t} \bvec p_i = \bvec F_i^{in} + \bvec F_i^{out}
\end{equation}
总动量的变化率为
\begin{equation}
\dv{\bvec P}{t} = \sum_i \dv{t} \bvec p_i  = \sum_i \bvec F_i^{in}  + \sum_i \bvec F_i^{out}
\end{equation}
由“质点系\upref{PSys}” 中的结论, 上式右边第一项求和是系统合内力, 恒为零. 于是我们得到系统的动量定理
\begin{equation}
\dv{\bvec P}{t} = \sum_i \bvec F_i^{out}
\end{equation}
可见当和外力(即等式右边)为零时, 动量 $\bvec P$ 不随时间变化, 也就是\textbf{动量守恒}.

\begin{example}{静止原子核的转变}
% 图未完成
一个原来静止的原子核,经放射性衰变,放出一个动量为$9.22×10^{-16}{\rm g\cdot cm/s}$的电子,同时该核在垂直方向上又放出一个动量为$5.33×10^{-16}{\rm g\cdot cm/s}$的中微子.问蜕变后原子核的动量的大小和方向.

解:由于这个静止的原子核在蜕变的全过程中没有受到其他外力,所以对该原子核构成的系统,总动量守恒.即有
$$\bvec p_{\rm B}+\bvec p_{\rm e}+\bvec p_{\rm \nu}=0$$
即有
$$p_{\rm B}=|\bvec p_{\rm B}|=|-\bvec p_{\rm e}-\bvec p_{\rm \nu}|=\sqrt{p_{\rm e}^{2}+p_{\rm \nu}^{2}}=10.65×10^{-16}{\rm g\cdot cm/s}$$
$$\theta=\arctan\frac{5.33}{9.22}=30^\circ$$
\end{example}
