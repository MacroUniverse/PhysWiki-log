% 动量定理 动量守恒
\pentry{动量和动量定理(单个质点)} %未完成
\subsection{结论}
系统总动量的变化率等于合外力,所以合外力为零时系统总动量守恒.

\subsection{推导}
任何系统都可以看做质点系,质点系中第 $i$ 个质点可能受到来自该质点系的其他质点的力(\textbf{系统内力}) $\vec F_i^{in}$以及来自系统外的力(\textbf{系统外力})$\vec F_i^{out}$. 由单个质点的动量定理,%未完成
\begin{equation}
\frac{d}{{dt}}{\vec p_i} = \vec F_i^{in} + \vec F_i^{out}
\end{equation}
总动量的变化率为
\begin{equation}
\frac{{d\vec P}}{{dt}} = \sum\limits_i {\frac{d}{{dt}}{{\vec p}_i}}  = \sum\limits_i {\vec F_i^{in}}  + \sum\limits_i {\vec F_i^{out}}
\end{equation}
右边的两项求和分别叫做系统的\textbf{和内力}与\textbf{合外力}.其中和内力又可以分解为不同的第 $j$ 个质点对第 $i$ 个质点的作用力
\begin{equation}
\sum\limits_i {\vec F_i^{in}}  = \sum\limits_{i,j}^{i \ne j} {{{\vec F}_{j \to i}}}
\end{equation}
任意两个质点 $k$ 和 $l$ 对该求和的贡献是一对相互作用力 ${\vec F_{k \to l}} + {\vec F_{l \to k}}$,而根据牛顿第三定律,相互作用力之和为零.所以上式求和为零,即\textbf{系统的和内力为零}.于是我们得到系统的动量定理
\begin{equation}
\frac{{d\vec P}}{{dt}} = \sum\limits_i {\vec F_i^{out}} 
\end{equation}

