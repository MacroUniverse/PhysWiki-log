%刚体的绕轴转动 转动惯量

\pentry{刚体\upref{RigBd}, 角动量定理\upref{AMLaw}}

设刚体绕光滑轴转动.这里令轴的方向为 $z$,假设轴光滑,则轴对刚体可施加 $x, y$ 两个方向的力矩,却不能施加 $z$ 方向的力矩. 所以根据角动量定理, 角动量的 $z$ 分量守恒.

对于单个质点,$L_z = ( \vec r \cross \vec p ) \vdot \uvec z$. 首先把质点的位矢在水平方向和竖直方向分解, $\vec r = {\vec r_z} + {\vec r_ \bot }$. 由于 $\vec p$ 一直沿水平方向, 根据叉乘的几何定义, ${\vec r_z} \cross \vec p$ 也是沿水平方向, 只有 ${\vec r_ \bot } \cross \vec p$ 沿 $z$ 方向.另外, 在圆周运动中, 半径始终与速度垂直, 所以 ${\vec r_ \bot }$ 始终与 $\vec p$ 垂直.得出结论
\begin{equation}
{L_z} = \abs{\vec r_\bot} \abs{\vec p} = m {r_ \bot } v = mr_ \bot ^2\omega 
\end{equation}
若把刚体分成无数小块, 每小块的质量分别为 $m_i$, 离轴的距离 $r_{\bot i} = \sqrt {x_i^2 + y_i^2} $, 则刚体的角动量 $z$ 分量为
\begin{equation}
L_z = \omega \sum_i m_i r_{ \bot i}^2
\end{equation}
用积分写成
\begin{equation}
{L_z} = \omega \int {r_ \bot ^2\D m} = \omega \int {r_ \bot ^2\rho  \D V} 
\end{equation}

定义刚体的\textbf{绕轴转动惯量}为
\begin{equation}
{\rm{I}} = \int {r_ \bot ^2\D m} 
\end{equation}
则刚体沿轴方向的角动量为
\begin{equation}\label{RigRot_eq5}
{L_z} = I\omega 
\end{equation}
 
现在来看“角动量定理\upref{AMLaw}” 的\autoref{AMLaw_eq1}, 注意等号两边是矢量, 所以各个分量必须相等, 我们有
\begin{equation}\label{RigRot_eq6}
\dv{L_z}{t} = M_z
\end{equation}
将\autoref{RigRot_eq5} 代入\autoref{RigRot_eq6}, 并利用角加速度的定义得
\begin{equation}
I\alpha = M_z
\end{equation}
这就是刚体绕轴转动的动力学方程, 其形式可类比牛顿第二定律\upref{New3}.











