% 位置矢量 位移

\pentry{矢量}

\textbf{位置矢量(位矢)}就是从坐标原点指向某一点的矢量,通常记为 $\vec r$. 当定义了一个坐标系,那么坐标系中一点的位置就可以用位矢表示.

有时候表示一个关于位置的函数, 通常将位矢 $\vec r$ 作为自变量. 例如一个物体内密度关于位置的分布可以表示为 $\rho(\vec r)$. 在直角坐标系中,就相当于 $\rho(x,y,z)$,在球坐标系中就相当于 $\rho(r,\theta,\phi)$ 这么做的好处是书写简洁,而且不需要指定坐标系的种类.

在物体运动过程中,可以把物体的位矢看做时间的矢量函数 $\vec r(t)$,则\textbf{位移} $\Delta \vec r$ 是一段时间 $[t_1,t_2]$ 内物体初末位矢的矢量差
\begin{equation}
\Delta \vec r = \vec r(t_2) - \vec r(t_1)
\end{equation}
注意位移只与一段时间内物体的初末位置有关,与路径无关.

\pentry{全微分\upref{TDiff},矢量的微分}%未完成链接
\begin{exam}{证明 $\dd{R} = \hat{\vec R} \vdot \dd{\vec R}$}\label{Disp_ex1}
这个证明的几何意义是, 位矢模长的微小变化等于位矢的微小变化在位矢正方向的投影.

这里以平面直角坐标系中的位矢为例证明. 令位矢 $\vec R$ 的坐标为 $(x, y)$, 模长为 $R = \sqrt{x^2 + y^2}$,
模长的全微分为
\begin{equation}
\dd{R} = \pdv{R}{x} \dd{x} + \pdv{R}{y} \dd{y} = \frac{x}{\sqrt{x^2 + y^2}} \dd{x} + \frac{y}{\sqrt{x^2 + y^2}} \dd{y}
\end{equation}
考虑到 $x/\sqrt{x^2 + y^2}$ 和 $y/\sqrt{x^2 + y^2}$ 分别为 $\hat {\vec R} = \vec R/R$ 的两个分量, $\dd{x}$ 和 $\dd{y}$ 分别为 $\dd{\vec R}$ 的两个分量, 根据点乘的定义\upref{Dot}上式变为
\begin{equation}
\dd{R} = \hat{\vec R} \vdot \dd{\vec R}
\end{equation}
\end{exam}

\rentry{速度和加速度(矢量)\upref{VnA}}








