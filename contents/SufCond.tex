% 充分必要条件

若由命题 $A$ 能推导出命题 $B$, 则 $A$ 是 $B$ 的充分条件, $B$ 是 $A$ 的必要条件.如何理解这个定义呢?下面举两个例子.

% 未完成: 这个例子太不数学了, 用一个欧式几何问题替代
\begin{exam}{}
命题 $A$:行星 $X$ 是地球.
命题 $B$:行星 $X$ 有一颗天然卫星.

首先我们考虑 $A$ 对 $B$ 的关系.显然,由 $A$ 可以推出 $B$, 说明 $A$ 中有充分的信息能得到 $B$, 所以叫做 $B$ 的\textbf{充分条件}. $A$ 中包括得到 $B$ 所必要的信息,还\textbf{可能}包括一些其他信息,例如由命题 $A$ 可以得出 $X$ 星球上面有氧气,有液态水,有生物,等等.这些多出来的信息并不一定是得到 $B$ 所必须的,因为还有许多其他行星有一颗天然卫星,但并没有液态水或者氧气等.

那如何判断 $A$ 中有没有多余的信息呢?我们可以反过来试图用 $B$ 推导命题 $A$, 若原则上得不出 $A$ (而不是因为我们逻辑水平不够),则证明 $A$ 中有多余的条件.这时我们说 $A$ 不是 $B$ 的\textbf{必要条件},因为 $A$ 中的一些信息是多余的,也就是没有必要的.综上, $A$ 是 $B$ 的\textbf{充分非必要条件}.

现在我们从 $B$ 的角度考虑.虽然由条件 $B$ 不能推导出条件 $A$, 但是 $B$ 是 $A$ 中信息的一部分, $B$ 必须要成立才有可能使 $A$ 成立,也就是说如果 $B$ 不成立 $A$ 就不可能成立(没有卫星的行星X绝对不可能是地球).所以说 $B$ 是 $A$ 的必要条件.另外,由 $B$ 中的少量信息不能得到 $A$, 所以 $B$ 不是 $A$ 的充分条件.综上, $B$ 是 $A$ 的\textbf{必要非充分条件}.
\end{exam}

\begin{exam}{}
命题 $A$ :三角形 $X$ 的其中两内个角分别为 90°和 45°.
命题 $B$ :三角形 $X$ 有两个 45°的内角.

利用三角形三个内角和为 180°的事实,可以从 $A$ 推出 $B$, 说明 $A$ 是 $B$ 的充分条件, $B$ 是 $A$ 的必要条件.但也可以从 $B$ 推出 $A$, 说明 $B$ 是 $A$ 的充分条件, $A$ 是 $B$ 的必要条件.所以 $A$ 和 $B$ 既是彼此的充分条件也是彼此的必要条件.所以我们说 $A$ 和 $B$ \textbf{互为充分必要条件}.若 $A$ 是 $B$ 的充分必要条件, $B$ 一定也是 $A$ 的充分必要条件.因为两种表述都意味着 $A$,  $B$ 命题\textbf{等效},所提供的信息都是一样的,两者都没有任何多余的或者缺失的信息.
\end{exam}

需要注意的是 
\begin{enumerate}
\item 充分/必要条件是两个命题之间的关系,若直说一个命题是充分/必要条件没有意义.
\item 讨论充分/必要条件需要在一定的前提下进行.以上的例子中,例1的前提如:我们讨论的是现在的地球和其它行星,而不是很久以前或以后.例2的前提如:我们讨论的是欧几里得几何中的平面三角形.
\item 在证明 $A$ 是 $B$ 的充分必要条件时,需要分别证明 $A$ (相对于 $B$)的充分性和必要性.充分性需要由 $A$ 证明 $B$, 必要性需要由 $B$ 证明 $A$. 
\item 在证明 $A$ 是 $B$ 的充分非必要条件时,除了需要证明 $A$ 的充分性,还需非必要性,即 $B$ 不能推出 $A$. 只要我们可以举出一个 $B$ 成立 $A$ 不成立的反例,就立刻证明了不可能由 $B$ 推出 $A$. 
\end{enumerate}
