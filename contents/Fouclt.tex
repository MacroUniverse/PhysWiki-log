% 傅科摆角速度的一种几何推导

傅科摆的介绍blablabla

\begin{equation}
\omega = \omega_0 \sin\alpha
\end{equation}

设 $\vec R$ 为地心指向傅科摆的矢量, 当地纬度为 $\alpha$, 地轴指向北的单位矢量为 $\uvec k$, 显然 $\uvec k\vdot \uvec R = \sin\alpha$.

若把任意矢量 $\vec P$ 围绕某单位矢量 $\uvec M$ 以右手定则旋转角微元 $\D\theta$, 有
\begin{equation}\label{Fouclt_eq2}
\D \vec P = \uvec M \cross \vec P \D\theta
\end{equation}
开始时, 令傅科摆在最低点的速度方向的单位向量为 $\uvec A$ ($\uvec A\vdot \vec R = 0$), 在傅科摆下方的水平地面上标记单位向量 $\uvec B$, 使开始时 $\uvec B = \uvec A$. 当傅科摆随地球在准静止状态下移动位移 $\D\vec s$ ($\D\vec s\vdot\vec R = 0$)后, 由\autoref{Fouclt_eq2} 可得

\begin{equation}
\D \uvec A = \uvec M\cross \uvec A\vdot\D\theta = 
\frac{\vec R\cross\D\vec s}{\abs{\vec R\cross\D\vec s}} \cross \uvec A \frac{\D s}{R}
\end{equation}
注意这只是一个比较符合物理直觉的假设, 这里并不给出证明. 当地球转动 $\D\theta$ 时, 上式中 $\D\vec s = \uvec k \cross \vec R \D\theta$, 而地面上的标记 $\uvec B$ 也围绕地轴转动, 所以 $\D\uvec B = \uvec k \cross \uvec A \D\theta$.

下面计算 $\D \uvec A - \D \uvec B$. 因为 $\vec R\vdot \D \vec s = 0$, 所以 $\abs{\vec R \cross \D \vec s} = R\D s$, 所以
\begin{equation}\ali{
\D \uvec A &= \frac{\vec R\cross\D\vec s}{R^2}\cross\uvec A =
\frac{1}{R^2}\vec R\cross(\uvec k\cross\vec R \D\theta)\cross \uvec A\\
&= \uvec R\cross(\uvec k\cross\uvec R)\cross\uvec A \D\theta =
[(\uvec R\vdot\uvec R)\uvec k - (\uvec R\vdot \uvec k)\uvec R] \cross \uvec A \D\theta\\
&= (\uvec k - \uvec R\sin\alpha) \cross \uvec A \D\theta
}\end{equation}
\begin{equation}
\D\uvec A - \D\uvec B = (\uvec k - \uvec R\sin\alpha)\cross \uvec A \D\theta - \uvec k \cross \uvec A \D\theta = -\sin\alpha \uvec R \cross \uvec A \D\theta
\end{equation}
所以地球转过 $\D\theta$ 角以后, $\uvec A$ 与 $\uvec B$ 之间的夹角为
\begin{equation}
\D\gamma = \abs{\D\uvec A - \D\uvec B} = \sin\alpha \D\theta
\end{equation}
两边除以 $\D t$ 得角速度
\begin{equation}
\omega = \omega_0 \sin\alpha
\end{equation}




