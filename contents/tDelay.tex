% 量子散射的延迟

\pentry{一维散射(量子)\upref{Sca1D}}
一个一维波包用傅里叶变换\upref{FTTri}表示为
\begin{equation}
\psi(x) = \int_{-\infty}^{+\infty} A(\omega) \exp(\I kx - \omega t) \dd{\omega}
\end{equation}
自由粒子情况下 $\omega = k^2/(2m)$. 如果经过一个势阱, 不同平面波透射后发生相移 $\phi(\omega)$, 经过后, 波包为
\begin{equation}\label{tDelay_eq1}
\psi'(x, t) = \int_{-\infty}^{+\infty} A(k) \exp[\I kx - \omega t + \phi(\omega)] \dd{k}
\end{equation}
想象一个特殊情况: 经过势阱后 $A(k)$ 不变, $\phi(\omega) = \Delta t \omega$, 那么波函数变为
\begin{equation}
\psi'(x, t) = \int_{-\infty}^{+\infty} A(k) \exp[\I kx - \omega (t - \Delta t) ] \dd{k}
= \psi(x, t - \Delta t)
\end{equation}
这样波包在时间轴上向右平移了 $\Delta t_{EWS}$, 即延迟. 近似来说, 如果波包带宽较窄, 频率中心为 $\omega_0$, 那么在带宽以内可以把 $\phi(\omega)$ 近似看成是线性的, 那么延迟近似为
\begin{equation}\label{tDelay_eq2}
\Delta t_{EWS}(\omega) = \dv{\phi}{\omega}
\end{equation}
如果要取一个与波包形状无关的延迟的定义, 那么这个定义是最佳选择. 注意这样定义的延迟与频率有关. 这个延迟被称为 \textbf{Wigner 延迟}或者 \textbf{EWS (Eisenbud-Wigner-Smith) 延迟} .

\subsection{驻相法}
使用\textbf{驻相法(stationary phase method)}可以分析出波包的近似位置, 速度以及时间延迟. 把\autoref{tDelay_eq1} 中的 $x, t$ 固定, 令相位 $kx - \omega t + \phi$ 对 $\omega$ 求导为零, 有
\begin{equation}
x = v_g \qty(t - \dv{\phi}{\omega})
\end{equation}
其中 $v_g = \dv*{\omega}{k}$ 是群速度. 同样可以得到\autoref{tDelay_eq2}.
