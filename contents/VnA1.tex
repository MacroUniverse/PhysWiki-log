% 速度 加速度(一维)

\pentry{位移\upref{Disp}, 基本初等函数的导数\upref{FunDer}, 复合函数求导\upref{ChainR}, 牛顿—莱布尼兹公式\upref{NLeib}}

速度和加速度都是矢量, 但如果我们考虑质点的一维运动(沿直线运动), 那么我们可以指定一个正方向并沿运动方向建立坐标轴. 这样一来, 我们就可以把一维情况下的位移、速度、加速度这些矢量用一个带正负号的标量来表示, 正号代表指向正方向, 负号代表指向负方向, 标量的绝对值就等于矢量的模长. 所以以下我们用坐标 $x$ 来表示一维位移, 实数 $v$ 和 $a$ 来表示一维速度和加速度.

物理学中, \bb{速度}和\bb{加速度}通常指瞬时值. 在一维运动中, \bb{瞬时速度}的定义为一段极短时间 $\Delta t$ 内质点的位移\upref{Disp} $\Delta x$ 除以这段时间, \bb{瞬时加速度}的定义为一段极短时间 $\Delta t$ 内质点的速度变化 $\Delta v$ 除以这段时间, 而这些恰好是导数\upref{Der}的定义. 用极限符号和导数来表示, 就是

\begin{equation}
v(t) = \lim_{\Delta t\to 0} \frac{x(t+\Delta t) - x(t)}{\Delta t} = \dv{x(t)}{t}
\end{equation}

\begin{equation}
a(t) = \lim_{\Delta t\to 0} \frac{v(t+\Delta t) - v(t)}{\Delta t} = \dv{v(t)}{t}
\end{equation}
根据高阶导数%未完成
的定义, 加速度就是位矢的二阶导数
\begin{equation}
a(t) = \dv[2]{x(t)}{t}
\end{equation}

\begin{exam}{匀加速运动}
已知匀加速运动的位移为 $x(t) = x_0 + v_0 t + a t^2/2$, 注意到这是一个幂函数, 求导得到速度为 $v(t) = v_0 + a t$, 再次求导(二阶导数)得到加速度为 $a(t) = a$. 可见这是一个匀加速运动.
\end{exam}

\begin{exam}{简谐振动}
已知简谐振动的位移函数为 $x(t) = A\cos(\omega t)$, 运用复合函数求导% 要有这个例子,并引用
得速度为 $v(t) = -A\omega\sin(\omega t)$, 加速度为 $a(t) = -A\omega^2\cos(\omega t)$.
\end{exam}


\subsection{由速度或加速度求位移}
既然一维速度是位置的导数(即 $x(t)$ 是速度的原函数)由牛顿—莱布尼兹公式得速度在一段时间的定积分等于初末位置之差, 即
\begin{equation}\label{VnA1_eq4}
x(t_2) - x(t_1) = \int_{t_1}^{t_2} v(t) \dd{t}
\end{equation}
所以若已知某时刻质点的位置 $x(t_0) = x_0$, 和速度函数 $v(t)$, 就可以求得任意时刻的位置(为了区分积分变量和积分上限, 我们把积分变量改成 $t'$)
\begin{equation}\label{VnA1_eq5}
x(t) = x_0 + \int_{t_0}^t v(t') \dd{t'}
\end{equation}

\begin{exam}{匀速直线运动}
若一维运动的质点速度始终为 $v_0$, 由\autoref{VnA1_eq5} 得
\begin{equation}
x(t) = x_0 + \int_{t_0}^t v_0 \dd{t} = x_0 + v_0(t-t_0)
\end{equation}
\end{exam}

与\autoref{VnA1_eq4} 和\autoref{VnA1_eq5} 同理,一维速度和加速度之间也有类似关系
\begin{equation}
v(t_2) - v(t_1) = \int_{t_1}^{t_2} a(t) \dd{t}
\end{equation}
\begin{equation}\label{VnA1_eq8}
v(t) = v_0 + \int_{t_0}^t a(t') \dd{t'}
\end{equation}

\begin{exam}{匀加速直线运动}
若质点在 $t_0$ 时的位置为 $x_0$, 速度为 $v_0$, 且加速度始终等于常数 $a_0$, 求任意时刻的速度和加速度 $x(t)$.

我们首先由\autoref{VnA1_eq8} 得到速度函数为
\begin{equation}
v(t) = v_0 + \int_{t_0}^t a_0 \dd{t'} = v_0 + a_0 (t - t_0)
\end{equation}
然后再次积分得到位置函数
\begin{equation}
x(t) = x_0 + \int_{t_0}^t [v_0 + a_0 (t' - t_0)] \dd{t'} = x_0 + v_0 (t - t_0) +  \frac12 a_0 (t - t_0)^2
\end{equation}
\end{exam}
