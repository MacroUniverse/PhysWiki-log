% Python模块
\pentry{Python 入门\upref{Python}}

模块是包含一组\textbf{函数}的文件, 希望在应用程序中引用.
\subsection{创建模块}
如需创建模块,只需将所需函数代码保存在文件扩展名为 \verb|.py| 的文件中. 例如
在名为 \verb|mymodule.py| 的文件中保存代码:
\begin{lstlisting}[language=python]
def greeting(name):
    print("Hello, " + name)

def saying():
    print("Hello! ")
\end{lstlisting}
在该模块中写了两个简单的函数.

\subsection{使用模块}
现在,我们就可以用 \verb|import| 语句来使用我们刚刚创建的模块.

导入名为 \verb|mymodule| 的模块,并调用 \verb|greeting| 函数与\verb|saying|函数.
\begin{lstlisting}[language=python]
import mymodule
mymodule.greeting("Bill")
mymodule.saying()
\end{lstlisting}
注释:如果使用模块中的函数时,请使用以下语法:\verb|模块名.函数名|.

\subsection{为模块命名}
可以随意对模块文件命名,但是文件扩展名必须是 \verb|.py|.

\subsection{重命名模块}
有时候模块名比较长,为了方便使用,可以在导入模块时使用 \verb|as| 关键字创建别名.
\begin{lstlisting}[language=python]
import mymodule as mx
mx.saying()
\end{lstlisting}

\subsection{内建模块}
Python 中有几个内建模块,可以随时导入.
\subsubsection{time模块}
在Python中,通常有这几种方式来表示时间:

时间戳(timestamp) :通常来说,时间戳表示的是从1970年1月1日00:00:00开始按秒计算的偏移量.我们运行“type(time.time())”,返回的是float类型.

格式化的时间字符串

元组(struct_time)   :struct_time元组共有9个元素共九个元素:(年,月,日,时,分,秒,一年中第几周,一年中第几天,夏令时)

\subsubsection{random模块}
随机生成内容.
\begin{lstlisting}[language=python]
import random
# 0,1之间时间生成的浮点数  float
print(random.random())
# 随机生成传入参数范围内的数字 即 1,2,3
print(random.randint(1, 3))
# 随机生成传入参数范围内的数字,range顾头不顾尾
print(random.randrange(1, 3))
# 随机选择任意一个数字
print(random.choice([1, '23', [4, 5]]))
\end{lstlisting}

\subsubsection{sys模块}
\begin{lstlisting}[language=python]
import sys
# 命令行参数List,第一个元素是程序本身路径
print(sys.argv)
# 结果['D:/Pycharm Community/python内置函数/sysTest.py']
# 获取Python解释程序的版本信息
print(sys.version)
\end{lstlisting}

\subsubsection{os模块}
\verb|os| 模块是与操作系统交互的一个接口.
\begin{lstlisting}[language=python]
os.getcwd() # 获取当前工作目录,即当前python脚本工作的目录路径
os.chdir("dirname")  # 改变当前脚本工作目录;相当于shell下cd
os.curdir  # 返回当前目录: ('.')
os.makedirs('dirname1/dirname2')    # 可生成多层递归目录
# 若目录为空,则删除,并递归到上一级目录,如若也为空,则删除,依此类推
os.removedirs('dirname1')
os.mkdir('dirname')    # 生成单级目录
os.rmdir('dirname')    # 删除单级空目录,若目录不为空则无法删除,报错;
# 列出指定目录下的所有文件和子目录,包括隐藏文件,并以列表方式打印
os.listdir('dirname')
os.remove()  # 删除一个文件
os.rename("oldname","newname")  # 重命名文件/目录
os.stat('path/filename')  # 获取文件/目录信息
os.path.abspath(path)  # 返回path规范化的绝对路径
os.path.split(path)  # 将path分割成目录和文件名二元组返回
os.path.exists(path) #  如果path存在,返回True;如果path不存在,返回False
os.path.isabs(path)  # 如果path是绝对路径,返回True
os.path.isfile(path)  # 如果path是一个存在的文件,返回True.否则返回False
os.path.isdir(path) # 如果path是一个存在的目录,则返回True.否则返回False
os.path.join(path1[, path2[, ...]])  将多个路径组合后返回
\end{lstlisting}
