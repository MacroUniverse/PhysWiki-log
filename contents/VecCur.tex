% 曲率 (向量丛)

\pentry{联络(向量丛)\upref{VecCon}}

本节采用爱因斯坦求和约定.

设$M$是$n$维微分流形, $E$是其上秩为$k$的光滑向量丛. 设给定了$E$上的联络$D$.

\subsection{定义}

给定截面$\xi\in\Gamma(E)$和切向量场$X,Y$, 联络$D$的沿着$X,Y$作用在$\xi$上的\textbf{曲率算子 (curvature operator)} 定义为
$$
R(X,Y)\xi=-D_XD_Y\xi+D_YD_X\xi+D_{[X,Y]}\xi.
$$
直观来说, 曲率算子是量度导数算子的可对易性的: 如果取$X,Y$为坐标向量$\partial_i,\partial_j$, 那么$[X,Y]=0$, 从而$R(\partial_i,\partial_j)\xi$就是沿着坐标方向的二阶导数之差.

\subsection{曲率方阵}
为计算$R(X,Y)\xi$, 设$\{s_\alpha\}_{\alpha=1}^k$是$E$的局部标架, $\xi=\xi^\alpha s_\alpha$为$\xi$在此标价下的局部表达式, $\omega$是此标架下的联络1-形式矩阵. 从而
$$
\begin{aligned}
D_X\xi=X(\xi^\alpha)\xi_\alpha+\xi^\beta\omega_\beta^\alpha(X) s_\alpha,
\end{aligned}
$$
$$
\begin{aligned}
D_YD_X\xi&=Y[X(\xi^\alpha)]s_\alpha+Y[\xi^\beta\omega_\beta^\alpha(X)] s_\alpha
+[X(\xi^\beta)+\xi^\gamma\omega_\gamma^\beta(X)]\omega_\beta^\alpha(Y) s_\alpha \\
&=Y[X(\xi^\alpha)]s_\alpha+Y(\xi^\beta)\omega_\beta^\alpha(X)s_\alpha+\xi^\beta Y[\omega_\beta^\alpha(X)]s_\alpha\\
&\quad+[X(\xi^\beta)+\xi^\gamma\omega_\gamma^\beta(X)]\omega_\beta^\alpha(Y) s_\alpha ,
\end{aligned}
$$
于是
$$
\begin{aligned}
(-&D_XD_Y+D_YD_X)\xi\\
&=[Y,X](\xi^\alpha)s_\alpha
+[Y(\xi^\beta)\omega_\beta^\alpha(X)-X(\xi^\beta)\omega_\beta^\alpha(Y)]s_\alpha+\left(\xi^\beta Y[\omega_\beta^\alpha(X)]-\xi^\beta X[\omega_\beta^\alpha(Y)]\right)s_\alpha\\
&\quad+\left(X(\xi^\beta)\omega_\beta^\alpha(Y)-Y(\xi^\beta)\omega_\beta^\alpha(X)+\xi^\gamma\omega_\gamma^\beta(X)\omega_\beta^\alpha(Y)-\xi^\gamma\omega_\gamma^\beta(Y)\omega_\beta^\alpha(X) \right)s_\alpha\\
&=[Y,X](\xi^\alpha)s_\alpha+\left(\xi^\beta Y[\omega_\beta^\alpha(X)]-\xi^\beta X[\omega_\beta^\alpha(Y)]\right)s_\alpha\\
&\quad+\left(\xi^\gamma\omega_\gamma^\beta(X)\omega_\beta^\alpha(Y)-\xi^\gamma\omega_\gamma^\beta(Y)\omega_\beta^\alpha(X) \right)s_\alpha\\
&=\left([Y,X](\xi^\alpha)+\xi^\beta\omega_\beta^\alpha[Y,X]\right)s_\alpha
+\left(\xi^\beta Y[\omega_\beta^\alpha(X)]-\xi^\beta X[\omega_\beta^\alpha(Y)]-\xi^\beta\omega_\beta^\alpha[Y,X]\right)s_\alpha\\
&\quad+\left(\xi^\gamma\omega_\gamma^\beta(X)\omega_\beta^\alpha(Y)-\xi^\gamma\omega_\gamma^\beta(Y)\omega_\beta^\alpha(X) \right)s_\alpha\\
&=D_{[Y,X]}\xi-(d\omega_\beta^\alpha-\omega_\beta^\gamma\wedge\omega_\gamma^\alpha)(X,Y)\xi^\beta s_\alpha.
\end{aligned}
$$
故
$$
R(X,Y)\xi=-(d\omega_\beta^\alpha-\omega_\beta^\gamma\wedge\omega_\gamma^\alpha)(X,Y)\xi.
$$
由
$$
\Omega_\beta^\alpha=d\omega_\beta^\alpha-\omega_\beta^\gamma\wedge\omega_\gamma^\alpha
$$
给出的1-形式的矩阵$\Omega=(\Omega_\beta^\alpha)$称为联络$D$的\textbf{曲率方阵(curvature matrix)}. 

\subsection{张量运算}
对于任何截面, $\xi\in\Gamma(E)$, 都有
$$
R(X,Y)\xi=-\Omega(X,Y)\xi.
$$
容易验证在新的局部标架$\{s_\alpha'\}$之下, 如果它与原来的标架之间的转换公式为$s_\alpha=b_\alpha^\beta s_\beta$, 则有
$$
\Omega'=b\cdot\Omega \cdot b^{-1},
$$
因此$(X,Y,\xi)\to R(X,Y)\xi$是张量运算: 它对于$X,Y,\xi$都是线性的, 而且只依赖于它们在某一点处的值. 

给定切丛的局部标架$\{e_i\}_{i=1}^n$后, 可写$R(X,Y)\xi=R^\alpha_{ij\beta}X^iY^js_\alpha$. 这个张量运算对于$X,Y$是反对称的, 而且可写(注意负号!)
$$
\Omega_\beta^\alpha=-\frac{1}{2}R^\alpha_{ij\beta}\theta^i\wedge\theta^j.
$$
这里$\{\theta^i\}_{i=1}^n$是同$\{e_i\}_{i=1}^n$对偶的局部标架.

有些文献中$R(X,Y)\xi$的定义比上文定义多一个负号, 这样在曲率方阵的定义中就不必出现负号.

\subsection{第二毕安基恒等式}
将等式$\Omega_\beta^\alpha=d\omega_\beta^\alpha-\omega_\beta^\gamma\wedge\omega_\gamma^\alpha$微分, 得到
$$
\begin{aligned}
d\Omega_\beta^\alpha&=-d\omega_\beta^\gamma\wedge\omega_\gamma^\alpha+\omega_\beta^\gamma\wedge d\omega_\gamma^\alpha\\
&=-(\Omega_\beta^\gamma+\omega_\beta^\lambda\wedge\omega_\lambda^\gamma)\wedge\omega_\gamma^\alpha
+\omega_\beta^\gamma\wedge(\Omega_\gamma^\alpha+\omega_\gamma^\lambda\wedge\omega_\lambda^\alpha)\\
&=\omega_\beta^\gamma\wedge\Omega_\gamma^\alpha-\Omega_\beta^\gamma\wedge\omega_\gamma^\alpha.
\end{aligned}
$$
这称为\textbf{第二毕安基恒等式(second Bianchi identity)}.
