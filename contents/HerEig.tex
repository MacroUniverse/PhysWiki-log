% 厄米矩阵的本征问题
% 线性代数|厄米矩阵|本征值|正交归一|本征矢|本征方程
% 未完成: 先把这个词条完成, 然后复制粘贴到对称矩阵词条里面, 符号替换一下就行

\pentry{厄米矩阵\upref{HerMat}, 矩阵的本征方程\upref{MatEig}}

本词条将 “对称矩阵的本征值问题\upref{SymEig}” 拓展到厄米矩阵, 结论和过程都相似. 我们来证明 $N$ 维厄米矩阵 $\mat A$ 存在 $N$ 个两两正交归一的本征矢 $\bvec v_1, \dots, \bvec v_N$. 矩阵的本征方程可以记为

\begin{equation}
(\mat A - \lambda \mat I) \bvec v = 0
\end{equation}

行列式 $\abs{\mat A - \lambda \mat I}$ 是一个关于 $\lambda$ 的 $N$ 阶多项式, 所以有 $N$ 个解. % 未完成: 这个结论在哪里提过?

\subsection{证明本征值为实数}
\begin{equation}
\bvec v_i\Her \mat A \bvec v_i = \lambda_i \bvec v_i\Her \bvec v_i
\end{equation}
将等式两边取厄米共轭(注意矢量也可以看成矩阵), 由\autoref{HerMat_eq2}\upref{HerMat} 和\autoref{HerMat_eq1}\upref{HerMat} 可得
\begin{equation}
\bvec v_i\Her \mat A\Her \bvec v_i = \bvec v_i\Her \mat A \bvec v_i = \lambda_i^* \bvec v_i\Her \bvec v_i
\end{equation}
对比两矢, 得 $\lambda_i = \lambda_i^*$, 所以 $\lambda_i$ 必为实数.

\subsection{证明本征矢的正交性}
简并空间内, 我们可以认为地指定正交归一基底, 所以只需要证明不同本征值对应的本征矢正交即可.
\begin{equation}
s = \bvec v_1\Her (\mat A \bvec v_2) = \bvec v_1\Her (\lambda_2 \bvec v_2) = \lambda_2 \bvec v_1\Her \bvec v_2
\end{equation}
使用矩阵乘法结合律\autoref{Mat_eq1}\upref{Mat} 以及\autoref{HerMat_eq2}\upref{HerMat} 得
\begin{equation}
s = (\mat A \bvec v_1)\Her \bvec v_2 = \lambda_1^* \bvec v_1\Her \bvec v_2 = \lambda_1 \bvec v_1\Her \bvec v_2
\end{equation}
以上两矢相等, 因为 $\lambda_1 \ne \lambda_2$, 所以 $\bvec v_1\Her \bvec v_2 = 0$.

事实上, 厄米矩阵也可以定义为满足
\begin{equation}
\bvec v_1\Her (\mat A \bvec v_2) = (\mat A \bvec v_1)\Her \bvec v_2
\end{equation}
的矩阵.
