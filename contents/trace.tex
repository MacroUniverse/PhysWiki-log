% 矩阵的迹
% 矩阵|迹|相似变换|本征值

\pentry{相似变换和相似矩阵\upref{MatSim}}

\begin{definition}{矩阵的迹}
令 $N$ 维方阵 $\mat A$ 的矩阵元为 $a_{ij}$, 它的\textbf{迹(trace)}定义为它对角线上矩阵元之和
\begin{equation}
\opn{tr}(A) = \sum_{i=0}^N a_{ii}
\end{equation}
\end{definition}

\begin{theorem}{}
相似变换不会改变矩阵的迹.
\end{theorem}

\begin{corollary}{}
矩阵的迹等于它的所有 $N$ 个本征值 $\lambda_i$ 相加. 如果某个本征值有 $n$ 重简并, 就视为 $n$ 个本征值.
\begin{equation}
\opn{tr}(\mat A) = \sum_{i=0}^N \lambda_i
\end{equation}
\end{corollary}

\subsection{性质}
\begin{itemize}
\item 迹运算是\textbf{线性}的, 即
\begin{equation}
\opn{tr}(c_1\mat A+c_2\mat B) = c_1\opn{tr}(\mat A) + c_2\opn{tr}(\mat B)
\end{equation}

\item 矩阵乘法的迹满足($\mat A$ 和 $\mat B$ 不必是方阵, 但要求乘积是方阵. 注意 $\mat A\mat B$ 的尺寸和 $\mat B\mat A$ 未必相同)
\begin{equation}\label{trace_eq1}
\opn{tr}(\mat A\mat B) = \opn{tr}(\mat B\mat A)
\end{equation}
\end{itemize}

\subsubsection{证明}
令\autoref{trace_eq1} 中 $\mat A$ 为 $M\times N$ 的矩阵, $\mat B$ 为 $N\times M$ 的矩阵
\begin{equation}
\opn{tr}(\mat A\mat B) = \sum_{i=1}^M \sum_{k=1}^N a_{ik}b_{ki} = \sum_{i=1}^M \sum_{k=1}^N b_{ki}a_{ik} = \sum_{i=1}^N \sum_{k=1}^M b_{ik}a_{ki} = \opn{tr}(\mat B\mat A)
\end{equation}
证毕.

\subsubsection{证明相似变换不变性}
根据性质\autoref{trace_eq1}
\begin{equation}
\opn{tr}(\mat P^{-1}\mat A\mat P) = \opn{tr}(\mat A\mat P\mat P^{-1}) = \opn{tr}(\mat A)
\end{equation}
证毕.
