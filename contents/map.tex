% 映射
\pentry{集合\upref{Set}}
给定集合$A$和$B$,我们可以从$A$中每一个元素上拉一根线连接到$B$中的某一个元素,这些线的分布形式就被称为一个从$A$到$B$的\textbf{映射(mapping)}.将这个映射记为$f$,$A$中拉出线的元素组成的集合,叫做$f$的\textbf{定义域(domain)};$B$中被线连接到的元素的集合,叫做$f$的\textbf{值域(image)}.我们一般用“$f:A\rightarrow B$”表示“$f$是从$A$到$B$的映射”,也就是说从$A$的元素上拉线到$B$的元素上.如果没有特殊说明,这样的表示方法都默认$f$的定义域是整个$A$集合.

注意,映射是有方向区分的,比如在上面的例子中,$A$中每个元素只能拉一根线出去,而且每个元素都要拉一根(默认定义域是整个$A$),但是$B$中某些元素是可以被多根线连接的,也可以没有连接(即不在值域中).

如果映射$f:A\rightarrow B$中每个$B$中元素只被1根或者0根线连接,那么称$f$是一个\textbf{单射(injective)};如果$f:A\rightarrow B$中每个$B$中元素都被至少1根线连接,那么称$f$是一个\textbf{满射}(surjection). 如果$f$既是单射又是满射,那么称它为一个\textbf{双射}(bijective),或者叫\textbf{一一对应}.

\textbf{Cantor-Bernstein定理}显示,如果集合$A$到集合$B$上存在一个单射$f$和一个满射$g$,那么总可以利用$f$和$g$来构造出一个双射.

注意,双射是没有方向性的.$f:A\rightarrow B$中如果$f$是一个双射,那么$A$中每一个元素都唯一地连接到$B$中某一个元素,并且$B$中每一个元素也都唯一被$A$中某一个元素所连接,因此很明显可以将这个过程反过来,从$B$中向$A$中拉连接线.另外,如果$A$和$B$存在双射,意味着$A$和$B$的元素数量应该一致.

给定集合$A, B$,定义$B^A$为“从$A$到$B$的所有可能的映射所构成的集合”.如果$B$是一个二元集合,即它只有两个元素,不妨记为$B=\{0,1\}$,那么$B^A$可以用来表示$A$的幂集,即由$A$的所有子集所构成的集合.这是因为对于任意的$f\in B^A$,我们可以把这个$f$对应到$A$的子集$S$,其中$S$的元素全都被$f$映射到1上,$A-S$的元素全都被$f$映射到0上.当然,0和1的地位反过来也可以.由于这个特点,我们简单地把$A$的幂集记为$2^A$. 

对于映射$f:A\rightarrow B$,可以定义其逆映射$f^{-1}:B\rightarrow 2^A$. 逆映射$f^{-1}$将$B$中的每个元素映射到$A$的某个子集上(而不是$A$的某个元素上).特别地,不在$f$的值域中的元素被$f^{-1}$映射到空集上,而空集也是$A$的一个子集.如果$f$是一个双射,那么$f^{-1}$都映射在$A$的单元素子集上,那么我们也可以认为此时$f^{-1}$实际上是映射在单个元素上,也是从$B$到$A$的映射.
