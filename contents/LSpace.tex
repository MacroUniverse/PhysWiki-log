% 矢量空间
% 到底应不应该讲矢量空间呢? 真的应该让他们知道坐标和矢量本身不是一回事.
% 但是,力学分册涉及到的无非就是旋转矩阵,傅里叶级数(说不定都不用)
% 所以,按照数学从简的原则,就不要了! 估计量子力学里面才一定要讲.

\bb{矢量空间}(向量空间,线性空间)是一种特殊的\bb{集合},该集合内的元素都称为\bb{矢量}.这里的矢量是广义的矢量,我们熟悉的几何矢量只是其中的一种.

\subsection{定义}
矢量空间内有矢量的\bb{加法}(用“+”表示)和\bb{数乘}(省略乘号)两种的\bb{闭合运算},可作用于集合内任意矢量,结果也是矢量.“闭合”是指\bb{运算得到的矢量也属于同一个矢量空间}.两种运算的性质如下( $a,b$ 为任意两个实数,$\vec u,\vec v,\vec w$ 为空间中任意三个矢量)
\begin{enumerate}
\item 加法运算
\begin{enumerate}
\item 满足加法交换律 $\vec u + \vec v = \vec v + \vec u$.
\item 满足加法结合律 $(\vec u + \vec v) + \vec w = \vec u + (\vec v + \vec w)$.
\item 存在零矢量,使得 $\vec v + \vec 0 = \vec v$.
\item 存在逆元素 $\tilde{\vec v}$,使得 $\vec v + \tilde{\vec v} = \vec 0$.
\end{enumerate}

\item 数乘运算
\begin{enumerate}
\item 乘法分配律 $a(\vec u + \vec v) = a\vec u + a\vec v$ 
\item 乘法分配律 $(a + b)\vec v = a\vec v + b\vec v$
\item 乘法结合律 $a (b \vec v) = (ab) \vec v$
\end{enumerate}
\end{enumerate}

\subsection{例1}
所有不大于 $n$ 阶的多项式 $c_n x^n + c_{n-1} x^{n-1} + \dots + c_1 x + c_0$ 可以构成一个矢量空间.加法定义为代数加法运算,数乘定义为“多项式乘以常数”运算. 

\subsection{例2}
三维空间中所有具有方向和长度的量的几何矢量成一个矢量空间\footnote{注意这个定义中并没有涉及任何坐标系的选取}(只有起点不同的矢量视为同一个矢量).加法用矢量的三角形法则或平行四边形法则定义,任意几何矢量的数乘就是把该矢量的长度乘以一个常数,方向保持不变.证明略.

\subsection{点乘}
原则上点乘运算不是矢量空间所必须的,但物理中的矢量空间几乎都定义了点乘运算.不同的空间有不同的定义两个矢量点乘

\subsection{证明}
两个不大于 $n$ 阶的多项式相加还是不大于 $n$ 阶的多项式(加法闭合性),多项式相加满足交换律和结合律(1a,1b),零向量取 0 阶多项式 0,任何多项式加 0 不会改变(1c),每个多项式取相反数就可得到其逆矢量,且与原多项式相加为零(1d).多项式的数乘不改变阶数,仍在集合内(数乘的闭合性),显然满足两种乘法分配律和乘法结合律(性质2a,2b,2c).点乘运算按照点乘的几何定义%未完成
矢量空间的证明略.

\subsection{推论}
矢量空间中必须包含无穷多的矢量,否则不可能在数乘
由于加法运算和数乘运算都具有闭合性,线性组合 $c_1\vec v_1+\ldots +\c_n\vec v_n$ 也具有闭合性.

\subsection{线性相关和线性无关}

如果矢量空间内的一个矢量 能表示成空间内其他几个矢量 $\vec v_1, \vec v_2 \dots \vec v_n$ 的线性组合,
\begin{equation}
\vec u = c_1 \vec v_1 + c_2 \vec v_2 + \ldots + c_n \vec v_n \qquad (c_i \ne 0)
\end{equation} 
就说 $\vec u$ 与 $\vec v_1, \vec v_2 \dots \vec v_n$ 线性相关.但事实上,上式中 $\vec u$ 并不特殊,因为只要上式成立, $\vec v_i$ 中的任何一个也都能表示成其他几个矢量的线性组合.所以线性相关的概念可以变为: 对 $\vec v_1, \vec v_2 \dots \vec v_n$, 如果能找到不为零的系数 $c_i$使
\begin{equation}
c_1 \vec v_1 + c_2 \vec v_2+ \ldots + c_n \vec v_n = \vec 0
\end{equation} 
那么这 $\vec v_1, \vec v_2 \dots \vec v_n$ 线性相关.如果不存在非零的 $c_1, c_2 \dots c_n$,那么 $\vec v_1, \vec v_2 \dots \vec v_n$ 线性无关,即任何一个 $\vec v_i$ 都不能用其他几个矢量的线性组合表示.

\subsection{基底和维数}
如果一个矢量空间中存在 $N$ 个线性无关的非零矢量 $\vec x_1, \vec x_2 \dots \vec x_N$,使得空间中的任何矢量都可以用 $\vec x_1, \vec x_2 \dots \vec x_N$ 的线性组合表示(系数可以为零),那么 $\vec x_1, \vec x_2 \dots \vec x_N$ 就是这个空间的一组基底,且这个空间是一个 $\mathbf{N}$ \bb{维空间}.这个定义并没有歧义,因为可以证明,一个空间只要存在一组基底是 $N$ 个线性无关的矢量,那么该空间的任何基底都只能是 $N$ 个线性无关的矢量(证明略).
