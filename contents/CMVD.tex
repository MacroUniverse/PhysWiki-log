%匀速圆周运动的速度(求导法)

\pentry{矢量的导数\upref{DerV}}

%未完成:缺少图片

如图,在平面直角坐标系(单位矢量分别为 $\uvec x$,  $\uvec y$ )中,令一个绕原点做逆时针匀速圆周运动的质点的位矢为 $\vec r$, 圆周运动的半径为 $R$, 角速度为 $\omega $ (逆时针为正). $t = 0$ 时刻与 $x$ 轴夹角为 $0$. 那么任意时刻 将位矢 $\vec r$ 沿着 $x$ 与 $y$ 轴方向分解,则
\begin{equation}\label{CMVD_eq1}
\vec r = R\left( {\uvec x\cos \omega t + \uvec y\,sin\omega t} \right)
\end{equation}

其中 $\uvec x$ 是 $x$ 轴正方向的单位矢量, $\uvec y$ 是 $y$ 轴正方向的单位矢量.这样, $\vec r$ 就成了时间 $t$ 的函数,可直接求导.根据速度的定义, $\vec v = \D\vec r / \D t$ 即
\begin{equation}\label{CMVD_eq2}
\begin{split}
\vec v &= \frac{\D}{{\D t}}\left( {\uvec xR\cos \omega t + \uvec yRsin\omega t} \right)\\
 &= \uvec xR\frac{{\D\cos \omega t}}{{\D t}} + \uvec yR\frac{{\D\sin \omega t}}{{\D t}}\\
 &= - \uvec xR\omega \sin \omega t + \uvec yR\omega \cos \omega t\\
 &= R\omega \left( { - \uvec x\sin \omega t + \uvec y\cos \omega t} \right)
\end{split}
\end{equation}
对比\autoref{CMVD_eq1} 和\autoref{CMVD_eq2}, 比较位矢 $\vec r$ 和速度矢量 $\vec v$, 可以发现速度的大小 $\left| {\vec v} \right| = R\omega  = \left| {\vec r} \right|\omega $, 多了一个 $\omega $ 因子. $\vec v$ 与 $\uvec x$ 轴的夹角比 $\vec r$ 与 $\uvec x$ 轴的夹角大了,即逆时针转了 $90°$. 
