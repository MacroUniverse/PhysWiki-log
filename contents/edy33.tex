% 磁多极矩
% 磁多极矩|磁矢势|多级展开|磁矩

\pentry{电多极展开\upref{EMulPo}}
\begin{lemma}{分子电流观点}
对于物质磁性的解释,把每个宏观体积元内的分子看成完全一样的电流环,及具有同样的面积 $a$ 和取向(由面元矢量 $\bvec a$ 表示),环内具有同样的电流 $I$,而磁性由分子电流引发.故而后面\autoref{edy33_eq4} 可以把电流密度的体积分转变为电流的环路积分.
\end{lemma}
\subsection{矢势的多级展开}
磁场矢势在空间中定义为 % 链接未完成
\begin{equation}
\bvec A(\bvec x)=\dfrac {\mu_0}{4\pi} \int_v \dfrac{ \bvec J^{\prime}(\bvec x)\dd V^{\prime}}{R}\label{edy33_eq1}
\end{equation}
势能零点在无穷远点,而电流分布在小区域 $V$ 内,则,可以对体积 $V$ 内的 $\bvec A(\bvec x)$ 进行多级泰勒展开( $R$ 为 $\bvec x$ 到体积 $V$ 内的势能零点的距离)
\begin{equation}
\bvec A(\bvec x)=\dfrac {\mu_0}{4\pi}\int_V \bvec J^{\prime}(\bvec x)\Big[\dfrac{1}{R}-\bvec x^{\prime} \cdot\div \dfrac{1}{R}+\dfrac{1}{2!}\sum_{i,j}x_i^{\prime}x_j^{\prime}\pdv{}{x_i}{x_j}{\dfrac 1 R}+\cdots\Big]\dd \bvec V^{\prime}\label{edy33_eq2}
\end{equation}
\autoref{edy33_eq2} 中,第一项
\begin{equation}
\bvec A^{(0)}(\bvec x)=\dfrac {\mu_0}{4\pi R}\int_V \bvec RJ^{\prime}(\bvec x)\dd{ V^{\prime}}\label{edy33_eq3}
\end{equation}
由于电流的连续性,% 引用未完成
把电流分为许多闭合的流管,对于每一个流管来说,有
\begin{equation}
\int_V \bvec J(\bvec x^{\prime}) \dd{V^{\prime}}=\oint_L I\dd{l}=I\oint_L \dd{l}=0\label{edy33_eq4}
\end{equation}
得\begin{equation}
\bvec A^{(0)}=0\label{edy33_eq5}
\end{equation}
对展开式第二项
\begin{equation}
\bvec A^{(1)}=-\dfrac{\mu_0 }{4\pi}\int_V \bvec J(\bvec x^{\prime})\bvec x^{\prime}\cdot\div\dfrac1 R\dd{V^{\prime}}\label{edy33_eq6}
\end{equation}
由\autoref{edy33_eq4} 得
\begin{equation}
\bvec A^{(1)}=-\dfrac{\mu_0 I }{4\pi}\int_l \bvec x^{\prime}\cdot\div\dfrac1 R\dd{V^{\prime}}=\dfrac{\mu_0 I }{4\pi}\int_l \bvec x^{\prime}\cdot\dfrac{\bvec R}R^3\dd l^{\prime}\label{edy33_eq7}
\end{equation}
其中有 $\dd \bvec x^{\prime}=\dd l^{\prime}$, 因为 $ \bvec x^{\prime}$ 为闭合流管上的坐标.
全微分绕闭合回路积分为 $0$
\begin{equation}
0=\oint_L \dd{(\bvec x^{\prime}\cdot \bvec R)\bvec x^{\prime}}=\oint_L (\bvec x^{\prime}\cdot\bvec R)\dd l^{\prime}+\oint_l (\dd l^{\prime}\cdot \bvec R)\bvec x^{\prime}\label{edy33_eq8}
\end{equation}
而
\begin{equation}
\oint_L (\bvec x^{\prime}\cdot \bvec R)\dd{l^{\prime}}=\dfrac1 2\oint_L (\bvec x^{\prime}\times \dd l^{\prime})\times \bvec R \label{edy33_eq9}
\end{equation}

\begin{equation}
\bvec A^{(1)}=\dfrac{\mu_0}{4\pi R^3}\cdot\dfrac{I}{2}\oint_L (\bvec x^{\prime}\times \dd{l^{\prime}})\times \bvec R=\dfrac{\mu_0}{4\pi}\dfrac{\bvec m \times \bvec R}{R^3}\label{edy33_eq10}
\end{equation}
其中 $\bvec m =\dfrac{1}{2}\oint_L (\bvec x^{\prime}\times \dd{l^{\prime}})$ 被称为磁矩,而对于一个小线圈(分子电流),他所围的面元 $\Delta\bvec S$ 可以表示为
\begin{equation}
\Delta\bvec S=\dfrac 1 2\oint_L \bvec x^{\prime}\times \dd{\bvec l^{\prime}}
\end{equation}
故而,有等式
\begin{equation}
\bvec m=I\Delta \bvec S
\end{equation}
