% testsss

% Hi,谢谢您的创作,为了更好地交流协作,百科需要申请称为志愿者才可以编辑,请先通过 wuli.wiki 申请

连续则极限值等于函数值.

\subsection{求连续区间}

若要考察一个函数的连续区间,必须要了解函数的所有部分,一般会给出分段函数,所以要了解分段函数的每段函数的性质.

对于函数$f(x)$是个极限表达形式,我们要简化这个极限,最好得到一个$x$的表达式,从而才能判断其连续区间.

\textbf{例题:}$f(x)=\lim\limits_{n\to\infty}\dfrac{x+x^2e^{nx}}{1+e^{nx}}$,求函数连续区间.

注意到函数的形式为一个极限值,其极限趋向的变量为$n$($n\to\infty$指$n\to+\infty$).所以在该极限式子中将$x$当作类似$t$的常数.

需要先求出极限形式的$f(x)$,而$x$变量的取值会影响到极限,且求的就是$x$的取值范围.所以将其分为三段:

当$x<0$时,$nx\to-\infty$,$\therefore e^{nx}\to 0$,$x^2$在这个极限式子为一个常数,$\therefore x^2e^{nx}\to 0$,$f(x)=\lim\limits_{n\to\infty}\dfrac{x+x^2e^{nx}}{1+e^{nx}}=\dfrac{x+0}{1+0}=x$.

当$x=0$时,$f(x)=\lim\limits_{n\to\infty}\dfrac{x+x^2e^{nx}}{1+e^{nx}}=\dfrac{0}{2}=0$.

当$x>0$时,$e^{nx}$在$n\to\infty$时为$\infty$,上下都有这个无穷大的因子,所以上下都除以$e^{nx}$,$f(x)=\lim\limits_{n\to\infty}\dfrac{x+x^2e^{nx}}{1+e^{nx}}=f(x)=\lim\limits_{n\to\infty}\dfrac{xe^{-nx}+x^2}{1+e^{-nx}}=\dfrac{0+x^2}{1}=x^2$.

从而得到了$f(x)$关于$x$的表达式:

$f(x)=\left\{\begin{array}{lcl}
        x,   &  & x<0 \\
        0,   &  & x=0 \\
        x^2, &  & x>0
    \end{array}
    \right.$

又$\lim\limits_{x\to 0^-}f(x)=\lim\limits_{x\to 0^-}x=\lim\limits_{x\to 0^+}f(x)=\lim\limits_{x\to 0^-}x^2=f(0)=0$.

$f(x)$在$R$上连续.

\subsection{已知连续区间求参数}

一般会给出带有参数的分段函数,要计算参数就必须了解连续区间与函数之间的关系.

\textbf{例题:}$f(x)=\left\{\begin{array}{lcl}
        6,                               &  & x\leqslant 0 \\
        \dfrac{e^{ax^3}-1}{x-\arcsin x}, &  & x>0
    \end{array}
    \right.$,$g(x)=\left\{\begin{array}{lcl}
        \dfrac{3\sin(x-1)}{x-1}, &  & x<1          \\
        e^{bx}+1,                &  & x\geqslant 1
    \end{array}
    \right.$ \\ 若$f(x)+g(x)$在$R$上连续,则求$a,b$.

解:已知$f(x)+g(x)$在$R$上连续,但是不能判断$f(x)$与$g(x)$的连续性.

所以分开讨论.

对于$f(x)$因为左侧为常数函数,所以若是$f(x)$连续,则必然:

$\lim\limits_{x\to 0^+}\dfrac{e^{ax^3}-1}{x-\arcsin x}=6$

$\therefore\lim\limits_{x\to 0^+}\dfrac{e^{ax^3}-1}{x-\arcsin x}$

$=\lim\limits_{x\to 0^+}\dfrac{ax^3}{x-\arcsin x}$

$\text{令}t=\arcsin x\Rightarrow=\lim\limits_{x\to 0^+}\dfrac{a\sin^3t}{\sin t-t}$

$=a\lim\limits_{x\to 0^+}\dfrac{t^3}{\sin t-t}$

$=a\lim\limits_{x\to 0^+}\dfrac{3t^2}{\cos t-1}=-6a=6$.

$\therefore a=-1$时$f(x)$在$R$上连续.

对于$g(x)$,当$x<1$时,$\lim\limits_{x\to 1^-}\dfrac{3\sin(x-1)}{x-1}=\lim\limits_{t\to 0^-}\dfrac{3\sin t}{t}=3$.

$\therefore\lim\limits_{x\to 1^+}e^{bx}+1=e^b+1=3$.

$\therefore b=\ln 2$时$g(x)$在$R$上连续.

$\therefore a=-1,b=\ln 2$时$f(x)+g(x)$在$R$上连续.而$a\neq -1$时$f(x)+g(x)$在$x=0$时不连续,$b\neq\ln 2$时$f(x)+g(x)$在$x=1$时不连续.

\subsection{求间断点}

求间断点需要首先分析函数的表达形式.

\textbf{例题:}设$f(x)=\lim\limits_{n\to\infty}\dfrac{1+x}{1+x^{2n}}$,求其间断点并分析其类型.

根据函数形式,我们需要首先回顾一下幂函数的性质,幂函数的变化趋势取决于底数.

当$x=1$时,$x^n\equiv 1$,当$x\in(-\infty,-1)\cup(1,+\infty)$时,当$n\to\infty$时,$x^n\to\infty$,而$x\in(-1,1)$时,当$n\to\infty$时,$x^n\to 0$.

$\therefore\lim\limits_{n\to\infty}\dfrac{1+x}{1+x^{2n}}=\left\{\begin{array}{lcl}
        0,   &  & x\in(-\infty,-1]\cup(1,+\infty) \\
        1,   &  & x=1                                        \\
        x+1, &  & x\in(-1,1)
    \end{array}
    \right.$

所以分段点为$x=\pm 1$.

当$x=-1$时,$f(-1^+)=f(-1^-)=f(-1)=0$,所以在此处连续.

当$x=1$时,$f(1^+)=0\neq f(1^-)=2$,所以在此处简短,为跳跃间断点.

\subsection{已知间断点求参数}

这种题目已知间断点,而未知式子中的参数,只用将间断点代入式子并利用极限计算间断点的类型就可以了.

\textbf{例题:}$f(x)=\dfrac{e^x-b}{(x-a)(x-b)}$有无穷间断点$x=e$,可去间断点$x=1$,求$ab$的值.

已知有两个间断点$x=a,x=b$,其中无穷间断点指极限值为无穷的点,可去间断点表示极限值存在且两侧相等,但是与函数值不相等的点.

已经给出两个间断点的值为$x=1$和$x=e$,所以$ab$必然对应其中一个,但是不清楚到底谁是谁.

当$a=1,b=e$时,$f(x)=\dfrac{e^x-e}{(x-1)(x-e)}$.

当$x\to 1$时,$\lim\limits_{x\to 1}\dfrac{e^x-e}{(x-1)(x-e)}$

$=\dfrac{1}{1-e}\lim\limits_{x\to 1}\dfrac{e^x-e}{x-1}$

$=\dfrac{e}{1-e}\lim\limits_{x\to 1}\dfrac{e^{x-1}-1}{x-1}$

$=\dfrac{e}{1-e}\lim\limits_{x\to 1}\dfrac{x-1}{x-1}$

$=\dfrac{e}{1-e}$.

$\therefore x=1$为可去间断点.

    当$x\to e$时,$\lim\limits_{x\to e}\dfrac{e^x-e}{(x-1)(x-e)}$
    
    $=\dfrac{1}{e-1}\lim\limits_{x\to e}\dfrac{e^x-e}{x-e}$
    
    $=\dfrac{e}{e-1}\lim\limits_{x\to e}\dfrac{e^{x-1}-1}{x-e}$ $=\dfrac{e}{e-1}\lim\limits_{x\to e}\dfrac{x-1}{x-e}$
    
    $=\dfrac{e(e-1)}{e-1}\lim\limits_{x\to e}\dfrac{1}{x-e}=\infty$.

$\therefore x=e$为无穷间断点.

    当$a=e,b=1$时,$f(x)=\dfrac{e^x-1}{(x-e)(x-1)}$.

    而作为分子的$e^x-1$必然为一个常数,当式子趋向$1$或$e$的时候分母两个不等式中的一个不等式必然为一个常数,从而另一个不等式则变为了无穷小,所以$\lim\limits_{x\to 1}f(x)=\lim\limits_{x\to e}f(x)=\infty$.

$\therefore a=1,b=e$.
