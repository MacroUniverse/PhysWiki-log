% 一元函数的微分

\pentry{导数\upref{Der}}

考察一个连续光滑的函数 $y = f\left( x \right)$, 在 ${x}$ 处函数值为 ${y}$, 若此时函数增加一个无穷小量 $\D x$, 函数值会相应增加无穷小量 $\D y$. 根据导数的定义\upref{Der} $f'({{x}}) = \D y/\D x$, 我们将 $\D y$ 与 $\D x$ 的关系记为
\begin{equation}\label{Diff_eq1}
\D y = f'\left( {{x}} \right)\D x
\end{equation}
这就是一元函数的\textbf{微分}.注意一元函数的求导和微分除了表达方式不同外并无太大区别.从形式上来看,微分是微小变化量之间的线性关系,而导数则强调变化率.

\subsection{微分近似}
严格来说,类似\autoref{Diff_eq1} 的微分关系式默认取极限 $\D x \to 0$ 才能使等号成立,但只要在一定范围 $\Delta x$ 内导函数 $f'(x)$ 的变化非常小,就可以将函数值的变化量 $\Delta y = f(x+\Delta x)-f(x)$ 近似为
\begin{equation}
\Delta y \approx f'(x) \Delta x
\end{equation}
注意在近似式中不能出现微分符号 $\D$, 也不能使用等号.
% 未完成:此处应有一张图

\begin{exam}{测量误差}\label{Diff_ex1}
若测得立方体的边长为 $a$, 测量的最大可能误差为 $\sigma_a$ (可以假设 $\sigma_a \ll a$), 估计立方体体积的最大误差 $\sigma_V$.

立方体的体积与边长的关系为 $V(a)=a^3$, 根据微分近似,有
\begin{equation}
\sigma_V \approx V'(a) \sigma_a = 3a^2 \sigma_x
\end{equation}
\end{exam}




