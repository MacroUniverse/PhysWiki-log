% 反开普勒问题

\pentry{双曲线的三种等效定义\upref{Hypb3},天体运动\upref{CelBd}}

考虑电荷为 $q$,质量为 $m$ 的轻质点从无穷远处以 $v_i$ 入射到电荷为 $Q$ (同种电荷)的重质点(假设始终静止).轻质点的运动轨迹为双曲线的一支,重质点的位置为离轨道较远的焦点(证明见下文).

定义\textbf{碰撞参量}为双曲线的渐近线到焦点的距离,若轻质点一直做匀速直线运动,则碰撞参量就是两质点的最近距离.由双曲线的性质,%链接未完成
碰撞参量等于双曲线的参数 $b$.

现在推导轨道参数 $a,b$ 与能量 $E$ 和角动量 $L$ 的关系.轨道离焦点的最近距离为
\begin{equation}\label{InvKep_eq1}
r_0 = a+c
\end{equation}
对应的速度为 $v_0$,由能量守恒得
\begin{equation}
\frac{1}{2} mv_i^2 = \frac{1}{2} mv_0^2 + \frac{kQq}{r_0}
\end{equation}
由角动量守恒得
\begin{equation}
m v_i b = m v_0 r_0
\end{equation}
双曲线的三个参数满足
\begin{equation}\label{InvKep_eq4}
a^2+b^2=c^2
\end{equation}
联立\autoref{InvKep_eq1} 到\autoref{InvKep_eq4} 解得
\begin{equation}
E = \frac{kQq}{2a} \qquad L = b\sqrt{\frac{2E}{m}} =b\sqrt{\frac{kQq}{ma}}
\end{equation}
注意若把平方反比力的系数 $kQq$ 换成 $GMm$,上式的结论与行星的双曲线轨道%链接未完成
一致.

\subsection{轨道方程推导}
%\pentry{比耐公式\upref{Binet}}
%由比耐公式得
% 未完成
