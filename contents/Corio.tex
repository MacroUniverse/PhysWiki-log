%科里奥利力

\pentry{惯性力\upref{Iner},离心力\upref{Centri},平面旋转矩阵,矢量的叉乘\upref{Cross}}

\subsection{结论}
\textbf{科里奥利力(Coriolis Force)}是匀速旋转的参考系中的由相对速度产生的惯性力.
\begin{equation}
{\vec F_{cori}} = 2m{\vec v_{abc}} \cross \vec \omega
\end{equation}
其中 $\vec v_{abc}$ 是质点相对于旋转系的瞬时速度, $\vec\omega$ 是旋转系的恒定角速度矢量.%未完成: 考虑使用脚注或链接
在匀速转动参考系(属于非惯性系)中,若质点保持相对静止,则惯性力只有离心力.然而当质点与转动参考系有相对速度时,惯性力中还会增加一个与速度垂直的力,这就是科里奥利力.地理中的地转偏向力就是科里奥利力,可用上式计算.其中 是地球的自转角速度.

\subsection{推导}
这里的主要思路是按照惯性力\upref{Iner}中的坐标法.设空间中存在一个惯性系 $xyz$ 和一个非惯性系 $abc$ 相对于 $xyz$ 绕 $z$ 轴以角速度 $\omega$ 逆时针匀速旋转(右手定则).%链接未完成
由于 $z$ 轴和 $c$ 轴始终重合( $z=c$), 只需要考虑 $x,y$ 坐标和 $a,b$ 坐标之间的关系即可.

令平面旋转矩阵为% 未完成:链接
\begin{equation}
\mat R(\theta) \equiv \begin{pmatrix}
{\cos \theta }&{ - \sin \theta }\\
{\sin \theta }&{\cos \theta }
\end{pmatrix}
\end{equation}
其意义是把坐标逆时针旋转 角 $\theta$.
则其逆矩阵等于转置矩阵
\begin{equation}
\mat R(- \theta) = \mat R (\theta)\Tr =  \begin{pmatrix}
{\cos \theta }&{\sin \theta }\\
{ - \sin \theta }&{\cos \theta }
\end{pmatrix}
\end{equation}
意义是顺时针旋转 $\theta$ 角.所以两坐标系之间的坐标变换为
% 未完成: 为什么矢量的括号那么粗?
\begin{equation}
\pmat{a\\b} = \mat R(\omega t)\Tr \pmat{x\\y}
\qquad
\pmat{x\\y} = \mat R(\omega t) \pmat{a\\b}
\end{equation}
为了得到质点在惯性系中的加速度,对上式的 $(x,y)\Tr$ 求二阶全导数得\footnote{本词条中,某个量上方加一点表示对时间的一阶导数,两点表示对时间的二阶导数.}
\begin{equation}\label{Corio_eq1}
\pmat{\ddot x \\ \ddot y} = 
\ddot{\mat R}(\omega t) \pmat{a\\b} + 2\dot{\mat R} (\omega t) \pmat{\dot a \\ \dot b} + \mat R\left( {\omega t} \right)\pmat{\ddot a \\ \ddot b}
\end{equation}
其中\footnote{\autoref{Corio_eq2} 和\autoref{Corio_eq3} 相当于用矩阵推导了圆周运动的速度和加速度公式\upref{CMVG}\upref{CMAG}.}
\begin{equation}\label{Corio_eq2}
\dot{\mat R}(\omega t) = \omega \begin{pmatrix}
{\cos \left( {\omega t + \pi /2} \right)}&{ - \sin \left( {\omega t + \pi /2} \right)}\\
{\sin \left( {\omega t + \pi /2} \right)}&{\cos \left( {\omega t + \pi /2} \right)}
\end{pmatrix} = \omega \mat R\left( {\omega t + \pi /2} \right)
\end{equation}
\begin{equation}\label{Corio_eq3}
\ddot{\mat R} (\omega t)  =  - {\omega ^2}\mat R (\omega t)
\end{equation}
 代入\autoref{Corio_eq1} 得
\begin{equation}
\pmat{\ddot x \\ \ddot y} =
- {\omega ^2}\mat R(\omega t)\pmat{a\\b} + 2\omega \mat R(\omega t + \pi /2)\pmat{\dot a \\ \dot b} + \mat R(\omega t)\pmat{\ddot a \\ \ddot b}
\end{equation}
上式中的每一项都是 $xyz$ 参考系中的坐标.所有坐标顺时针旋转 $\omega t$, 得到 $abc$ 参考系中的坐标.首先由旋转矩阵的几何意义得 $\mat R(\theta_1)\mat R(\theta_2) = \mat R(\theta_1 + \theta_2)$
\begin{equation}
\pmat{\ddot x \\ \ddot y}_{abc} =
\mat R(-\omega t ) \pmat{\ddot x \\ \ddot y} =
- {\omega^2} \pmat{a\\b} + 2\omega \mat R(\pi /2)\pmat{\dot a\\ \dot b} + \pmat{\ddot a\\ \ddot b}
\end{equation}
所以旋转参考系中的总惯性力(\autoref{Iner_eq9} \upref{Iner})为(用 $abc$ 系中的坐标表示)
\begin{equation}\label{Corio_eq10}
\vec f = m({\vec a}_{abc} - {\vec a}_{xyz})
= \pmat{\ddot a\\ \ddot b} - \pmat{\ddot x \\ \ddot y}_{abc} = m{\omega ^2}\pmat{a\\b} - 2m\omega \mat R(\pi /2)\pmat{\dot a\\ \dot b}
\end{equation}
其中第一项是已知的离心力\upref{Centri}
\begin{equation}
m{\omega ^2} \begin{pmatrix} a\\b \end{pmatrix} = m{\omega ^2}\vec r
\end{equation}
我们将第二项定义为\textbf{科里奥利力}.由叉乘的几何意义\upref{Cross},科里奥利力可以用叉乘记为
\begin{equation}
- 2m\omega \mat R(\pi /2)\pmat{\dot a\\ \dot b} = 2m{\vec v_{abc}} \cross \vec \omega
\end{equation}
其中 $\vec\omega$ 是 $abc$ 系旋转的角速度矢量, $\vec v_{abc}$ 是质点相对于 $abc$ 系的速度.最后,矢量形式的\autoref{Corio_eq10} 为
\begin{equation}
\vec f = m{\omega ^2}\vec r + 2m{\vec v_{abc}} \cross \vec \omega 
\end{equation}
 




