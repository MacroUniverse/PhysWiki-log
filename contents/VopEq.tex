% 矢量算符常用公式
% 矢量算符|微积分|叉乘|点乘|内积

\begin{issues}
\issueAbstract
\end{issues}

% 为了避免抄袭, 可以换一些符号以及顺序

\pentry{梯度\upref{Grad}, 旋度\upref{Curl}}

\footnote{参考教材: David Griffiths, Introduction to Electrodynamics}这里列举一些常用的矢量算符恒等式.

\subsection{一阶}
\subsubsection{梯度}
\begin{equation}\label{VopEq_eq2}
\grad (fg) = f\grad g + g \grad f
\end{equation}

\begin{equation}
\grad (\bvec A \vdot \bvec B) = \bvec A \cross (\curl \bvec B) + \bvec B \cross (\curl \bvec A) + (\bvec A \vdot \grad) \bvec B + (\bvec B \vdot \grad) \bvec A
\end{equation}

\subsubsection{散度}

\begin{equation}\label{VopEq_eq1}
\div (f \bvec A) = f (\div \bvec A) + \bvec A \vdot (\grad f)
\end{equation}

\begin{equation}
\div (\bvec A \cross \bvec B) = \bvec B \vdot (\curl \bvec A) - \bvec A \vdot (\curl \bvec B)
\end{equation}

\subsubsection{旋度}
\begin{equation}
\curl(f\bvec A) = f (\curl\bvec A) + (\grad f) \cross\bvec A
\end{equation}

\begin{equation}
\curl (\bvec A \cross \bvec B) = (\bvec B \vdot \grad) \bvec A - (\bvec A \vdot \grad) \bvec B + \bvec A (\div \bvec B) - \bvec B (\div \bvec A)
\end{equation}

\subsection{二阶}

\begin{equation}
\div (\grad f) = \laplacian f
\end{equation}

\begin{equation}
\laplacian \bvec v = (\laplacian v_x) \uvec x + (\laplacian v_y) \uvec y + (\laplacian v_z)\uvec z
\end{equation}

\begin{equation}
\curl (\grad f) = \bvec 0
\end{equation}

\begin{equation}
\div (\curl \bvec v) = 0
\end{equation}

\begin{equation}
\curl(\curl \bvec v) = \grad(\div\bvec v) - \laplacian \bvec v
\end{equation}

\subsection{证明}
理论上, 我们可以直接根据定义, 将各个矢量记为分量的形式证明, 但直接写出来非常繁琐. 一种简单的记号是使用狄拉克 delta 函数 $\delta_{i,j}$ 和 Levi-Civita 符号 $\epsilon_{i,j,k}$, 再结合爱因斯坦求和约定% 未完成
.

我们另外推荐一种不需要写出分量的推导方法, 见 “一种矢量算符的运算方法\upref{MyNab}”.
