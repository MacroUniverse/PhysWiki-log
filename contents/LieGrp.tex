% 李群
% Lie group|李代数|lie 代数|lie algebra|切空间|群|微分几何|转动|对称

\pentry{拓扑群\upref{TopGrp},流形\upref{Manif}}

\begin{issues}
\issueDraft
\end{issues}


\subsection{李群的概念}

当拓扑群的拓扑部分构成光滑流形时,所得到的拓扑群被称为李群.

\begin{definition}{实李群}
给定一个实光滑流形 $G$,若在 $G$ 上定义了一个群运算“$\cdot$”,且满足流形空间之间的映射 $f:G\times G\rightarrow G$ 是一个光滑映射,其中 $f(g_1, g_2)=g_1\cdot g_2^{-1}$,那么称 $G$ 是一个\textbf{李群(Lie group)}.
\end{definition}

\begin{example}{实李群的例子}
\begin{enumerate}
\item \textbf{拓扑群}\upref{TopGrp}词条中\autoref{TopGrp_ex1}~\upref{TopGrp}所举的两个例子,实数轴 $\mathbb{R}$ 和单位圆 $S^1$ 都是实流行,都构成实李群.
\item \textbf{拓扑群}\upref{TopGrp}词条中\autoref{TopGrp_ex2}~\upref{TopGrp}所举的例子,一般线性群,是一个实李群.
\item 一般线性群 $\opn{GL}(k, \mathbb{R})$ 的子群,正交群 $\opn{O}(k, \mathbb{R})$,以及特殊正交群 $\opn{O}(k, \mathbb{R})$\footnote{正交群指的是全体保度量线性变换构成的群,等价于全体行列式的绝对值是 $1$ 的矩阵构成的乘法群.特殊正交群是正交群的子群,是由其中行列式为 $+1$ 的矩阵构成的.}都是实李群.
\end{enumerate}
\end{example}

为了加深对李群的印象,我们再举出一个重要的反例:

\begin{definition}{环面\footnote{参考 Wikipedia \href{https://en.wikipedia.org/wiki/Lie_group}{相关页面}.}}
记 $\mathbb{T}=S^1\times S^1$ 为李群 $S^1$ 的积李群,即其群乘法为 $S^1$ 的群乘法的积,流形为 $S^1$ 的流形的积.取 $\mathbb{T}^2$ 的子群 $H=\{(\E^{\theta\I}, \E^{a_0\theta\I})|\theta\in\mathbb{R}\}$,其中 $a_0$ 是一个无理数.

按以上方式定义的群 $H$ 是一个拓扑群,但不是流形,因而\textbf{不是李群}.这是因为无理数 $a_0$ 导致 $H$ 的图像是环面上密绕的一条线,在任何一个点的任何一个开邻域里都有无数条不连通的线段,因此任何点处都无法找到同构于欧几里得空间的\textbf{图}.

不过我们依然可以在群 $H$ 上构造出李群来.把 $H$ 的拓扑进行如下改动:定义映射 $\phi:\mathbb{R}\to H$,其中 $\phi(\theta)=(\E^{\theta\I}, \E^{a_0\theta\I})$;再利用 $\phi$ 来构造 $H$ 的拓扑为“$U\subseteq H$ 为开集当且仅当 $\phi^{-1}(U)$ 为 $\mathbb{R}$ 的开集”.这样定义的拓扑空间 $H$ 就是一个同胚于 $\mathbb{R}$ 的实李群了.
\end{definition}






\subsection{和李代数的联系}

李代数虽然是对李群性质的抽象,但是其本身可以脱离李群来定义,参见\textbf{李代数}\upref{LieAlg}词条.事实上,学有余力的学生完全可以在熟悉了线性代数的概念之后直接进入李代数的学习,无需流形的知识\footnote{我国李群和李代数专家朱富海教授一直致力于向低年级本科生科普李代数知识,他在bilibili.com和知乎都有发布“给大一学生的 Lie 代数”系列,可参见\href{https://space.bilibili.com/509086270?from=search&seid=2394735306274350134和https://zhuanlan.zhihu.com/p/161735986}{这里}.(2021年1月更新).}.

李代数是李群在单位元 $e$ 处的切空间.




