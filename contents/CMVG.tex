%匀速圆周运动的速度(几何法)

\pentry{小角正弦值极限\upref{LimArc},速度的定义\upref{VnA}}

设一个点 $A$ 做半径为 $R$ 的圆周运动,恒定的角速度为 $\omega $, 那么经过一段微小时间 $\Delta t$ 以后,点 $A$ 转过的角度为 $\Delta \theta  = \omega \Delta t$. 粗略地说,当 $\omega $ 不算太大时, $\Delta \theta $ 也是一个微小量.这样,根据小角正弦值极限\upref{LimArc},当 $\Delta t$ 趋近于 $0$ 时,物体在 $\Delta t$ 内走过的位移长度(线段的长度)趋近于弧的长度,即 $\abs{\Delta \vec s}$ 趋近于 $R\omega \Delta t$. 

根据速度的定义 
\begin{equation}
\vec v = \lim_{\Delta t \to 0} \dv{\vec s}{t}
\end{equation}
速度的大小为
\begin{equation}
\vec v = \lim_{\Delta t \to 0} \frac{\abs{\Delta \vec s}}{\Delta t} = \frac{R\omega \Delta t}{\Delta t} = R\omega 
\end{equation}
速度的方向显然与过 $A$ 点的圆的切线重合.











