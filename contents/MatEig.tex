% 矩阵的本征方程

\pentry{线性方程组\upref{GAUSS}, 行列式\upref{Deter}}

若已知矩阵 $\mat A$, 我们把线性方程组
\begin{equation}\label{MatEig_eq1}
\mat A \bvec v = \lambda \bvec v
\end{equation}
称为矩阵 $\mat A$ 的\textbf{本征方程}. 通常 $\mat A$ 是已知的, 而 $\lambda$ 和 $\bvec v$ 是未知的. 显然, 当 $\bvec v = \bvec 0$ 时方程恒成立, 所以我们通常只对非零解感兴趣. 也就是说, 我们希望找到一些非零矢量 $\bvec v$, 使得矩阵 $\mat A$ 乘以该矢量以后方向不变\footnote{“方向” 只是从几何矢量中沿用过来的一个习惯说法, 注意\autoref{MatEig_eq1} 中的所有量都可以是复数. 两个矢量方向相同意味着一个矢量乘以标量可以得到另一个.}. 对于每个这样的矢量, 我们用一个标量 $\lambda$ 来描述其模长的改变. 我们把这些矢量叫做\textbf{本征矢量}, 把对应的 $\lambda$ 叫做\textbf{本征值}.

若令 $\mat I$ 为 $N\times N$ 的单位矩阵\footnote{即对角线上的元为 1, 其他元为 0, 见“矩阵\upref{Mat}”}, 则本征方程本质上是一个其次方程
\begin{equation}
(\mat A - \lambda\mat I)\bvec v = \bvec 0
\end{equation}
括号中的矩阵相当于把矩阵 $\mat A$ 的对角线上的元都减去 $\lambda$ 得到的方阵. 要判断方程有没有非零解, 只需看该方阵的行列式是否为零. 当且仅当 $\abs{\mat A - \lambda\mat I} = 0$ 时, 方程存在 $N$ 个非零解.
