% 偏导数
\pentry{导数\upref{Der}}

对一个多元函数 $y = f({x_1},{x_2}\dots{x_i}\dots)$,如果求导时只把 $x_i$ 看成自变量,剩下的 $x_{j \ne i}$ 都看做常数,得到的导数就叫函数(关于 $x_i$)的\textbf{偏导数}.以二元函数 $z=f(x,y)$ 为例,对 $x$ 的偏导数常记为
\begin{equation}\label{ParDer_eq1}
\frac{\partial z}{\partial x} \qquad \frac{\partial f}{\partial x} \qquad f_x  \qquad \left(\frac{\partial f}{\partial x}\right)_y
\end{equation}
最后一种记号在括号右下角声明了保持不变的自变量, 这在许多情况下能避免混淆.

\begin{exam}{}\label{ParDer_ex1}
对于函数 $f(x,y) = {x^2} + 2{y^2} + 2xy$, 两个偏导数分别为
\begin{equation}
\frac{{\partial f}}{{\partial x}} = 2x + 2y  \quad  \frac{{\partial f}}{{\partial y}} = 4y + 2x
\end{equation}
\end{exam}

\begin{exam}{}\label{ParDer_ex2}
对于函数 $z = \sin (y\cos x) + {\cos ^2}x$
\begin{align}
\frac{{\partial z}}{{\partial x}} &=  - y\cos (y\cos x)\sin x - 2\cos x\sin x =  - y\cos (y\cos x) - \sin 2x\\
\frac{{\partial z}}{{\partial y}} &= \cos (y\cos x)\cos x
\end{align}
\end{exam}

\subsection{几何意义}
类比导数的几何意义(曲线的斜率), 若在三维直角坐标系中画出曲面 $f(x,y)$,则 $\partial f/\partial x$ 和 $\partial f/\partial y$ 分别是是某点处曲面延 $x$ 方向和 $y$ 方向的斜率.所以从某点 $({x_0},{y_0})$ 延 $x$ 方向移动一个微小量 $\Delta x$,假设曲面平滑,则函数值增加
\begin{equation}
\Delta f \approx \frac{{\partial f}}{{\partial x}}\Delta x
\end{equation}
写成微分关系就是
\begin{equation}
\D f = \frac{{\partial f}}{{\partial x}}\D x \quad (y \text{\ 不变})
\end{equation}

\subsection{高阶偏导}
与一元函数的高阶导数类似,多元函数也可以求高阶偏导数,不同的是,由于每求一次偏导都需要指定对哪个变量.例如二元函数的二阶偏导有
\begin{equation}
\frac{{{\partial ^2}f}}{{\partial {x^2}}} \qquad
\frac{{{\partial ^2}f}}{{\partial x\partial y}} \qquad
\frac{{{\partial ^2}f}}{{\partial y\partial x}} \qquad
\frac{{{\partial ^2}f}}{{\partial {y^2}}}
\end{equation}
若高阶偏导的分母中出现不止一个变量,我们就称其为\textbf{混合偏导}.混合偏导的一个重要性质就是偏导的顺序可以任意改变,例如上式中有 ${\partial ^2}f/\partial x\partial y = {\partial ^2}f/\partial y\partial x$. 这点本书不做证明,可以通过以上的例子验证.








