% 多通道散射

多通道散射(multi-channel scattering)研究的是非弹性散射(inelastic scattering)。 例如我们用一个电子或入射到一个原子或者粒子上, 将总波函数表示为
\begin{equation}
\Psi = \sum_i \Phi_i (l, m, \vec r_2, \vec r_3, \dots) \psi_i (r)
\end{equation}
其中 $\psi_i(r)$ 表示入射电子的径向波函数, $\Phi_i$ 表示其他田子的波函数以及入射电子的量子数 $l, m$。

关于费米子波函数的反对称, 我们假设 $\Phi$ 满足 $\vec r_2, \vec r_3 \dots$ 等电子的反对称条件。 先忽略总波函数反对称的要求, 如果需要反对称, 只需用反对称化算符处理即可。

对于弹性散射,虽然我们我们解微分方程得到的两种线性无关的球面波通常是实函数, 如 $j_l(kr), y_l(kr)$ (代表驻波), 但这里我们需要的是行波, 就是类似 $h^{(\pm)}(kr) = j_l(kr) \pm y_l(kr)$ 这种, 同样是两个。 而 $h_l^{(+)}(kr)$ 代表出射波, $h_l^{(-)}$ 代表入射波。 $j_l = (h_l^{(+)} + h_l^{(-)})/2$, 这样就是一进一出。

对于弹性散射, 我们可以把 $h_l^{(+)} + h_l^{(-)}$ 叠加得到球面波。 对非弹性散射的本征态, 我们也首先希望得到从某个通道(channel)$i$ 入射的球面波 $f_i^{(-)}$ 加上所有通道的出射波 $f_f^{(+)}$。
