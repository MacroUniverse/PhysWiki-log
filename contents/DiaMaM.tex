% 抗磁质的磁化
% 抗磁质|轨道磁矩|分子

在抗磁质中,每个原子或分子中所有电子的\textbf{轨道磁矩(orbital magnetic moment)}和\textbf{自旋磁矩(spin magnetic moment)}的矢量和等于零,在外磁场$\mathbf B_0$中电子轨道运动的平面在磁场中会发生进动,而且其轨道角动量进动的方向在任何情况下都是沿着磁场的方向,和电子轨道运动的速度方向无关,并在同一外磁场$\mathbf B_0$中都以相同的角速度进动.因此,这时抗磁质中每个分子或原子中所有的电子形成一个整体绕外磁场的进动,从而产生一个附加磁矩$\Delta\mathbf{m}_{mole}$,$\Delta\mathbf{m}_{mole}$的方向与$\mathbf B_0$的方向相反,大小与$\mathbf B_0$的大小成正比.这样,抗磁材料在外磁场的作用下,磁体内任一体积元中大量分子或原子的附加磁矩的矢量和$\sum \Delta\mathbf{m}_{mole}$有一定的量值,结果在磁体内激发一个和外磁场方向相反的附加磁场,这就是抗磁性的起源.抗磁性既然起源于外磁场对电子轨道运动作用的结果,应该在任何原子或分子的结构中都会产生,因此它是一切磁介质所共有的性质.