% 装备希尔伯特空间

     按照通常的说法, 量子力学的基本舞台是\textbf{Hilbert 空间}, 即完备的内积空间. 本节中都用$\mathcal{H}$来代表所要考虑的 Hilbert 空间, 用$\langle\cdot,\cdot\rangle$来代表其上的内积. 然而在这里, 我们要考虑的数学对象比 Hilbert 空间更精细一些: 我们的基本舞台不再是一般的 Hilbert 空间, 而是所谓的\textbf{装备 Hilbert 空间 (rigged Hilbert space, RHS)}. 为了定义装备 Hilbert 空间, 首先要定义\textbf{核空间 (nuclear space)}. 它是 Alexander Grothendieck 的博士论文研究的内容. 我们这里采用如下较狭窄但已经足够用的定义:

\begin{definition}{核空间 (Nuclear Space)}
一个核空间$\Phi$是一族递降的 Hilbert 空间$\{\Phi_n\}_{n=0}^\infty$的交, 赋予自然的拓扑, 这族Hilbert 空间满足如下条件: (1) $\Phi_{n+1}$的范数比$\Phi_n$的范数更强, 而且$\Phi_n$是$\Phi_{n+1}$在$\Phi_n$的范数之下的完备化; (2) 对于任何$n$, 总有$m>n$使得嵌入映射$T_{m,n}:\Phi_{m}\hookrightarrow\Phi_n$是核算子 (nuclear operator), 即它的迹范数是良好定义的.
\end{definition}

      从此以后, 沿用 Dirac 的左右矢记号: 以$\langle\phi|$代表$\Phi$的元素, $|u\rangle$代表其对偶空间$\Phi^*$的元素, 二者之间的配对就以$\langle\phi|u\rangle$代表. 今后还要考虑包含映射$\Phi\hookrightarrow\mathcal{H}$, 它自然诱导出包含映射$\mathcal{H}\hookrightarrow\Phi^*$. 为方便计, 对于$f\in\Phi\subset\mathcal{H}$, 以$\langle f|$表示之; 对于$g\in\mathcal{H}$, 以$|g\rangle$表示$g$在包含映射$\mathcal{H}\hookrightarrow\Phi^*$下的像.

      

\begin{definition}{装备 Hilbert 空间}
装备 Hilbert 空间包含一可分 Hilbert 空间$\mathcal{H}$与一稠密子空间$\Phi\subset\mathcal{H}$, 满足如下条件: $\Phi$上有一更强的拓扑$\tau_\Phi$, 使得$(\Phi,\tau_\Phi)$成为一核空间, 而且有$\Phi_0=\mathcal{H}$, 且自然的包含映射$\Phi\hookrightarrow\mathcal{H}$在此拓扑下连续.
\end{definition}

    注意到自然的三重偶$\Phi\hookrightarrow\mathcal{H}\hookrightarrow\Phi^*$与$\mathcal{H}$的内积相容, 即如果$\langle f|\in\Phi\subset\mathcal{H},g\in\mathcal{H}$, 则$\langle f,g\rangle_{\mathcal{H}}=\langle f|g\rangle$. 此三重偶称为 \textbf{Gelfand 三重偶 (Gelfand triple)}.

    为什么要如此大费周章地考虑 Gelfand 三重偶? 原来, 它的原型乃是 Hilbert 空间语言中的位置表象空间$L^2(\mathbb{R}^n)$以及其上定义的算子$x,i\nabla$即位置算子和动量算子; 这两个算子的定义域都不是整个$L^2(\mathbb{R}^n)$, 它们最自然的定义域乃是 Schwartz 函数空间$\mathcal{S}(\mathbb{R}^n)$, 即所有光滑且各阶导数都迅速衰减的函数的空间, 赋予 Fréchet 空间拓扑. 位置算子和动量算子在 Schwartz 空间上都是连续算子. 而 Schwartz 空间$\mathcal{S}(\mathbb{R}^n)$本身是加权的 Sobolev 空间系
$$\Phi_k := \qty{f \in L^2 (\mathbb{R}^n): (-\Delta + |x|^ 2)^{k} f \in L^2 (\mathbb{R}^n) }$$的交, 而包含映射$T_{m+n,m}:\Phi_{m+n}\to\Phi_m$是核算子. 这时就有 Gelfand 三重偶$$
\mathcal{S}(\mathbb{R}^n)\hookrightarrow L^2(\mathbb{R}^n)\hookrightarrow \mathcal{S}'(\mathbb{R}^n).
$$也可以注意到, 上述定义的 Sobolev 范数中的算子$-\Delta+|x|^2$正是谐振子的薛定谔算符.
      这个三重偶应该就包含了所有物理上感兴趣的对象. 实际上, 它显然囊括了平方可积的束缚态, 即可归一化的波函数; 也囊括了不平方可积的散射态, 例如自由单色波$e^{ik\cdot x}$. 前者属于$L^2(\mathbb{R}^n)$, 而后者应该用 Schwartz 分布即$\mathcal{S}'(\mathbb{R}^n)$的元素来代表. 另外, 物理上的"位置本征态"即$\delta(x-x_0)$显然并不是数学意义下的函数, 但它属于对偶空间$\mathcal{S}'(\mathbb{R}^n)$, 而且当然满足$$
x^i\delta(x-x_0)=x_0^i,
$$因为对于任何$\phi\in\mathcal{S}(\mathbb{R}^n)$, 按照定义皆有$$
\langle\phi|\left[x^i|\delta(x-x_0)\rangle\right]:=\left[\langle\phi|x^i\right]|\delta(x-x_0)\rangle
=x_0^i\phi(x_0)=\langle\phi|\left[x_0^i|\delta(x-x_0)\rangle\right].
$$由此可见, 动量本征态和位置本征态正是"不可归一化的态矢量"的例子.