% 刚体的运动方程

\pentry{刚体的平面运动方程\upref{RBEM}}

一般情况下下刚体的运动方程要比平面运动复杂得多, 但我们仍然可以将运动分解为质心的运动以及刚体绕质心的旋转, 前者由合力决定, 所以仍然有(\autoref{RBEM_eq1}\upref{RBEM})
\begin{equation}
M\bvec a_c = \sum_i \bvec F_i
\end{equation}
所以相对于平面运动, 该问题的困难在于绕质心转动的计算. 虽然角动量定理仍然满足, 但转动惯量将有可能随时间变化. 这是因为刚体有可能沿不同的转轴转动, 而不同的转轴对应的角动量一般不同.

% 未完成: 这个话题我真的不熟啊. 瞬时转轴, 
% 能否先用的思想来说明? 就是把时间分成许多 \Delta t, 每个 dt 假设某些量为常数, 等等
% 陀螺问题: 之前的假设是转动惯量很大, 但事实上陀螺是除了进动还会张动的.

\subsection{平动动能与转动动能}
\pentry{柯尼希定理\upref{Konig}}
由柯尼希定理, 刚体的动能可以分为\textbf{平动动能}和\textbf{转动动能}两部分. 下面来证明一段时间内, 平动动能的增加等于合外力关于质心位移的做功, 而转动动能的增加等于质心系中合外力矩的做功.
(未完成!)

\subsection{运动方程}
我们仍然可以用角动量定理来推导刚体的运动方程, 但这里的角动量要用惯性张量来表示(\autoref{ITensr_eq3}\upref{ITensr} 和\autoref{ITensr_eq4}\upref{ITensr})
\begin{equation}
\bvec L = \mat I \bvec \omega = \mat R \mat I_0 \mat R\Tr \bvec \omega
\end{equation}
其中 $\mat I_0$ 不随时间变化, $\bvec L$, $\bvec \omega$ 和 $\bvec R$ 都是时间的函数. 带入角动量定理得(\autoref{AMLaw_eq1}\upref{AMLaw})
\begin{equation}\label{RBEqM_eq3}
\bvec \tau = \dv{\bvec L}{t} = \dv{t} (\bvec R\mat I_0 \mat R\Tr) \bvec \omega + \mat R \mat I_0 \mat R\Tr \dv{\bvec \omega}{t}
\end{equation}
注意这里对矩阵求导就是对每个元分别求导.

我们把力矩 $\bvec \tau$ 看作是一个已知函数, 把旋转矩阵 $\mat R$ 和角速度 $\bvec \omega$ 看做未知的(微分方程的解). $\mat R$ 和 $\bvec \omega$ 完整描述了刚体绕固定点转动的\bb{状态}, 就像位置和动量可以完整描述了一个质点运动的状态.

另外, $\bvec \omega$ 和 $\bvec R$ 之间的关系就像位移和速度的关系, 假设体坐标系中固定在刚体上的任意一点为 $\bvec r$, 变换到实验室坐标系中为 $\mat R \bvec r$. 对时间求导得该点得速度为
\begin{equation}\label{RBEqM_eq1}
\bvec v = \dv{\mat R}{t} \bvec r
\end{equation}
而角速度和速度之间有 $\bvec v = \bvec \omega \cross \bvec r$(\autoref{CMVD_eq5}\upref{CMVD}). 我们可以把叉乘用矩阵乘法表示为 % 未完成: 再叉乘词条里面提及这点
\begin{equation}\label{RBEqM_eq2}
\bvec v = \mat\Omega \bvec r
\qquad
\mat\Omega = \pmat{0 & -\omega_z & \omega_y \\ \omega_z & 0 & -\omega_x\\ -\omega_y & \omega_x & 0}
\end{equation}
由于 $\bvec r$ 是任意的, 对比\autoref{RBEqM_eq1} 和\autoref{RBEqM_eq2} 得
\begin{equation}\label{RBEqM_eq4}
\dv{\mat R}{t} = \mat\Omega
\end{equation}
将\autoref{RBEqM_eq3} 整理得(注意 $\mat R \mat I_0 \mat R\Tr$ 得逆矩阵是 $\mat R \mat I_0^{-1} \mat R\Tr$)% 链接未完成
\begin{equation}\label{RBEqM_eq5}
\dv{\bvec \omega}{t} = \mat R \mat I_0^{-1} \mat R\Tr \qty[\bvec \tau  - \dv{t} (\bvec R\mat I_0 \mat R\Tr) \bvec \omega]
\end{equation}
\autoref{RBEqM_eq4} 和\autoref{RBEqM_eq5} 就是完整的运动方程. 这是一个一阶常微分方程组\upref{ODEsys}, 写成标量的形式共有 12 条, 未知数分别为 $\omega_x, \omega_y, \omega_z$, $R_{i,j}$ 共 12 个.

事实上旋转矩阵 $\mat R$ 其实只有三个独立的自由度, 如果我们能用三个变量表示 $\mat R$, 就可以得到只含 6 个未知数的 6 个方程. 一种方法是使用欧拉角, 但列出来后式子会比较复杂. 另一种方法是用 4 元数, 即用 4 个变量表示 $\mat R$, 可以得到形式相对简单的方程(见\upref{RBEMQt}).
