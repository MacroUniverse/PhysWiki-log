% 高阶同伦群
\pentry{基本群\upref{HomT3}}

基本群的符号被定义为$\pi_1$.你可能好奇为什么要有一个下标$1$——这是因为基本群只是同伦群中的一种,而仅仅用$\pi_1$群来描述一个拓扑空间往往是不够的.

考虑三个拓扑空间:二维球面(如地球表面)$S^2$,挖去中心的三维球体(如挖去地心的地球)$D^3-x_0$以及$n$维几何空间$\mathbb{R}^n$.容易看出,这三个空间中的任意道路都彼此同伦,因此对任何基点来说都只存在一个回路类,意味着它们的基本群都是平凡群(只有一个元素).这样一来,对于空间中有几个洞、空间的维度是什么等信息就丢失了,我们就需要推广基本群的概念来描述这些性质.这就是高阶同伦群.

\subsection{球和球面}
在$n$维欧几里得空间中,集合$\{(x_1, x_2,\cdots,x_n)|\sum^n_{i=1}x_i^2\leq 1\}$也被称为半径为$1$的$n$维球体,记为$B^n$;集合$\{x_1, x_2, \cdots, x_n|\sum^n_{i=1}x_i^2=1\}$是$B^n$的表面,记为$\partial B^n$,或$S^{n-1}$.记号$\partial B^n$表达的是“$B^n$的边界\upref{Topo0}”,而$S^{n-1}$表达的是“$n$维球面”.



\subsection{$2$阶同伦群}



