%定积分

\pentry{导数\upref{Der},极限\upref{Lim}}

首先以不均匀细绳的质量为例,引入定积分的思想

\begin{exam}{不均匀细绳的质量}\label{DefInt_ex1}
一条密度不均匀的绳子长为 $L$, 横截面积是 $S$, 细绳距离 O 端 $x (x<L) $ 处的密度为 $\rho(x)$. 求绳子的质量.

\begin{figure}[ht]
\centering
\includegraphics[width=9cm]{./figures/Defint1.pdf}
\caption{密度不均匀的绳子}
\end{figure}

如果题目中,密度是恒定的,那么直接可以写出绳子的质量为 $m = LS\rho$. 但是题中 $\rho(x)$是关于 $x$ 的函数,所以我们要寻找另外的做法.假设绳子的密度变化是连续且“平滑”的,我们可以通过把绳子分割成 $n$ 小节(注意这些小节必须严格地首尾相接,不能有重合或者空隙).第 $i$ 节取 $x_i$ 到 $x_i +1$, 令其长度为 $x_{i + 1} - {x_i} = \Delta x_i$ 使每一个小节内,密度可以近似看成是恒定的,这样我们可以用 $\rho(\xi _i)\,\, (x_i \le \xi _i \le x_{i + 1})$ 来代替第 $i$ 节的密度,当每一节足够小时,可以认为 ${\xi _i}$ 在 ${x_i} \le {\xi _i} \le {x_{i + 1}}$ 约束下的取值并不会影响结果.
第 $i$ 小节的质量为
\begin{equation}
\Delta {m_i} = \rho (\xi _i)\Delta {x_i}S 
\end{equation}
所以总的质量用求和符号来表示,就是
\begin{equation}
m = \sum_{i = 1}^n \Delta m_i  \approx \sum_{i = 1}^n \rho(\xi _i)\Delta x_i S   = S \sum_{i = 1}^n \rho (\xi _i)\Delta x_i
\end{equation}
由于当 $n$ 取有限值时,上式并不精确成立,所以只能使用约等号,但是 $n$ 越大,约等号两边就越精确成立.这是极限的思想,用极限符号来表,就是
\begin{equation}
m = \lim_{n \to \infty } \sum_{i = 1}^n {\Delta {m_i}}  = \lim_{n \to \infty } \sum_{i = 1}^n {\rho(\xi _i)\Delta {x_i}S  }  = S   \lim_{n \to \infty } \sum_{i = 1}^n {\rho(\xi _i)\Delta {x_i} } 
\end{equation}
这种表达式在物理中反复出现,所以使用积分符号 $\int {} $ 用于代替极限和求和符号.另外把 ${\xi _i}$ 写成 $x$ (当 $n$ 趋近于无穷大时,参量 $i$ 和 $\Delta {x_i}$ 具体是多少就不重要了),把表示增量的 $\Delta $ 变为表示微小量的 $\D$, 上式就写为
\begin{equation}
m = \int \dd{m}  = \int S\rho(x) \dd{x}  = S\int \rho(x) \dd{x}
\end{equation}
下面先看看 $\int \rho(x) \dd{x}$, 即 $\lim\limits_{n \to \infty } \sum\limits_{i = 1}^n \rho(\xi _i)\Delta {x_i}$的另一种理解.画出 $\rho (x)$图像.例如 $\rho(x) = x + 1$, 则 $\rho(\xi _i)\Delta {x_i}$可以表示左图的第 $i$ 个小长方形的面积, $\sum\limits_{i = 1}^n \rho(\xi _i)\Delta {x_i}$表示长方形面积之和.如果 $n$ 非常大且每个 $\Delta {x_i}$ 取得非常小,左图看起来就会像右图. 所以 $\int \rho(x) \dd{x}$ 可以用来表示右图阴影部分的面积.

\begin{figure}[ht]
\centering
\includegraphics[width=12cm]{./figures/Defint2.pdf}
\end{figure}

但 $\int \rho(x) \dd{x}$ 里面显然不包含 $0$ 和 $L$ 的信息,我们根据题目中的情况,说这个积分是“从 $0$ 积到 $L$”,其中 $0$ 是积分下限,$L$ 是积分上限.为了表示这个信息,把它写到积分号右边变为
\begin{equation}
\int_0^L \rho(x) \dd{x}
\end{equation}
这就是定积分的标准形式,但有时候为了书写方便,在不混淆的情况下可以把积分上下限省略.
\end{exam}

这样的写法是很形象的,可以想象,积分号就是函数的曲线需要积分的部分,下标的位置代表曲线的起点,上标代表曲线的终点.这样,物理中很多问题就可以用积分表示了.

要注意的是,根据上面积分的定义,如果曲线在 $x$ 轴的下方,面积应该表示成负值.但根据\autoref{DefInt_ex1} 的物理情景,可知密度不可能是负值.

至于计算积分的具体方法,比求导要复杂得多,甚至很多积分的结果不能用初等函数表示,只能表示为级数等形式.然而对于基本初等函数的积分,用牛顿-莱布尼兹公式\upref{NLeib} 即可马上求解.

% 未完成:这个词条应该放很多很多物理实例!这么重要的概念!
