% 共振

% not finished
% 提示: 测试阶段, 请自行做好备份。 等开放注册后使用账户名登录后写文章。

我们知道原子存在能级,可以用特定频率的光去激发,而这种行为也可以一般的描述为共振。Wikipedia说

In mechanical systems, resonance is a phenomenon that only occurs when the frequency at which a force is periodically applied is equal or nearly equal to one of the natural frequencies of the system on which it acts. This causes the system to oscillate with larger amplitude than when the force is applied at other frequencies.

而对应到原子里面,这个force就是周期的电磁场,而oscillation则是原子在基态和激发态之间振荡。对应的,这个振荡不会仅仅发生在共振频率,其它频率也会存在振荡只不过振幅不如共振点大。那么,当频率偏离共振后,振幅的变化其实反映了这个共振的“好坏”:如果稍微偏离一点,振幅就大幅衰减,那么说明这个共振非常锐利而精确,几乎只发生在共振点,这样的共振我们说它有更高的“品质”。具体说来,我们用品质因数(quality factor)来定义:
\begin{align}
Q = \frac{f}{\Delta f}
\end{align}
其中,而$f$是共振频率,$\Delta f$是共振的带宽(bandwidth),对于不同的共振线性有不同的定义\footnote{比如,对于Lorentzian这种有无限大的方差的线型来说就是半高全宽(Full Width at Half Maximum, FWHM),而对于Gaussian线型来说就是它的标准差。}。

举一个例子:对于一般的室温原子气体,这个参数基本由室温的热运动造成的多普勒展宽决定。原子的能级一般在可见光波段,即$f\sim 500$THz。对应的,假如原子序数为100,室温$T=300$K,对应的最可几速率为$\bar{v}=\sqrt{2k_BT/m}\sim300$m/s,从而对应了一个$\Delta f=f\bar{v}/c\sim500\times10^{-6}$THz$=500$MHz的展宽。这个对应了品质因子$Q\sim10^6$。这个展宽一般大于常规原子能级的展宽(在$10$MHz量级),从而会把原子本身能级的性质掩盖掉。

常见的品质好($Q>100$)的共振有石英石的共振$Q\sim10^4-10^6$,也就是说现在的石英表在一年里一般也就误差$5-10$分钟量级,而机械表就要差一个档次了,很可能到一个小时。此外,人工制造的微观结构,比如回音壁腔,能达到$10^9$的品质因子,是目前人类构造出的最高的品质因子层级。而地球自转的品质因子在$10^7$水平,脉冲星自转在$10^{10}$。

共振有不同的线型。刚才说的热展宽,由于原子速度是玻尔兹曼分布,所以是Gaussian线型;而原子自己的能级则是Lorentzian线型,它对应于谐振子的阻尼振荡\cite{SHOfF}。

















