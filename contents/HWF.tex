% 类氢原子的波函数

\pentry{玻尔原子模型\upref{BohrMd}} % 预备知识未完成

\bb{类氢原子} 被定义为原子核有 $Z$ 个质子(电荷为 $+Ze$, 有一个核外电子的原子/离子, 例如氢原子和失去一个电子的氦原子, 失去两个电子的锂离子. % 这个定义应该放在玻尔模型里面

类氢原子的定态薛定谔方程为
\begin{equation}\label{HWF_eq1}
-\frac{\hbar^2}{2m} \laplacian \psi(\vec r) - \frac{Z}{r} \psi(\vec r) = E \psi(\vec r)
\end{equation}
类氢原子是唯一存在解析解的原子(离子).

我们这里只讨论束缚态, 即 $E < 0$ 的解.  从数学上, $E$ 取小于零的任意值时我们都能找到解, 但只有当 $E$ 取特定离散值的时候这些波函数才能归一化. 由于类氢原子具有球对称性, 球坐标下的波函数具有最简单的形式. 波函数的表达式为
\begin{equation}\label{HWF_eq3}
\psi_{nlm} (r,\theta ,\phi) = R_{nl}(r) Y_{l,m}(\theta, \phi)
\end{equation}
其中 $n$ 是\bb{主量子数}($n = 1, 2, \dots$), $l$ 是\bb{角量子数}($l = 0, 1, \dots, n - 1$), $m$ 是\bb{磁量子数}($m = -l, -l+1, \dots, l$). $R_{nl}(r)$ 是归一化的\bb{径向波函数}, $Y_{l,m}(\theta, \phi)$ 是归一化的\bb{球谐函数}(见“球谐函数列表\upref{YlmTab}”).

如果忽略原子核的运动, 以下的 $a$ 是玻尔半径, 如果不忽略, $a$ 就是约化玻尔半径.% 两个链接未完成  

\subsubsection{径向波函数 $R_{nl}(r)$}

注意 $Z$ 和 $a$ 的作用是把径向波函数关于原点收缩 $Z/a$ 倍(并保持波函数归一化).
\begin{equation}\label{HWF_eq2}
R_{nl}(r) = \sqrt{\qty(\frac{2 Z}{na})^3 \frac{(n - l - 1)!}{2n (n + l)!}} \E^{-Zr/(na)} \qty(\frac{2Zr}{na})^l  L_{n-l-1}^{2l+1}\qty(\frac{2Zr}{na})
\end{equation}
其中 $L_n^l(x)$ 是\bb{连带拉盖尔多项式(associated Laguerre polynomial)}. 以下给出前几个径向波函数

\begin{equation}
n = 1 \qquad
R_{10}(r) = 2\qty(\frac{Z}{a})^{3/2}\exp(-Zr/a)
\end{equation}
\begin{equation}
n = 2 \qquad
\leftgroup{
R_{20}(r) &= \frac{1}{\sqrt 2} \qty(\frac{Z}{a})^{3/2} \qty(1 - \frac12 \frac{Zr}{a}) \exp(-\frac{Zr}{2a})\\
R_{21}(r) &= \frac{1}{\sqrt{24}} \qty(\frac{Z}{a})^{3/2} \frac{Zr}{a} \exp(-\frac{Zr}{2a})
}\end{equation}
\begin{equation}
n = 3 \qquad
\leftgroup{
R_{30}(r) &= \frac{2}{\sqrt {27}} \qty(\frac{Z}{a})^{3/2} \qty(1 - \frac23 \frac{Zr}{a} + \frac{2}{27} \frac{Z^2r^2}{a^2}) \exp(-\frac{Zr}{3a})\\
R_{31}(r) &= \frac{8}{27\sqrt 6} \qty(\frac{Z}{a})^{3/2} \qty(1 - \frac16 \frac {Zr}{a}) \frac {Zr}{a} \exp(-\frac{Zr}{3a})\\
R_{32}(r) &= \frac{4}{81\sqrt {30}} \qty(\frac{Z}{a})^{3/2} \frac{Z^2r^2}{a^2} \exp(-\frac{Zr}{3a})}
\end{equation}

\subsection{动量表象\ 动量分布}
要求动量表象下的波函数, 我们需要将位置表象的波函数投影到归一化的动量的本征矢上, 即三维傅里叶变换
\begin{equation}
\psi_{nlm}(\vec p) = \braket{\vec p}{\psi} = \frac{1}{\sqrt{2\pi}} \int \exp(-\I \vec p \vdot \vec r/\hbar) \psi(\vec r) \dd[3]{r}
\end{equation}
这个积分在球坐标中完成才是最方便的, 具体方法我们将举例子说明(见\autoref{Pl2Ylm_ex1}\upref{Pl2Ylm}).

正如位置表象下位置的分布函数是 $\abs{\psi(\vec r)}^2$, 动量表象下动量的分布函数是 $\abs{\psi(\vec p)}^2$ (也符合测量理论).
