%氢原子的波函数

\pentry{波函数简介}%未完成链接
氢原子的波函数在球坐标中表示为
\begin{equation}
\psi_{nlm} (r,\theta ,\phi) = R_{nl}(r) Y_l^n(\theta, \phi)
\end{equation}
其中 $n$ 是\bb{主量子数}, $l$ 是\bb{角量子数}, $m$ 是\bb{磁量子数}. $R_{nl}$ 是归一化的\bb{径向波函数}, $Y_l^m$ 是归一化的\bb{球谐函数}.

\subsubsection{径向波函数 $R_{nl}(r)$}
\begin{equation}
R_{nl}(r) = \sqrt{\qty(\frac{2}{na})^3 \frac{(n - l - 1)!}{2n [(n + l)!]^3}} \E^{-r/na} \qty(\frac{2r}{na})^l  L_{n-l-1}^{2l+1}\qty(\frac{2r}{na})
\end{equation}
\begin{equation}
n = 1 \qquad
R_{10} = 2a^{-3/2}\exp(-r/a)
\end{equation}
\begin{equation}
n = 2 \qquad
\leftgroup{
R_{20} &= \frac{1}{\sqrt 2} a^{-3/2} \qty(1 - \frac12 \frac{r}{a}) \exp(-\frac{r}{2a})\\
R_{21} &= \frac{1}{\sqrt{24}} a^{-3/2} \frac{r}{a} \exp(-\frac{r}{2a})
}\end{equation}
\begin{equation}
n = 3 \qquad
\leftgroup{
R_{30} &= \frac{2}{\sqrt {27}} a^{-3/2} \qty(1 - \frac23 \frac ra + \frac{2}{27} \frac{r^2}{a^2}) \exp(-\frac{r}{3a})\\
R_{31} &= \frac{8}{27\sqrt 6} a^{-3/2} \qty(1 - \frac16 \frac ra) \frac ra \exp(-\frac{r}{3a})\\
R_{32} &= \frac{4}{81\sqrt {30}} a^{- 3/2} \frac{r^2}{a^2} \exp(-\frac{r}{3a})}
\end{equation}

\subsubsection{球谐函数 $Y_l^m$}
\begin{equation}
Y_l^m(\theta, \phi) = \varepsilon\sqrt{\frac{2l + 1}{4\pi} \frac{(l - \abs{m})!}{(l + \abs{m})!}} \E^{\I m\phi} P_l^m (\cos\theta)
\end{equation}
\begin{equation}
l = 0 \qquad
Y_0^0 = \qty(\frac{1}{4\pi})^{1/2}
\end{equation}
\begin{equation}
l = 1 \qquad
\leftgroup{
Y_1^0 &= \qty(\frac{3}{4\pi})^{1/2} \cos\theta \\
Y_1^{\pm 1} &= \mp\qty(\frac{3}{8\pi})^{1/2} \sin\theta \cdot \E^{\pm\I\phi}
}\end{equation}
\begin{equation}
l = 2 \qquad
\leftgroup{
Y_2^0 &= \qty(\frac{5}{16\pi})^{1/2} (3\cos^2 \theta  - 1)\\
Y_2^{\pm1} &= \mp \qty(\frac{15}{8\pi})^{1/2} \sin\theta \cos\theta \cdot \E^{ \pm \I\phi}\\
Y_2^{\pm 2} &= \qty(\frac{15}{32\pi})^{1/2} \sin ^2\theta  \cdot \E^{\pm 2\I\phi}
}\end{equation}

