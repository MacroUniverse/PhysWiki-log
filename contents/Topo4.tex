% 道路连通性
\pentry{连续映射和同胚\upref{Topo1},连通性\upref{Topo3}}
连通性的概念还不够直观,因此我们还有一个更强的性质,道路连通性.连通性的定义思路是用不相交的开集来分割不连通的各部分,而道路连通性的定义思路是用一条连续的曲线来连接任意两点,如果不连通则这条曲线应该会被截断.

\subsection{道路连通性的概念}
\begin{definition}{道路}\label{Topo4_def1}
在区间$I=[0,1]$上取通常的度量(子)拓扑.对于任意的拓扑空间$X$,如果存在一个连续映射$f: I\rightarrow X$,那么称该映射的像$f(I)$为$X$中的一条\textbf{道路(path)}.
\end{definition}

道路的定义要更直观一些,而且在别的数学分支也会出现,比如复变函数中会经常讨论复平面上的道路.需要注意的是,这里定义道路所用的区间是闭区间,也就是说它有起点和终点.可以把$I$看成是从$0$到$1$流逝的时间,而对应的$f(I)$就是这段时间里,一个点在$X$中连续运动的轨迹.这个点运动的“速度”不会是无穷大,否则就不是连续映射了;这样,在有限时间内,这个运动轨迹必然也是“有限”的.这里打双引号是因为“速度”和“有限”在一般情况下只是一个类比,只有在度量空间里面我们才可以讨论长度,进而有了“速度”的概念.

\begin{example}{道路的例子}\label{Topo4_ex1}
\begin{itemize}
\item 在度量空间$\mathbb{R}$中,$I$本身就是一条道路.任何闭区间都是一条道路.
\item 在$\sin{\frac{1}{x}}$\footnote{见\autoref{Topo3_ex3}.}空间中,对于任何$a, b>0$,从$x=a$到$x=b$的一段函数图像也构成道路
\item 还是在$\sin{\frac{1}{x}}=S$空间中,不过考虑的是$\bar{S}=S\cup\{(0, y)|y\in [-1,1]\}$,就是在$S$上再添加$x=0$处的一条长度为$2$的线段.如果选择$\{(0, y)|y\in [-1,1]\}$上的任意一点作为起点/终点,而选择$S$上的一点作为终点/起点,那么中间需要走一条无限长的曲线才能让二者相连,而道路都是有限的,所以这样的两点之间就没有道路.

\end{itemize}
\end{example}

道路的概念可以用来引出一个更直观的连通性:

\begin{definition}{道路连通性}\label{Topo4_def2}
设拓扑空间$X$和它的子集$A$.如果$\forall a, b\in A$,总能找到一条道路连接$a$和$b$,那么称$A$是\textbf{道路连通(path-connected)}的.
\end{definition}

但是道路连通不等价于连通,连通的子集是可能不道路连通的.我们在\autoref{Topo3_ex3}中定义的$\sin{\frac{1}{x}}$空间的闭包空间$\bar{S}$就是一个反例.$\bar{S}$是连通的,因为$S$连通,根据闭包的连通性\autoref{Topo3_the2},$\bar{S}$也得连通;但是由\autoref{Topo4_ex1}给出的反例,我们知道$\bar{S}$不是道路连通的.

但是道路连通的子集一定连通.

\begin{theorem}{道路连通性推出连通性}\label{Topo4_the1}
对于拓扑空间$X$及其子集$A$,如果$A$是道路连通的,那么$A$必是连通的.
\end{theorem}

\textbf{证明}

设$A$是道路连通但不是连通的.那么必然有$A\subset U\cup V$,其中$U, V$和$A$的交集都非空.
取$a\in U$和$b\in V$,那么由道路连通性,必然存在一条道路$f:I\rightarrow A$,使得$f(0)=a, f(1)=b$.由于道路是连续映射,我们有$f^{-1}(U)$和$f^{-1}(V)$都是$I$的非空开子集,而且$I=f^{-1}(U)\cup f^{-1}(V)$.由于$I$是连通的,它不可能是两个非空开子集的并.因此矛盾,所以$A$必须是连通的.

\textbf{证毕}

从这个证明的思路可以看出,道路连通性本质上是在描述每一条道路的连通性,由于道路的端点是任意取的,这就要求$A$本身必须连通了.

\subsection{道路连通性的性质}

道路连通性也是一个同胚不变量.

\begin{theorem}{道路连通性的同胚不变性}
设有拓扑空间$X$和$Y$,令$f:X\rightarrow Y$是一个连续映射,那么,对于$X$的任何道路连通子集$A$,$f(A)$也是$Y$的道路连通子集.
\end{theorem}