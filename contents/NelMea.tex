% Nelder-Mead 算法

\pentry{Matlab}% 未完成

Nelder-Mead 算法是一种求多元函数局部最小值的算法, 其优点是不需要函数可导并能较快收敛到局部最小值. Matlab 自带的 \x{fminsearch} 函数就是使用该算法. 对 $N$ 元函数 $f(\vec x)$ (这里把函数自变量用 $N$ 维矢量来表示), 该算法需要提供函数自变量空间中的一个初始点, $\vec x_1$, 算法从该点出发寻找局部最小值. 以下讲解具体算法.

我们先根据初始点另外生成另外 $N$ 个初始点 $\vec x_2\dots\vec x_{N + 1}$, 使 $\vec x_{1 + i}$ 在第的第 $i$ 个分量比 $\vec x_1$ 的大 5\%, 其他分量保持相同. 如果 $\vec x_1$ 的第 $i$ 个分量为零, 那么 $\vec x_{1 + i}$ 的第 $i$ 个分量设为 $0.00025$. 得到 $N+1$ 个初始点后, 开始按照以下步骤进行循环, 直到满足特定的精度条件时退出循环.

\begin{enumerate}
\item 先给这些点按照 $f(\vec x_i)$ 从小到大的顺序重新排序并重命名, 使 $i$ 越大 $f(\vec x_i)$ 越大.

\item 计算前 $N$ 个点的平均位置为
\begin{equation}
\vec m = \frac 1N \sum_{i=1}^N \vec x_i
\end{equation}

\item 计算 $\vec x_{N + 1}$ 关于点 $\vec m$ 的\bb{反射点}为
\begin{equation}
\vec r = 2\vec m - \vec x_{N + 1}
\end{equation}

\item 如果 $f(\vec x_1) \le f(\vec r) < f(\vec x_N)$, 令 $\vec x_{N+1} = \vec r$, 并进入下一个循环.

\item 如果 $f(\vec r) < f(\vec x_1)$, 计算\bb{拓展点}为
\begin{equation}
\vec s = \vec m + 2(\vec m - \vec x_{N+1})
\end{equation}
如果 $f(\vec s) < f(\vec r)$, 令 $\vec x_{N+1} = \vec s$ 并进入下一个循环, 否则令 $\vec x_{N+1} = \vec r$. 并进入下一循环.

\item 如果 $f(\vec x_N) \le f(\vec r) < f(\vec x_{N+1})$, 令
\begin{equation}
\vec c_1 = \vec m + (\vec r - \vec m)/2
\end{equation}
如果 $f(\vec c_1) < f(\vec r)$, 令 $\vec x_{N + 1} = \vec c_1$ 并进入下一循环, 否则执行最后一步.

\item 如果 $f(\vec x_{N+1}) \le f(\vec r)$ 令
\begin{equation}
\vec c_2 = \vec m + (\vec x_{N+1} - \vec m)/2
\end{equation}
如果 $f(\vec c_2) < f(\vec x_{N+1})$, 令 $\vec x_{N+1}  = \vec c_2$ 并进入下一循环, 否则执行最后一步.

\item 令
\begin{equation}
\vec v_i = \vec x_1 + (\vec x_i - \vec x_1)/2 \qquad (i = 2\dots N+1)
\end{equation}
并用 $\vec v_i$ 赋值给 $\vec x_i$, 进入下一循环.
\end{enumerate}

观察以上步骤可知, 当局部最小值的位置在 $N+1$ 个围成的图形以外时, 图形倾向于变大且加速向最小值移动. 当最小值的位置在图形内部时, 图形倾向于缩小. 随着循环次数增加, 这 $N+1$ 个点最终将向局部最小值聚拢. 

我们可以在每个循环的第一步之后计算 $\vec x_1$ 和 $\vec x_{N+1}$ 的距离来估算自变量的误差, 如果该误差小于某个值, 即可结束循环并使用 $\vec x_1$ 作为最终结果. 作为另一种方法, 我们也可以在每个循环的第一步之后计算 $f(\vec x_{N+1}) - f(\vec x_1)$ 来估算最小值的误差.

以下是该算法的 Matlab 代码.

\subsubsection{NelderMead.m}
% f 是函数句柄,只接受一个 N 维行矢量作为输入变量, 并返回一个函数值
% x0 是 N 维行矢量, xerr 是一个标量
function [xmin, fmin] = NelderMead(f, x0, xerr)
N = numel(x0); % f 是 N 元函数
x = zeros(N+1,N); % 预赋值
y = zeros(1,N+1);
% 计算 N+1 个初始点
x(1,:) = x0;
for ii = 1:N
    x(ii+1,:) = x(1,:);
    if x(1,ii) == 0
        x(ii+1,ii) = 0.00025;
    else
        x(ii+1,ii) = 1.05 * x(1,ii);
    end
end
% 主循环
for kk = 1:10000
    % 求值并排序
    for ii = 1 : N + 1
        y(ii) = f(x(ii,:));
    end
    [y, order] = sort(y);
    x = x(order,:);
    if norm(x(N+1,:) - x(1,:)) < xerr % 判断误差
        break;
    end
    m = mean(x(1:N,:)); % 平均位置
    r = 2*m - x(N+1,:); % 反射点
    f_r = f(r);
    if y(1) <= f_r && f_r < y(N) % 第 4 步
        x(N+1,:) = r; continue;
    elseif f_r < y(1) % 第 5 步
        s = m + 2*(m - x(N+1,:));
        if f(s) < f_r
            x(N+1,:) = s;
        else
            x(N+1,:) = r;
        end
        continue;
    elseif f_r < y(N+1) % 第 6 步
        c1 = m + (r - m)*0.5;
        if f(c1) < f_r
            x(N+1,:) = c1; continue;
        end
    else % 第 7 步
        c2 = m + (x(N+1,:) - m)*0.5;
        if f(c2) < y(N+1)
            x(N+1,:) = c2; continue;
        end
    end
    for jj = 2:N+1 % 第 8 步
        x(jj,:) = x(1,:) + (x(jj,:) - x(1,:))*0.5;
    end
end
% 输出变量
xmin = x(1,:);
fmin = f(xmin);
end


该程序中有几个需要注意的地方. 首先, 主循环并没有用 \x{while 1}, 而是用 \x{for} 循环指定一个最大循环次数. 这是为了避免少数情况下可能发生的死循环(例如 $f(\vec x)$ 在某个区域中的值处处相等时). 第二, 4-7 步中对 $f(\vec r)$ 的判断有且仅有一个成立, 所以我们可以用 \x{if, elseif, else} 结构来选择. 最后, 4-5 步的情况下程序必定会执行 \x{continue} 语句而跳过第 8 步, 只有 6-7 步中的 \x{if} 判断失败程序才会执行第 8 步.

% 未完成:需要举一个应用的例子.


