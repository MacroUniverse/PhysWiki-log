% Nelder-Mead 算法

\pentry{Matlab}% 未完成

Nelder-Mead 算法是一种求多元函数最小值的算法, 其优点是不需要函数可导. 介绍 blablabla

Matlab 自带的 \x{fminsearch} 函数就是使用该算法. 这里进行介绍.需要提供一个初始点.

对于 $N$ 元函数 $f(\vec x)$, 把初始点记为 $\vec x_1$, 然后根据初始点另外生成 $N$ 个点 $\vec x_2\dots\vec x_{N + 1}$, 使 $\vec x_{1 + i}$ 在第的第 $i$ 个分量比 $\vec x_1$ 的大 5\%, 其他分量保持相同. 如果 $\vec x_1$ 的第 $i$ 个分量为零, 那么 $\vec x_{1 + i}$ 的第 $i$ 个分量设为 $0.00025$.

先给这些点按照 $f(x_i)$ 从小到大的顺序重新排序并重命名, 使 $i$ 越大 $f(x_i)$ 越大.

计算前 $N$ 个点的平均位置为
\begin{equation}
\vec m = \frac 1N \sum_{i=1}^N \vec x_i
\end{equation}


计算 $\vec x_{N + 1}$ 关于点 $\vec m$ 的反射点为
\begin{equation}
\vec r = 2\vec m - \vec x_{N + 1}
\end{equation}

如果 $f(\vec x_1) \le f(\vec r) < f(\vec x_N)$, 令 $\vec x_{N+1} = \vec r$, 并进入下一个循环.

如果 $f(\vec r) < f(\vec x_1)$, 计算拓展点为
\begin{equation}
\vec s = \vec m + 2(\vec m - \vec x_{N+1})
\end{equation}
如果 $f(\vec s) < f(\vec r)$, 令 $\vec x_{N+1} = \vec s$ 并进入下一个循环.