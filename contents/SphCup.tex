% 球谐函数 CG 系数 角动量加法

Bransden 附录 A4 给出了一个很有用的公式(方括号代表 CG 系数, 圆括号代表 $3j$ 符号,这里也体现了 $3j$ 符号的优美)
\begin{equation}\label{SphCup_eq1}
\ali{
&\quad \int Y_{l_1 m_1} (\uvec r) Y_{l_2 m_2} (\uvec r) Y_{l_3 m_3}(\uvec r) \dd{\Omega}\\
&= (-1)^{m_3} \sqrt{\frac{(2l_1+1)(2l_2+1)}{4\pi(2l_3+1)}} \bmat{l_1& l_2& l_3\\ 0 & 0 & 0}\bmat{l_1 & l_2 & l_3\\  m_1 & m_2 & -m_3}\\
&= \sqrt{\frac{(2l_1+1)(2l_2+1)(2l_3+1)}{4\pi}}  \pmat{l_1& l_2& l_3\\ 0 & 0 & 0}\pmat{l_1 & l_2 & l_3\\  m_1 & m_2 & m_3}
}\end{equation}

任何算符的一组归一化本征基底各自乘以一个任意相位因子 $\E^{\I \phi_I}$ 仍然是一组归一化的本征基底(本征值不变). 所以基底变换矩阵(如 CG 矩阵), 每一行或每一列分别乘以一个相位因子, 仍然表示相同的基底变换. 这些相位怎么取被称为相位约定(Phase Convention). 一般的约定是使所有 CG 系数为实数, 且第一行和第一列(对应最大的 $m_1$ 和最大的 $L$)的矩阵元大于零(Wikipedia,Griffiths).

CG 系数有解析表达式(符合上面的相位约定)
\begin{equation}
\ali{
&\braket{l_1 m_1 l_2 m_2}{l_1 l_2 L M} =
\sqrt{\frac{(2L+1)(L+l_1-l_2)!(L-l_1+l_2)!(l_1+l_2-L)!}{(l_1+l_2+L+1)!}}\\
&\times\sqrt{(L+M)!(L-M)!(l_1-m_1)!(l_1+m_1)!(l_2-m_2)!(l_2+m_2)!}\\
&\times\sum_k \frac{(-1)^k}{k!(l_1+l_2-L-k)!(l_1-m_1-k)!(l_2+m_2-k)!}\\
&\times \frac{1}{(L-l_2+m_1+k)!(L-l_1-m_2+k)!}
}\end{equation}

CG 系数的对称性有(见 $3j$ 符号的对称性)
\begin{equation}
\ali{
\bmat{l_1 &l_2 &L\\ m_1 &m_2 &M}
&= (-1)^{l_1+l_2+L}\bmat{l_1 &l_2 &L\\ -m_1 &-m_2 &-M}\\
&= (-1)^{l_1+l_2+L}\bmat{l_2 &l_1 &L\\ m_2 &m_1 &M}
}\end{equation}
所以当 $l_1 + l_2 + L$ 为奇数时,  parity CG coefficient
\begin{equation}
\bmat{l_1 &l_2 & L\\ 0 & 0 & 0} = 0
\end{equation}
