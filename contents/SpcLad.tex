% 太空电梯

\begin{issues}
\issueDraft
\end{issues}

在地球所在的旋转参考系中, 缆绳上某点的张力为
\begin{equation}\label{SpcLad_eq1}
F(r) = \int_{r_0}^{r} \qty(\frac{GM}{r'^2} \dd{r'} - \omega^2 r') \lambda(r') \dd{r'}
\end{equation}
$\lambda(r)$ 是绳的线密度, $r_0$ 是地球半径, $\omega$ 是地球自转角速度, $r$ 是绳索上某点与地心的距离. 我们需要 $r$ 达到地球同步轨道的高度.

绳截面可承受的最大张力压强为 $p$, 那么截面为
\begin{equation}
A(r) = F(r)/p
\end{equation}
绳的密度为常数 $\rho$, 那么线密度为
\begin{equation}
\lambda(r) = \rho A(r) = \rho F(r)/p
\end{equation}
带回\autoref{SpcLad_eq1} 有积分方程
\begin{equation}
F(r) = \frac{GM\rho}{p} \int_{r_0}^{r} \frac{F(r')}{r'^2} \dd{r'} - \frac{\omega^2\rho}{p}\int_{r_0}^r F(r') r' \dd{r'}
\end{equation}
令两个积分前面的常数为 $\alpha, \beta$, 两边求导
\begin{equation}
\dv{F}{r} = \alpha \frac{F(r)}{r^2} - \beta r F(r)
\end{equation}
分离变量, 解得
\begin{equation}
F = C\exp(-\frac{\alpha}{r} + \frac{\beta}{2} r^2)
\end{equation}
初始条件是 $F(0) = F_0$, $F_0$ 是载重. 代入得到方程的解为
\begin{equation}
F(r) = F_0 \exp[\frac{GM\rho}{p} \qty(\frac{1}{r_0} - \frac{1}{r}) - \frac{\omega^2\rho}{2p}(r^2 - r_0^2)]
\end{equation}

目前最强的材料碳纳米管有 $p = 6.2\times 10^{10} \Si{Pa}$, 密度使用 $\rho = 1.34\times 10^{3} \Si{kg/m^3}$. 地球半径取 $6.370\times 10^6\Si{m}$, 同步轨道半径为 $4.2164\times 10^7 \Si{m}$. 地球质量 $M = 5.972 \times 10^{24} \Si{kg}$.

代入上式得 $F = 2.85 F_0$ !!!! 卧槽竟然有这么好的事情. 碳纳米管的大规模生产完全可以开启新的航天时代.
