% 群作用
% 群作用|同态
%未完成
\pentry{群同态\upref{Group2}}

\subsection{群在自身上的作用}

给定一个群$G$,我们任意拿出一个元素$a\in G$,用$a$去左乘$G$中的所有元素(包括$a$自己),那么我们可以把$a$看成一种$G$到自身上的映射:$f_a:G\rightarrow G$,使得对于任意$x\in G$,$f_a(x)=ax$.

群$G$中的每一个元素都可以像这样生成一个映射,把这些映射全部放在一起,我们也可以整体上看成一个映射$f:G\times G\rightarrow G$,满足:对于任意的$a, x\in G$,有$f(a,x)=f_a(x)=ax$.这样的$G\times G$到$G$上的映射,被称为一个\textbf{群作用(group operation)}.

\subsection{群作用}

更一般地,对于任何集合$X$,群$G$中每个元素都可以代表$X\rightarrow X$的一个映射.我们当然可以任意规定这些映射,但如果这些映射满足一定条件的话,就会构造出一个很有意思的结构:

\begin{definition}{群作用}
设群$G$和集合$X$,$G$中每个元素都是$X$到自身的映射,记$g\in G$将$x\in X$映射为$g\cdot x\in X$.如果所有这些映射满足满足下面两条公理:
\begin{itemize}
\item \textbf{结合律}:对于$g_1, g_2\in G, x\in X$,$(gh)\cdot x=g\cdot (h\cdot x)$.
\item \textbf{单位元是恒等映射}:$G$的单位元$e$将任何$x\in X$映射到自身:$e\cdot x=x$.
\end{itemize}

那么我们称群$G$ \textbf{作用(operate)}于集合$X$上.

\end{definition}