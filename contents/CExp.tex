%指数函数(复数)

%未完成
%(计划内容: 定义,复平面上的特征,是解析函数,反函数,物理上最广的应用).

\pentry{指数函数,复数简介}%未完成:词条
复数范围内的指数函数被定义为
 \begin{equation}\label{CExp_eq1}
w = {\E^z} = {\E^{x + \I y}} = {\E^x}(\cos y + \I\sin y)
\end{equation}
在复平面上表示这个函数,则指数的实部 $x$ 控制因变量 $w$ 的模长, 虚部 $y$ 控制 $w$ 的幅角
 \begin{gather}
\abs{w} = {\E^x} \\
\arg(w) = y
\end{gather}
当指数为纯虚数时,\autoref{CExp_eq1} 变为著名的\textbf{欧拉公式}
\begin{equation}\label{CExp_eq2}
{\E^{\I \theta}} = {\cos \theta + \I\sin \theta}
\end{equation}
虽然这里的 $\theta$ 只能是实数(物理中应用得最多的情况),但根据复数域三角函数的定义\upref{CTrig}, 对于任何复数 $z$,都有欧拉公式
\begin{equation}
{\E^{\I z}} = {\cos z + \I\sin z}
\end{equation}
将“三角函数(复数)\upref{CExp}”中的\autoref{CTrig_eq1} 和\autoref{CTrig_eq2} 代入即可证明.