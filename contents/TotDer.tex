% 多元函数的全导数
\pentry{全微分\upref{TDiff}}

若多元函数包含若干个变量(以下以 $f(x,y,t)$ 为例),我们知道它可以对其中任意一个变量求偏导, 即 $\partial f/\partial x$,  $\partial f/\partial y$,  $\partial f/\partial t$, 注意求偏导时其余两个变量不变.现在若把 $x,y$ 看做 $t$ 的函数, 那么 $f$ 归根结底也是 $t$ 的函数 $f(x(t),y(t),t)$, 我们可以将其对 $t$ 求导.为了强调这与对 $t$ 求偏导有所不同,我们把得到的函数叫做\textbf{全导数}.

与偏微分中的链式法的推导类似,我们先来看函数的全微分\upref{TDiff}
\begin{equation}
\D f = \frac{{\partial f}}{{\partial x}}\D x + \frac{{\partial f}}{{\partial y}}\D y + \frac{{\partial f}}{{\partial t}}\D t
\end{equation}
而根据 $x,y$ 与 $t$ 的微分关系
\begin{equation}
\D x = \frac{{\D x}}{{\D t}}\D t \qquad  \D y = \frac{{\D y}}{{\D t}}\D t
\end{equation}
代入上式得
\begin{equation}
\D f = \left( {\frac{{\partial f}}{{\partial x}}\frac{{\partial x}}{{\partial t}} + \frac{{\partial f}}{{\partial y}}\frac{{\partial y}}{{\partial t}} + \frac{{\partial f}}{{\partial t}}} \right)\D t
\end{equation}
而若把 $f$ 看做 $t$ 的一元函数,又应该有全微分关系
\begin{equation}
\D f = \frac{\D f}{\D t} \D t
\end{equation}
对比以上两式可得 $f$ 关于 $t$ 的全导数为
\begin{equation}
\frac{{\D f}}{{\D t}} = \frac{{\partial f}}{{\partial x}}\frac{{\partial x}}{{\partial t}} + \frac{{\partial f}}{{\partial y}}\frac{{\partial y}}{{\partial t}} + \frac{{\partial f}}{{\partial t}}
\end{equation}

% 未完成 此处应该有例子
% 未完成 要介绍显含 $t$ 和不显含 $t$ 的区别