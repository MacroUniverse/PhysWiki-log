%球谐函数

\pentry{球坐标的拉普拉斯方程\upref{SphLap}}

在球坐标的拉普拉斯方程分离变量后, 关于极角 $\theta$ 的函数为连带勒让德多项式 $P_l^m(\cos\theta)$, 方向角函数为 $\E^{\I m\phi}$. 我们定义\footnote{有时也将球谐函数记为 $Y_{lm}$}
\begin{equation}
Y_l^m(\uvec r) = Y_l^m(\theta, \phi) = A_{l,m} P_l^m(\cos \theta) \E^{\I m\phi}
\end{equation}
$A_{l,m}$ 是归一化系数, 使得 $\abs{Y_l^m (\uvec r)}^2$ 在单位球面上的面积分等于 1\footnote{\autoref{SphHar_eq2} 是 Mathematica 中的定义, 也有一种定义在前面加 $(-1)^m$, 同样满足归一化条件.}.
\begin{equation}\label{SphHar_eq2}
A_{l,m} =  \sqrt{\frac{2l + 1}{4\pi }\frac{(l - m)!}{(l + m)!}}
\end{equation}

球谐函数满足
\begin{equation}
\qty[\frac{1}{\sin\theta}\pdv{\theta} \qty(\sin \theta \pdv{\theta}) + \frac{1}{\sin^2 \theta} \qty(\pdv[2]{\phi})]Y_{lm}(\theta, \phi) = -l(l+1)Y_{lm}(\theta, \phi)
\end{equation}
中括号中的算符是球坐标系拉普拉斯算子 $\laplacian$ 中的角向部分(记为 $\laplacian_\Omega$)乘以 $r^2$.

\subsection{归一化系数}
由球谐函数的归一化条件, % 未完成:哪里讲解一下立体角是单位球面上的面积元, 以及极坐标中的表达式. % 链接未完成
\begin{equation}\ali{
1 &= \int \abs{Y_l^m(\uvec r)}^2 \dd{\Omega} = \int_0^\pi  \int_0^{2\pi}  \abs{Y_l^m(\theta, \phi)}^2 \sin\theta\dd{\theta}\dd{\phi} \\
&= \abs{A_{l,m}}^2 \int_{-1}^1  \abs{P_l^m(\cos\theta)}^2 \dd{(\cos \theta)} \int_0^{2\pi } \abs{\E^{\I m\theta}}^2  \dd{\phi}\\
&= \frac{2\pi}{\abs{A_{l,m}'}^2} \abs{A_{l,m}}^2
}\end{equation}
其中 $A_{l,m}'$ 是 $P_l^m(x)$ 的归一化系数(见\autoref{AsLgdr_eq3}\upref{AsLgdr}), 代入后可得 $A_{l,m}$.

%未完成 列出一些球谐函数!

\subsection{正交归一性}
由勒让德函数的正交归一性(\autoref{AsLgdr_eq4}\upref{AsLgdr})以及 $\E^{\I m \phi}$  的正交归一性,% 未完成: 引用复数傅里叶级数中的公式
 不难证明球谐函数的正交归一性
\begin{equation}
\int Y_{l'}^{m'}(\uvec r) Y_l^m(\uvec r) \dd{\Omega} = \delta_{ll'}\delta_{mm'}
\end{equation}

\subsection{其他性质}
\begin{equation}\label{SphHar_eq6}
Y_l^{m*} = (-1)^m Y_l^{-m}
\end{equation}

旋转变换
\begin{equation}
Y_{l,m}(\uvec r') = \sum_{m'=-l}^l D_{m,m'}^{(l)} (\mathcal R)^* Y_{l,m}'(\uvec r)
\end{equation}
其中 $D$ 是 Wigner-D 矩阵. 从量子力学的角度来说, 总角动量是与方向无关的, 只有角动量分量有关.

特殊地
\begin{equation}
Y_{l,m}(-\uvec r) = (-1)^l Y_{l,m}(\uvec r)
\end{equation}
