% 一维散射(量子)

\begin{issues}
\issueDraft
\end{issues}

\pentry{一维自由粒子(量子)\upref{FreeP1}}

在量子力学中, 散射是一个十分重要的过程. 例如为了探索例如原子的结构, 一种重要的手段就是用其他粒子轰击原子, 并探测出射粒子在不同方向上的动量或能量分布等. 著名的卢瑟福散射就是用电子轰击原子的方法解开了原子结构之谜. 虽然卢瑟福最初使用经典力学来分析这类散射过程(当时量子力学还没有出现), 但原子尺度下量子力学比经典力学要精确得多.

在经典力学中, 散射\upref{Scater}过程考虑一个粒子从无穷远处入射, 经过一个势能 $V(\bvec r)$ 后发生偏折. 量子力学中, 这一过程可以用波包来描述: 一个波包 $\psi(x, t)$ 在初始时刻从远处入射, 我们想知道经过 $V(\bvec r)$ 散射后, $\psi(x, t)$ 会如何变化, 例如不同出射方向的概率流密度\upref{PrbJ}以及动量如何分布?

由于三维空间的量子散射所需的数学较为复杂, 我们先学习一维散射. 三维散射的许多性质都可以从一维情况类比.

我们已经学习了一维自由粒子\upref{FreeP1}如何随时间演化, 例如一维高斯波包\upref{GausWP}在自由演化过程中, 他的中心会像经典粒子一样以恒定速度移动, 但同时波包还会慢慢变宽. 在此基础上, 如果在波包前进方向上添加一个不为零的 $V(x)$ (例如一个势垒或势阱\autoref{QM0_fig3}~\upref{QM0}), 那么波包将会在各个方向上发生散射(一维运动只有左右两个方向). 我们希望能计算该过程中波函数如何变化. 我们要求势能函数 $V(x)$ 只在局部不为零, 即满足 $V(\pm\infty) \to 0$. 

原则上, 我们可以直接把初始波函数和势能代入含时薛定谔方程\upref{TDSE}进行求解
\begin{equation}\label{Sca1D_eq1}
\frac{1}{2m}\pdv[2]{t}\psi(x, t) + V(x) \psi(x, t) = \I \pdv{t} \psi(x, t)
\end{equation}
即使没有解析解, 也可以通过数值方法, 依次求出每个时间步长 $t_n$ ($n = 1, 2, \dots$)的波函数 $\psi(x, t_n)$.

解析解能帮我们更好地理解问题. 和 “一维自由粒子(量子)\upref{FreeP1}” 中的过程类似, 分离变量法可以得到\autoref{Sca1D_eq1} 的通解(\autoref{TDSE_eq4}~\upref{TDSE}). 首先看定态薛定谔方程定态薛定谔方程
\begin{equation}
\frac{1}{2m}\dv[2]{t}\psi_E + V(x) \psi_E = E \psi_E \qquad (E > 0)
\end{equation}
其中 $E$ 可以连续取值, 每个 $E$ 都对应不同的解. 这是一个二阶常微分方程, 每个 $E$ 有两个线性无关的解, 记为 $\psi_{E,1}(x)$ 和 $\psi_{E,2}(x)$, 于是通解可以表示为
\begin{equation}
\Psi(x, t) = \int [C_1(E) \psi_{E,1}(x) + C_2(E) \psi_{E,2}(x)] \E^{-\I E t} \dd{E}
\end{equation}
系数 $C_i(E)$ 由初始波函数决定, 满足
\begin{equation}
\Psi(x, t_0) = \int [C_1(E) \psi_{E,1}(x) + C_2(E) \psi_{E,2}(x)] \E^{-\I E t_0} \dd{E}
\end{equation}
