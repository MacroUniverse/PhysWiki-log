% 库仑定律
% 库仑力|库仑定律|平方反比力|电介质常数

\pentry{万有引力\upref{Gravty}}

学习万有引力以后, 库仑力就可以轻而易举地类比过来, 所以我们这里不再做重复的推导.

两个点电荷间的库仑力的矢量表达式为
\begin{equation}\label{ClbFrc_eq1}
\bvec F_{12} = k\frac{q_1 q_2}{r_{12}^2} \uvec r_{12} = \frac{1}{4\pi\epsilon_0} \frac{q_1 q_2}{\abs{\bvec r_2 - \bvec r_1}^3}(\bvec r_2 - \bvec r_1)
\end{equation}
其中 $q_1, q_2$ 分别是两个点电荷的电荷量. $k$ 是库仑常数, 但在大学物理中我们几乎不会见到这个符号, 它通常被替换为
\begin{equation}
k = \frac{1}{4\pi\epsilon_0} = 8.9876\times 10^9 \Si{N m^2/C^2}
\end{equation}
$\epsilon_0$ 是\textbf{真空中的电介质常数(vacuum permittivity)}, 我们以后会见到\upref{EGauss}.

注意比起万有引力的\autoref{Gravty_eq1}\upref{Gravty}, \autoref{ClbFrc_eq1} 没有负号, $q_1, q_2$ 可以是负数(代表负电荷). 我们容易看出两电荷同号相吸, 异号相斥.

至于\textbf{点电荷}的概念, 就是在质点的基础上(忽略物体的大小与形状), 增加了一个总电荷量的属性.

\subsection{电势能}
两个点电荷之间的库仑力产生的势能为
\begin{equation}\label{ClbFrc_eq2}
V(r) = k \frac{q_1 q_2}{r_{12}} = \frac{1}{4\pi\epsilon_0} \frac{q_1 q_2}{r_{12}}
\end{equation}

\begin{example}{}
两个电荷量为 $Q$ 的电荷分别被固定在 $(-c, 0)$ 和 $(c, 0)$ 两点处, 另一质量为 $m$ 电荷量为 $q$ 的点电荷从 $(0, a)$ 延直线移动到 $(0, b)$, 试问它的动能增加了多少?

解: 由\autoref{ClbFrc_eq2} 可知, 点电荷 $q$ 在 $(0, y)$ 处的电势能分别为
\begin{equation}
V = \frac{2kQq}{\sqrt{y^2 + c^2}}
\end{equation}
所以动能增加等于势能减少, 即
\begin{equation}
\Delta E_k = V_a - V_b = 2kQq \qty(\frac{1}{\sqrt{a^2 + c^2}} - \frac{1}{\sqrt{b^2 + c^2}})
\end{equation}
\end{example}
