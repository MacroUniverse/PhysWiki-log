% 稀疏矩阵
% 系数矩阵|数据结构|计算物理|数值计算

\textbf{稀疏矩阵(Sparse Matrix)}有不同的储存方式(数据结构), 这里介绍几种常见的\footnote{参考 Wikipedia 的 Sparse Matrix 词条}.

\subsection{Banded}
Banded 矩阵只储存矩阵主对角线上下的若干条对角线, 上带宽和下带宽分别指定主对角线上面和下面有几条对角线, 例如三对角矩阵的上带宽和下带宽都是 1. 带内即使有矩阵元为零也必须储存. 这样就可以按照 row major 或者 column major 来储存.

\subsection{Coordinate List (COO)}
COO 格式列出非零矩阵元和对应的行标列标. 通常将它们储存为三个数组 \x{a}, \x{ia}, \x{ja}, 顺序任意. 除此之外, 有时还需要储存三个数组的长度 \x{nnz} (none zero) 以及矩阵的尺寸.

\subsection{Compressed Sparse Row (CSR)}
也叫 Compressed Row Storage (CRS), 这种格式做矩阵与矢量相乘较快.

CRS 格式储存为三个数组 \x{a}, \x{ia}, \x{ja}, 非零矩阵元按照 row major 的顺序储存, 第 \x{n} 行矩阵元是 \x{a(ia(n) : ia(n+1)-1)}, \x{ja} 是对应矩阵元的列标.

\subsection{Compressed Sparse Column (CSC)}
也叫 Compressed Column Storage (CCS), 与 CRS 一样, 只是改为 column major.
