% 光的多普勒效应
% 相对论|光速|多普勒|频率|波长
%注意不可以无脑用尺缩效应来推导,因为同时性的相对性.
%基本完成

\pentry{多普勒效应\upref{Dopler},洛伦兹变换\upref{SRLrtz}}

不论是牛顿力学框架下还是相对论意义下,光的频率都可能在不同参考系之间有所不同.由于观察者和光源之间相对运动,造成观察者所测得的光的频率与光源的频率不一致,这个现象被称为光的多普勒效应.由于光速不变原理,在相对论框架下讨论光的多普勒效应反而更为简单.




\subsection{光的多普勒效应的推导}
由于在任何参考系中,光的速度都一样,因此只需要知道光的波长就可以根据以下公式得到光的频率:
\begin{equation}
f=\frac{1}{\lambda}
\end{equation}
其中,$f$是光的频率,$\lambda$是光的波长,$1$是光速.

假设光源所在的参考系为$K_1$,观察者所在的参考系为$K_2$,而$K_2$相对$K_1$的运动速度为$\vec{v}=(v, 0, 0)^T$.设光的频率在$K_1$为$f$,在$K_2$中为$f'$,那么光的波长在$K_1$中为$\lambda=1/f$,在$K_2$中为$\lambda'=1/f'$.

\subsubsection{一维情况}
设在$K_1$看来,观察者和光的运动方向是一致的.

取$K_1$中光的两个相邻的波峰\footnote{也可以是相邻波谷,或者任何相邻的两个同相位的点.},作为两个以光速运动的点.为简化讨论,不妨设其中一个点的世界线\footnote{见\textbf{时空的四维表示}\upref{SR4Rep}.}表示为
\begin{equation}
\pmat{t_1\\t_1\\0\\0}
\end{equation}
而另一个点为
\begin{equation}
\pmat{t_2\\t_2+\lambda\\0\\0}
\end{equation}

代入洛伦兹变换可得,在$K_2$中两波峰的世界线分别为
\begin{equation}
\pmat{\frac{t_1-vt_1}{\sqrt{1-v^2}}\\\frac{t_1-vt_1}{\sqrt{1-v^2}}\\0\\0}
\end{equation}
和
\begin{equation}
\pmat{\frac{t_2-v(t_2+\lambda)}{\sqrt{1-v^2}}\\\frac{(t_2+\lambda)-vt_2}{\sqrt{1-v^2}}\\0\\0}=\pmat{\frac{t_2-vt_2-v\lambda}{\sqrt{1-v^2}}\\\frac{t_2-vt_2+\lambda}{\sqrt{1-v^2}}\\0\\0}
\end{equation}

在各参考系中计算波长,就要计算对应参考系中同一时间下两波峰的空间坐标的差.对于$K_1$,“同一时间”意味着$t_1=t_2$;对于$K_2$,“同一时间”意味着$t_1-vt_1=t_2-v(t_2+\lambda)$,

将“同一时间”条件分别代入两个参考系中两个点的世界线,消去$t_1$计算后发现,两波峰在$K_1$中的距离始终是$\lambda$,而在$K_2$中的距离始终是$\frac{(1+v)\lambda}{\sqrt{1-v^2}}$.由于已经设定这束光在$K_2$中的波长是$\lambda'$,因此有
\begin{equation}
\lambda'=\frac{(1+v)\lambda}{\sqrt{1-v^2}}=\frac{\sqrt{1+v}}{\sqrt{1-v}}\lambda
\end{equation}

因此在$K_2$中,光的频率变为

\begin{equation}\label{RelDop_eq2}
f'=\frac{1}{\lambda'}=\frac{\sqrt{1-v}}{\sqrt{1+v}\lambda}=\frac{\sqrt{1-v}}{\sqrt{1+v}}f
\end{equation}

这就是一维情况下光的多普勒效应.

\subsubsection{三维情况}

三维情况下,我们要考虑在$K_1$看来光的传播方向和$K_2$的运动方向不一致的情况.

只要在垂直$x$轴的方向上,$y$轴和$z$轴的方向可以任意选取,因此不妨设在$K_1$看来,光的传播方向在$x-y$平面的第一象限,与$x$轴夹角为$\theta$.

类似一维情况中,取相邻两个波峰,考虑它们在$K_1$中的世界线:
\begin{equation}
\pmat{t_1\\t_1\cos{\theta}\\t_1\sin{\theta}\\0}
\end{equation}
和
\begin{equation}
\pmat{t_2\\(t_2+\lambda)\cos{\theta}\\(t_2+\lambda)\sin{\theta}\\0}
\end{equation}

代入洛伦兹变换,得到它们在$K_2$中的世界线:

\begin{equation}
\pmat{\frac{t_1-vt_1\cos{\theta}}{\sqrt{1-v^2}}\\\frac{t_1\cos{\theta}-vt_1}{\sqrt{1-v^2}}\\t_1\sin{\theta}\\0}
\end{equation}
和
\begin{equation}
\pmat{\frac{t_2-v(t_2+\lambda)\cos{\theta}}{\sqrt{1-v^2}}\\\frac{(t_2+\lambda)\cos{\theta}-vt_2}{\sqrt{1-v^2}}\\(t_2+\lambda)\sin{\theta}\\0}
\end{equation}

在$K_2$中,“同一时间”条件为$t_1-vt_1\cos{\theta}=t_2-v(t_2+\lambda)\cos{\theta}$.类似一维情况,代入该条件并求两波峰的空间坐标之差,所得到的差是一个向量
\begin{equation}
\lambda\pmat{\frac{\cos\theta\sqrt{1-v^2}}{1-v\cos\theta}\\\frac{\sin\theta}{1-v\cos\theta}\\0}
\end{equation}
该向量的模长就是
\begin{equation}
\lambda'=\lambda\cdot\frac{\cos\theta\sqrt{1-v^2}}{1-v\cos\theta}
\end{equation}

因此在$K_2$中,光的频率变为

\begin{equation}\label{RelDop_eq1}
f'=\frac{1}{\lambda'}=\frac{1-v\cos\theta}{\cos\theta\sqrt{1-v^2}}f
\end{equation}

这就是三维空间中一般情况下光的多普勒效应.

如果$\theta=0$,那么情况退化为一维空间的问题.作为验证将$\theta=0$代入\autoref{RelDop_eq1} 后结果和\autoref{RelDop_eq2} 一致;事实上,容易验证,三维情况下的所有式子在$\theta=0$时都退化为对应的一维情况下的式子.
