% Lanczos 算法
% Lanczos|本征值问题

\subsection{背景介绍}

\footnote{参考 Renate Pazourek 的论文}在计算含时薛定谔方程时,若已知某时刻 $t$ 的波函数 $\Psi(\bvec r,t)$,要求 $\Psi(\bvec r,t+\Delta t)$,通常使用传播算符 % 链接未完成
\begin{equation}
U(\Delta t) = \exp(-\frac{\I}{\hbar} \int_{t}^{t+\Delta t} \dd{t} \vdot H(t))
\approx \exp(-\I H(t+\Delta t/2) \Delta t)
\end{equation}
其中 $H$ 为哈密顿算符.若 $H$ 不含时,约等号变为等号
\begin{equation}
\Psi(\bvec r, t+\Delta t) = \exp(-\I H\Delta t/\hbar)\Psi(\bvec r, t)
\end{equation}
通常情况下,用有限个正交归一基底 $\Psi_0,\dots\Psi_{K-1}$ 近似展开 $\Psi(\bvec r, t+\Delta t)$,这时,$H$ 可以表示成矩阵的形式.
\begin{equation}
H_{ij} = \bra{\Psi_i} H \ket{\Psi_j}
\end{equation}
这样,幺正算符 $\exp(-\I H\Delta t/\hbar)$ 也可以表示成矩阵.根据定义 % 链接未完成
\begin{equation}
\exp(-\I H\Delta t/\hbar) = 1+ (-\I H\Delta t/\hbar) + \frac{1}{2!} (-\I H\Delta t/\hbar)^2\dots
\end{equation}
此时若把 $H$ 对角化(求解本征方程),得到本征矢列矩阵 $\mat P$ (单位正交阵)以及本征值矩阵 $\mat \Lambda$ (对角矩阵),则可进行基底变换变到 $H$ 矩阵的本征空间求解上式 %(矩阵指数的对角化化简,未完成)
\begin{equation}\ali{
\exp(-\I H\Delta t/\hbar) &= \mat P\times \rm{diag}(\E^{-\I E_1 t/\hbar},\E^{-\I E_2 t/\hbar}\dots \E^{-\I E_N t/\hbar})\times \mat P^{-1}\\
&= \sum_{j=0}^{K-1} P_{ij} \E^{-\I E_j t/\hbar} P_{jk}^{-1}
= \sum_{j=0}^{K-1} P_{ij}P_{kj} \E^{-\I E_j t/\hbar} 
}\end{equation}
要说明的是,这种算法的误差(除数值计算误差)来源于两个方面,第一是用 $\exp(\I H\Delta t/\hbar)$ 代替 $\exp( -\frac{\I}{\hbar} \int_{t}^{t+\Delta t} \dd{t} \cdot H(t) )$, 但如果 不含时就没有该误差,第二是有限数量的基底不具有完备性,这个误差可以随着基底数量增加而减小.

\subsection{Lanczos算法}
在以上的算法中,选取施密特正交归一化 %(链接未完成)
的 Krilov 基底作为基底.令 $\Psi_0 \equiv \Psi(\bvec r, t)$,  $K$ 阶的 Krilov 基底为
 \begin{equation}
\qty{ \ket{\Psi_0}, H \ket{\Psi_0}, H^2 \ket{\Psi_0} \dots H^{K-1} \ket{\Psi_0}}
\end{equation}
我们把正交归一化后的基底记为
\begin{equation}
\qty{ \ket{\Psi_0}, \ket{\Psi_1}, \ket{\Psi_2} \dots \ket{\Psi_{K-1}} }
\end{equation}
这样,用其展开哈密顿算符
\begin{equation}
H_{ij}=\mel{\Psi_i}{H}{\Psi_j}
\end{equation}
在算法上,本来我们可以按部就班地按照以上步骤做,然而由于 Krilov 基底的性质,可以通过一些定理(见词条最后)大大减少计算量.正交归一化步骤如下

\begin{enumerate}
\item 如果 $\Psi(\bvec r,t)$ 没有归一化,将其进行归一化 $\ket{\Psi_0}=\Psi(\bvec r,t)/\sqrt{\braket{\Psi(\bvec r,t)}}$.
\item 把 $\Psi_1$ 进行施密特正交化, 令
\begin{equation}
\ket*{\tilde\Psi_1} = (1 - \ket{\Psi_0}\bra{\Psi_0})H \ket{\Psi_0}
\qquad
\ket{\Psi_1}=\ket*{\tilde\Psi_1}/\sqrt{\braket*{\tilde\Psi_1}}
\end{equation}
% 未完成
\item 现在起我们令 $\beta_j = \sqrt{\braket*{\tilde\Psi_j}}$. % 未完成
\end{enumerate}

\subsubsection{定理1}
对以上定义的基底
\begin{equation}
\qty(1 - \ket{\Psi_j} \bra{\Psi_j} - \ket{\Psi_{j - 1}} \bra{\Psi_{j - 1}} - \ldots - \ket{\Psi_0} \bra{\Psi_0}) H^{j + 1} \ket{\Psi_0}
\end{equation}
与
\begin{equation}
\qty( 1 - \ket{\Psi_j}\bra{\Psi_j} - \ket{\Psi_{j - 1}}\bra{\Psi_{j - 1}} - \ldots - \ket{\Psi_0}\bra{\Psi_0}) H \ket{\Psi_j}
\end{equation}
共线.故现在起采用后者进行正交化.

\subsubsection{定理 2}
\begin{equation}
\bra{\Psi_{j-1}} H \ket{\Psi_j}  = \sqrt{\braket*{\tilde\Psi_j}} = \beta_j
\end{equation}

\begin{enumerate}[resume]
\item 把 $H^2 \ket{\Psi_0}$ 进行施密特正交化, 令
\begin{equation}
\alpha_i = \mel{\Psi_i}{H}{\Psi_i}
\end{equation}
根据两个定理,正交化结果也可以写成
\begin{equation}\ali{
\ket*{\tilde\Psi_2}  &= \qty(1 -\ket{\Psi_1}\bra {\Psi_1}  - \ket{\Psi_0}\bra{\Psi_0}) H \ket{\Psi_1}\\
&= H \ket{\Psi_1} - \alpha_1 \ket{\Psi_1} - \beta_1 \ket{\Psi_0} \\ 
}\end{equation}
\item 再次归一化  $\ket{\Psi_2} = \ket*{\tilde\Psi_2} / \sqrt{\braket*{\tilde\Psi_2}} = \ket*{\tilde\Psi_2}/\beta_2$
\end{enumerate}

\subsubsection{定理 3}
\begin{equation}\ali{
\ket*{\tilde\Psi_{j+1}} &= \qty(1 - \ket{\Psi_j}\bra{\Psi_j} - \ldots - \ket{\Psi_0}\bra{\Psi_0}) H \ket{\Psi_j}\\
&= \qty(1 - \ket{\Psi_j}\bra{\Psi_j} - \ket{\Psi_{j-1}}\bra{\Psi_{j-1}}) H \ket{\Psi_j} \\
&= H \ket{\Psi_j} - \alpha_j \ket{\Psi_j} - \beta_j \ket{\Psi_{j-1}}
}\end{equation}
这就是最简洁的正交化公式.比起最原始的正交化
\begin{equation}
\ket*{\tilde\Psi_{j+1}} = \qty(1 - \ket{\Psi_j}\bra{\Psi_j}  - \ldots - \ket{\Psi_0}\bra{\Psi_0}) H^{j+1} \ket{\Psi_0}
\end{equation}
定理 3 只用到了一个矩阵-矢量乘法,两个矢量数乘和两个矢量减法.

接下来只要对 $j = 2,4,5...K - 1$ 不断重复定理 3 中的正交化和归一化,就可以得到所有正交归一基底 $\ket{\Psi_j} $.

现在我们来求该基底下的 $\mat H$ 矩阵.根据 $\alpha_i$ 的定义以及定理 2 可以发现它们分别是矩阵的主对角元和副对角元.

\subsubsection{定理4}
在正交归一基底 $\ket{\Psi_j}$ 中, 矩阵 $\mat H$ 是对称三对角矩阵.所以我们已经顺便求出了所有的矩阵元!
\begin{equation}
\mat H =
\begin{pmatrix}
\alpha_0 & \beta_1 &&& \\ 
\beta_1 & \alpha_1 & \beta_2 && \\ 
 & \beta_2 & \alpha_2 & \ddots&  \\ 
& & \ddots& \ddots & \beta_{K-1} \\
&&&\beta_{K-1} & \alpha_{K-1}
\end{pmatrix}\end{equation}
该矩阵具有维数小,易求本征问题的优势.

\subsection{定理1证明}

根据施密特正交化的性质,Krilov 基底的前 $j$ 项的与 $\ket{\Psi_0}\dots\ket{\Psi_j}$ 展开同一空间.所以 $H^j\ket{\Psi_0} = c_j\ket{\Psi_j} +\ldots + c_0 \ket{\Psi_0}$,所以(以下所有系数 只是用来表示线性组合,具体值不重要)
\begin{equation}\ali{
H^{j+1} \ket{\Psi_0} &= c_j H \ket{\Psi_j} + H \qty(c_{j-1} \ket{\Psi_{j-1}} \dots + c_0 \ket{\Psi_0}) \\
& = c_j H \ket{\Psi_j} + H \qty(c'_{j-1} H^{j-1} \ket{\Psi_0} + c'_{j-2} H^{j - 2} \ket{\Psi_0} \dots + c'_0 \ket{\Psi_0})  \\
&= c_j H \ket{\Psi_j} + \qty(c'_{j-1} H^j \ket{\Psi_0} + c'_{j-2} H^{j-1} \ket{\Psi_0} \dots + c'_0 \ket{\Psi_0})  \\
&= c_j H \ket{\Psi_j} + \qty(c''_{j-1} \ket{\Psi_j} + c''_{j-2} \ket{\Psi_{j-1}}+ \ldots + c''_0 \ket{\Psi_0}) \\ 
}\end{equation}
把 $H^{j+1}\ket{\Psi_0}$ 施密特正交归一化
\begin{equation}\ali{
&\phantom{=} \qty(1 - \ket{\Psi_j}\bra{\Psi_j} - \ldots - \ket{\Psi_0}\bra{\Psi_0}) H^{j+1} \ket{\Psi_0} \\
&= c_j \qty(1 - \ket{\Psi_j}\bra{\Psi_j} - \ldots - \ket{\Psi_0}\bra{\Psi_0}) H \ket{\Psi_j}\\
&\phantom{=} + \qty(1 - \ket{\Psi_j} \bra{\Psi_j} - \ldots - \ket{\Psi_0} \bra{\Psi_0}) \qty(c''_{j-1} \ket{\Psi_j} + c''_{j - 2} \ket{\Psi_{j - 1}} + \ldots + c''_0 \ket{\Psi_0}) \\
&= c_j \qty(1 - \ket{\Psi_j}\bra{\Psi_j} - \ldots - \ket{\Psi_0}\bra{\Psi_0}) H \ket{\Psi_j}
}\end{equation}

\subsection{定理 2 证明}
要证 $\bra{\Psi_{j-1}} H \ket{\Psi_j}  = \sqrt{\braket*{\tilde\Psi_j}}$, 即证 $\mel*{\Psi_{j - 1}}{H}{\tilde \Psi_j} = \braket*{\tilde\Psi_j}$, 即证$\braket*{\tilde\Psi_j} = \braket*{\tilde\Psi_j}{H\Psi_{j-1}}$ 而
\begin{equation}
\braket*{\tilde\Psi_j} = \mel*{\tilde\Psi_j}{\qty(1-\ket{\Psi_{j-1}}\bra{\Psi_{j-1}} - \ldots - \ket{\Psi_0}\bra{\Psi_0}) H}{\Psi_{j-1}}
\end{equation}
由于上式中 $\bra*{{\tilde \Psi }_j} \qty(\ket{\Psi_{j - 1}} \bra{\Psi_{j - 1}} - \ldots - \ket{\Psi_0} \bra{\Psi_0}) = 0$
\begin{equation}
\braket*{\tilde\Psi_j} = \mel*{\tilde\Psi_j}{H}{\Psi_{j-1}}
\end{equation}

\subsection{定理 3 证明}

要证明定理 3,即证,对 $n \leqslant j - 2$, 有 $\bra{\Psi_n} H \ket{\Psi_j} = 0$, $H$ 为厄米算符时即证 $\bra{\Psi_j} H \ket{\Psi_n} = 0$,即证对 $m \geqslant j + 2$,  $\bra{\Psi_m} H \ket{\Psi_j} = 0$. 在定理 1 的证明类似,我们知道
\begin{equation}\ali{
H \ket{\Psi_j}  &= H \qty(c_j H^j \ket{\Psi_0} + \ldots + c_0 \ket{\Psi_0}) = c_j H^{j+1} \ket{\Psi_0} + \ldots + c_0 H \ket{\Psi_0}\\
&= c'_j \ket{\Psi_{j+1}} + \ldots + c'_0 \ket{\Psi_0}\\ 
}\end{equation}
所以对 $m \geqslant j + 2$ 有
\begin{equation}
\bra{\Psi_m} H \ket{\Psi_j} = \bra{\Psi_m} \qty(c'_j \ket{\Psi_{j+1}} + \ldots + c'_0 \ket{\Psi_0}) = 0
\end{equation}

\subsection{定理 4 证明}

首先证明 $\mat H$ 是三对角矩阵.即证 $m \geqslant j + 2$ 或 $m \leqslant j - 2$ 时 $\bra{\Psi_m} H \ket{\Psi_j} = 0$.
$m \geqslant j + 2$ 的情况在证明3中已经证明,只需证明另一种情况. 对厄米矩阵, $\braket*{H\Psi_m}{\Psi_j} = \mel{\Psi_m}{H}{\Psi_j} = 0$, 取复共轭,即 $\mel{\Psi_j}{H}{\Psi_m} = 0$. 可见左边的角标的确小于等于右边的角标减 2,证毕.

然后证明 $\mat H$ 是实矩阵,考虑到厄米矩阵 $H_{mn} = H_{nm}^* $ 的性质,只需要证明 $H_{j, j+1}$ 是实数.对任意 $j = 2 \dots K-1$
\begin{equation}\ali{
H_{j, j+1} &= \mel{\Psi_j}{H}{\Psi_{j+1}} = \bra{\Psi_j} H \qty(H \ket{\Psi_j} - \alpha_j \ket{\Psi_j} - \beta_j \ket{\Psi_{j-1}}) \\
&= \mel{\Psi_j}{H^2}{\Psi_j} - \alpha_j \bra{\Psi_j} H \ket{\Psi_j} - \beta_j \bra{\Psi_j}  H \ket{\Psi_{j-1}}
}\end{equation}
易证 $\ev{H^2}{\Psi_j}$, $\alpha_j$ 及 $\beta_j$ 是实数,只需要证明 $\mel{\Psi_j}{H}{\Psi_{j-1}}$ 是实数即可,即证 $\mel{\Psi_{j-1}}{H}{\Psi_j}$ 是实数,重复上述论证,即证 $\mel{\Psi_{j-2}}{H}{\Psi_{j - 1}}$ 是实数,\dots\dots,即证 $\mel{\Psi_1}{H}{\Psi_0}$ 是实数,即证 $\mel{\Psi_0}{H}{\Psi_1}$ 是实数,即证 $\mel*{\Psi_0}{H}{\tilde\Psi_1}$ 是实数.而
\begin{equation}
\mel*{\Psi_0}{H}{\tilde\Psi_1} = \bra{\Psi_0} H \qty(H \ket{\Psi_0} - \alpha_0 \ket{\Psi_0}) = \ev{H^2}{\Psi_0} - \alpha_0 \ev{H}{\Psi_0} 
\end{equation}
显然是实数.以上的思路是数学归纳法的逆过程.
