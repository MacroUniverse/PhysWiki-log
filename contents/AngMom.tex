% 角动量

\pentry{质点的角动量\upref{AMLaw1}}

\subsection{系统的角动量}
角动量是矢量,若把系统看做质点系,则系统的角动量等于所有质点的角动量矢量叠加.
\begin{equation}
\vec L = \sum_i \vec L_i = \sum_i \vec r_i \cross\vec p_i
\end{equation}

\subsection{角动量的坐标系变换}
可类比力矩的坐标系变换(\autoref{Torque_eq5}),坐标系 $A$ 中总角动量为
\begin{equation}
\vec L_A = \sum_i \vec r_{Ai} \cross \vec p_i 
\end{equation}
变换到坐标系 $B$ 中,总角动量为
\begin{equation}\label{AngMom_eq4}
\vec L_B = \sum_i (\vec r_{BA} + \vec r_{Ai})\cross \vec p_i = \vec r_{BA}\cross \sum_i \vec p_i + \vec L_A
\end{equation}

\subsection{角动量的分解}
质心系中的角动量为
\begin{equation}\label{AngMom_eq5}
\vec L_0 = \sum_i \vec r_{ci} \cross \vec p_i
\end{equation}
定义\bb{质心角动量}为“ 质心处具有系统总质量 $M$ 的质点的角动量” (类比质心动量的定义, \autoref{SysP_eq2}\upref{SysP})
\begin{equation}\label{AngMom_eq6}
\vec L_c  = \vec r_c \cross (M \vec v_c) = \vec r_c \cross \vec p_c
\end{equation}

现在我们变换到任意坐标系中,令总角动量为 $\vec L$,由\autoref{AngMom_eq4} 得
\begin{equation}
\vec L = \vec r_{c} \cross \sum_i \vec p_i + \vec L_0
\end{equation}
由于系统总动量 $\sum_i \vec p_i$ 等于质心动量 $\vec p_c$,右边第一项等于质心角动量\autoref{AngMom_eq6}.最后得到
\begin{equation}
\vec L = \vec L_c + \vec L_0
\end{equation}
所以\bb{任何坐标系中,系统的总角动量等于其质心的角动量加上相其相对质心的角动量}.
