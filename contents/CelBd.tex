% 开普勒问题

\pentry{中心力场问题\upref{CenFrc}, 万有引力\upref{Gravty}, 开普勒三定律\upref{Keple}, 双曲线的三种定义\upref{Hypb3}}

在中心力场问题\upref{CenFrc} 中, 若中心力场是平方反比力, 即
\begin{equation}
F(r) = -\frac{k}{r^2}  \qquad V(r) = -\frac{k}{r}
\end{equation}
(其中 $k$ 为大于零的常数) 则该问题被称为\bb{开普勒问题}. 对于万有引力, 有 $k = GMm$, 对于电荷间的库伦引力, 有\footnote{高中所学的库伦定律的系数 $k$ 在大学物理中通常记为 $1/(4\pi\epsilon_0)$, 其中 $\epsilon_0$ 为真空中的电介质常数.} $k = Qq/(4\pi\epsilon_0)$.

\subsection{结论}
在开普勒问题中, 能量和角动量可以唯一地确定行星轨道. 当质点能量小于零时轨道为椭圆($e<1$)
\begin{equation}
E=-\frac{k}{2a} \qquad  L = b \sqrt{\frac{mk}{a}}
\end{equation}
能量等于零时轨道为抛物线($e=1$), 焦距与角动量的关系为($p$ 为焦距)
\begin{equation}
L = \sqrt{2mkp}
\end{equation}
能量大于零时轨道为双曲线($e>1$).
\begin{equation}
E=\frac{k}{2a}  \qquad  L = b \sqrt{\frac{mk}{a}}
\end{equation}
根据圆锥曲线的性质, 对椭圆有 $a^2=b^2+c^2$,双曲线有 $c^2=a^2+b^2$
现在我们假设已知轨道是椭圆,抛物线和双曲线的一种(见开普勒第一定律的证明\upref{Keple1}),证明以上关系

\subsection{椭圆轨道}
令椭圆轨道距离焦点的最近和最远距离分别为 $r_1$ 和 $r_2$,总能量(动能加势能)守恒
\begin{equation}\label{CelBd_eq4}
\frac12 m v_1^2 - \frac{GMm}{r_1} = \frac12 mv_2^2 - \frac{GMm}{r_2}
\end{equation}
总角动量守恒
\begin{equation}\label{CelBd_eq5}
mv_1 r_1 = mv_2 r_2
\end{equation}
把\autoref{CelBd_eq5} 中的 $v_2$ 代入\autoref{CelBd_eq4},可得
\begin{equation}\label{CelBd_eq6}
v_1^2 = \frac{2GM}{r_1 + r_2} \frac{r_2}{r_1}
\end{equation}
再代入\autoref{CelBd_eq4} 的左边,并使用 $r_1+r_2=2a$ %未完成, 此处该引用公式
得到总能量 $E=-GMm/(2a)$.把\autoref{CelBd_eq6} 代入\autoref{CelBd_eq5} 的左边,并使用 $r_1 r_2 = (a+c)(a-c) =b^2$ %未完成: 此处该引用公式
得总角动量 $L = bm \sqrt{GM/a}$.

\subsection{双曲线}
注意双曲线轨道只可能是双曲线离中心天体所在焦点较近的一支,另一支这里没有物理意义(当引力变为斥力,只能取双曲线的另一支)

总能量守恒
\begin{equation}
\frac12 mv_0^2 = \frac12 mv_1^2 - \frac{GMm}{r_1}
\end{equation}
总角动量守恒
\begin{equation}
m v_0 b = m v_1 r_1
\end{equation}
% 未完成

\subsection{抛物线}
抛物线轨道离焦点的最近距离为焦距 $p$,

% 未完成