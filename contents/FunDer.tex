% 基本初等函数的导数

\pentry{导数\upref{Der}}

\subsection{基本初等函数}
\textbf{基本初等函数}由以下五类函数构成 ($a$ 是常数)

\begin{itemize}
\item 幂函数
\begin{equation}
{x^a} \quad(a \in R)
\end{equation}
\item 指数函数
\begin{equation}
{a^x} \quad(a > 0 , a \ne 1)
\end{equation}
\item 对数函数
\begin{equation}
{\log _a}x \quad(a > 0 , a \ne 1)
\end{equation}
当底为 $a = \E$ 时,叫做\textbf{自然对数函数}, 记为 $\ln x$. \\
\item 三角函数
\begin{equation}
\sin x \qquad  \cos x \qquad \tan x
\end{equation}
\item 反三角函数
\begin{equation}
y = \arcsin x  \qquad \arccos x  \qquad \arctan x
\end{equation}
\end{itemize}

由以上函数经过有限次四则运算和有限次函数复合所构成并可用一个式子表示的函数,称为\textbf{初等函数}.例如
\begin{equation}
y = \sqrt {1 - {x^2}}\qquad y = {\sin ^2}x\qquad y = \sqrt {\cot \frac{x}{2}} 
\end{equation}



\subsection{基本初等函数的导数}

基本初等函数在其定义域内都是可导的,导函数如下\\
\textbf{幂函数}
\begin{equation}
\left( {{x^a}} \right)' = a {x^{a - 1}}   \left( {a \in R} \right)
\end{equation}
\textbf{三角函数}
% 公式不美丽
\begin{equation}
\sin' x = \cos x \qquad \cos' x =  - \sin x \qquad
\tan'x = 1/{\cos ^2}x = {\sec ^2}x
\end{equation}
\textbf{指数函数}
\begin{equation}
{\left( {{a^x}} \right)^\prime } = \ln \left( a \right){a^x}
\end{equation}
特殊地,${\left( {{\E^x}} \right)^\prime } = {\E^x}$\\
\textbf{对数函数}
\begin{equation}
{\left( {{{\log }_a}x} \right)^\prime } = \frac{1}{{\ln \left( a \right)x}}
\end{equation}
特殊地,$\ln' x= 1/x$.

\subsection{幂函数证明}
由导数的代数定义, $f'\left( x \right) = \mathop {\lim }\limits_{h \to 0} [{f\left( {x + h} \right) - f\left( x \right)}]/{h}$,而
\begin{equation}
{\left( {x + h} \right)^a} - {x^a} = {x^a}[{\left( {1 + h/x} \right)^a} - 1]
\end{equation}
由于 $h \to 0$,  $h/x \to 0$. 令 $\varepsilon  = h/x$, 由非整数二项式定理\upref{BiNor},
\begin{equation}
{\left( {1 + \varepsilon } \right)^a} = 1 + a\varepsilon  + \frac{{a\left( {a - 1} \right)}}{{2!}}{\varepsilon ^2} + \frac{{a\left( {a - 1} \right)\left( {a - 2} \right)}}{{3!}}{\varepsilon ^3}\dots
\end{equation}
所以
\begin{equation}\begin{aligned}
{\left( {{x^a}} \right)^\prime } &= {x^a}\mathop {\lim }\limits_{\varepsilon  \to 0} \frac{{{{\left( {1 + \varepsilon } \right)}^a} - 1}}{{\varepsilon x}} \\
&= {x^{a - 1}}\mathop {\lim }\limits_{\varepsilon  \to 0} \left( {a + \frac{{a\left( {a - 1} \right)}}{{2!}}\varepsilon  + \frac{{a\left( {a - 1} \right)\left( {a - 2} \right)}}{{3!}}{\varepsilon ^2}\dots} \right) = a{x^{a - 1}}
\end{aligned}\end{equation}

\subsection{正弦函数证明}
使用三角函数和差化积公式%未完成:词条
%\begin{equation}
%\sin a + \sin b = 2\sin \frac{{a + b}}{2}\cos \frac{{a - b}}{2}
%\end{equation}
化简极限
\begin{equation}
\sin'x = \mathop {\lim }\limits_{h \to 0} \frac{{\sin (x + h) - \sin x}}{h} = \mathop {\lim }\limits_{h \to 0} \frac{{\sin (h/2)}}{{h/2}}\cos \left( {x + \frac{h}{2}} \right)
\end{equation}
由小角正弦值极限\upref{LimArc}中的结论,其中
\begin{equation}
\lim\limits_{h \to 0} \frac{\sin (h/2)}{h/2} = 1
\end{equation}
所以
\begin{equation}
\sin'x =  \mathop {\lim }\limits_{h \to 0}\cos \left( {x + \frac{h}{2}} \right) = \cos x
\end{equation}

\subsection{余弦函数证明}
若 $f'\left( x \right) = g\left( x \right)$, 且 $b$ 为任意常数,根据导数的定义 $f'(x + b) = g(x + b)$ 同样成立(证明略).所以 $\sin'(x + \pi/2) = \cos(x + {\pi }/{2})$. 而 $\sin(x + {\pi }/{2}) = \cos x$,  $\cos(x + {\pi }/{2}) =  - \sin x$ 所以 $\cos' x =  - \sin x$

\subsection{正切函数证明}
根据求导法则\upref{DerRul}%未完成:词条
,因为 $\tan x = {{\sin x}}/{{\cos x}}$, 所以
 \begin{equation}
{\mathop{\rm ta}\nolimits} n'x = \frac{{{\mathop{\rm si}\nolimits} n'x\cos x - {\mathop{\rm co}\nolimits} s'x\sin x}}{{{{\cos }^2}x}} = \frac{{{{\cos }^2}x + {{\sin }^2}x}}{{{{\cos }^2}x}} = \frac{1}{\cos ^2 x} = {\sec ^2}x
\end{equation}
\subsection{对数函数证明}
先证明 $\ln' x = {1}/{x}$.  $\ln \left( {x + h} \right) - \ln x = \ln \left( {1 + h/x} \right)$, 所以
 \begin{equation}
\ln 'x = \mathop {\lim }\limits_{h \to 0} \frac{{\ln \left( {x + h} \right) - \ln x}}{h} = \frac{1}{x}\mathop {\lim }\limits_{h \to 0} \frac{{\ln \left( {1 + h/x} \right)}}{{h/x}}
\end{equation}
令 $\varepsilon  = h/x$, 则
\begin{equation}
\ln' x = \frac{1}{x}\mathop {\lim }\limits_{\varepsilon  \to 0} \frac{{\ln \left( {1 + \varepsilon } \right)}}{\varepsilon } = \frac{1}{x}\mathop {\lim }\limits_{\varepsilon  \to 0} \ln {\left( {1 + \varepsilon } \right)^{\frac{1}{\varepsilon }}} 
\end{equation}
 
由自然对数底的定义, $\mathop {\lim }\limits_{\varepsilon  \to 0} {\left( {1 + \varepsilon } \right)^{\frac{1}{\varepsilon }}} = \E$, 所以
 \begin{equation}
\ln 'x = \frac{{\ln \E}}{x} = \frac{1}{x}
\end{equation}
再证明 ${\log'_a}x = {1}/{(x\ln a)}$. 
由对数函数的性质 ${\log _a}b = {{\ln b}}/{{\ln a}}$
\begin{equation}
{\mathop{\rm lo}\nolimits} {g'_a}x = {\left( {\frac{{\ln x}}{{\ln a}}} \right)^\prime } = \frac{1}{{\ln a}}{\mathop{\rm l}\nolimits} n'x = \frac{1}{{x\ln a}}
\end{equation}

\subsection{指数函数证明}
先证明 ${\left( {{\E^x}} \right)^\prime } = {\E^x}$. 
由于上面已经证明了 $ \ln'x = {1}/{x}$, 而 ${\E^x}$ 是 $\ln x$ 的反函数.所以令 $f\left( x \right) = \ln x$, $f'\left( x \right) = 1/x$,  ${f^{ - 1}}\left( x \right) = {\E^x}$, 代入反函数的求导法则\upref{InvDer}%未完成:词条
\begin{equation}
[{f^{ - 1}}\left( x \right)]' = \frac{1}{{f'[{{f^{ - 1}}(x)} ]}} 
\end{equation} 
得
\begin{equation}
{\left( {{\E^x}} \right)^\prime } = \frac{1}{{1/{\E^x}}} = {\E^x} 
\end{equation}
再证明 ${\left( {{a^x}} \right)^\prime } = {a^x}\ln a$.  ${\left( {{a^x}} \right)^\prime } = {\left[ {{{\left( {{\E^{\ln a}}} \right)}^x}} \right]^\prime } = {\left( {{\E^{\left( {\ln a} \right)x}}} \right)^\prime }$. 把 ${\E^{\left( {\ln a} \right)x}}$ 看成是 ${\E^u}$ 和 $u = \left( {\ln a} \right)x$ 的复合函数,根据复合函数的求导法则\upref{DerRul}%未完成:词条
, ${\left( {{a^x}} \right)^\prime } = \left( {\ln a} \right){a^x}$ 



















