% 电感
% 电感|法拉第|电磁感应|安培环路|感生电动势

\pentry{安培环路定理\upref{AmpLaw}, 法拉第电磁感应定律\upref{FaraEB}}

\textbf{电感器}是电路中的一个原件, 当通过它的电流增加时, 它的两端会产生反向电动势. 我们用物理量\textbf{电感}来描述电感器的这种性质, 记为 $L$, 定义如下
\begin{equation}
U = -L\dv{I}{t}
\end{equation}
也就是说反向电动势与电流的时间导数成正比, 比例系数就是电感.

\subsection{简单的电感模型}
圆柱形均匀缠绕的线圈, 单位长度线圈数 $n$, 长度为 $l$, 忽略边缘效应. 取方形回路(图未完成), 由安培环路定理(\autoref{AmpLaw_eq1}\upref{AmpLaw})
\begin{equation}
Bl = \mu_0nlI \Rightarrow B = \mu_0nI
\end{equation}
磁通量为
\begin{equation}
\Phi = BS = \mu_0nSI
\end{equation}
感生电动势为
\begin{equation}
U = -nL\dv{\Phi}{t} = -\mu_0n^2lS \dv{I}{t}
\end{equation}
令电感为
\begin{equation}
L = \mu_0n^2lS
\end{equation}

\begin{equation}
U = -L\dv{I}{t}
\end{equation}
