% 点集的内部、外部和边界

拓扑空间定义了什么样的集合算开集,而开集的补集被称为闭集.除此之外,还有一些集合是既不开也不闭的.比如说,在$\mathcal{R}$上定义的度量拓扑中,开区间都是开集,闭区间和孤立点都是闭集,但是半开半闭区间两者都不属于.

\begin{definition}{}
给定拓扑空间$(X, \mathcal{T})$以及它的一个子集$A$.
\begin{itemize}
\item 如果对于$x\in X$,存在一个开集$V\in \mathcal{T}$使得$x\in V\subset A$,那么称这个$x$是$A$的一个\textbf{内点}.
\item 如果存在一个开集$V\in \mathcal{T}$使得$x\in V\subset A^C$,那么称这个$x$是$A$的一个\textbf{外点}.
\item 如果一个$x$既不是$A$的外点也不是内点,那么称$x$是$A$的一个\textbf{边界点.}

\item 全体内点构成的集合,称为$A$的\textbf{内部},记为$A^\circ$.
\item 全体外点构成的集合,称为$A$的\textbf{外部}.
\item 全体边界点构成的集合,称为$A$的\textbf{边界}.

\end{itemize}
\end{definition}{}
