% 积分表
% 同济高数上面的积分例题拿一些典型的来演示积分方法就好了!另外就是书中出现的积分! 其他的积分一律不要给出!
% 数学从简!不要浪费时间!
% 然而书中的任何积分一律要引用积分表!要么就说明 Wolfram

\pentry{不定积分\upref{Int}}

这里给出一个基本积分表和一个常用积分表,前者建议熟记.部分积分有的给出计算步骤,没有给出则是由基本初等函数的导数\upref{FunDer}直接逆向得出.所有的不定积分公式都可以通过求导验证.

应用换元积分法\upref{IntCV}, 表中任何积分都可以拓展为
\begin{equation}\label{ITable_eq1}
\int{f(ax+b) \D x} = \frac 1a F(ax+b)
\end{equation}

\subsection{基本积分表}

\begin{align}
&\int {{x^a}\D x}  = \frac{1}{{a + 1}}{x^{a + 1}} + C \quad(a \in R, a \ne  - 1)
\\
&\int {\frac{1}{x}\D x} = \ln\abs{x} + C \quad\text{(\autoref{ITable_ex11})}\label{ITable_eq10}
\\
&\int \cos x\D x = \sin x + C \label{ITable_eq4}
\\
&\int \sin x\D x =  - \cos x + C
\\
&\int \tan x\D x =  -\ln\abs{\cos x} + C \quad\text{(\autoref{ITable_ex2})}
\\
&\int {\cot x\D x} = \ln \abs{\sin x} + C \quad\text{(\autoref{ITable_ex7})}
\\
&\int \frac{1}{\cos^2 x}\D x = \tan x + C
\\
&\int \frac{1}{1 + x^2}\D x = \arctan x + C
\\
&\int {{\E^x}\D x} = {\E^x} + C \label{ITable_eq9}
\\
&\int {x{\E^x}\D x} =\E^x (x-1) + C \quad\text{(\autoref{ITable_ex5})}
\\
&\int {{a^x}\D x} = \frac{1}{\ln a}{a^x} + C \quad\text{(\autoref{ITable_ex1})}
\end{align}

\subsection{常用积分表}

\begin{align}
&\int \sin^2 x \D x = \frac 12 (x - \sin x\cos x) + C \quad\text{(\autoref{ITable_ex3})}
\\
&\int \cos^2 x \D x = \frac 12 (x + \sin x \cos x) + C \quad\text{(\autoref{ITable_ex4})}
\\
&\int \sec x \D x = \ln\abs{\tan x + \sec x} + C \quad\text{(\autoref{ITable_ex10})}
\\
&\int \ln x \D x = x\ln x - x + C \quad\text{(\autoref{ITable_ex6})}
\\
&\int \frac{1}{\sqrt{1-x^2}} \D x = \arcsin(x) + C \quad\text{(\autoref{ITable_ex8})}
\\
&\int \frac{1}{\sqrt{1+x^2}} \D x = \ln(1+\sqrt{1+x^2}) = \sinh^{-1}(x) + C \quad\text{(\autoref{ITable_ex9})}
\end{align}

\begin{exam}{}\label{ITable_ex1}
\begin{equation}
\int {{a^x}\D x}
\end{equation}
我们已经知道如何算 $\E^x$ 的积分, 而 $a = \E ^{\ln a}$, 再根据\autoref{ITable_eq1} 就有
\begin{equation}
\int \E^{\ln (a) x}\D x = \frac{1}{\ln a}\E^{\ln (a) x} + C = \frac{1}{\ln a}a^x + C
\end{equation}
\end{exam}

\begin{exam}{}\label{ITable_ex2}
\begin{equation}
\int \tan x\D x
\end{equation}
这个积分用第一类换元积分法(\autoref{IntCV_eq2}\upref{IntCV})
\begin{equation}
\int f[u(x)]u'(x) \D x  = F[u(x)] + C
\end{equation}
首先 $\tan x = \sin x/ \cos x$ , 令 $u(x) = \cos x$, 则 $\sin x = -u'(x)$, 对比得 $f(x) = -1/x$ 其原函数为 $F(x) = -\ln\abs{x}$, 所以
\begin{equation}
\int \tan x\D x = \int f[u(x)] u'(x) \D x = F[u(x)] + C = -\ln\abs{\cos x} + C
\end{equation}
\end{exam}

\begin{exam}{}\label{ITable_ex7}
类似\autoref{ITable_ex2}, $\cot x = \cos x/\sin x$, 令 $u(x) = \sin x$, 则 $\cos x = u'(x)$, 对比得 $f(x) = 1/x$, 原函数为 $F(x) = \ln\abs{x}$ (\autoref{ITable_eq10}) , 所以
\begin{equation}
\int \cot x\D x = F[u(x)] + C = \ln\abs{\sin x} + C
\end{equation}
\end{exam}

\begin{exam}{}\label{ITable_ex3}
\begin{equation}
\int \sin^2 x  \dd{x}
\end{equation}
用降幂公式(\autoref{TriEqv_eq5}\upref{TriEqv}) 和不定积分的线性(\autoref{Int_eq4}\upref{Int}) 把上式变为常数的积分和 $\cos 2x$ 的积分, 再利用\autoref{ITable_eq4} 和\autoref{ITable_eq1} 计算后者即可
\begin{equation}\begin{aligned}
\int \sin^2 x \D x &=  \int \frac 12 \D x - \frac 12\int \cos 2x \dd{x} \\
&=  \frac {x}{2} - \frac 14\sin(2x) = \frac 12 (x - \sin x \cos x) + C
\end{aligned}\end{equation}
\end{exam}

\begin{exam}{}\label{ITable_ex4}
\begin{equation}
\int \cos^2(x) \D x
\end{equation}
与\autoref{ITable_ex3} 类似, 用三角恒等式 $\cos^2(x) =  [1 + \cos(2x)]/2$ 得
\begin{equation}\begin{aligned}
\int \cos^2 x \D x &=  \int \frac 12 \D x + \frac 12\int \cos(2x) \D x \\
&=  \frac {x}{2} + \frac 14\sin(2x) = \frac 12 (x + \sin x \cos x) + C
\end{aligned}\end{equation}
\end{exam}

\begin{exam}{}\label{ITable_ex11}
\begin{equation}
\int \frac 1x \D x
\end{equation}
首先在区间 $(0,+\infty)$ 内, 由于 $\ln x$ 的导数是 $1/x$, 所以积分结果为 $\ln x + C$. 现在再来考虑区间 $(-\infty, 0)$, 注意 $\ln x$ 在这里没有定义, 不妨看看 $\ln(-x)$, 由复合函数求导% 未完成
, 其导数恰好为 $1/x$. 所以在除去原点的实数范围内, 有
\begin{equation}
\int \frac 1x \D x = \ln\abs{x} + C
\end{equation}
事实上, 由于 $1/x$ 在 $x=0$ 没有定义, 更广义的原函数可以取
\begin{equation}
\int \frac 1x \D x = \leftgroup{
&\ln x + C_1 \quad (x>0)\\
&\ln (-x) + C_2 \quad (x<0)
}
\end{equation}
其中 $C_1$ 和 $C_2$ 是两个不相同的待定常数.
\end{exam}

\begin{exam}{}\label{ITable_ex5}
\begin{equation}
\int x\E^x \D x
\end{equation}
使用用分部积分\autoref{IntBP_eq1}\upref{IntBP}
\begin{equation}
\int {F(x)g(x)\D x}  = F(x)G(x) - \int {f(x)G(x)\D x}
\end{equation}
令 $F(x) = x$, 求导得 $f(x) = 1$, 令 $g(x) = \E^x$, 由\autoref{ITable_eq9}, $G(x) = \E^x$. 代入分部积分得
\begin{equation}
\int x\E^x \D x = x\E^x - \int 1\cdot \E^x \D x = \E^x(x - 1) + C
\end{equation}
\end{exam}

\begin{exam}{}\label{ITable_ex6}
\begin{equation}
\int \ln x \D x
\end{equation}
\textbf{方法一:} 使用第二类换元法\autoref{IntCV_eq6}\upref{IntCV}
\begin{equation}
\int f(x)\D x = \int f[x(t)]\D [x(t)] = \int f[x(t)]x'(t)\D t
\end{equation}
令\footnote{注意被积函数只在 $x>0$ 区间有定义, 否则使用 $x = \E^t$ 将会自动忽略 $x\le 0$ 的情况.} $x = \E^t$, 求导得 $x'(t) = \E^t$, 换元得
\begin{equation}
\int \ln x \D x = \int \ln(\E^t) \E^t  \D t = \int t \E^t  \D t
\end{equation}
由\autoref{ITable_ex5} 中的分部积分得
\begin{equation}
\int \ln x \D x = \E^t (t-1) + C = \E^{\ln x} (\ln x -1) + C = x (\ln x-1) + C
\end{equation}
\textbf{方法二:} 直接使用分部积分法\autoref{IntBP_eq1}\upref{IntBP}, 对常数 1 积分, 对 $\ln x$ 求导, 得
\begin{equation}
\int \ln x \D x = x\ln x - \int x\cdot \frac 1x \D x = x\ln x - x + C
\end{equation}
\end{exam}

\begin{exam}{}\label{ITable_ex8}
\begin{equation}
\int \frac{1}{\sqrt{1-x^2}} \D x 
\end{equation}
使用第二类换元法\autoref{IntCV_eq6}\upref{IntCV}, 令 $x = \sin t$ 得
\begin{equation}
\int \frac{1}{\sqrt{1-\sin^2 t}} \dd(\sin t) = \int \dd{t} = t + C = \arcsin x + C
\end{equation}
\end{exam}

\begin{exam}{}\label{ITable_ex10}
\begin{equation}
\int \sec x \D x
\end{equation}
分子分母同时乘以 $\sec x + \tan x$, 可以发现分子是分母的导数. 再用第一类换元积分法(\autoref{IntCV_eq2}\upref{IntCV}) , 令 $u(x) = \sec x + \tan x$, 再使用\autoref{ITable_eq10} 即可
\begin{equation}\begin{aligned}
\int \sec x \D x &= \int \frac{\sec^2 x + \sec x\tan x}{\sec x + \tan x} \D x = \int \frac{u'(x)}{u} \D x = \int \frac 1u \D u \\
&= \ln\abs{u}+C = \ln\abs{\sec x + \tan x}+C
\end{aligned}\end{equation}
\end{exam}


\begin{exam}{}\label{ITable_ex9}
\begin{equation}
\int \frac{1}{\sqrt{1+x^2}} \D x
\end{equation}
使用第二类换元法\autoref{IntCV_eq6}\upref{IntCV}, 令 $x = \tan t$, 再利用三角恒等式\autoref{TriEqv_eq13}\upref{TriEqv} 和 \autoref{ITable_eq10} 得
\begin{equation}
\int \frac{1}{\sqrt{1+\tan^2 t}} \dd(\tan t) = \int \frac{1}{\sec t} \sec^2 t\dd{t}
 = \ln\abs{\tan t + \sec t} + C
\end{equation}
由同一三角恒等式, $\sec t = \sqrt{1+\tan^2 t} = \sqrt{1+x^2}$, 所以
\begin{equation}
\int \frac{1}{\sqrt{1+x^2}} \dd{x} = \ln(x + \sqrt{1+x^2}) + C
\end{equation}
注意上式中 $\ln$ 后面的绝对值符号消失是因为 $x + \sqrt{1+x^2}\ge 0 $ 恒成立. 另外由 $\sinh^{-1} x$ 函数的定义可知上式又等于 $\sinh^{-1} x + C$.
\end{exam}