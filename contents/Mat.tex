%矩阵

\pentry{线性变换\upref{LTrans}}

本书中\bb{矩阵}符号用加粗的正体字母来表示,而对应的\bb{矩阵元}一般用斜体加\bb{行标}和\bb{列标}表示.例如矩阵 $\mat A$ 的第 $i$ 行第 $j$ 列的的\bb{矩阵元}表示为 $A_{ij}$. 特殊地, 行数等于列数的矩阵叫做\bb{方阵}. 只有一行的矩阵和只有一列的矩阵分别叫做\bb{行矢量}和\bb{列矢量}.

\subsection{矩阵的转置}

我们先定义矩阵的\bb{对角线}是从左上角到右下角的所有矩阵元, 即行标等于列标的所有矩阵元. 则任意矩阵 $\mat A$ 的\bb{转置(Transpose)}记为 $\mat A\Tr$. 转置操作把 $\mat A$ 的第 $i$ 行变为 $\mat A\Tr$ 的第 $i$ 列,相当于把矩阵沿对角线翻转. 即任意矩阵元满足
\begin{equation}
A\Tr_{ij} = A_{ji}
\end{equation}
注意转置操作不影响对角线上的矩阵元. 另外行矢量转置后变为列矢量,反之亦然.
\begin{equation}\label{Mat_eq2}
(x_1,x_2 \dots, x_n)\Tr = \begin{pmatrix} x_1\\x_2\\ \vdots\\x_n \end{pmatrix}
\end{equation}
为了排版方便,本书在正文中通常用 $(x_1,x_2 \dots, x_n)\Tr$ 表示列矢量.

\subsection{矩阵的乘法}

矩阵最常见的运算是矩阵的乘法.线性变换\upref{LTrans}
\begin{equation}
\leftgroup{
y_1 &= A_{11}x_1 + A_{12}x_2 + \ldots + A_{1n}x_n\\
y_2 &= A_{21}x_1 + A_{22}x_2 + \ldots + A_{2n}x_n\\
&\;\;\vdots \\
y_m &= A_{m1}x_1 + A_{m2}x_2 + \ldots + A_{mn}x_n
}\end{equation}
可用矩阵与列矢量的乘法表示为
\begin{equation}
\begin{pmatrix} y_1 \\ y_2\\ \vdots \\ y_m \end{pmatrix}
= \begin{pmatrix}
A_{11}  & A_{12} & \ldots & A_{1n} \\
A_{21}  & A_{22} & \ldots & A_{2n} \\
 \vdots & \vdots  & \ddots & \vdots \\
A_{m1}  & A_{m2} & \ldots & A_{mn}
\end{pmatrix}
\begin{pmatrix} x_1 \\ x_2 \\ \vdots \\ x_n \end{pmatrix}
\end{equation}

令列矢量 $\mat x = (x_1,x_2 \dots, x_n)\Tr $,  $\mat y = (y_1,y_2 \dots, y_n)\Tr$, 系数矩阵为 $\mat A$, 上式可记为
\begin{equation}\label{Mat_eq5}
\mat y = \mat A\mat x
\end{equation} 
注意 $\mat A$ 的列数必须和 $\mat x$ 的行数相等.由此可以定义\bb{矩阵乘以列矢量}的运算规则:$m \times n$ 矩阵乘以 $n \times 1$ 列矢量会得到 $m \times 1$ 的列矢量.要计算 $y_i$, 就用 $m \times n$ 矩阵的第 $i$ 行的 $n$ 个数和 $x_1 \dots x_n$ 分别相乘再相加,即点乘\upref{Dot} 的代数定义
\begin{equation}
y_i = \sum_{j = 1}^n A_{ij} x_j 
\end{equation}
若有 $l$ 个不同的 $\vec x$ 和 $\vec y$, 第 $k$ 个记为 $\vec x_k = (x_{1k},\dots, x_{nk})\Tr$ 和 $\vec y_k = (y_{1k},\dots, y_{mk})\Tr$,对应的变换为
\begin{equation}
\begin{pmatrix} y_{1k} \\ y_{2k}\\ \vdots \\ y_{mk} \end{pmatrix}
= \begin{pmatrix}
A_{11}  & A_{12} & \ldots & A_{1n} \\
A_{21}  & A_{22} & \ldots & A_{2n} \\
 \vdots & \vdots  & \ddots & \vdots \\
A_{m1}  & A_{m2} & \ldots & A_{mn}
\end{pmatrix}
\begin{pmatrix} x_{1k} \\ x_{2k} \\ \vdots \\ x_{nk} \end{pmatrix}
\end{equation}
可以将所有的 $\vec x_k$ 和 $\vec y_k$ 分别横向拼成 $n \times l$ 和 $m \times l$ 的矩阵
\begin{equation}
\mat X =
\begin{pmatrix}
x_{11} & \cdots & x_{1l} \\
 \vdots & \ddots & \vdots \\
x_{n1} & \cdots & x_{nl}
\end{pmatrix}
\qquad
\mat Y =
\begin{pmatrix}
y_{11} & \cdots & y_{1l} \\
 \vdots & \ddots & \vdots \\
y_{m1} & \cdots & y_{ml}
\end{pmatrix}
\end{equation}
现在把 $l$ 组线性变换用一条式子表示为
\begin{equation}
\begin{pmatrix}
y_{11} & \cdots & y_{1l} \\
 \vdots & \ddots & \vdots \\
y_{m1} & \cdots & y_{ml}
\end{pmatrix}
=
\begin{pmatrix}
A_{11} & \cdots & A_{1n} \\
 \vdots & \ddots & \vdots \\
A_{m1} & \cdots & A_{mn}
\end{pmatrix}
\begin{pmatrix}
x_{11} & \cdots & x_{1l} \\
 \vdots & \ddots & \vdots \\
x_{n1} & \cdots & x_{nl}
\end{pmatrix}
\end{equation}
由此,可以定义一般的矩阵乘法: $m \times n$ 的矩阵 $\mat A$ 和 $n \times l$ 的矩阵 $\mat X$ 相乘得到 $m \times l$ 的矩阵 $\mat Y$,  $Y_{ij}$ 等于 $\mat A$ 的第 $i$ 行和 $\mat X$ 的第 $j$ 列点乘.
\begin{equation}
\mat Y=\mat A\mat X
\end{equation}
矩阵元公式为
\begin{equation}
Y_{ij} = \sum_{k = 1}^n A_{ik} X_{kj}
\end{equation}
再次注意两个相乘的矩阵,左边矩阵的列数必须等于右边矩阵的行数.
% 未完成:用图快速记忆矩阵的乘法.

根据定义,容易证明矩阵乘法\bb{满足分配律} $\mat A(\mat B+\mat C) = \mat A\mat B+\mat A\mat C$, 但一般\bb{不满足交换律},举一个反例:
\begin{equation}
\pmat{ 1 & 1\\ 0 & 0 }
\pmat{ 1 & 0\\ 1 & 0 } =
\pmat{ 2 & 0\\ 0 & 0 } \ne
\pmat{ 1 & 1\\ 1 & 1 } =
\pmat{ 1 & 0\\ 1 & 0 }
\pmat{ 1 & 1\\ 0 & 0 }
\end{equation}

\subsection{矩阵的乘法分配律}
下面证明矩阵的乘法分配律
\begin{equation}\label{Mat_eq13}
\mat A (\mat B + \mat C) = \mat A \mat B + \mat A \mat C
\end{equation}
\begin{equation}\label{Mat_eq14}
(\mat A + \mat B) \mat C = \mat A \mat C + \mat B \mat C
\end{equation}
令\autoref{Mat_eq13} 左边等于矩阵 $\mat D$, 则其矩阵元为
\begin{equation}
D_{ij} = \sum_k A_{ik} (B_{kj} + C_{kj})
\end{equation}
拆括号得
\begin{equation}
D_{ij} = \sum_k A_{ik}B_{kj} + \sum_k A_{ik}C_{kj}
\end{equation}
而这恰好是 $\mat A \mat B + \mat A \mat C$ 的矩阵元. 证毕.\autoref{Mat_eq14} 的证明类似.

\subsection{矩阵乘法的结合律}
现在来看三个矩阵相乘,令
\begin{equation}
\mat D = \mat A (\mat B\mat C)
\end{equation}
这里的括号是为了强调顺序.即使没有括号,习惯上也是从右向左计算. $\mat D$ 的矩阵元为
\begin{equation}
D_{ij} = \sum_l A_{il} (BC)_{lj} = \sum_l A_{il} \qty(\sum_k B_{lk} C_{kj} )
\end{equation}
拆括号,得
\begin{equation}
D_{ij} = \sum_k\sum_l  ( A_{il} B_{lk} C_{kj} )
\end{equation}
对 $C_{kj}$ 进行合并同类项,得
\begin{equation}
D_{ij} = \sum_k \qty(\sum_l A_{il} B_{lk} ) C_{kj} 
\end{equation}
括号中恰好是 $\mat A$ 乘以 $\mat B$ 所得矩阵的矩阵元 $(AB)_{ik}$ 所以
\begin{equation}
D_{ij} = \sum_k (AB)_{ik} C_{kj}
\end{equation}
即
\begin{equation}
\mat D = (\mat A\mat B)\mat C
\end{equation}
证毕.

\subsection{单位矩阵}
\bb{单位矩阵}就是对角线上的元素全为 1, 非对角线上的元素全为 0 的方阵. 通常记为通常记为$\mat I$. 为了强调矩阵的维数 $N$, 也可记为 $\mat I_N$. 单位矩阵的矩阵元可用克罗内克 $\delta$ 函数(\autoref{OrNrB_eq2}\upref{OrNrB})表示为
\begin{equation}
I_{ij} = \delta_{ij}
\end{equation} 
任何矩阵左乘或右乘单位矩阵, 仍然得到矩阵本身. 单位矩阵的转置仍为单位矩阵.


